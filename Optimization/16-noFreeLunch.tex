\providecommand{\slides}{
  \newcommand{\slideshead}{
  \newcommand{\thepage}{\arabic{mypage}}
  %beamer
%  \documentclass[t,hyperref={bookmarks=true}]{beamer}
%  \geometry{papersize={171mm,96mm}}
  \documentclass[t,hyperref={bookmarks=true},aspectratio=169]{beamer}
  \setbeamersize{text margin left=5mm}
  \setbeamersize{text margin right=5mm}
  \usetheme{default}
  \usefonttheme[onlymath]{serif}
  \setbeamertemplate{navigation symbols}{}
  \setbeamertemplate{itemize items}{{\color{black}$\bullet$}}

  \newwrite\keyfile

  %\usepackage{palatino}
  \stdpackages
  %\usepackage{tikz} \usetikzlibrary {shapes.geometric} 
  \usepackage{multimedia}
  \usepackage[utf8]{inputenc}

  %%% geometry/spacing issues
  %
  \definecolor{bluecol}{rgb}{0,0,.5}
  \definecolor{greencol}{rgb}{0,.6,0}
  \definecolor{citcol}{rgb}{.4,.4,.4}
  %\renewcommand{\baselinestretch}{1.1}
  \renewcommand{\arraystretch}{1.2}
  \columnsep 0mm

  \columnseprule 0pt
  \parindent 0ex
  \parskip 0ex
  %\setlength{\itemparsep}{3ex}
  %\renewcommand{\labelitemi}{\rule[3pt]{10pt}{10pt}~}
  %\renewcommand{\labelenumi}{\textbf{(\arabic{enumi})}}
  \setbeamertemplate{enumerate item}{(\roman{enumi})}
  \newcommand{\headerfont}{\helvetica{14}{1.5}{b}{n}}
  \newcommand{\slidefont} {\helvetica{11}{1.4}{m}{n}}
  %\newcommand{\codefont} {\helvetica{8}{1.2}{m}{n}}
  \newcommand{\urlfont} {\fontsize{6}{1.1}\selectfont}
  \renewcommand{\small} {\helvetica{10}{1.4}{m}{n}}
  \renewcommand{\tiny} {\helvetica{8}{1.3}{m}{n}}
  \newcommand{\ttiny} {\helvetica{7}{1.3}{m}{n}}

  %%% count pages properly and put the page number in bottom right
  %
  \newcounter{mypage}
  \newcommand{\incpage}{\addtocounter{mypage}{1}\setcounter{page}{\arabic{mypage}}}
  \setcounter{mypage}{0}
  \resetcounteronoverlays{page}

  \pagestyle{fancy}
  %\setlength{\headsep}{10mm}
  %\addtolength{\footheight}{15mm}
  \renewcommand{\headrulewidth}{0pt} %1pt}
  \renewcommand{\footrulewidth}{0pt} %.5pt}
  \cfoot{}
  \rhead{}
  \lhead{}
  %% \lfoot{\vspace*{-3mm}\hspace*{-3mm}\helvetica{5}{1.3}{m}{n}{\texttt{github.com/MarcToussaint/AI-lectures}}}
\lfoot{}
%  \rfoot{{\tiny\textsf{AI -- \topic -- \subtopic -- \arabic{mypage}/\pageref{lastpage}}}}
  %\lfoot{\raisebox{5mm}{\tiny\textsf{\slideauthor}}}
  %\rfoot{\raisebox{5mm}{\tiny\textsf{\slidevenue{} -- \arabic{mypage}/\pageref{lastpage}}}}
  %\rfoot{~\anchor{30,12}{\tiny\textsf{\thepage/\pageref{lastpage}}}}
  %\lfoot{\small\textsf{Marc Toussaint}}
%  \rfoot{\vspace*{-4.5mm}{\tiny\textsf{\color{gray}\topic\ -- \subtopic\ -- \arabic{mypage}/\pageref{lastpage}}}\hspace*{-4mm}}
  \rfoot{\vspace*{-4.5mm}{\tiny\textsf{\color{gray}\topic\ -- \arabic{mypage}/\pageref{lastpage}}}\hspace*{-4mm}}
%  \rfoot{~\anchor{-10,12}{\tiny\textsf{\color{gray}\topic\ --  \arabic{mypage}/\pageref{lastpage}}}}
  \lfoot{\vspace*{-4.5mm}{\hspace*{-3mm}\includegraphics[height=4mm]{LIS-logo-longText}}}

  \definecolor{grey}{rgb}{.8,.8,.8}
  \definecolor{head}{rgb}{.85,.9,.9}
%  \definecolor{blue}{rgb}{.0,.0,.5}
%  \definecolor{green}{rgb}{.0,.5,.0}
  \definecolor{red}{rgb}{.8,.0,.0}
  \newcommand{\inverted}{
    \definecolor{main}{rgb}{1,1,1}
    \color{main}
    \pagecolor[rgb]{.3,.3,.3}
  }
  \input{../latex/macros}

  \graphicspath{{pics/}{../pics/}{pics-local/}}
}

\newcommand{\slidestitle}{
  \title{\course \topic}
  \author{Marc Toussaint}
  \institute{Learning \& Intelligent Systems Lab, TU Berlin}

  \begin{document}


  %% title slide!
  \slide{}{
    \thispagestyle{empty}

    \twocol{.35}{.55}{
      %\vspace*{-5mm}%
      \hspace*{-5mm}%
      \includegraphics[width=1.\columnwidth]{\coursepicture}
    }{\center

      \textbf{\fontsize{17}{20}\selectfont \course}

      ~

      %Lecture
      \topic\\

      \vspace{1cm}

      {\tiny~\emph{\keywords}~\\}

      \vspace{1cm}

      \teacher
      
      Technical University of Berlin

      \coursedate

      ~

    }
  }
}

\newcommand{\slide}[2]{
  \slidefont
  \incpage\begin{frame}
  \addcontentsline{toc}{section}{#1}
  \vfill
  {\headerfont #1} \vspace*{-2ex}
  \begin{itemize}\item[]~\\
    #2
  \end{itemize}
  \vfill
  \end{frame}
}

% use \begin{frame}[fragile] around slidecore!
\newenvironment{slidecore}[1]{
  \slidefont\incpage
  \addcontentsline{toc}{section}{#1}
  \vfill
  {\headerfont #1} \vspace*{-2ex}
  \begin{itemize}\item[]~\\
}{
  \end{itemize}
  \vfill
}

\newcommand{\titleslide}[4][Marc Toussaint]{
  \newcommand{\slideauthor}{#1}
  \newcommand{\slidevenue}{#3}
  \slidefont
  \incpage
  \begin{frame}
  \begin{center}
    \vspace*{15mm}

    {\headerfont #2\\}
        
    \vspace*{10mm}

    #1 \\

    \vspace*{5mm}

    {\small
      Learning \& Intelligent Systems Lab, TU Berlin\\
%      Science of Intelligence Cluster of Excellence, Berlin\\
      Max Planck Fellow, Institute for Intelligent Systems\\ %Physical Reasoning \& Manipulation Lab -- 
%      MIT CSAIL\\
%      mtoussai@mit.edu,~ marc.toussaint@informatik.uni-stuttgart.de

      \vspace*{10mm}

      \emph{#3}
    }

    \vspace*{0mm}

    %\includegraphics[scale=.1]{pics/eushield-fullcolour}

  \end{center}
  \begin{itemize}\item[]~\\
    #4
  \end{itemize}
  \end{frame}
}

\newcommand{\titleslideempty}[2]{
  \slidefont
  \incpage
  \begin{frame}
  \begin{center}
    \vspace*{15mm}

    {\headerfont #1\\}
        
    %% \vspace*{5mm}

    %% {\small\emph{#2}} \\

  \end{center}
  \begin{itemize}\item[]~\\
    #2
  \end{itemize}
  \end{frame}
}

\providecommand{\key}[1]{
  \addtocounter{mypage}{1}
% \immediate\write\keyfile{#1}
  \addtocontents{toc}{\hyperref[key:#1]{#1 (\arabic{mypage})}}
%  \phantomsection\label{key:#1}
%  \index{#1@{\hyperref[key:#1]{#1 (\arabic{mysec}:\arabic{mypage})}}|phantom}
  \addtocounter{mypage}{-1}
}

\providecommand{\course}{}

\providecommand{\subtopic}{}

\providecommand{\sublecture}[2]{
  \renewcommand{\subtopic}{#1}
  \slide{#1}{#2}
}

\providecommand{\sublectureHide}[2]{
  \renewcommand{\subtopic}{#1}
}

\providecommand{\story}[1]{
~

Motivation: {\tiny #1}\clearpage
}

\newenvironment{items}[1][9]{
\par\setlength{\unitlength}{1pt}\fontsize{#1}{#1}\linespread{1.2}\selectfont
\begin{list}{--}{\leftmargin4ex \rightmargin0ex \labelsep1ex \labelwidth2ex
\topsep.7ex \parsep0ex \itemsep3pt}
}{
\end{list}
}

\newenvironment{itemS}[1][4ex]{
\par
\tiny
\begin{list}{--}{\leftmargin#1 \rightmargin0ex \labelsep1ex
  \labelwidth2ex \topsep0pt \parsep0ex \itemsep2pt}
}{
\end{list}
}

\providecommand{\slidesfoot}{
  \end{document}
}


  \slideshead
  %\slidestitle
}

\providecommand{\exercises}{
  \input{../latex/style-exercises}
  \exerciseshead
}

\providecommand{\script}{
  \newcommand{\scripthead}{
  \documentclass[9pt,fleqn,twoside]{article}
  \stdpackages

  \usepackage{makeidx}
  \makeindex

  \usepackage{thmtools}
  \definecolor{shadecolor}{gray}{0.85}
  \declaretheoremstyle[
    %headfont=\normalfont\bfseries,
    %notefont=\mdseries, notebraces={(}{)},
    %bodyfont=\normalfont,
    %postheadspace=0.5em,
    %spaceabove=6pt,
    mdframed={
      %  skipabove=8pt,
      %  skipbelow=6pt,
      hidealllines=true,
      backgroundcolor={shadecolor},
      innertopmargin=8pt,
      %  innerleftmargin=3pt,
      %  innerrightmargin=3pt
    }
  ]{shaded}
  \declaretheorem[style=shaded,within=section,name=Definition]{myDefinition}
  \declaretheorem[style=shaded,within=section,name=Theorem]{myTheorem}
  \declaretheorem[style=shaded,within=section,name=Identities]{Identities}
  \declaretheorem[style=shaded,within=section,name=Example]{myExample}

  \definecolor{grey}{rgb}{.8,.8,.8}
  \definecolor{bluecol}{rgb}{0,0,.5}
  \definecolor{greencol}{rgb}{0,.4,0}
  \definecolor{shadecolor}{gray}{0.9}
  \definecolor{citcol}{rgb}{.4,.4,.4}
  \usepackage[
    %    pdftex%,
    %%    letterpaper,
    %bookmarks,
    bookmarksnumbered,
    colorlinks,
    urlcolor=bluecol,
    citecolor=black,
    linkcolor=bluecol,
    %    pagecolor=bluecol,
    pdfborder={0 0 0},
    %pdfborderstyle={/S/U/W 1},
    %%    backref,     %link from bibliography back to sections
    %%    pagebackref, %link from bibliography back to pages
    %%    pdfstartview=FitH, %fitwidth instead of fit window
    pdfpagemode=UseOutlines, %bookmarks are displayed by acrobat
    pdftitle={\course},
    pdfauthor={Marc Toussaint},
    pdfkeywords={}
  ]{hyperref}
  \DeclareGraphicsExtensions{.pdf,.png,.jpg,.eps}

  %\usepackage{multirow}
  \usepackage{multimedia}
  %\usepackage{marginnote}
  %\setbeamercolor{background canvas}{bg=}

  \usepackage[round]{natbib}
  \bibliographystyle{abbrvnat}

  \renewcommand{\r}{\varrho}
  \renewcommand{\l}{\lambda}
  \renewcommand{\L}{\Lambda}
  \renewcommand{\b}{\beta}
  \renewcommand{\d}{\delta}
  \renewcommand{\k}{\kappa}
  \renewcommand{\t}{\theta}
  \renewcommand{\O}{\Omega}
  \renewcommand{\o}{\omega}
  \renewcommand{\SS}{{\cal S}}
  \renewcommand{\=}{\!=\!}
  %\renewcommand{\boldsymbol}{}
  %\renewcommand{\Chapter}{\chapter}
  %\renewcommand{\Subsection}{\subsection}

  \renewcommand{\baselinestretch}{1.0}
  \geometry{a5paper,headsep=6mm,hdivide={10mm,*,10mm},vdivide={13mm,*,7mm}}

  \fancyhead[OL,ER]{\course, \textit{Marc Toussaint}}
  \fancyhead[OR,EL]{\thepage}
  \fancyhead[C]{}
  \fancyfoot{}
  \pagestyle{fancy}

  \renewcommand{\labelenumi}{{(\roman{enumi})}}
  \renewcommand{\theenumi}{(\roman{enumi})} %for ref
  \parindent 0pt
  \parskip .5pc

  \columnsep 6ex

  \renewcommand{\familydefault}{\sfdefault}

  \newcommand{\headerfont}{\large}%helvetica{12}{1}{b}{n}}
  \newcommand{\slidefont} {}%\helvetica{9}{1.3}{m}{n}}
  \newcommand{\storyfont} {}
  %  \renewcommand{\small}   {\helvetica{8}{1.2}{m}{n}}
  \renewcommand{\tiny}    {\footnotesize}%helvetica{7}{1.1}{m}{n}}
  \newcommand{\ttiny} {\footnotesize}%fontsize{7}{7}\selectfont}
%  \newcommand{\codefont}{\fontsize{6}{6}\selectfont}%helvetica{8}{1.2}{m}{n}}
  \newcommand{\codefont} {\helvetica{8}{1.2}{m}{n}}

  \input{../latex/macros}

  \usepackage{comment}
  \specialcomment{solution}{
    \small
    \begin{shaded}
  }{
    \end{shaded}
  }

  \graphicspath{{pics/}{../pics/}{pics-local/}}

  \mytitle{\course\\Lecture Script}
  \myauthor{Marc Toussaint}
  \date{\coursedate}
}

%%%%%%%%%%%%%%%%%%%%%%%%%%%%%%%%%%%%%%%%%%%%%%%%%%%%%%%%%%%%%%%%%%%%%%%%%%%%%%%%

\newcommand{\scripttitle}{
  \begin{document}
  \maketitle
  %\anchor{100,10}{\includegraphics[width=4cm]{optim}}
%  \vspace*{1cm}
}

%%%%%%%%%%%%%%%%%%%%%%%%%%%%%%%%%%%%%%%%%%%%%%%%%%%%%%%%%%%%%%%%%%%%%%%%%%%%%%%%

\newcounter{mypage}
\setcounter{mypage}{0}
\newcommand{\incpage}{\addtocounter{mypage}{1}}

\newcommand{\subtopic}{}
\newcommand{\pause}{}
\newcommand{\only}[1]{#1}

\renewcommand{\slides}[1][]{
  %  \clearpage
  \subsection{\topic}
  \index{\topic}
  {\small #1}
  \setcounter{mypage}{0}
  \smallskip\nopagebreak\hrule\medskip
}

\newcommand{\slidesfoot}{
  \bigskip
}

\newcommand{\sublecture}[2]{
  \phantomsection\addcontentsline{toc}{subsubsection}{#1}
  \index{#1}
}

\newcommand{\sublectureHide}[2]{
  \renewcommand{\subtopic}{#1}
}

\newcommand{\key}[1]{
  \phantomsection\addcontentsline{toc}{subsubsection}{#1}
  %\subsubsection{#1}
  \index{#1}
}

\providecommand{\defn}[1]{%
  \textbf{#1}\index{#1}%
}

\newenvironment{slidecore}[1]{
  \incpage
  \subsubsection*{#1}%{\headerfont\noindent\textbf{#1}\\}%
  \vspace{-6ex}%
  \begin{list}{$\bullet$}{\leftmargin4ex \rightmargin0ex \labelsep1ex
    \labelwidth2ex \partopsep0ex \topsep0ex \parsep.5ex \parskip0ex \itemsep0pt}\item[]~\\\nopagebreak%
}{
  \end{list}\nopagebreak%
  {\hfill\tiny \textsf{\arabic{section}.\arabic{subsection}:\arabic{mypage}}}\nopagebreak%
  \smallskip\nopagebreak\hrule
}

\newcommand{\slide}[2]{
  \begin{slidecore}{#1}
    #2
  \end{slidecore}
}

\renewcommand{\exercises}{}
\newcommand{\exercisestitle}{}
\newcommand{\exsection}[1]{\subsubsection{#1}}
\newcommand{\exsubsection}[1]{\paragraph{#1}}
\newcommand{\exerfoot}{\bigskip}

\newcommand{\story}[1]{
  \subsection*{Motivation \& Outline}
  {\storyfont\sf #1}
  \medskip\nopagebreak\hrule
}

\newcounter{savedsection}
\newcommand{\subappendix}{\setcounter{savedsection}{\arabic{section}}\appendix}
\newcommand{\noappendix}{
  \setcounter{section}{\arabic{savedsection}}% restore section number
  \setcounter{subsection}{0}% reset section counter
%  \gdef\@chapapp{\sectionname}% reset section name
  \renewcommand{\thesection}{\arabic{section}}% make section numbers arabic
}

\newenvironment{items}[1][9]{
\par\setlength{\unitlength}{1pt}\fontsize{#1}{#1}\linespread{1.2}\selectfont
\begin{list}{--}{\leftmargin4ex \rightmargin0ex \labelsep1ex \labelwidth2ex
\topsep0pt \parsep0ex \itemsep3pt}
}{
\end{list}
}

\newenvironment{itemS}[1][4ex]{
\par
\tiny
\begin{list}{--}{\leftmargin#1 \rightmargin0ex \labelsep1ex
  \labelwidth2ex \topsep0pt \parsep0ex \itemsep2pt}
}{
\end{list}
}

\newcommand{\Def}[1]{%
\textbf{#1}\index{#1}}%\marginnote{#1}}

  \scripthead
}

\providecommand{\paper}{
  \input{../latex/style-paper}
  \paperhead
}

\providecommand{\note}[1][9pt]{
  \providecommand{\notehead}[2]{
  \documentclass[#1,fleqn,twoside]{article}
  \stdpackages
  \renewcommand{\labelenumi}{{(\roman{enumi})}}
  \renewcommand{\theenumi}{(\roman{enumi})} %for ref

  \renewcommand{\baselinestretch}{#2}
  \renewcommand{\arraystretch}{1.2}
  \renewcommand{\topfraction}{1}
  \renewcommand{\bottomfraction}{1}
  \renewcommand{\textfraction}{0}
  \columnsep 5ex
  \parindent 3ex
  \parskip 1ex

  % Lists and paragraphs
  \parindent 0pt
  \topsep 4pt plus 1pt minus 2pt
  \partopsep 1pt plus 0.5pt minus 0.5pt
  \itemsep 2pt plus 1pt minus 0.5pt
  \parsep 2pt plus 1pt minus 0.5pt
  \parskip .5pc %add _in_ {thebibliography} environment in *.bbl

  \setcounter{tocdepth}{3}
  \setcounter{secnumdepth}{3}

  \geometry{a4paper,hdivide={25mm,*,25mm},vdivide={25mm,*,25mm}}

  \renewcommand{\headrulewidth}{.0pt}\renewcommand{\footrulewidth}{.0pt}\cfoot{}
  \fancyhead[OL,EC]{\it\theauthor---\today}
  \fancyhead[ER]{\leftmark}
  \fancyhead[OR,EL]{\thepage}
  \fancyfoot[EL,OR]{}
  \setlength{\headsep}{10mm}
  %\fancyhead[OL]{\rightmark}
  %\fancyfoot[EL,OR]{}


  %  \usepackage{palatino}

  \newcommand{\codefont}{\helvetica{8}{1.2}{m}{n}}

  \renewenvironment{abstract}{
    \vspace*{5ex}\begin{list}{}{
      \leftmargin3ex
      \rightmargin3ex
      \topsep-\parskip}\item[]
     \hrule\vspace{1.5ex}{\bf Abstract.~}\small}
    {\vspace{2ex}\hrule\end{list}\vspace{5ex}}
    
  \newenvironment{keyword}
    {\par{\it Keywords:~}}
    {}

  \def\makemytitle{%
    \thispagestyle{empty}
    \begin{list}{}{\leftmargin3ex \rightmargin3ex \topsep0ex \parsep0ex}\item[]
      \begin{center}
        {\fontsize{18}{25}\selectfont{\thetitle\\}}\vspace{5ex}

        {\fontsize{14}{16}\selectfont{\theauthor\\}}\vspace{1ex}

        {\footnotesize{\sl \addressFUB}\\ \emailBerlin}

        {\footnotesize \today}

        \vspace{1ex}
        {\small \published}
      \end{center}
    \end{list}
    \renewcommand{\maketitle}{\chapter{\thetitle}}
  }

  \input{../latex/macros}
  \pdflatex

  \graphicspath{{pics/}{../pics/}{pics-local/}}

  \myauthor{Marc Toussaint}
  \date{\today}
}

%%%%%%%%%%%%%%%%%%%%%%%%%%%%%%%%%%%%%%%%%%%%%%%%%%%%%%%%%%%%%%%%%%%%%%%%%%%%%%%%

\newcommand{\notetitle}{
  \begin{document}
  \thispagestyle{empty}
    
  \maketitle

}

\newenvironment{items}[1][9]{
\par\setlength{\unitlength}{1pt}\fontsize{#1}{#1}\linespread{1.2}\selectfont
\begin{list}{--}{\leftmargin4ex \rightmargin0ex \labelsep1ex \labelwidth2ex
\topsep0pt \parsep0ex \itemsep3pt}
}{
\end{list}
}

  \notehead{#1}{1.1}
}

\providecommand{\course}{NO COURSE}
\providecommand{\coursepicture}{NO PICTURE}
\providecommand{\coursedate}{NO DATE}
\providecommand{\topic}{NO TOPIC}
\providecommand{\keywords}{}
\providecommand{\exnum}{NO NUMBER}
\providecommand{\teacher}{Marc Toussaint}

\providecommand{\stdpackages}{
  \usepackage{amsmath}
  \usepackage{amssymb}
  \usepackage{amsfonts}
  \allowdisplaybreaks
  \usepackage{amsthm}
  \usepackage{eucal}
  \usepackage{graphicx}
%  \usepackage{color}
  \usepackage{geometry}
  \usepackage{framed}
  \usepackage{xcolor}
  \definecolor{shadecolor}{gray}{0.9}
  \setlength{\FrameSep}{3pt}
  \usepackage{fancyvrb}
  \fvset{numbers=none,xleftmargin=5ex,fontsize=\small}

  \usepackage{pdfpages}

  \usepackage{multicol} 
  \usepackage{fancyhdr}
}

\providecommand{\addressUSTT}{
  Machine~Learning~\&~Robotics~lab, U~Stuttgart\\\small
  Universit{\"a}tsstra{\ss}e 38, 70569~Stuttgart, Germany
}

\providecommand{\addressTUB}{
  Learning~\&~Intelligent~Systems~Lab, TU~Berlin\\\small
  Marchstr. 23, 10587 Berlin, Germany
}


\renewcommand{\course}{Optimization Algorithms}
\renewcommand{\coursepicture}{optim}
\renewcommand{\coursedate}{Winter 2024/25}

\renewcommand{\topic}{No Free Lunch}
\renewcommand{\keywords}{}

\slides

\slidestitle

%%%%%%%%%%%%%%%%%%%%%%%%%%%%%%%%%%%%%%%%%%%%%%%%%%%%%%%%%%%%%%%%%%%%%%%%%%%%%%%%

\slide{References}{

\item Toussaint: \emph{The Bayesian Search Game}. In Theory and Principled Methods for Designing Metaheuristics, Springer, 2012.

\item Igel \& Toussaint: \emph{On Classes of Functions for which No Free Lunch Results Hold}. Information Processing Letters, 86, p.\ 317-321, 2003.

\item Wolpert \& Macready. \emph{No free lunch theorems for optimization}. IEEE Transactions on Evolutionary Computation, 1(1):67–82, 1997.

}

%%%%%%%%%%%%%%%%%%%%%%%%%%%%%%%%%%%%%%%%%%%%%%%%%%%%%%%%%%%%%%%%%%%%%%%%%%%%%%%%

\key{No Free Lunch (NFL) Theorem}
\slide{No Free Lunch Theorem -- Problem Setting}{

{\tiny [Following \emph{The Bayesian Search Game} (2012)]}

\item Finite(!) space $X$

\item Distribution $P(f)$ over functions $f:~ X \to Y$

\item A \textbf{non-revisiting} algorithm $\AA$ generates queries $x_t$ and observations $y_t = f(x_t)$. Formally, a probabilistic algorithm is defined by
$$P(x_t \| x_{1:t\1}, y_{1:t\1}; \AA)$$
and $P(x_1 ; \AA)$.

~\pause\small

\item Therefore, $\AA$ interacting with random function $f$ generates the joint process:
\begin{align*}
P(f,x_{1:T},y_{1:T};\AA)
&= P(f)~ P(y_1 \| x_1,f)~ P(x_1;\AA)~
 \prod_{t=2}^T P(y_t \| x_t,f)~ P(x_t \| x_{1:t\1},y_{1:t\1};\AA)
\end{align*}

}

%%%%%%%%%%%%%%%%%%%%%%%%%%%%%%%%%%%%%%%%%%%%%%%%%%%%%%%%%%%%%%%%%%%%%%%%%%%%%%%%

\slide{No Free Lunch Theorem}{

\item \textbf{Theorem:}
\begin{align} 
&\exists h:Y\to\RRR ~~\text{s.t.}~~ \forall K\in\NN^+, \{x_1,..,x_K\} \subset X:~
 P(f_{x_1},..,f_{x_K}) = \prod_{k=1}^K h(f_{x_k}) \label{nflL}\\
 &\qquad \Longleftrightarrow \qquad \forall_\AA, \forall_T:~ P(y_{1:T};\AA) = \prod_{i=1}^T h(y_i) \quad \text{(independent of $\AA$) } \label{nflR}
\end{align}

\item In words:

\cen{$P(f)$ factorizes $\quad\iff\quad$ all $\AA$ generate the same random observations}

~\tiny

[Proof later]

}

%%%%%%%%%%%%%%%%%%%%%%%%%%%%%%%%%%%%%%%%%%%%%%%%%%%%%%%%%%%%%%%%%%%%%%%%%%%%%%%%

\slide{No Free Lunch Theorem -- Comments}{

\item Interpreting the LHS:
\begin{items}
\item $P(f_{x_1},..,f_{x_K}) = \prod_{k=1}^K h(f_{x_k})$ factorizes i.i.d.\
\item \textbf{There is no mutual information between any $f(x_1), f(x_2), x_1\not=x_2$}, $I(f(x_1), f(x_2)) = 0$
\item Observing $f(x_1)$ reveals no information whatsoever on what $f(x_2)$ might be
\item Any (non-repeating!) algorithm is equally blind and uninformed about what future observations might be, not matter how it collected past information $(x_{1:t\1}, y_{1:t\1})$
\end{items}

~\pause\small

\item Often we have a performance metric (see later); but ``all observations $P(y_t\|...;\AA)$ are indep.\ of $\AA$'' is stronger and implies equal expected performance with whatever metric

\item Traditional statement: ``Averaged over \emph{all} problem instances, any algorithm performs equally. (E.g.\ equal to random.)''

\item ``there is no one algorithm that works best for every problem''

}

%%%%%%%%%%%%%%%%%%%%%%%%%%%%%%%%%%%%%%%%%%%%%%%%%%%%%%%%%%%%%%%%%%%%%%%%%%%%%%%%

\slide{No Free Lunch Theorem -- Comments}{

\item The classical citation is Wolpert \& Macready (1997), but is less general than the above and proof overly complicated and less clear in my view.

\begin{items}
\item ``Averaging over all problems'' $\to$ expectation w.r.t.\ $P(f)$
\item ``set of functions closed under permutation'' $\to$ $P(f)$ factorizes
\item Our Theorem is strong $\iff$, not just $\To$ (Igel \& Toussaint, 2004)
\end{items}

}

%%%%%%%%%%%%%%%%%%%%%%%%%%%%%%%%%%%%%%%%%%%%%%%%%%%%%%%%%%%%%%%%%%%%%%%%%%%%%%%%

\key{NFL Proof}
\slide{NFL Proof}{

\item We defined the process $P(f,x_{1:T},y_{1:T};\AA)$ previously

\item Basic definitions of probabilities to prove $\To$:
\begin{align*} \label{eq4}
P(y_t \| x_{1:t\1},y_{1:t\1};\AA)
&= \sum_{x_t \in X} \[\sum_f P(y_t \| x_t,f)~ P(f \| x_{1:t\1},y_{1:t\1})\]~ P(x_t \| x_{1:t\1},y_{1:t\1};\AA) \feed
&= \sum_{x_t \in X} P(f_{x_t}\=y_t \| x_{1:t\1},y_{1:t\1})~ P(x_t \| x_{1:t\1},y_{1:t\1};\AA) \feed
&= \sum_{x_t \in X} h(y_t)~ P(x_t \| x_{1:t\1},y_{1:t\1};\AA) = h(y_t) ~.
\end{align*}
Last line: $\AA$ is non-revisiting, and $P(f_{x_t}\=y_t \|
x_{1:t\1},y_{1:t\1}) = P(f_{x_t}\=y_t) = h(y_t)$.

~\pause

\item Prove $\oT$ by explicitly constructing algorithms that generate different outputs when $P(f)$ is non-factored.
{\tiny [Details in \emph{The Bayesian Search Game}, 2012]}

}

%%%%%%%%%%%%%%%%%%%%%%%%%%%%%%%%%%%%%%%%%%%%%%%%%%%%%%%%%%%%%%%%%%%%%%%%%%%%%%%%

\slide{No Free Lunch for Optimization}{

\item Consider the problem $\min_{x\in X} f(x)$ for finite $X$

\item Also here, an algorithm $\AA$ is defined by $P(x_k \| x_{1:t\1}, y_{1:t\1}; \AA)$

\item A typical performance metric could be \defn{regret}
$$R(T) = \sum_{t=1}^T y_t - y^*$$

~\pause

\item But if for a non-repeating(!) $\AA$, $P(y_t)$ is indep.\ of $\AA$, so is the expected regret

}

%%%%%%%%%%%%%%%%%%%%%%%%%%%%%%%%%%%%%%%%%%%%%%%%%%%%%%%%%%%%%%%%%%%%%%%%%%%%%%%%

\slide{No Free Lunch for Machine Learning}{

\item Given data $D = \{ (x_i, y_i) \}_{i=1}^n$, find a predictor $\hat f:~ X \to y$ that minimizes expected loss $\Exp{\ell(\hat f(x^*), f(x^*))}$ for a future query $x^*$, where $f(x^*)$ is the ground truth

\item A learning algorithm $\AA$ is a predictive distribution $P(y \| x^*, D; \AA)$

(i.e., a mapping from $D$ to a prediction $P(y \| x^*)$ for a new query $x^*$)

\item Assume $X$ is finite and $x^* \not\in D$ (non-repeating!)

~

\item But if $P(f)$ factorizes so that $P(f(x^*)\=y) = h(y)$ is fully independent from $D$ (zero mutual information), then no algorithm can learn anything or predict better than the prior.

}

%%%%%%%%%%%%%%%%%%%%%%%%%%%%%%%%%%%%%%%%%%%%%%%%%%%%%%%%%%%%%%%%%%%%%%%%%%%%%%%%

\slide{Bayes' Theorem}{

{\Large\centering

$$
P(X|D) = \frac{P(D|X)}{P(D)}~ P(X)
$$

~

{$\text{posterior} ~=~
\frac{\text{likelihood} ~\cdot~ \text{prior}}{\text{normalization}}$\quad}

}

~\normalsize

\item But if $X$ is indep.\ from $D$, then there is nothing to learn or predict better than the prior $P(X)$

}

%%%%%%%%%%%%%%%%%%%%%%%%%%%%%%%%%%%%%%%%%%%%%%%%%%%%%%%%%%%%%%%%%%%%%%%%%%%%%%%%

\sublecture{Conclusions from NFL?}{
}

%%%%%%%%%%%%%%%%%%%%%%%%%%%%%%%%%%%%%%%%%%%%%%%%%%%%%%%%%%%%%%%%%%%%%%%%%%%%%%%%

\slide{}{

\item NFL is an almost trivial theorem, what is non-trivial is what to make of it

~\pause

\item First pressing question:
\begin{items}
\item Does NFL also hold for continuous $X$? What would it mean that $P(f)$ is factorized, or $I(f(x_1), f(x_2)) = 0$, for any $x_1\not= x_2$ in continous $X$?
\end{items}

\pause

\item Thoughts on conclusions from NFL:
\begin{items}
\item Become aware, in your methods, what actually you are assuming - you must assume something
\item Fight back if anybody ever states ``we don't (want to) make assumptions'' (e.g.\ in a talk on RL that claims it can solve any problem without assumptions)
\item There is no Artificial General Intelligence if general would mean ``making NO assumptions''. So, the AGI community (say, Marcus Hutter) must make some assumptions -- what are they \emph{exactly}?
\item What are assumptions we would ``generally'' accept to make in our physical universe? (In case we care about AI specifically in our physical universe.)
\item What are algorithms that literally start by making assumptions about $P(f)$ and then derive an optimal algorithm for that $P(f)$? (see Bayesian Search Game...)
\end{items}

}

%%%%%%%%%%%%%%%%%%%%%%%%%%%%%%%%%%%%%%%%%%%%%%%%%%%%%%%%%%%%%%%%%%%%%%%%%%%%%%%%

\key{NFL in continuous domains}
\key{Gaussian Processes}
\slide{NFL in continuous domains}{

\item The LHS describes $P(f)$ with $I(f(x_1), f(x_2)) = 0$ for any $x_1\not= x_2$
\begin{items}
\item How can we define probability distributions over functions (over continuous $X$) in the first place?
\end{items}

~\pause

\item A typical way to define distributions over $f:\RRR^n \to \RRR$ is as a \defn{Gaussian Process}:
\begin{items}
\item For every finite set $\{x_1,..,x_M\}$, the function values
$f(x_1),..,f(x_M)$ are Gaussian distributed with mean and covariance
\begin{align*}
&\Exp{f(x_i)} = \mu(x_i) \qquad\text{(often zero)} \\
&\Exp{[f(x_i)-\m(x_i)][f(x_j)-\mu(x_j)]} = k(x_i,x_j)
\end{align*}
where, $\mu(x)$ is called \textbf{mean function}, and $k(x,x')$ is called \defn{covariance function}
\item $\mu$ and $k$ generalize the notion of \emph{mean vector} $\mu_x$ and \emph{covariance matrix} $\S_{xx'}$ from finite $x\in\{1,..,n\}$ to continuous $x\in\RRR^n$
\end{items}

}

%%%%%%%%%%%%%%%%%%%%%%%%%%%%%%%%%%%%%%%%%%%%%%%%%%%%%%%%%%%%%%%%%%%%%%%%%%%%%%%%

\slide{GP examples}{

\show[.6]{gaussianProcess1}
{\tiny\hfill(from Rasmussen \& Williams)}

}

%%%%%%%%%%%%%%%%%%%%%%%%%%%%%%%%%%%%%%%%%%%%%%%%%%%%%%%%%%%%%%%%%%%%%%%%%%%%%%%%

%% \slide{}{
%% \show[.6]{BayesianPredictiveDistribution}
%% {\tiny\hfill(from Bishop)}
%% }

%%%%%%%%%%%%%%%%%%%%%%%%%%%%%%%%%%%%%%%%%%%%%%%%%%%%%%%%%%%%%%%%%%%%%%%%%%%%%%%%

\slide{GP examples: different covariance functions}{

~

\show[.6]{gaussianProcess2}
{\tiny\hfill(from Rasmussen \& Williams)}

~

\item These are examples from the $\g$-exponential covariance function

$$k(x,x') = \exp\{-|(x-x')/l|^\g\}$$

}

%%%%%%%%%%%%%%%%%%%%%%%%%%%%%%%%%%%%%%%%%%%%%%%%%%%%%%%%%%%%%%%%%%%%%%%%%%%%%%%%

\slide{NFL in continuous domains}{

\item Back to NFL: the LHS requires $I(f(x_1), f(x_2)) = 0$, which would mean, for GPs, zero covariance function $k(x,x')=0$ for any $x\not=x'$

~\pause

\item At first sight this might seem ok, but
\begin{items}
\item Auger \& Teytaud clarify that ``zero-covariance GP'' is not a proper Lebesgue measure over function
\item Conversely, they state that for any Lebesgue meassure the LHS does not hold (and claim that Lebesgue meassures are the only sensible kind of $P(f)$)
\end{items}

\cit{A. Auger and O. Teytaud}{Continuous lunches are free plus the design of optimal optimization
algorithms}{Algorithmica, 2008}

~\pause\small

\item Beyond my expertise as non-mathematician
\item But the point of NFL remains the same: one would only have to replace ``non-revisiting'' by ``non-near-revisiting'' or so.

}

%%%%%%%%%%%%%%%%%%%%%%%%%%%%%%%%%%%%%%%%%%%%%%%%%%%%%%%%%%%%%%%%%%%%%%%%%%%%%%%%

\slide{NFL in continuous domains -- conclusions}{

\item Whether NFL holds in continuous domains depends on what $P(f)$ you consider mathematically sound

\item The core point remains that if $I(f(x_1), f(x_2))=0$ (for non-close $x_1,x_2$), no non-(near)-revisiting algorithm can be smart

~

\item Gaussian Processes are the simplest instance for assuming non-zero $I(f(x_1), f(x_2)) \not= 0$, by assuming Gaussian dependencies between $x\not=x'$

$\To$ GPs became a standard assumption to explicitly design algorithms exploiting that assumption and evading NFL

}

%%%%%%%%%%%%%%%%%%%%%%%%%%%%%%%%%%%%%%%%%%%%%%%%%%%%%%%%%%%%%%%%%%%%%%%%%%%%%%%%

\slide{}{

\emph{Become aware, in your methods, what actually you are assuming - you must assume something}

~

\item What did our optimization algorithms assume so far?

}

%%%%%%%%%%%%%%%%%%%%%%%%%%%%%%%%%%%%%%%%%%%%%%%%%%%%%%%%%%%%%%%%%%%%%%%%%%%%%%%%

\slide{Assumptions in continuous optimization}{

\item $f$ is continously differentiable $f\in C^1$!
\begin{items}
\item The limits exist! Clearly there are ``correlations'' when approaching infinitesimally!
\item Sure we can predict to (infinitesimally close) points: The gradient gives an accurate 1st order Taylor prediction (in the vicinity)
\item We can predict to go downhill following the gradient.
\item All this would not be possible with NFL assumptions.
\end{items}


\pause

\item Lipschitz continuity of $\na f(x)$ ~ (assumption of SGD convergence)

\pause

\item Strong convexity assumption (eigenvalues $\l$ of the Hessian $\he f(x)$ bounded by $m < \l < M$) ~ (exponential convergence of line search)

~\pause

\item All assumptions are \emph{local}, and were used to characterize local convergence behavior

}

%%%%%%%%%%%%%%%%%%%%%%%%%%%%%%%%%%%%%%%%%%%%%%%%%%%%%%%%%%%%%%%%%%%%%%%%%%%%%%%%

\slide{Assumptions made in AGI}{

\item Kolmogorov \& Solomonoff complexity

(also not my expertise...)


\cit{Lattimore \& Hutter}{No free lunch versus Occam's razor in supervised learning}{In Algorithmic Probability and Friends. Bayesian Prediction and Artificial Intelligence, 2013}

\cit{Baum, Hutter, \& Kitzelmann}{Artificial general intelligence}{In Proceedings of the Third Conference on Artificial General Intelligence, 2010}

~

\item Occam's rasor: $P(f)$ is higher for ``simpler'' functions $f$. Assuming all (relevant) $f$ are computable, simpler = of lower Kolmogorov/Solomonoff complexity.

\item Obvious algorithm to exploit this universal prior: Sort all $f$ by complexity, test each in order -- will be better than random.

\item Can also define optimal algorithms (optimal AGI) under this universal complexity prior

}

%%%%%%%%%%%%%%%%%%%%%%%%%%%%%%%%%%%%%%%%%%%%%%%%%%%%%%%%%%%%%%%%%%%%%%%%%%%%%%%%

\slide{}{

\emph{What are assumptions we would ``generally'' accept to make in our physical universe? (In case
we care about AI specifically in our physical universe.)}

~\pause

\item Beyond full discussion here. Some thoughts:
\begin{items}
\item physics $\oto$ space$\times$time; things (fields/objects); local(!) interactions between things; invariances(!)
\pause
\item images $\oto$ invariances; neighboring pixels correlated $\oto$ convolutional features, hierarchies, CNN
\item time series $\oto$ Markovian, maybe smooth $\oto$ HMMs, MDPs, control, etc, etc
\item Robotics, Language, Text, humans, animals, etc etc
\end{items}

}

%%%%%%%%%%%%%%%%%%%%%%%%%%%%%%%%%%%%%%%%%%%%%%%%%%%%%%%%%%%%%%%%%%%%%%%%%%%%%%%%

\slide{}{

\emph{What are algorithms that literally start by making assumptions about $P(f)$ and then derive an
optimal algorithm for that $P(f)$?}

}

%%%%%%%%%%%%%%%%%%%%%%%%%%%%%%%%%%%%%%%%%%%%%%%%%%%%%%%%%%%%%%%%%%%%%%%%%%%%%%%%

\key{Optimal Optimization}
\slide{Optimal Optimization}{

\item Optimization can be formalized as a sequential decision problem (MDP):
\begin{items}
\item Start with a prior $b_0 = P(f)$
\item Choose a query $x_t$ based on $b_t$ ~ (policy, acquisition function)
\item Query $x_t$, observe $y_t$, update data $D$, update belief $b_t \gets P(f\|D)$, iterate
\end{items}
\show[.6]{bsg}
{\tiny\hfill[Bayesian Search Game]}

\pause

\item This defines a \emph{known} decision process, for which we can define an optimal policy
\begin{items}
\item Can in principle be computed using Dynamic Programming -- but intractable
\end{items}


}

%%%%%%%%%%%%%%%%%%%%%%%%%%%%%%%%%%%%%%%%%%%%%%%%%%%%%%%%%%%%%%%%%%%%%%%%%%%%%%%%

\slide{Bayesian Optimization in a nutshell}{

\item We maintain a particular belief $b_t = P(f\|D)$, namely a \emph{Gaussian Process}

~\pause

\item Don't plan an optimal query policy, but use a 1-step heuristic:

\item An \defn{acquisition function} $\a(x, b_t)$ characterizes how ``interesting'' it is to query $x$ next, and defines the policy
$$x_t = \argmax_x \a(x, b_t)$$

\item Analogies:
\begin{items}
\item $\a(x, b_t)$ is a descriminative function for the next decision
\item $\a(x, b_t)$ is like a $Q$-function $Q(b_t,x)$ for the next decision (but not learned)
\end{items}

}

%%%%%%%%%%%%%%%%%%%%%%%%%%%%%%%%%%%%%%%%%%%%%%%%%%%%%%%%%%%%%%%%%%%%%%%%%%%%%%%%

\slide{}{\label{lastpage}

to be continued with Bayesian Optimization...

}

%%%%%%%%%%%%%%%%%%%%%%%%%%%%%%%%%%%%%%%%%%%%%%%%%%%%%%%%%%%%%%%%%%%%%%%%%%%%%%%%

\slidesfoot
