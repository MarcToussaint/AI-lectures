\providecommand{\slides}{
  \newcommand{\slideshead}{
  \newcommand{\thepage}{\arabic{mypage}}
  %beamer
%  \documentclass[t,hyperref={bookmarks=true}]{beamer}
%  \geometry{papersize={171mm,96mm}}
  \documentclass[t,hyperref={bookmarks=true},aspectratio=169]{beamer}
  \setbeamersize{text margin left=5mm}
  \setbeamersize{text margin right=5mm}
  \usetheme{default}
  \usefonttheme[onlymath]{serif}
  \setbeamertemplate{navigation symbols}{}
  \setbeamertemplate{itemize items}{{\color{black}$\bullet$}}

  \newwrite\keyfile

  %\usepackage{palatino}
  \stdpackages
  %\usepackage{tikz} \usetikzlibrary {shapes.geometric} 
  \usepackage{multimedia}
  \usepackage[utf8]{inputenc}

  %%% geometry/spacing issues
  %
  \definecolor{bluecol}{rgb}{0,0,.5}
  \definecolor{greencol}{rgb}{0,.6,0}
  \definecolor{citcol}{rgb}{.4,.4,.4}
  %\renewcommand{\baselinestretch}{1.1}
  \renewcommand{\arraystretch}{1.2}
  \columnsep 0mm

  \columnseprule 0pt
  \parindent 0ex
  \parskip 0ex
  %\setlength{\itemparsep}{3ex}
  %\renewcommand{\labelitemi}{\rule[3pt]{10pt}{10pt}~}
  %\renewcommand{\labelenumi}{\textbf{(\arabic{enumi})}}
  \setbeamertemplate{enumerate item}{(\roman{enumi})}
  \newcommand{\headerfont}{\helvetica{14}{1.5}{b}{n}}
  \newcommand{\slidefont} {\helvetica{11}{1.4}{m}{n}}
  %\newcommand{\codefont} {\helvetica{8}{1.2}{m}{n}}
  \newcommand{\urlfont} {\fontsize{6}{1.1}\selectfont}
  \renewcommand{\small} {\helvetica{10}{1.4}{m}{n}}
  \renewcommand{\tiny} {\helvetica{8}{1.3}{m}{n}}
  \newcommand{\ttiny} {\helvetica{7}{1.3}{m}{n}}

  %%% count pages properly and put the page number in bottom right
  %
  \newcounter{mypage}
  \newcommand{\incpage}{\addtocounter{mypage}{1}\setcounter{page}{\arabic{mypage}}}
  \setcounter{mypage}{0}
  \resetcounteronoverlays{page}

  \pagestyle{fancy}
  %\setlength{\headsep}{10mm}
  %\addtolength{\footheight}{15mm}
  \renewcommand{\headrulewidth}{0pt} %1pt}
  \renewcommand{\footrulewidth}{0pt} %.5pt}
  \cfoot{}
  \rhead{}
  \lhead{}
  %% \lfoot{\vspace*{-3mm}\hspace*{-3mm}\helvetica{5}{1.3}{m}{n}{\texttt{github.com/MarcToussaint/AI-lectures}}}
\lfoot{}
%  \rfoot{{\tiny\textsf{AI -- \topic -- \subtopic -- \arabic{mypage}/\pageref{lastpage}}}}
  %\lfoot{\raisebox{5mm}{\tiny\textsf{\slideauthor}}}
  %\rfoot{\raisebox{5mm}{\tiny\textsf{\slidevenue{} -- \arabic{mypage}/\pageref{lastpage}}}}
  %\rfoot{~\anchor{30,12}{\tiny\textsf{\thepage/\pageref{lastpage}}}}
  %\lfoot{\small\textsf{Marc Toussaint}}
%  \rfoot{\vspace*{-4.5mm}{\tiny\textsf{\color{gray}\topic\ -- \subtopic\ -- \arabic{mypage}/\pageref{lastpage}}}\hspace*{-4mm}}
  \rfoot{\vspace*{-4.5mm}{\tiny\textsf{\color{gray}\topic\ -- \arabic{mypage}/\pageref{lastpage}}}\hspace*{-4mm}}
%  \rfoot{~\anchor{-10,12}{\tiny\textsf{\color{gray}\topic\ --  \arabic{mypage}/\pageref{lastpage}}}}
  \lfoot{\vspace*{-4.5mm}{\hspace*{-3mm}\includegraphics[height=4mm]{LIS-logo-longText}}}

  \definecolor{grey}{rgb}{.8,.8,.8}
  \definecolor{head}{rgb}{.85,.9,.9}
%  \definecolor{blue}{rgb}{.0,.0,.5}
%  \definecolor{green}{rgb}{.0,.5,.0}
  \definecolor{red}{rgb}{.8,.0,.0}
  \newcommand{\inverted}{
    \definecolor{main}{rgb}{1,1,1}
    \color{main}
    \pagecolor[rgb]{.3,.3,.3}
  }
  \input{../latex/macros}

  \graphicspath{{pics/}{../pics/}{pics-local/}}
}

\newcommand{\slidestitle}{
  \title{\course \topic}
  \author{Marc Toussaint}
  \institute{Learning \& Intelligent Systems Lab, TU Berlin}

  \begin{document}


  %% title slide!
  \slide{}{
    \thispagestyle{empty}

    \twocol{.35}{.55}{
      %\vspace*{-5mm}%
      \hspace*{-5mm}%
      \includegraphics[width=1.\columnwidth]{\coursepicture}
    }{\center

      \textbf{\fontsize{17}{20}\selectfont \course}

      ~

      %Lecture
      \topic\\

      \vspace{1cm}

      {\tiny~\emph{\keywords}~\\}

      \vspace{1cm}

      \teacher
      
      Technical University of Berlin

      \coursedate

      ~

    }
  }
}

\newcommand{\slide}[2]{
  \slidefont
  \incpage\begin{frame}
  \addcontentsline{toc}{section}{#1}
  \vfill
  {\headerfont #1} \vspace*{-2ex}
  \begin{itemize}\item[]~\\
    #2
  \end{itemize}
  \vfill
  \end{frame}
}

% use \begin{frame}[fragile] around slidecore!
\newenvironment{slidecore}[1]{
  \slidefont\incpage
  \addcontentsline{toc}{section}{#1}
  \vfill
  {\headerfont #1} \vspace*{-2ex}
  \begin{itemize}\item[]~\\
}{
  \end{itemize}
  \vfill
}

\newcommand{\titleslide}[4][Marc Toussaint]{
  \newcommand{\slideauthor}{#1}
  \newcommand{\slidevenue}{#3}
  \slidefont
  \incpage
  \begin{frame}
  \begin{center}
    \vspace*{15mm}

    {\headerfont #2\\}
        
    \vspace*{10mm}

    #1 \\

    \vspace*{5mm}

    {\small
      Learning \& Intelligent Systems Lab, TU Berlin\\
%      Science of Intelligence Cluster of Excellence, Berlin\\
      Max Planck Fellow, Institute for Intelligent Systems\\ %Physical Reasoning \& Manipulation Lab -- 
%      MIT CSAIL\\
%      mtoussai@mit.edu,~ marc.toussaint@informatik.uni-stuttgart.de

      \vspace*{10mm}

      \emph{#3}
    }

    \vspace*{0mm}

    %\includegraphics[scale=.1]{pics/eushield-fullcolour}

  \end{center}
  \begin{itemize}\item[]~\\
    #4
  \end{itemize}
  \end{frame}
}

\newcommand{\titleslideempty}[2]{
  \slidefont
  \incpage
  \begin{frame}
  \begin{center}
    \vspace*{15mm}

    {\headerfont #1\\}
        
    %% \vspace*{5mm}

    %% {\small\emph{#2}} \\

  \end{center}
  \begin{itemize}\item[]~\\
    #2
  \end{itemize}
  \end{frame}
}

\providecommand{\key}[1]{
  \addtocounter{mypage}{1}
% \immediate\write\keyfile{#1}
  \addtocontents{toc}{\hyperref[key:#1]{#1 (\arabic{mypage})}}
%  \phantomsection\label{key:#1}
%  \index{#1@{\hyperref[key:#1]{#1 (\arabic{mysec}:\arabic{mypage})}}|phantom}
  \addtocounter{mypage}{-1}
}

\providecommand{\course}{}

\providecommand{\subtopic}{}

\providecommand{\sublecture}[2]{
  \renewcommand{\subtopic}{#1}
  \slide{#1}{#2}
}

\providecommand{\sublectureHide}[2]{
  \renewcommand{\subtopic}{#1}
}

\providecommand{\story}[1]{
~

Motivation: {\tiny #1}\clearpage
}

\newenvironment{items}[1][9]{
\par\setlength{\unitlength}{1pt}\fontsize{#1}{#1}\linespread{1.2}\selectfont
\begin{list}{--}{\leftmargin4ex \rightmargin0ex \labelsep1ex \labelwidth2ex
\topsep.7ex \parsep0ex \itemsep3pt}
}{
\end{list}
}

\newenvironment{itemS}[1][4ex]{
\par
\tiny
\begin{list}{--}{\leftmargin#1 \rightmargin0ex \labelsep1ex
  \labelwidth2ex \topsep0pt \parsep0ex \itemsep2pt}
}{
\end{list}
}

\providecommand{\slidesfoot}{
  \end{document}
}


  \slideshead
  %\slidestitle
}

\providecommand{\exercises}{
  \input{../latex/style-exercises}
  \exerciseshead
}

\providecommand{\script}{
  \newcommand{\scripthead}{
  \documentclass[9pt,fleqn,twoside]{article}
  \stdpackages

  \usepackage{makeidx}
  \makeindex

  \usepackage{thmtools}
  \definecolor{shadecolor}{gray}{0.85}
  \declaretheoremstyle[
    %headfont=\normalfont\bfseries,
    %notefont=\mdseries, notebraces={(}{)},
    %bodyfont=\normalfont,
    %postheadspace=0.5em,
    %spaceabove=6pt,
    mdframed={
      %  skipabove=8pt,
      %  skipbelow=6pt,
      hidealllines=true,
      backgroundcolor={shadecolor},
      innertopmargin=8pt,
      %  innerleftmargin=3pt,
      %  innerrightmargin=3pt
    }
  ]{shaded}
  \declaretheorem[style=shaded,within=section,name=Definition]{myDefinition}
  \declaretheorem[style=shaded,within=section,name=Theorem]{myTheorem}
  \declaretheorem[style=shaded,within=section,name=Identities]{Identities}
  \declaretheorem[style=shaded,within=section,name=Example]{myExample}

  \definecolor{grey}{rgb}{.8,.8,.8}
  \definecolor{bluecol}{rgb}{0,0,.5}
  \definecolor{greencol}{rgb}{0,.4,0}
  \definecolor{shadecolor}{gray}{0.9}
  \definecolor{citcol}{rgb}{.4,.4,.4}
  \usepackage[
    %    pdftex%,
    %%    letterpaper,
    %bookmarks,
    bookmarksnumbered,
    colorlinks,
    urlcolor=bluecol,
    citecolor=black,
    linkcolor=bluecol,
    %    pagecolor=bluecol,
    pdfborder={0 0 0},
    %pdfborderstyle={/S/U/W 1},
    %%    backref,     %link from bibliography back to sections
    %%    pagebackref, %link from bibliography back to pages
    %%    pdfstartview=FitH, %fitwidth instead of fit window
    pdfpagemode=UseOutlines, %bookmarks are displayed by acrobat
    pdftitle={\course},
    pdfauthor={Marc Toussaint},
    pdfkeywords={}
  ]{hyperref}
  \DeclareGraphicsExtensions{.pdf,.png,.jpg,.eps}

  %\usepackage{multirow}
  \usepackage{multimedia}
  %\usepackage{marginnote}
  %\setbeamercolor{background canvas}{bg=}

  \usepackage[round]{natbib}
  \bibliographystyle{abbrvnat}

  \renewcommand{\r}{\varrho}
  \renewcommand{\l}{\lambda}
  \renewcommand{\L}{\Lambda}
  \renewcommand{\b}{\beta}
  \renewcommand{\d}{\delta}
  \renewcommand{\k}{\kappa}
  \renewcommand{\t}{\theta}
  \renewcommand{\O}{\Omega}
  \renewcommand{\o}{\omega}
  \renewcommand{\SS}{{\cal S}}
  \renewcommand{\=}{\!=\!}
  %\renewcommand{\boldsymbol}{}
  %\renewcommand{\Chapter}{\chapter}
  %\renewcommand{\Subsection}{\subsection}

  \renewcommand{\baselinestretch}{1.0}
  \geometry{a5paper,headsep=6mm,hdivide={10mm,*,10mm},vdivide={13mm,*,7mm}}

  \fancyhead[OL,ER]{\course, \textit{Marc Toussaint}}
  \fancyhead[OR,EL]{\thepage}
  \fancyhead[C]{}
  \fancyfoot{}
  \pagestyle{fancy}

  \renewcommand{\labelenumi}{{(\roman{enumi})}}
  \renewcommand{\theenumi}{(\roman{enumi})} %for ref
  \parindent 0pt
  \parskip .5pc

  \columnsep 6ex

  \renewcommand{\familydefault}{\sfdefault}

  \newcommand{\headerfont}{\large}%helvetica{12}{1}{b}{n}}
  \newcommand{\slidefont} {}%\helvetica{9}{1.3}{m}{n}}
  \newcommand{\storyfont} {}
  %  \renewcommand{\small}   {\helvetica{8}{1.2}{m}{n}}
  \renewcommand{\tiny}    {\footnotesize}%helvetica{7}{1.1}{m}{n}}
  \newcommand{\ttiny} {\footnotesize}%fontsize{7}{7}\selectfont}
%  \newcommand{\codefont}{\fontsize{6}{6}\selectfont}%helvetica{8}{1.2}{m}{n}}
  \newcommand{\codefont} {\helvetica{8}{1.2}{m}{n}}

  \input{../latex/macros}

  \usepackage{comment}
  \specialcomment{solution}{
    \small
    \begin{shaded}
  }{
    \end{shaded}
  }

  \graphicspath{{pics/}{../pics/}{pics-local/}}

  \mytitle{\course\\Lecture Script}
  \myauthor{Marc Toussaint}
  \date{\coursedate}
}

%%%%%%%%%%%%%%%%%%%%%%%%%%%%%%%%%%%%%%%%%%%%%%%%%%%%%%%%%%%%%%%%%%%%%%%%%%%%%%%%

\newcommand{\scripttitle}{
  \begin{document}
  \maketitle
  %\anchor{100,10}{\includegraphics[width=4cm]{optim}}
%  \vspace*{1cm}
}

%%%%%%%%%%%%%%%%%%%%%%%%%%%%%%%%%%%%%%%%%%%%%%%%%%%%%%%%%%%%%%%%%%%%%%%%%%%%%%%%

\newcounter{mypage}
\setcounter{mypage}{0}
\newcommand{\incpage}{\addtocounter{mypage}{1}}

\newcommand{\subtopic}{}
\newcommand{\pause}{}
\newcommand{\only}[1]{#1}

\renewcommand{\slides}[1][]{
  %  \clearpage
  \subsection{\topic}
  \index{\topic}
  {\small #1}
  \setcounter{mypage}{0}
  \smallskip\nopagebreak\hrule\medskip
}

\newcommand{\slidesfoot}{
  \bigskip
}

\newcommand{\sublecture}[2]{
  \phantomsection\addcontentsline{toc}{subsubsection}{#1}
  \index{#1}
}

\newcommand{\sublectureHide}[2]{
  \renewcommand{\subtopic}{#1}
}

\newcommand{\key}[1]{
  \phantomsection\addcontentsline{toc}{subsubsection}{#1}
  %\subsubsection{#1}
  \index{#1}
}

\providecommand{\defn}[1]{%
  \textbf{#1}\index{#1}%
}

\newenvironment{slidecore}[1]{
  \incpage
  \subsubsection*{#1}%{\headerfont\noindent\textbf{#1}\\}%
  \vspace{-6ex}%
  \begin{list}{$\bullet$}{\leftmargin4ex \rightmargin0ex \labelsep1ex
    \labelwidth2ex \partopsep0ex \topsep0ex \parsep.5ex \parskip0ex \itemsep0pt}\item[]~\\\nopagebreak%
}{
  \end{list}\nopagebreak%
  {\hfill\tiny \textsf{\arabic{section}.\arabic{subsection}:\arabic{mypage}}}\nopagebreak%
  \smallskip\nopagebreak\hrule
}

\newcommand{\slide}[2]{
  \begin{slidecore}{#1}
    #2
  \end{slidecore}
}

\renewcommand{\exercises}{}
\newcommand{\exercisestitle}{}
\newcommand{\exsection}[1]{\subsubsection{#1}}
\newcommand{\exsubsection}[1]{\paragraph{#1}}
\newcommand{\exerfoot}{\bigskip}

\newcommand{\story}[1]{
  \subsection*{Motivation \& Outline}
  {\storyfont\sf #1}
  \medskip\nopagebreak\hrule
}

\newcounter{savedsection}
\newcommand{\subappendix}{\setcounter{savedsection}{\arabic{section}}\appendix}
\newcommand{\noappendix}{
  \setcounter{section}{\arabic{savedsection}}% restore section number
  \setcounter{subsection}{0}% reset section counter
%  \gdef\@chapapp{\sectionname}% reset section name
  \renewcommand{\thesection}{\arabic{section}}% make section numbers arabic
}

\newenvironment{items}[1][9]{
\par\setlength{\unitlength}{1pt}\fontsize{#1}{#1}\linespread{1.2}\selectfont
\begin{list}{--}{\leftmargin4ex \rightmargin0ex \labelsep1ex \labelwidth2ex
\topsep0pt \parsep0ex \itemsep3pt}
}{
\end{list}
}

\newenvironment{itemS}[1][4ex]{
\par
\tiny
\begin{list}{--}{\leftmargin#1 \rightmargin0ex \labelsep1ex
  \labelwidth2ex \topsep0pt \parsep0ex \itemsep2pt}
}{
\end{list}
}

\newcommand{\Def}[1]{%
\textbf{#1}\index{#1}}%\marginnote{#1}}

  \scripthead
}

\providecommand{\paper}{
  \input{../latex/style-paper}
  \paperhead
}

\providecommand{\note}[1][9pt]{
  \providecommand{\notehead}[2]{
  \documentclass[#1,fleqn,twoside]{article}
  \stdpackages
  \renewcommand{\labelenumi}{{(\roman{enumi})}}
  \renewcommand{\theenumi}{(\roman{enumi})} %for ref

  \renewcommand{\baselinestretch}{#2}
  \renewcommand{\arraystretch}{1.2}
  \renewcommand{\topfraction}{1}
  \renewcommand{\bottomfraction}{1}
  \renewcommand{\textfraction}{0}
  \columnsep 5ex
  \parindent 3ex
  \parskip 1ex

  % Lists and paragraphs
  \parindent 0pt
  \topsep 4pt plus 1pt minus 2pt
  \partopsep 1pt plus 0.5pt minus 0.5pt
  \itemsep 2pt plus 1pt minus 0.5pt
  \parsep 2pt plus 1pt minus 0.5pt
  \parskip .5pc %add _in_ {thebibliography} environment in *.bbl

  \setcounter{tocdepth}{3}
  \setcounter{secnumdepth}{3}

  \geometry{a4paper,hdivide={25mm,*,25mm},vdivide={25mm,*,25mm}}

  \renewcommand{\headrulewidth}{.0pt}\renewcommand{\footrulewidth}{.0pt}\cfoot{}
  \fancyhead[OL,EC]{\it\theauthor---\today}
  \fancyhead[ER]{\leftmark}
  \fancyhead[OR,EL]{\thepage}
  \fancyfoot[EL,OR]{}
  \setlength{\headsep}{10mm}
  %\fancyhead[OL]{\rightmark}
  %\fancyfoot[EL,OR]{}


  %  \usepackage{palatino}

  \newcommand{\codefont}{\helvetica{8}{1.2}{m}{n}}

  \renewenvironment{abstract}{
    \vspace*{5ex}\begin{list}{}{
      \leftmargin3ex
      \rightmargin3ex
      \topsep-\parskip}\item[]
     \hrule\vspace{1.5ex}{\bf Abstract.~}\small}
    {\vspace{2ex}\hrule\end{list}\vspace{5ex}}
    
  \newenvironment{keyword}
    {\par{\it Keywords:~}}
    {}

  \def\makemytitle{%
    \thispagestyle{empty}
    \begin{list}{}{\leftmargin3ex \rightmargin3ex \topsep0ex \parsep0ex}\item[]
      \begin{center}
        {\fontsize{18}{25}\selectfont{\thetitle\\}}\vspace{5ex}

        {\fontsize{14}{16}\selectfont{\theauthor\\}}\vspace{1ex}

        {\footnotesize{\sl \addressFUB}\\ \emailBerlin}

        {\footnotesize \today}

        \vspace{1ex}
        {\small \published}
      \end{center}
    \end{list}
    \renewcommand{\maketitle}{\chapter{\thetitle}}
  }

  \input{../latex/macros}
  \pdflatex

  \graphicspath{{pics/}{../pics/}{pics-local/}}

  \myauthor{Marc Toussaint}
  \date{\today}
}

%%%%%%%%%%%%%%%%%%%%%%%%%%%%%%%%%%%%%%%%%%%%%%%%%%%%%%%%%%%%%%%%%%%%%%%%%%%%%%%%

\newcommand{\notetitle}{
  \begin{document}
  \thispagestyle{empty}
    
  \maketitle

}

\newenvironment{items}[1][9]{
\par\setlength{\unitlength}{1pt}\fontsize{#1}{#1}\linespread{1.2}\selectfont
\begin{list}{--}{\leftmargin4ex \rightmargin0ex \labelsep1ex \labelwidth2ex
\topsep0pt \parsep0ex \itemsep3pt}
}{
\end{list}
}

  \notehead{#1}{1.1}
}

\providecommand{\course}{NO COURSE}
\providecommand{\coursepicture}{NO PICTURE}
\providecommand{\coursedate}{NO DATE}
\providecommand{\topic}{NO TOPIC}
\providecommand{\keywords}{}
\providecommand{\exnum}{NO NUMBER}
\providecommand{\teacher}{Marc Toussaint}

\providecommand{\stdpackages}{
  \usepackage{amsmath}
  \usepackage{amssymb}
  \usepackage{amsfonts}
  \allowdisplaybreaks
  \usepackage{amsthm}
  \usepackage{eucal}
  \usepackage{graphicx}
%  \usepackage{color}
  \usepackage{geometry}
  \usepackage{framed}
  \usepackage{xcolor}
  \definecolor{shadecolor}{gray}{0.9}
  \setlength{\FrameSep}{3pt}
  \usepackage{fancyvrb}
  \fvset{numbers=none,xleftmargin=5ex,fontsize=\small}

  \usepackage{pdfpages}

  \usepackage{multicol} 
  \usepackage{fancyhdr}
}

\providecommand{\addressUSTT}{
  Machine~Learning~\&~Robotics~lab, U~Stuttgart\\\small
  Universit{\"a}tsstra{\ss}e 38, 70569~Stuttgart, Germany
}

\providecommand{\addressTUB}{
  Learning~\&~Intelligent~Systems~Lab, TU~Berlin\\\small
  Marchstr. 23, 10587 Berlin, Germany
}


\renewcommand{\course}{Optimization Algorithms}
\renewcommand{\coursepicture}{optim}
\renewcommand{\coursedate}{Winter 2024/25}

\renewcommand{\topic}{Implicit Functions \& Differentiable Optimization}
\renewcommand{\keywords}{}

\slides

\slidestitle

%%%%%%%%%%%%%%%%%%%%%%%%%%%%%%%%%%%%%%%%%%%%%%%%%%%%%%%%%%%%%%%%%%%%%%%%%%%%%%%%

\slide{Outline}{

\item Implicit Functions
\begin{items}
\item Definition
\item Implicit Function Theorem and differentiation
\end{items}

\item Differentiable Optimization

}

%%%%%%%%%%%%%%%%%%%%%%%%%%%%%%%%%%%%%%%%%%%%%%%%%%%%%%%%%%%%%%%%%%%%%%%%%%%%%%%%

\sublecture{Implicit Functions}{
}

%%%%%%%%%%%%%%%%%%%%%%%%%%%%%%%%%%%%%%%%%%%%%%%%%%%%%%%%%%%%%%%%%%%%%%%%%%%%%%%%

\key{Implicit Functions}
\slide{What is an Implicit Function?}{

\item A function $F: \RRR^d \to Y$ can be defined \textbf{implicitly}, e.g.\ via
$$F(x) = \argmin_y f(x,y) \qquad \text{optimality formulation}$$
or alternatively via
$$F(x) = y \st f(x,y) = 0 \qquad \text{standard (root) formulation}$$

\item $F$ is called \emph{implicit function}, $f$ is sometimes called \textbf{discriminative function}, as it discriminates ``correct'' outputs $y$
from others. \pause Examples:
\begin{items}
\item \textbf{ML classification}: A classifier $F: \RRR^d \to \{A,B,C\}$ is represented via a discriminative function $f(x,y)$ that assignes different neg-likelihoods to the three possible outputs $y\in\{A,B,C\}$ (cf.\ logistic regression, multi-class classification, conditional random fields).
\item \textbf{Implicit Surface Functions}: A 3D surface is implicitly defined as the \emph{set} of points $y\in\RRR^3$ for which $f(y)=0$ (often no parameter $x$ here) (cf. recent work in CV and robotics to use neural implicit functions (NIF) to represent objects and scenes).
\item \textbf{Control \& Robot Motion}: Optimal control and robot are described via optimality principles, e.g., motion such that various constraints $h(\text{environment},\text{motion})=0$ are fulfilled.
\end{items}

%% ~\pause

%% \item Both formulations (optimality, root) are of course related. The standard is the root formulation $F(x) = y \st f(x,y) = 0$.

}

%%%%%%%%%%%%%%%%%%%%%%%%%%%%%%%%%%%%%%%%%%%%%%%%%%%%%%%%%%%%%%%%%%%%%%%%%%%%%%%%

\slide{Implicit Function Theorem}{

$$F: x \mapsto y \st f(x,y) = 0$$
where $f: \RRR^d \times \RRR^n \to \RRR^n$ has $n$-dimensional output

~

\item Is $F$ really well-defined? E.g., what if no $y$ solves $f(x,y) = 0$? What if multiple $y$ solve $f(x,y) = 0$?

}

%%%%%%%%%%%%%%%%%%%%%%%%%%%%%%%%%%%%%%%%%%%%%%%%%%%%%%%%%%%%%%%%%%%%%%%%%%%%%%%%

\key{Implicit Function Theorem}
\slide{Implicit Function Theorem}{

\small

\item \textbf{Theorem:} Let $f(x,y)$, $x\in\RRR^d, y\in\RRR^n$ be a continuously differentiable $\RRR^n$-valued function (in $C^1$). Assume we have a point $(x^*,y^*)\in\RRR^{d+n}$ where
$$f(x^*,y^*)=0 \quad\text{and}\quad \det \Del y f(x^*,y^*) \not=0 ~.$$

a) Then there exists a radius $r$ such that for each $x$, $|x-x^*|<r$, there exists a \textbf{unique} $y = F(x)$ such that $f(x,y)=0$.

b) The implicit function $F$ is continuously differentiable, and

\cen{$f(x,F(x)) = 0 \quad\To \quad \Del x f(x,y) + \Del y f(x,y) \Del x F(x) = 0$ \quad at $y=F(x)$,}

and since $\Del y f$ is invertible, we have
$$\Del x F(x) = - [\Del y f(x,y)]^\1 \Del x f(x,y)~.$$

~\tiny \pause

\item  $\det \Del y f(x^*,y^*) \not=0 \iff$ Jacobian w.r.t.\ $y$ has full rank $\iff$ $f(x,y)=0$ has non-zero gradient in all $y$-directions

}

%%%%%%%%%%%%%%%%%%%%%%%%%%%%%%%%%%%%%%%%%%%%%%%%%%%%%%%%%%%%%%%%%%%%%%%%%%%%%%%%

\slide{Interpretation in view of Newton step*}{

\small

(Same statement, just derived as Newton step for root finding)

\item Assume you already found $y^*$ to solve $f(x^*,y^*)=0$ for a given $x^*$. But now the parameter/input $x$ varies slightly. How does the solution $y$ vary?

\item Consider the 1st order Taylor approximation of $f$:
$$f(x,y) = \underbrace{f(x^*,y^*)}_{=0} + \Del x f(x^*,y^*)~ (x-x^*) + \Del y f(x^*,y^*)~ (y-y^*)$$

If we also want $f(x,y)=0$, then we need
$$ (y-y^*) = - [\Del y f]^\1~ \Del x f~ (x-x^*) ~,$$

which is the Newton step for root finding, and coincides with the Implicit Function Theorem.

}

%%%%%%%%%%%%%%%%%%%%%%%%%%%%%%%%%%%%%%%%%%%%%%%%%%%%%%%%%%%%%%%%%%%%%%%%%%%%%%%%

\sublecture{Differentiable Optimization}{
}

%%%%%%%%%%%%%%%%%%%%%%%%%%%%%%%%%%%%%%%%%%%%%%%%%%%%%%%%%%%%%%%%%%%%%%%%%%%%%%%%

\key{Differentiable Optimization}
\slide{The KKT Implicit Function}{

\item Consider a \textbf{parameterized} problem $$x^*(\t) = \argmin_x f(\t,x) \st g(\t,x)\le0,~ h(\t,x)=0$$

\pause

\item We define the \textbf{implicit function} $F: \t \mapsto (x^*,\k^*,\l^*) \st r(\t,x,\k,\l)=0$ for the KKT residual
\begin{align*}
& r(\t,x,\k,\l) = 
\mat{c}{
\na {~} [f(\t,x) + \l^\T g(\t,x) + \k^\T h(\t,x)] \\
h(\t,x) \\
\diag(\l) g(\t,x)
}
\end{align*}
(i.e., for any $\t$, $F$ outputs the primal and dual solution to the KKT conditions.)

\pause

\item In particular,  at $(x,\k,\l)=F(\t)$ we have
$$
\Del \t F = -[\Del {{}_{x\k\l}} r]^\1~ \Del \t r ~.
$$

}

%%%%%%%%%%%%%%%%%%%%%%%%%%%%%%%%%%%%%%%%%%%%%%%%%%%%%%%%%%%%%%%%%%%%%%%%%%%%%%%%

\slide{The KKT Implicit Function}{

$$
\Del \t F = -[\Del {{}_{x\k\l}} r]^\1~ \Del \t r ~.
$$
\begin{items}
\item The matrix $\Del {x\k\l} r \in \RRR^{(n+l+m) \times (n+l+m)}$ is the \defn{KKT Jacobian} (cf. Primal-Dual Newton!)
$$\Del {{}_{x\k\l}} r = \mat{ccc}{
\he {} [f + \l^\T g + \k^\T h] & \del_x h^\T & \del_x g^\T \\
\del_x h & 0 & 0 \\
\diag(\l) \del_x g & 0 & \diag(g) \\
}$$
\item The vector $\Del \t r \in \RRR^{n+l+m}$ describes how the KKT residual depends on $\t$:
$$
\Del \t r
= \mat{c}{
\del_\t \na {~} [f + \l^\T g +\k^\T h]\\
\del_\t h \\
\diag(\l) \del_\t g
}
$$
\end{items}

\item E.g., for a small variation $(\t-\t^*)$, the new optimum is (in linear approx.) at
$$(x,\k,\l) = (x^*,\k^*,\l^*) -[\Del {{}_{x\k\l}} r]^\1~ \Del \t r~ (\t-\t^*)$$

}

%%%%%%%%%%%%%%%%%%%%%%%%%%%%%%%%%%%%%%%%%%%%%%%%%%%%%%%%%%%%%%%%%%%%%%%%%%%%%%%%

%% \slide{The KKT Implicit Function}{

%% \item Consider variation $\tilde f = f+\e \hat f$,
%% $\tilde g = g+\e \hat g$, $\tilde h = h+\e \hat h$; how does $x^*$ vary?

%% \item The KKT-residual, and corresponding primal-dual Newton step are
%% \begin{align*}
%% \hat r
%% &= \e\mat{c}{
%% \na \hat f + \del \hat g^\T \l + \del \hat h^\T \k\\
%% \hat h \\
%% \diag(\l) \hat g
%% } \\
%% \e\mat{c}{
%% \hat x \\
%% \hat \l \\
%% \hat \k
%% }
%% &=
%% - \mat{ccc}{
%% \he f & \del g^\T & \del h^\T \\
%% \del h & 0 & 0 \\
%% \diag(\l) \del g & \diag(g) & 0 \\
%% }^\1~
%% \e\mat{c}{
%% \na \hat f + \del \hat g^\T \l + \del \hat h^\T \k\\
%% \hat h \\
%% \diag(\l) \hat g
%% }
%% \end{align*}

%% \item The new optimum is at $x^* + \e\hat x$
%% %% \begin{items}
%% %% \item Insight: This derivation implies stability of constraint activity, which is ``standard constraint qualification'' in the optimization literature
%% %% \end{items}

%% }

%%%%%%%%%%%%%%%%%%%%%%%%%%%%%%%%%%%%%%%%%%%%%%%%%%%%%%%%%%%%%%%%%%%%%%%%%%%%%%%%

\slide{Example}{

\item Assume $\phi(x;\t)$ is a NN with parameters $\t\in\RRR^d$, inputs $x\in\RRR^n$, outputs $\phi(x;\t)\in\RRR^o$

\item For given $\t$, a Newton method converges to $x^* = \argmin_x \phi(x;\t)^2$

(We assume a least squares form $f(\t,x) = \phi(x;\t)^2$, it could be $o=1$)

\item What is $\frac{d x^*}{d \t} = \Del \t F$?

\item Since we have no $\k,\l$ here, we have
\begin{align*}
\Del \t F &= -[\Del x r]^\1~ \Del \t r \\
\Del x r &= \he f \comma \Del \t r = \del_\t \na f \\
\Del \t F &= -[\he f]^\1~ \del_\t \na f
\end{align*}
\tiny

where we could approximate $\he f(x) \approx 2 J^\T J$, with the NN's Jacobian $J = \del_x \phi(x;\t)$.

%% {\tiny ($ \del_\t \na f$ describes how the \emph{gradient} of $f$ at $x^*$ changes with $\t$ -- if the gradient does not change, the optimum $x^*$ does not vary.)

}

}

%%%%%%%%%%%%%%%%%%%%%%%%%%%%%%%%%%%%%%%%%%%%%%%%%%%%%%%%%%%%%%%%%%%%%%%%%%%%%%%%

\slide{Switching Constraints Example}{

\item For $x\in\RRR$, Consider the problem
$$\min_x (x-\t)^2 \st x\ge 0 ~.$$
What is the implicit function $F(\t) = x^*$?

~\pause

$$F(\t)= x^* = \max \{0,\t\}$$
\anchor{300,20}{\showh[.15]{relu}}

~

which is non-differentiable at $\t=0$.

}


%%%%%%%%%%%%%%%%%%%%%%%%%%%%%%%%%%%%%%%%%%%%%%%%%%%%%%%%%%%%%%%%%%%%%%%%%%%%%%%%

\slide{Limitation -- Constraint Activity Switching}{

\small

\item Note that the KKT residual $r(\t,x,\k,\l)=0$ neglects the conditions $g(\t,x)\le 0, \l\ge 0$

\item The Implicit Function Theorem assumes $r \in C^1$ and $\det\del_{x\k\l}r \not= 0$, but when constraint activity switches, $r$ changes in a non-differentiable manner.

~\pause

\item[$\to$] In a \textbf{vicinity} of a solution $x^*,\k^*,\l^*$, we may assume that constraint activity is stable, the inequalities $g(x)\le 0, \l\ge 0$ remain fulfilled, and that the Jacobian of active constraints have full rank (aka.\ \emph{constraint qualification assumption}).

THEN, \textbf{locally}, the implicit function theorem holds and we have the correct gradient.

\item However, in general, constraint activity switches somewhere -- then we have a discontinuity in the active constraint Jacobians, and in the implicit function gradient.

~\pause

\cen{\textbf{$\To$ NLPs with inequalities are \emph{piece-wise} differentiable!}}

}


%%%%%%%%%%%%%%%%%%%%%%%%%%%%%%%%%%%%%%%%%%%%%%%%%%%%%%%%%%%%%%%%%%%%%%%%%%%%%%%%

\slide{Classical Literature: ``Sensitivity Analysis''}{

\item Lot's of classical literature on differentiation through NLP solutions:
  
\begin{items}\tiny
\item Ralph \& Dempe. \textbf{Directional derivatives of the solution of a parametric nonlinear program.} \textbf{1994}. Research Report.

\item Fiacco \& Kyparisis.  \textbf{Sensitivity analysis in nonlinear programming} under second order assumptions. Lecture Notes in Control and Information Sciences, 74-97, \textbf{1985}.

\item Kyparisis. Sensitivity analysis for nonlinear programs and variational inequalities with nonunique multipliers. Mathematics of Operations Research, 15:286–298, 1990.

%% \item Levy \& Rockafellar. Sensitivity analysis of solutions to generalized equations. Trans. Amer. Math. Soc. 1993.

%% \item Poliquin \& Rockafellar. \textbf{Proto-derivative} formulas for basic \textbf{subgradient mappings} in mathematical programming.
%% Set-valued Analysis, 2:275–290, 1994.

\item Levy \& Rockafellar. Sensitivity of solutions in nonlinear programs with nonunique multiplier. Recent Adv.\ in Nonsmooth Optimzation: 215-223, 1995

\end{items}

(More recent publications at NeurIPS (keyword ``Differentiable Optimization'') ignore this classical literature.)

}

%%%%%%%%%%%%%%%%%%%%%%%%%%%%%%%%%%%%%%%%%%%%%%%%%%%%%%%%%%%%%%%%%%%%%%%%%%%%%%%%

\slide{Classical Literature: ``Sensitivity Analysis''}{

\item The implicit function $F(\t)$ is also called \emph{quasi-solution mapping:} Assume a parameterized NLP $\PP(\t)$
\begin{align*}
F: \t \mapsto \{ x : \text{KKT hold for } \PP(\t) \}
\end{align*}

\emph{``We show \textbf{under a standard constraint qualification}, not requiring uniqueness of the
 multipliers, that the quasi-solution mapping is differentiable in a
 generalized sense, and we present a formula for its derivative.''}

\item Constant rank constraint qualification (CRCQ): For each subset of the gradients of the active inequality constraints and the gradients of the equality constraints the rank at a vicinity of $x^*$ is constant. 


}

%%%%%%%%%%%%%%%%%%%%%%%%%%%%%%%%%%%%%%%%%%%%%%%%%%%%%%%%%%%%%%%%%%%%%%%%%%%%%%%%

\slide{Conclusions}{\label{lastpage}

\small

\item We can analyze how changes in the optimization problem translate to changes of the optimium $x^*$

\item Using the KKT Jacobian, we can provide the gradient of $x^*$ w.r.t.\ problem parameters $\t$

\item We can embed optimization algos in auto-differentiation computation graphs (torch, tensorflow)

\item Important implications for Differentiable Physics

\item \textbf{But:} Gradients can be discontinuous across constraint activations

}

%%%%%%%%%%%%%%%%%%%%%%%%%%%%%%%%%%%%%%%%%%%%%%%%%%%%%%%%%%%%%%%%%%%%%%%%%%%%%%%%

\slidesfoot
