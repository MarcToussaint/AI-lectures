\providecommand{\slides}{
  \newcommand{\slideshead}{
  \newcommand{\thepage}{\arabic{mypage}}
  %beamer
%  \documentclass[t,hyperref={bookmarks=true}]{beamer}
%  \geometry{papersize={171mm,96mm}}
  \documentclass[t,hyperref={bookmarks=true},aspectratio=169]{beamer}
  \setbeamersize{text margin left=5mm}
  \setbeamersize{text margin right=5mm}
  \usetheme{default}
  \usefonttheme[onlymath]{serif}
  \setbeamertemplate{navigation symbols}{}
  \setbeamertemplate{itemize items}{{\color{black}$\bullet$}}

  \newwrite\keyfile

  %\usepackage{palatino}
  \stdpackages
  %\usepackage{tikz} \usetikzlibrary {shapes.geometric} 
  \usepackage{multimedia}
  \usepackage[utf8]{inputenc}

  %%% geometry/spacing issues
  %
  \definecolor{bluecol}{rgb}{0,0,.5}
  \definecolor{greencol}{rgb}{0,.6,0}
  \definecolor{citcol}{rgb}{.4,.4,.4}
  %\renewcommand{\baselinestretch}{1.1}
  \renewcommand{\arraystretch}{1.2}
  \columnsep 0mm

  \columnseprule 0pt
  \parindent 0ex
  \parskip 0ex
  %\setlength{\itemparsep}{3ex}
  %\renewcommand{\labelitemi}{\rule[3pt]{10pt}{10pt}~}
  %\renewcommand{\labelenumi}{\textbf{(\arabic{enumi})}}
  \setbeamertemplate{enumerate item}{(\roman{enumi})}
  \newcommand{\headerfont}{\helvetica{14}{1.5}{b}{n}}
  \newcommand{\slidefont} {\helvetica{11}{1.4}{m}{n}}
  %\newcommand{\codefont} {\helvetica{8}{1.2}{m}{n}}
  \newcommand{\urlfont} {\fontsize{6}{1.1}\selectfont}
  \renewcommand{\small} {\helvetica{10}{1.4}{m}{n}}
  \renewcommand{\tiny} {\helvetica{8}{1.3}{m}{n}}
  \newcommand{\ttiny} {\helvetica{7}{1.3}{m}{n}}

  %%% count pages properly and put the page number in bottom right
  %
  \newcounter{mypage}
  \newcommand{\incpage}{\addtocounter{mypage}{1}\setcounter{page}{\arabic{mypage}}}
  \setcounter{mypage}{0}
  \resetcounteronoverlays{page}

  \pagestyle{fancy}
  %\setlength{\headsep}{10mm}
  %\addtolength{\footheight}{15mm}
  \renewcommand{\headrulewidth}{0pt} %1pt}
  \renewcommand{\footrulewidth}{0pt} %.5pt}
  \cfoot{}
  \rhead{}
  \lhead{}
  %% \lfoot{\vspace*{-3mm}\hspace*{-3mm}\helvetica{5}{1.3}{m}{n}{\texttt{github.com/MarcToussaint/AI-lectures}}}
\lfoot{}
%  \rfoot{{\tiny\textsf{AI -- \topic -- \subtopic -- \arabic{mypage}/\pageref{lastpage}}}}
  %\lfoot{\raisebox{5mm}{\tiny\textsf{\slideauthor}}}
  %\rfoot{\raisebox{5mm}{\tiny\textsf{\slidevenue{} -- \arabic{mypage}/\pageref{lastpage}}}}
  %\rfoot{~\anchor{30,12}{\tiny\textsf{\thepage/\pageref{lastpage}}}}
  %\lfoot{\small\textsf{Marc Toussaint}}
%  \rfoot{\vspace*{-4.5mm}{\tiny\textsf{\color{gray}\topic\ -- \subtopic\ -- \arabic{mypage}/\pageref{lastpage}}}\hspace*{-4mm}}
  \rfoot{\vspace*{-4.5mm}{\tiny\textsf{\color{gray}\topic\ -- \arabic{mypage}/\pageref{lastpage}}}\hspace*{-4mm}}
%  \rfoot{~\anchor{-10,12}{\tiny\textsf{\color{gray}\topic\ --  \arabic{mypage}/\pageref{lastpage}}}}
  \lfoot{\vspace*{-4.5mm}{\hspace*{-3mm}\includegraphics[height=4mm]{LIS-logo-longText}}}

  \definecolor{grey}{rgb}{.8,.8,.8}
  \definecolor{head}{rgb}{.85,.9,.9}
%  \definecolor{blue}{rgb}{.0,.0,.5}
%  \definecolor{green}{rgb}{.0,.5,.0}
  \definecolor{red}{rgb}{.8,.0,.0}
  \newcommand{\inverted}{
    \definecolor{main}{rgb}{1,1,1}
    \color{main}
    \pagecolor[rgb]{.3,.3,.3}
  }
  \input{../latex/macros}

  \graphicspath{{pics/}{../pics/}{pics-local/}}
}

\newcommand{\slidestitle}{
  \title{\course \topic}
  \author{Marc Toussaint}
  \institute{Learning \& Intelligent Systems Lab, TU Berlin}

  \begin{document}


  %% title slide!
  \slide{}{
    \thispagestyle{empty}

    \twocol{.35}{.55}{
      %\vspace*{-5mm}%
      \hspace*{-5mm}%
      \includegraphics[width=1.\columnwidth]{\coursepicture}
    }{\center

      \textbf{\fontsize{17}{20}\selectfont \course}

      ~

      %Lecture
      \topic\\

      \vspace{1cm}

      {\tiny~\emph{\keywords}~\\}

      \vspace{1cm}

      \teacher
      
      Technical University of Berlin

      \coursedate

      ~

    }
  }
}

\newcommand{\slide}[2]{
  \slidefont
  \incpage\begin{frame}
  \addcontentsline{toc}{section}{#1}
  \vfill
  {\headerfont #1} \vspace*{-2ex}
  \begin{itemize}\item[]~\\
    #2
  \end{itemize}
  \vfill
  \end{frame}
}

% use \begin{frame}[fragile] around slidecore!
\newenvironment{slidecore}[1]{
  \slidefont\incpage
  \addcontentsline{toc}{section}{#1}
  \vfill
  {\headerfont #1} \vspace*{-2ex}
  \begin{itemize}\item[]~\\
}{
  \end{itemize}
  \vfill
}

\newcommand{\titleslide}[4][Marc Toussaint]{
  \newcommand{\slideauthor}{#1}
  \newcommand{\slidevenue}{#3}
  \slidefont
  \incpage
  \begin{frame}
  \begin{center}
    \vspace*{15mm}

    {\headerfont #2\\}
        
    \vspace*{10mm}

    #1 \\

    \vspace*{5mm}

    {\small
      Learning \& Intelligent Systems Lab, TU Berlin\\
%      Science of Intelligence Cluster of Excellence, Berlin\\
      Max Planck Fellow, Institute for Intelligent Systems\\ %Physical Reasoning \& Manipulation Lab -- 
%      MIT CSAIL\\
%      mtoussai@mit.edu,~ marc.toussaint@informatik.uni-stuttgart.de

      \vspace*{10mm}

      \emph{#3}
    }

    \vspace*{0mm}

    %\includegraphics[scale=.1]{pics/eushield-fullcolour}

  \end{center}
  \begin{itemize}\item[]~\\
    #4
  \end{itemize}
  \end{frame}
}

\newcommand{\titleslideempty}[2]{
  \slidefont
  \incpage
  \begin{frame}
  \begin{center}
    \vspace*{15mm}

    {\headerfont #1\\}
        
    %% \vspace*{5mm}

    %% {\small\emph{#2}} \\

  \end{center}
  \begin{itemize}\item[]~\\
    #2
  \end{itemize}
  \end{frame}
}

\providecommand{\key}[1]{
  \addtocounter{mypage}{1}
% \immediate\write\keyfile{#1}
  \addtocontents{toc}{\hyperref[key:#1]{#1 (\arabic{mypage})}}
%  \phantomsection\label{key:#1}
%  \index{#1@{\hyperref[key:#1]{#1 (\arabic{mysec}:\arabic{mypage})}}|phantom}
  \addtocounter{mypage}{-1}
}

\providecommand{\course}{}

\providecommand{\subtopic}{}

\providecommand{\sublecture}[2]{
  \renewcommand{\subtopic}{#1}
  \slide{#1}{#2}
}

\providecommand{\sublectureHide}[2]{
  \renewcommand{\subtopic}{#1}
}

\providecommand{\story}[1]{
~

Motivation: {\tiny #1}\clearpage
}

\newenvironment{items}[1][9]{
\par\setlength{\unitlength}{1pt}\fontsize{#1}{#1}\linespread{1.2}\selectfont
\begin{list}{--}{\leftmargin4ex \rightmargin0ex \labelsep1ex \labelwidth2ex
\topsep.7ex \parsep0ex \itemsep3pt}
}{
\end{list}
}

\newenvironment{itemS}[1][4ex]{
\par
\tiny
\begin{list}{--}{\leftmargin#1 \rightmargin0ex \labelsep1ex
  \labelwidth2ex \topsep0pt \parsep0ex \itemsep2pt}
}{
\end{list}
}

\providecommand{\slidesfoot}{
  \end{document}
}


  \slideshead
  %\slidestitle
}

\providecommand{\exercises}{
  \input{../latex/style-exercises}
  \exerciseshead
}

\providecommand{\script}{
  \newcommand{\scripthead}{
  \documentclass[9pt,fleqn,twoside]{article}
  \stdpackages

  \usepackage{makeidx}
  \makeindex

  \usepackage{thmtools}
  \definecolor{shadecolor}{gray}{0.85}
  \declaretheoremstyle[
    %headfont=\normalfont\bfseries,
    %notefont=\mdseries, notebraces={(}{)},
    %bodyfont=\normalfont,
    %postheadspace=0.5em,
    %spaceabove=6pt,
    mdframed={
      %  skipabove=8pt,
      %  skipbelow=6pt,
      hidealllines=true,
      backgroundcolor={shadecolor},
      innertopmargin=8pt,
      %  innerleftmargin=3pt,
      %  innerrightmargin=3pt
    }
  ]{shaded}
  \declaretheorem[style=shaded,within=section,name=Definition]{myDefinition}
  \declaretheorem[style=shaded,within=section,name=Theorem]{myTheorem}
  \declaretheorem[style=shaded,within=section,name=Identities]{Identities}
  \declaretheorem[style=shaded,within=section,name=Example]{myExample}

  \definecolor{grey}{rgb}{.8,.8,.8}
  \definecolor{bluecol}{rgb}{0,0,.5}
  \definecolor{greencol}{rgb}{0,.4,0}
  \definecolor{shadecolor}{gray}{0.9}
  \definecolor{citcol}{rgb}{.4,.4,.4}
  \usepackage[
    %    pdftex%,
    %%    letterpaper,
    %bookmarks,
    bookmarksnumbered,
    colorlinks,
    urlcolor=bluecol,
    citecolor=black,
    linkcolor=bluecol,
    %    pagecolor=bluecol,
    pdfborder={0 0 0},
    %pdfborderstyle={/S/U/W 1},
    %%    backref,     %link from bibliography back to sections
    %%    pagebackref, %link from bibliography back to pages
    %%    pdfstartview=FitH, %fitwidth instead of fit window
    pdfpagemode=UseOutlines, %bookmarks are displayed by acrobat
    pdftitle={\course},
    pdfauthor={Marc Toussaint},
    pdfkeywords={}
  ]{hyperref}
  \DeclareGraphicsExtensions{.pdf,.png,.jpg,.eps}

  %\usepackage{multirow}
  \usepackage{multimedia}
  %\usepackage{marginnote}
  %\setbeamercolor{background canvas}{bg=}

  \usepackage[round]{natbib}
  \bibliographystyle{abbrvnat}

  \renewcommand{\r}{\varrho}
  \renewcommand{\l}{\lambda}
  \renewcommand{\L}{\Lambda}
  \renewcommand{\b}{\beta}
  \renewcommand{\d}{\delta}
  \renewcommand{\k}{\kappa}
  \renewcommand{\t}{\theta}
  \renewcommand{\O}{\Omega}
  \renewcommand{\o}{\omega}
  \renewcommand{\SS}{{\cal S}}
  \renewcommand{\=}{\!=\!}
  %\renewcommand{\boldsymbol}{}
  %\renewcommand{\Chapter}{\chapter}
  %\renewcommand{\Subsection}{\subsection}

  \renewcommand{\baselinestretch}{1.0}
  \geometry{a5paper,headsep=6mm,hdivide={10mm,*,10mm},vdivide={13mm,*,7mm}}

  \fancyhead[OL,ER]{\course, \textit{Marc Toussaint}}
  \fancyhead[OR,EL]{\thepage}
  \fancyhead[C]{}
  \fancyfoot{}
  \pagestyle{fancy}

  \renewcommand{\labelenumi}{{(\roman{enumi})}}
  \renewcommand{\theenumi}{(\roman{enumi})} %for ref
  \parindent 0pt
  \parskip .5pc

  \columnsep 6ex

  \renewcommand{\familydefault}{\sfdefault}

  \newcommand{\headerfont}{\large}%helvetica{12}{1}{b}{n}}
  \newcommand{\slidefont} {}%\helvetica{9}{1.3}{m}{n}}
  \newcommand{\storyfont} {}
  %  \renewcommand{\small}   {\helvetica{8}{1.2}{m}{n}}
  \renewcommand{\tiny}    {\footnotesize}%helvetica{7}{1.1}{m}{n}}
  \newcommand{\ttiny} {\footnotesize}%fontsize{7}{7}\selectfont}
%  \newcommand{\codefont}{\fontsize{6}{6}\selectfont}%helvetica{8}{1.2}{m}{n}}
  \newcommand{\codefont} {\helvetica{8}{1.2}{m}{n}}

  \input{../latex/macros}

  \usepackage{comment}
  \specialcomment{solution}{
    \small
    \begin{shaded}
  }{
    \end{shaded}
  }

  \graphicspath{{pics/}{../pics/}{pics-local/}}

  \mytitle{\course\\Lecture Script}
  \myauthor{Marc Toussaint}
  \date{\coursedate}
}

%%%%%%%%%%%%%%%%%%%%%%%%%%%%%%%%%%%%%%%%%%%%%%%%%%%%%%%%%%%%%%%%%%%%%%%%%%%%%%%%

\newcommand{\scripttitle}{
  \begin{document}
  \maketitle
  %\anchor{100,10}{\includegraphics[width=4cm]{optim}}
%  \vspace*{1cm}
}

%%%%%%%%%%%%%%%%%%%%%%%%%%%%%%%%%%%%%%%%%%%%%%%%%%%%%%%%%%%%%%%%%%%%%%%%%%%%%%%%

\newcounter{mypage}
\setcounter{mypage}{0}
\newcommand{\incpage}{\addtocounter{mypage}{1}}

\newcommand{\subtopic}{}
\newcommand{\pause}{}
\newcommand{\only}[1]{#1}

\renewcommand{\slides}[1][]{
  %  \clearpage
  \subsection{\topic}
  \index{\topic}
  {\small #1}
  \setcounter{mypage}{0}
  \smallskip\nopagebreak\hrule\medskip
}

\newcommand{\slidesfoot}{
  \bigskip
}

\newcommand{\sublecture}[2]{
  \phantomsection\addcontentsline{toc}{subsubsection}{#1}
  \index{#1}
}

\newcommand{\sublectureHide}[2]{
  \renewcommand{\subtopic}{#1}
}

\newcommand{\key}[1]{
  \phantomsection\addcontentsline{toc}{subsubsection}{#1}
  %\subsubsection{#1}
  \index{#1}
}

\providecommand{\defn}[1]{%
  \textbf{#1}\index{#1}%
}

\newenvironment{slidecore}[1]{
  \incpage
  \subsubsection*{#1}%{\headerfont\noindent\textbf{#1}\\}%
  \vspace{-6ex}%
  \begin{list}{$\bullet$}{\leftmargin4ex \rightmargin0ex \labelsep1ex
    \labelwidth2ex \partopsep0ex \topsep0ex \parsep.5ex \parskip0ex \itemsep0pt}\item[]~\\\nopagebreak%
}{
  \end{list}\nopagebreak%
  {\hfill\tiny \textsf{\arabic{section}.\arabic{subsection}:\arabic{mypage}}}\nopagebreak%
  \smallskip\nopagebreak\hrule
}

\newcommand{\slide}[2]{
  \begin{slidecore}{#1}
    #2
  \end{slidecore}
}

\renewcommand{\exercises}{}
\newcommand{\exercisestitle}{}
\newcommand{\exsection}[1]{\subsubsection{#1}}
\newcommand{\exsubsection}[1]{\paragraph{#1}}
\newcommand{\exerfoot}{\bigskip}

\newcommand{\story}[1]{
  \subsection*{Motivation \& Outline}
  {\storyfont\sf #1}
  \medskip\nopagebreak\hrule
}

\newcounter{savedsection}
\newcommand{\subappendix}{\setcounter{savedsection}{\arabic{section}}\appendix}
\newcommand{\noappendix}{
  \setcounter{section}{\arabic{savedsection}}% restore section number
  \setcounter{subsection}{0}% reset section counter
%  \gdef\@chapapp{\sectionname}% reset section name
  \renewcommand{\thesection}{\arabic{section}}% make section numbers arabic
}

\newenvironment{items}[1][9]{
\par\setlength{\unitlength}{1pt}\fontsize{#1}{#1}\linespread{1.2}\selectfont
\begin{list}{--}{\leftmargin4ex \rightmargin0ex \labelsep1ex \labelwidth2ex
\topsep0pt \parsep0ex \itemsep3pt}
}{
\end{list}
}

\newenvironment{itemS}[1][4ex]{
\par
\tiny
\begin{list}{--}{\leftmargin#1 \rightmargin0ex \labelsep1ex
  \labelwidth2ex \topsep0pt \parsep0ex \itemsep2pt}
}{
\end{list}
}

\newcommand{\Def}[1]{%
\textbf{#1}\index{#1}}%\marginnote{#1}}

  \scripthead
}

\providecommand{\paper}{
  \input{../latex/style-paper}
  \paperhead
}

\providecommand{\note}[1][9pt]{
  \providecommand{\notehead}[2]{
  \documentclass[#1,fleqn,twoside]{article}
  \stdpackages
  \renewcommand{\labelenumi}{{(\roman{enumi})}}
  \renewcommand{\theenumi}{(\roman{enumi})} %for ref

  \renewcommand{\baselinestretch}{#2}
  \renewcommand{\arraystretch}{1.2}
  \renewcommand{\topfraction}{1}
  \renewcommand{\bottomfraction}{1}
  \renewcommand{\textfraction}{0}
  \columnsep 5ex
  \parindent 3ex
  \parskip 1ex

  % Lists and paragraphs
  \parindent 0pt
  \topsep 4pt plus 1pt minus 2pt
  \partopsep 1pt plus 0.5pt minus 0.5pt
  \itemsep 2pt plus 1pt minus 0.5pt
  \parsep 2pt plus 1pt minus 0.5pt
  \parskip .5pc %add _in_ {thebibliography} environment in *.bbl

  \setcounter{tocdepth}{3}
  \setcounter{secnumdepth}{3}

  \geometry{a4paper,hdivide={25mm,*,25mm},vdivide={25mm,*,25mm}}

  \renewcommand{\headrulewidth}{.0pt}\renewcommand{\footrulewidth}{.0pt}\cfoot{}
  \fancyhead[OL,EC]{\it\theauthor---\today}
  \fancyhead[ER]{\leftmark}
  \fancyhead[OR,EL]{\thepage}
  \fancyfoot[EL,OR]{}
  \setlength{\headsep}{10mm}
  %\fancyhead[OL]{\rightmark}
  %\fancyfoot[EL,OR]{}


  %  \usepackage{palatino}

  \newcommand{\codefont}{\helvetica{8}{1.2}{m}{n}}

  \renewenvironment{abstract}{
    \vspace*{5ex}\begin{list}{}{
      \leftmargin3ex
      \rightmargin3ex
      \topsep-\parskip}\item[]
     \hrule\vspace{1.5ex}{\bf Abstract.~}\small}
    {\vspace{2ex}\hrule\end{list}\vspace{5ex}}
    
  \newenvironment{keyword}
    {\par{\it Keywords:~}}
    {}

  \def\makemytitle{%
    \thispagestyle{empty}
    \begin{list}{}{\leftmargin3ex \rightmargin3ex \topsep0ex \parsep0ex}\item[]
      \begin{center}
        {\fontsize{18}{25}\selectfont{\thetitle\\}}\vspace{5ex}

        {\fontsize{14}{16}\selectfont{\theauthor\\}}\vspace{1ex}

        {\footnotesize{\sl \addressFUB}\\ \emailBerlin}

        {\footnotesize \today}

        \vspace{1ex}
        {\small \published}
      \end{center}
    \end{list}
    \renewcommand{\maketitle}{\chapter{\thetitle}}
  }

  \input{../latex/macros}
  \pdflatex

  \graphicspath{{pics/}{../pics/}{pics-local/}}

  \myauthor{Marc Toussaint}
  \date{\today}
}

%%%%%%%%%%%%%%%%%%%%%%%%%%%%%%%%%%%%%%%%%%%%%%%%%%%%%%%%%%%%%%%%%%%%%%%%%%%%%%%%

\newcommand{\notetitle}{
  \begin{document}
  \thispagestyle{empty}
    
  \maketitle

}

\newenvironment{items}[1][9]{
\par\setlength{\unitlength}{1pt}\fontsize{#1}{#1}\linespread{1.2}\selectfont
\begin{list}{--}{\leftmargin4ex \rightmargin0ex \labelsep1ex \labelwidth2ex
\topsep0pt \parsep0ex \itemsep3pt}
}{
\end{list}
}

  \notehead{#1}{1.1}
}

\providecommand{\course}{NO COURSE}
\providecommand{\coursepicture}{NO PICTURE}
\providecommand{\coursedate}{NO DATE}
\providecommand{\topic}{NO TOPIC}
\providecommand{\keywords}{}
\providecommand{\exnum}{NO NUMBER}
\providecommand{\teacher}{Marc Toussaint}

\providecommand{\stdpackages}{
  \usepackage{amsmath}
  \usepackage{amssymb}
  \usepackage{amsfonts}
  \allowdisplaybreaks
  \usepackage{amsthm}
  \usepackage{eucal}
  \usepackage{graphicx}
%  \usepackage{color}
  \usepackage{geometry}
  \usepackage{framed}
  \usepackage{xcolor}
  \definecolor{shadecolor}{gray}{0.9}
  \setlength{\FrameSep}{3pt}
  \usepackage{fancyvrb}
  \fvset{numbers=none,xleftmargin=5ex,fontsize=\small}

  \usepackage{pdfpages}

  \usepackage{multicol} 
  \usepackage{fancyhdr}
}

\providecommand{\addressUSTT}{
  Machine~Learning~\&~Robotics~lab, U~Stuttgart\\\small
  Universit{\"a}tsstra{\ss}e 38, 70569~Stuttgart, Germany
}

\providecommand{\addressTUB}{
  Learning~\&~Intelligent~Systems~Lab, TU~Berlin\\\small
  Marchstr. 23, 10587 Berlin, Germany
}


\renewcommand{\course}{Optimization Algorithms}
\renewcommand{\coursepicture}{optim}
\renewcommand{\coursedate}{Winter 2024/25}

\renewcommand{\topic}{Bayesian Optimization}
\renewcommand{\keywords}{}

\slides

\slidestitle

%%%%%%%%%%%%%%%%%%%%%%%%%%%%%%%%%%%%%%%%%%%%%%%%%%%%%%%%%%%%%%%%%%%%%%%%%%%%%%%%

\slide{References}{

\item \emph{Information-theoretic regret bounds for gaussian process
optimization in the bandit setting}
Srinivas, Krause, Kakade \& Seeger, Information Theory, 2012.

\item \emph{A taxonomy of global optimization
methods based on response surfaces} Jones, Journal of Global
Optimization, 2001.

\item \emph{Explicit local models: Towards optimal optimization
algorithms}, Poland, Technical Report No. IDSIA-09-04, 2004.

}

%%%%%%%%%%%%%%%%%%%%%%%%%%%%%%%%%%%%%%%%%%%%%%%%%%%%%%%%%%%%%%%%%%%%%%%%%%%%%%%%

\sublecture{Global Optimization}{

\item Let $x\in\RRR^n$, $f:~ \RRR^n \to \RRR$, ~ find
\begin{align*}
\min_x~ & f(x)
\end{align*}

\item Blackbox optimization: find a global optimium by sampling values
$y_t = f(x_t)$
\begin{items}
\item No access to $\na f$ or $\he f$
\item Observations may be noisy $y \sim \NN(y \| f(x_t), \s^2)$
\end{items}

~\pause

\item Global Optimization = infinite Bandits, with infinite decision space,
$x\in\RRR^n$
\begin{items}
\item Bandit problems are archetype for sequential decision making under uncertainty
\item Upper Confidence Bound (UCB) decisions have provably bounded regret!
\item Resolves exploration/exploitation ``dilemma''
\item Bayesian Optimization (GP-UCB) transfers bandits to continuous decisions $x\in\RRR^n$
\end{items}

}

%%%%%%%%%%%%%%%%%%%%%%%%%%%%%%%%%%%%%%%%%%%%%%%%%%%%%%%%%%%%%%%%%%%%%%%%%%%%%%%%

\key{Random restarts}
\slide{Random Restarts ~ (run downhill multiple times)}{

\item first the most basic approach...

~\pause

\item We assume to have a start distribution $q(x)$, and restart greedy search:

\begin{algo}
\Repeat
\State Sample $x\sim q(x)$
\State $x \gets \texttt{GreedySearch}(x)$ or
$\texttt{StochasticSearch}(x)$
\State \textbf{If} $f(x)<f(x^*)$ \textbf{then} $x^*\gets x$
\Until run out of budget
\end{algo}

\small

\item When gradients are available, replace greedy search by BFGS or Newton

~\pause

\item Can we not \emph{learn} more from all the evaluated points and found local optima?

}

%%%%%%%%%%%%%%%%%%%%%%%%%%%%%%%%%%%%%%%%%%%%%%%%%%%%%%%%%%%%%%%%%%%%%%%%%%%%%%%%

\key{Optimization and Learning}
\slide{Optimizing and Learning}{

\item Blackbox optimization is often related to learning:
\begin{items}
\item When we have local a gradient or Hessian, we can take that local
   information and run downhil -- no need to keep track of the history or
   learn (exception: BFGS, momentum)
\item In the Blackbox case we have no local information directly
   accessible $\to$ one needs to account of the history in some way or another to have an idea where to continue search
\end{items}
   
\item ``Accounting for the history'' often means learning or maintaining data:
\begin{items}
\item Learning a local or global model of $f$ itself, learning which steps
have been successful recently (gradient estimation), or which step
directions, or other heuristics
\item Maintaining data: populations, evolutionary algorithms, EDAs, etc.
\end{items}

}

%%%%%%%%%%%%%%%%%%%%%%%%%%%%%%%%%%%%%%%%%%%%%%%%%%%%%%%%%%%%%%%%%%%%%%%%%%%%%%%%

\slide{}{

\item Where we left when discussing No Free Lunch:

~

\emph{What are algorithms that literally start by making assumptions about $P(f)$ and then derive an
optimization algorithm for that $P(f)$?}

~

\item In Bayesian Optimization we maintain a particular belief $b_t = P(f\|D)$, namely a \emph{Gaussian Process}, and choose the next query based on that.

}

%%%%%%%%%%%%%%%%%%%%%%%%%%%%%%%%%%%%%%%%%%%%%%%%%%%%%%%%%%%%%%%%%%%%%%%%%%%%%%%%

\sublecture{Gaussian Processes}{

~\small

\item In my ML lectures, I introduce Gaussian Processes as Bayesian Kernel Ridge Regression

But here, the function space view of GPs relates more directly to NLF

{\tiny (see also Welling: ``Kernel Ridge Regression'' Lecture Notes;  Rasmussen \& Williams sections 2.1 \& 6.2; Bishop sections 3.3.3 \& 6)\\}

}

%%%%%%%%%%%%%%%%%%%%%%%%%%%%%%%%%%%%%%%%%%%%%%%%%%%%%%%%%%%%%%%%%%%%%%%%%%%%%%%%

\slide{Gaussian Process definition}{

\item The function space view: We have a prior $P(f)$ and data, then
$$P(f|\text{Data}) = \frac{P(\text{Data}|f)~ P(f)}{P(\text{Data})}$$

\item Gaussian Processes define a probability distribution over
functions:
\begin{items}
\item A function is an infinite dimensional thing -- how could we
define a Gaussian distribution over functions?
\item For every finite set $\{x_1,..,x_M\}$, the function values
$f(x_1),..,f(x_M)$ are Gaussian distributed with mean and covariance
\begin{align*}
\Exp{f(x_i)} &= \mu(x_i) \qquad\text{(often zero)} \\
\Exp{[f(x_i)-\m(x_i)][f(x_j)-\mu(x_j)]} &= k(x_i,x_j)
\end{align*}
where, $\mu(x)$ is called \textbf{mean function}, and $k(x,x')$ is called \defn{covariance function}
\item $\mu$ and $k$ generalize the notion of \emph{mean vector} $\mu_x$ and \emph{covariance matrix} $\S_{xx'}$ from finite $x\in\{1,..,n\}$ to continuous $x\in\RRR^n$
\end{items}

\item Second, Gaussian Processes define an observation probability
$$P(y|x,f) = \NN(y | f(x),\s_0^2)$$

}

%%%%%%%%%%%%%%%%%%%%%%%%%%%%%%%%%%%%%%%%%%%%%%%%%%%%%%%%%%%%%%%%%%%%%%%%%%%%%%%%

\slide{Gaussian Process posterior}{

\small

\item Given a Gaussian Process prior $GP(f|\mu,k)$ over $f$ and data
$D = \{ (x_i,y_i) \} _{i=1}^n$,
the posterior $P(f\|D)$ has new posterior mean and variance:
\begin{align*}
\Exp{f(x)\|D}
 &= \mu(x|D) = \k(x)^\T (K + \s_0^2 \Id)^\1 y\\
\Exp{[f(x)-\hat f(x)]^2\|D}
 &= \s^2(x|D) = k(x,x)
  - \k(x)^\T (K + \s_0^2 \Id_n)^\1 \k(x)
\end{align*}
{\tiny where $\k(x) = (k(x,x_1), \ldots, k(x,x_n))^\T\in\RRR^n$
contains covariances of $x$ to all data points; $K
= \left(k(x_i,x_j)\right)_{i,j=1}^{n,n}$ contains covariances between
all data points; and $y = (y_1,\ldots, y_n)^\T\in\RRR^n$
contains all data output values; the choice of kernel $k(\cdot,\cdot)$ and the observation sdv $\s_0$ are parameters

}

~

\item Side notes:
\begin{items}
\item Note: Don't forget that
$\text{Var}(y^*|x^*,D) = \s_0^2 + \text{Var}(f(x^*)|D)$

\item Gaussian Processes ~=~ Bayesian Kernel Ridge Regression

\item GP classification ~=~ Bayesian Kernel Logistic Regression

%% \item We can also handle discrete-valued functions $f$ using GP
%%   classification
\end{items}


}

%%%%%%%%%%%%%%%%%%%%%%%%%%%%%%%%%%%%%%%%%%%%%%%%%%%%%%%%%%%%%%%%%%%%%%%%%%%%%%%%

\slide{GP examples}{

\show[.6]{gaussianProcess1}
{\tiny\hfill(from Rasmussen \& Williams)}

}

%%%%%%%%%%%%%%%%%%%%%%%%%%%%%%%%%%%%%%%%%%%%%%%%%%%%%%%%%%%%%%%%%%%%%%%%%%%%%%%%

%% \slide{}{
%% \show[.6]{BayesianPredictiveDistribution}
%% {\tiny\hfill(from Bishop)}
%% }

%%%%%%%%%%%%%%%%%%%%%%%%%%%%%%%%%%%%%%%%%%%%%%%%%%%%%%%%%%%%%%%%%%%%%%%%%%%%%%%%

\slide{GP examples: different covariance functions}{

~

\show[.6]{gaussianProcess2}
{\tiny\hfill(from Rasmussen \& Williams)}

~

\item These are examples from the $\g$-exponential covariance function

$$k(x,x') = a~ \exp\{-|(x-x')/l|^\g\}$$

{\tiny with $a$ the prior variance of function values}

}

%%%%%%%%%%%%%%%%%%%%%%%%%%%%%%%%%%%%%%%%%%%%%%%%%%%%%%%%%%%%%%%%%%%%%%%%%%%%%%%%

\slide{GP examples: derivative observations}{

~

\show[.6]{gaussianProcess3}
{\tiny\hfill(from Rasmussen \& Williams)}

}

%%%%%%%%%%%%%%%%%%%%%%%%%%%%%%%%%%%%%%%%%%%%%%%%%%%%%%%%%%%%%%%%%%%%%%%%%%%%%%%%

\sublecture{Heuristics / Acquisition Functions}{
}

%%%%%%%%%%%%%%%%%%%%%%%%%%%%%%%%%%%%%%%%%%%%%%%%%%%%%%%%%%%%%%%%%%%%%%%%%%%%%%%%

\slide{Bayesian Optimization Algorithm}{

\begin{algo}
\Require GP prior given as $\mu(x)$ and $k(x,x')$, black-box function $f(x)$
\Ensure $x$
\State initialize empty data $D=\{\}$
\Repeat
\State find optimal query $x \gets \argmax_x \a(x|D)$  ~ (where $\a$ depends on $\mu(x|D), \s^2(x|D)$)
\State query $y \gets f(x)$
\State add to data $D \gets D \cup \{(x,y)\}$, update GP posterior $\mu(x|D), \s^2(x|D)$
\Until resources
\end{algo}

~

\item $\a(x;D)$ is called \textbf{acquisition function}
\begin{items}
\item $\a(x;D)$ characterizes how ``interesting'' it is to query $x$ next, given $D$
\item $\a(x;D)$ is a descriminative function for the next decision
\item $\a(x;D)$ analogous to a $Q$-function $Q(D,x)$ for the next decision $x$ in state $D$
\end{items}

}

%%%%%%%%%%%%%%%%%%%%%%%%%%%%%%%%%%%%%%%%%%%%%%%%%%%%%%%%%%%%%%%%%%%%%%%%%%%%%%%%

\slide{Acquisition Functions}{

\twocol[-.05]{.6}{.5}{

~

\item Maximize Probability of Improvement ~ (MPI)
$$\a(x;D) = \int_{-\infty}^{y^*} \NN(y|\mu_D(x),\s^2_D(x))$$

\item Maximize Expected Improvement ~ (EI)
$$\a(x;D) = \int_{-\infty}^{y^*} \NN(y|\mu_D(x),\s^2_D(x))~ (y^*-y)$$

\item Maximize UCB
$$\a(x;D) = \mu_D(x) + \b_t \s^2_D(x)$$


}{
\showh[1]{jones01}

\cen{\qquad\tiny(from Jones, 2001)}

}

~

~\tiny

(Often, $\b_t=1$ is chosen. UCB theory allows for
better choices. See Srinivas et al.\ citation.)

}

%%%%%%%%%%%%%%%%%%%%%%%%%%%%%%%%%%%%%%%%%%%%%%%%%%%%%%%%%%%%%%%%%%%%%%%%%%%%%%%%

\slide{Each step requires solving an optimization problem}{

\item Note: each $\argmax_x \a(x)$ on the previous slide is an optimization
problem!

\item As $\mu(x|D),\s^2(x|D)$ are given analytically, we have gradients and
Hessians. BUT: multi-modal problem!

\item In practice:
\begin{items}
\item Many restarts of gradient/2nd-order optimization runs
\item Restarts from a grid; from many random points
\end{items}

~\pause

\item We traded a \emph{blackbox} global optimization problem by solving an \emph{analytical} global optimization problem in each iteration:
\begin{items}
\item Assumes evaluating the real $f(x)$ is very expensive
\item The inner problem is analytical, can exploit gradients/Hessian, can run without real-world queries
\end{items}

}

%%%%%%%%%%%%%%%%%%%%%%%%%%%%%%%%%%%%%%%%%%%%%%%%%%%%%%%%%%%%%%%%%%%%%%%%%%%%%%%%

\key{GP-UCB}
\slide{GP-UCB}{

{\tiny From: \emph{Information-theoretic regret bounds for gaussian process
optimization in the bandit setting}
Srinivas, Krause, Kakade \& Seeger, Information Theory, 2012.

}

~

%\show[1]{GP-UCB1}
\show[.75]{GP-UCB2}
}

%%%%%%%%%%%%%%%%%%%%%%%%%%%%%%%%%%%%%%%%%%%%%%%%%%%%%%%%%%%%%%%%%%%%%%%%%%%%%%%%

\slide{}{
\show[.75]{GP-UCB3}

~

\show[.75]{GP-UCB4}
}

%%%%%%%%%%%%%%%%%%%%%%%%%%%%%%%%%%%%%%%%%%%%%%%%%%%%%%%%%%%%%%%%%%%%%%%%%%%%%%%%


\slide{Pitfall of using GPs as belief}{

\item A real issue, in my view, is the choice of kernel (i.e.\ prior $P(f)$)
\begin{items}
\item 'small' kernel: almost exhaustive search
\item 'wide' kernel: miss local optima
\item adapting/choosing kernel online (with CV): might fail
\item real $f$ might be non-stationary
\item non RBF kernels? Too strong prior, strange extrapolation
\end{items}

~

\item Assuming that we have the right prior $P(f)$ is really a strong assumption

}

%%%%%%%%%%%%%%%%%%%%%%%%%%%%%%%%%%%%%%%%%%%%%%%%%%%%%%%%%%%%%%%%%%%%%%%%%%%%%%%%

\slide{Further reading}{

\item Classically, such methods are known as \emph{Kriging}

~

\item \emph{Information-theoretic regret bounds for gaussian process
optimization in the bandit setting}
Srinivas, Krause, Kakade \& Seeger, Information Theory, 2012.

~

\item \emph{Efficient global optimization of expensive black-box functions.} Jones, Schonlau, \& Welch, Journal of Global Optimization, 1998.

\item \emph{A taxonomy of global optimization
methods based on response surfaces} Jones, Journal of Global
Optimization, 2001.

\item \emph{Explicit local models: Towards optimal optimization
algorithms}, Poland, Technical Report No. IDSIA-09-04, 2004.

%\show{GP-UCB}

}

%%%%%%%%%%%%%%%%%%%%%%%%%%%%%%%%%%%%%%%%%%%%%%%%%%%%%%%%%%%%%%%%%%%%%%%%%%%%%%%%

\slide{Further reading: Entropy Search}{

\item P. Hennig \& C. Schuler: \emph{Entropy Search for
Information-Efficient Global Optimization}, JMLR 13 (2012).

~

\item \textbf{Predictive Entropy Search}

\item Hern{\'a}ndez-Lobato, Hoffman \& Ghahraman: \emph{Predictive Entropy
Search for Efficient Global Optimization of Black-box Functions}, NIPS
2014.

\item Also for constraints!

\item Code: \url{https://github.com/HIPS/Spearmint/}

}

%%%%%%%%%%%%%%%%%%%%%%%%%%%%%%%%%%%%%%%%%%%%%%%%%%%%%%%%%%%%%%%%%%%%%%%%%%%%%%%%

\slide{Note: beyond Gaussian Processes}{

~

\item Use emsembles (e.g.\ bootstrap ensembles) of models and their discrepancy to decide on information gain, rather than variance!
\begin{items}
\item Can be realized also with more complicated function models (NNs)
\item covariance function is implicit and more structured
\end{items}

}

%%%%%%%%%%%%%%%%%%%%%%%%%%%%%%%%%%%%%%%%%%%%%%%%%%%%%%%%%%%%%%%%%%%%%%%%%%%%%%%%

\sublecture{Appendix}{

Other basic approaches...

}

%%%%%%%%%%%%%%%%%%%%%%%%%%%%%%%%%%%%%%%%%%%%%%%%%%%%%%%%%%%%%%%%%%%%%%%%%%%%%%%%

\key{Iterated local search}
\slide{Iterated Local Search}{

\small

\item Iterated Local Search (in discrete spaces) restarts in a \textbf{meta-neighborhood} $\NN^*(x)$ of the last visited local minimum $x$

\item Iterated Local Search (Variant 1):\\
\begin{algo}
\Require initial $x$, function $f(x)$
\Repeat
\State $x \gets \argmin_{y'\in\{\texttt{GreedySearch}(y)\,:\,y\in\NN^*(x)\}} f(y')$
\Until $x$ converges
\end{algo}
\begin{items}
\item This evalutes a GreedySearch for all meta-neighbors
$y\in\NN^*(x)$ of the last local optimum $x$
\item The inner GreedySearch uses another neighborhood function
$\NN(x)$
\end{items}

\item Variant 2: $x \gets $ the ``first'' $y\in\NN^*(x)$ such that $f(\texttt{GS}(y)) < f(x)$

\item In \textbf{continuous space}: $\NN(x)$ and $\NN^*(x)$ are replaced by
transition proposals $q(y|x)$ and $q^*(y|x)$

}

%%%%%%%%%%%%%%%%%%%%%%%%%%%%%%%%%%%%%%%%%%%%%%%%%%%%%%%%%%%%%%%%%%%%%%%%%%%%%%%%

\slide{Iterated Local Search}{\label{lastpage}

\item Application to Travelling Salesman Problem:

$k$-opt neighbourhood: solutions which differ by at most k edges

~

\show[.6]{TSP-neighborhoods}
{\tiny\hfill from Hoos \& St{\"u}tzle: \emph{Tutorial: Stochastic
Search Algorithms}}

~

\item GreedySearch uses 2-opt or 3-opt neighborhood

Iterated Local Search uses 4-opt meta-neighborhood (double bridges)

}

%%%%%%%%%%%%%%%%%%%%%%%%%%%%%%%%%%%%%%%%%%%%%%%%%%%%%%%%%%%%%%%%%%%%%%%%%%%%%%%%

%% \key{Variable neighborhood search}
%% \slide{Very briefly...}{

%% \item Variable Neighborhood Search:
%% \begin{items}
%% \item Switch the neighborhood function in different phases
%% \item Similar to Iterated Local Search
%% \end{items}

%% ~

%% \item Tabu Search:
%% \begin{items}
%% \item Maintain a \emph{tabu list} points (or points features) which
%% may not be visited again
%% \item The list has a fixed finite size: FILO
%% \item Intensification and diversification heuristics make it more global
%% \end{items}

%% }

%%%%%%%%%%%%%%%%%%%%%%%%%%%%%%%%%%%%%%%%%%%%%%%%%%%%%%%%%%%%%%%%%%%%%%%%%%%%%%%%

\slidesfoot
