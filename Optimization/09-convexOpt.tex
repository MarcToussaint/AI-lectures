\providecommand{\slides}{
  \newcommand{\slideshead}{
  \newcommand{\thepage}{\arabic{mypage}}
  %beamer
%  \documentclass[t,hyperref={bookmarks=true}]{beamer}
%  \geometry{papersize={171mm,96mm}}
  \documentclass[t,hyperref={bookmarks=true},aspectratio=169]{beamer}
  \setbeamersize{text margin left=5mm}
  \setbeamersize{text margin right=5mm}
  \usetheme{default}
  \usefonttheme[onlymath]{serif}
  \setbeamertemplate{navigation symbols}{}
  \setbeamertemplate{itemize items}{{\color{black}$\bullet$}}

  \newwrite\keyfile

  %\usepackage{palatino}
  \stdpackages
  %\usepackage{tikz} \usetikzlibrary {shapes.geometric} 
  \usepackage{multimedia}
  \usepackage[utf8]{inputenc}

  %%% geometry/spacing issues
  %
  \definecolor{bluecol}{rgb}{0,0,.5}
  \definecolor{greencol}{rgb}{0,.6,0}
  \definecolor{citcol}{rgb}{.4,.4,.4}
  %\renewcommand{\baselinestretch}{1.1}
  \renewcommand{\arraystretch}{1.2}
  \columnsep 0mm

  \columnseprule 0pt
  \parindent 0ex
  \parskip 0ex
  %\setlength{\itemparsep}{3ex}
  %\renewcommand{\labelitemi}{\rule[3pt]{10pt}{10pt}~}
  %\renewcommand{\labelenumi}{\textbf{(\arabic{enumi})}}
  \setbeamertemplate{enumerate item}{(\roman{enumi})}
  \newcommand{\headerfont}{\helvetica{14}{1.5}{b}{n}}
  \newcommand{\slidefont} {\helvetica{11}{1.4}{m}{n}}
  %\newcommand{\codefont} {\helvetica{8}{1.2}{m}{n}}
  \newcommand{\urlfont} {\fontsize{6}{1.1}\selectfont}
  \renewcommand{\small} {\helvetica{10}{1.4}{m}{n}}
  \renewcommand{\tiny} {\helvetica{8}{1.3}{m}{n}}
  \newcommand{\ttiny} {\helvetica{7}{1.3}{m}{n}}

  %%% count pages properly and put the page number in bottom right
  %
  \newcounter{mypage}
  \newcommand{\incpage}{\addtocounter{mypage}{1}\setcounter{page}{\arabic{mypage}}}
  \setcounter{mypage}{0}
  \resetcounteronoverlays{page}

  \pagestyle{fancy}
  %\setlength{\headsep}{10mm}
  %\addtolength{\footheight}{15mm}
  \renewcommand{\headrulewidth}{0pt} %1pt}
  \renewcommand{\footrulewidth}{0pt} %.5pt}
  \cfoot{}
  \rhead{}
  \lhead{}
  %% \lfoot{\vspace*{-3mm}\hspace*{-3mm}\helvetica{5}{1.3}{m}{n}{\texttt{github.com/MarcToussaint/AI-lectures}}}
\lfoot{}
%  \rfoot{{\tiny\textsf{AI -- \topic -- \subtopic -- \arabic{mypage}/\pageref{lastpage}}}}
  %\lfoot{\raisebox{5mm}{\tiny\textsf{\slideauthor}}}
  %\rfoot{\raisebox{5mm}{\tiny\textsf{\slidevenue{} -- \arabic{mypage}/\pageref{lastpage}}}}
  %\rfoot{~\anchor{30,12}{\tiny\textsf{\thepage/\pageref{lastpage}}}}
  %\lfoot{\small\textsf{Marc Toussaint}}
%  \rfoot{\vspace*{-4.5mm}{\tiny\textsf{\color{gray}\topic\ -- \subtopic\ -- \arabic{mypage}/\pageref{lastpage}}}\hspace*{-4mm}}
  \rfoot{\vspace*{-4.5mm}{\tiny\textsf{\color{gray}\topic\ -- \arabic{mypage}/\pageref{lastpage}}}\hspace*{-4mm}}
%  \rfoot{~\anchor{-10,12}{\tiny\textsf{\color{gray}\topic\ --  \arabic{mypage}/\pageref{lastpage}}}}
  \lfoot{\vspace*{-4.5mm}{\hspace*{-3mm}\includegraphics[height=4mm]{LIS-logo-longText}}}

  \definecolor{grey}{rgb}{.8,.8,.8}
  \definecolor{head}{rgb}{.85,.9,.9}
%  \definecolor{blue}{rgb}{.0,.0,.5}
%  \definecolor{green}{rgb}{.0,.5,.0}
  \definecolor{red}{rgb}{.8,.0,.0}
  \newcommand{\inverted}{
    \definecolor{main}{rgb}{1,1,1}
    \color{main}
    \pagecolor[rgb]{.3,.3,.3}
  }
  \input{../latex/macros}

  \graphicspath{{pics/}{../pics/}{pics-local/}}
}

\newcommand{\slidestitle}{
  \title{\course \topic}
  \author{Marc Toussaint}
  \institute{Learning \& Intelligent Systems Lab, TU Berlin}

  \begin{document}


  %% title slide!
  \slide{}{
    \thispagestyle{empty}

    \twocol{.35}{.55}{
      %\vspace*{-5mm}%
      \hspace*{-5mm}%
      \includegraphics[width=1.\columnwidth]{\coursepicture}
    }{\center

      \textbf{\fontsize{17}{20}\selectfont \course}

      ~

      %Lecture
      \topic\\

      \vspace{1cm}

      {\tiny~\emph{\keywords}~\\}

      \vspace{1cm}

      \teacher
      
      Technical University of Berlin

      \coursedate

      ~

    }
  }
}

\newcommand{\slide}[2]{
  \slidefont
  \incpage\begin{frame}
  \addcontentsline{toc}{section}{#1}
  \vfill
  {\headerfont #1} \vspace*{-2ex}
  \begin{itemize}\item[]~\\
    #2
  \end{itemize}
  \vfill
  \end{frame}
}

% use \begin{frame}[fragile] around slidecore!
\newenvironment{slidecore}[1]{
  \slidefont\incpage
  \addcontentsline{toc}{section}{#1}
  \vfill
  {\headerfont #1} \vspace*{-2ex}
  \begin{itemize}\item[]~\\
}{
  \end{itemize}
  \vfill
}

\newcommand{\titleslide}[4][Marc Toussaint]{
  \newcommand{\slideauthor}{#1}
  \newcommand{\slidevenue}{#3}
  \slidefont
  \incpage
  \begin{frame}
  \begin{center}
    \vspace*{15mm}

    {\headerfont #2\\}
        
    \vspace*{10mm}

    #1 \\

    \vspace*{5mm}

    {\small
      Learning \& Intelligent Systems Lab, TU Berlin\\
%      Science of Intelligence Cluster of Excellence, Berlin\\
      Max Planck Fellow, Institute for Intelligent Systems\\ %Physical Reasoning \& Manipulation Lab -- 
%      MIT CSAIL\\
%      mtoussai@mit.edu,~ marc.toussaint@informatik.uni-stuttgart.de

      \vspace*{10mm}

      \emph{#3}
    }

    \vspace*{0mm}

    %\includegraphics[scale=.1]{pics/eushield-fullcolour}

  \end{center}
  \begin{itemize}\item[]~\\
    #4
  \end{itemize}
  \end{frame}
}

\newcommand{\titleslideempty}[2]{
  \slidefont
  \incpage
  \begin{frame}
  \begin{center}
    \vspace*{15mm}

    {\headerfont #1\\}
        
    %% \vspace*{5mm}

    %% {\small\emph{#2}} \\

  \end{center}
  \begin{itemize}\item[]~\\
    #2
  \end{itemize}
  \end{frame}
}

\providecommand{\key}[1]{
  \addtocounter{mypage}{1}
% \immediate\write\keyfile{#1}
  \addtocontents{toc}{\hyperref[key:#1]{#1 (\arabic{mypage})}}
%  \phantomsection\label{key:#1}
%  \index{#1@{\hyperref[key:#1]{#1 (\arabic{mysec}:\arabic{mypage})}}|phantom}
  \addtocounter{mypage}{-1}
}

\providecommand{\course}{}

\providecommand{\subtopic}{}

\providecommand{\sublecture}[2]{
  \renewcommand{\subtopic}{#1}
  \slide{#1}{#2}
}

\providecommand{\sublectureHide}[2]{
  \renewcommand{\subtopic}{#1}
}

\providecommand{\story}[1]{
~

Motivation: {\tiny #1}\clearpage
}

\newenvironment{items}[1][9]{
\par\setlength{\unitlength}{1pt}\fontsize{#1}{#1}\linespread{1.2}\selectfont
\begin{list}{--}{\leftmargin4ex \rightmargin0ex \labelsep1ex \labelwidth2ex
\topsep.7ex \parsep0ex \itemsep3pt}
}{
\end{list}
}

\newenvironment{itemS}[1][4ex]{
\par
\tiny
\begin{list}{--}{\leftmargin#1 \rightmargin0ex \labelsep1ex
  \labelwidth2ex \topsep0pt \parsep0ex \itemsep2pt}
}{
\end{list}
}

\providecommand{\slidesfoot}{
  \end{document}
}


  \slideshead
  %\slidestitle
}

\providecommand{\exercises}{
  \input{../latex/style-exercises}
  \exerciseshead
}

\providecommand{\script}{
  \newcommand{\scripthead}{
  \documentclass[9pt,fleqn,twoside]{article}
  \stdpackages

  \usepackage{makeidx}
  \makeindex

  \usepackage{thmtools}
  \definecolor{shadecolor}{gray}{0.85}
  \declaretheoremstyle[
    %headfont=\normalfont\bfseries,
    %notefont=\mdseries, notebraces={(}{)},
    %bodyfont=\normalfont,
    %postheadspace=0.5em,
    %spaceabove=6pt,
    mdframed={
      %  skipabove=8pt,
      %  skipbelow=6pt,
      hidealllines=true,
      backgroundcolor={shadecolor},
      innertopmargin=8pt,
      %  innerleftmargin=3pt,
      %  innerrightmargin=3pt
    }
  ]{shaded}
  \declaretheorem[style=shaded,within=section,name=Definition]{myDefinition}
  \declaretheorem[style=shaded,within=section,name=Theorem]{myTheorem}
  \declaretheorem[style=shaded,within=section,name=Identities]{Identities}
  \declaretheorem[style=shaded,within=section,name=Example]{myExample}

  \definecolor{grey}{rgb}{.8,.8,.8}
  \definecolor{bluecol}{rgb}{0,0,.5}
  \definecolor{greencol}{rgb}{0,.4,0}
  \definecolor{shadecolor}{gray}{0.9}
  \definecolor{citcol}{rgb}{.4,.4,.4}
  \usepackage[
    %    pdftex%,
    %%    letterpaper,
    %bookmarks,
    bookmarksnumbered,
    colorlinks,
    urlcolor=bluecol,
    citecolor=black,
    linkcolor=bluecol,
    %    pagecolor=bluecol,
    pdfborder={0 0 0},
    %pdfborderstyle={/S/U/W 1},
    %%    backref,     %link from bibliography back to sections
    %%    pagebackref, %link from bibliography back to pages
    %%    pdfstartview=FitH, %fitwidth instead of fit window
    pdfpagemode=UseOutlines, %bookmarks are displayed by acrobat
    pdftitle={\course},
    pdfauthor={Marc Toussaint},
    pdfkeywords={}
  ]{hyperref}
  \DeclareGraphicsExtensions{.pdf,.png,.jpg,.eps}

  %\usepackage{multirow}
  \usepackage{multimedia}
  %\usepackage{marginnote}
  %\setbeamercolor{background canvas}{bg=}

  \usepackage[round]{natbib}
  \bibliographystyle{abbrvnat}

  \renewcommand{\r}{\varrho}
  \renewcommand{\l}{\lambda}
  \renewcommand{\L}{\Lambda}
  \renewcommand{\b}{\beta}
  \renewcommand{\d}{\delta}
  \renewcommand{\k}{\kappa}
  \renewcommand{\t}{\theta}
  \renewcommand{\O}{\Omega}
  \renewcommand{\o}{\omega}
  \renewcommand{\SS}{{\cal S}}
  \renewcommand{\=}{\!=\!}
  %\renewcommand{\boldsymbol}{}
  %\renewcommand{\Chapter}{\chapter}
  %\renewcommand{\Subsection}{\subsection}

  \renewcommand{\baselinestretch}{1.0}
  \geometry{a5paper,headsep=6mm,hdivide={10mm,*,10mm},vdivide={13mm,*,7mm}}

  \fancyhead[OL,ER]{\course, \textit{Marc Toussaint}}
  \fancyhead[OR,EL]{\thepage}
  \fancyhead[C]{}
  \fancyfoot{}
  \pagestyle{fancy}

  \renewcommand{\labelenumi}{{(\roman{enumi})}}
  \renewcommand{\theenumi}{(\roman{enumi})} %for ref
  \parindent 0pt
  \parskip .5pc

  \columnsep 6ex

  \renewcommand{\familydefault}{\sfdefault}

  \newcommand{\headerfont}{\large}%helvetica{12}{1}{b}{n}}
  \newcommand{\slidefont} {}%\helvetica{9}{1.3}{m}{n}}
  \newcommand{\storyfont} {}
  %  \renewcommand{\small}   {\helvetica{8}{1.2}{m}{n}}
  \renewcommand{\tiny}    {\footnotesize}%helvetica{7}{1.1}{m}{n}}
  \newcommand{\ttiny} {\footnotesize}%fontsize{7}{7}\selectfont}
%  \newcommand{\codefont}{\fontsize{6}{6}\selectfont}%helvetica{8}{1.2}{m}{n}}
  \newcommand{\codefont} {\helvetica{8}{1.2}{m}{n}}

  \input{../latex/macros}

  \usepackage{comment}
  \specialcomment{solution}{
    \small
    \begin{shaded}
  }{
    \end{shaded}
  }

  \graphicspath{{pics/}{../pics/}{pics-local/}}

  \mytitle{\course\\Lecture Script}
  \myauthor{Marc Toussaint}
  \date{\coursedate}
}

%%%%%%%%%%%%%%%%%%%%%%%%%%%%%%%%%%%%%%%%%%%%%%%%%%%%%%%%%%%%%%%%%%%%%%%%%%%%%%%%

\newcommand{\scripttitle}{
  \begin{document}
  \maketitle
  %\anchor{100,10}{\includegraphics[width=4cm]{optim}}
%  \vspace*{1cm}
}

%%%%%%%%%%%%%%%%%%%%%%%%%%%%%%%%%%%%%%%%%%%%%%%%%%%%%%%%%%%%%%%%%%%%%%%%%%%%%%%%

\newcounter{mypage}
\setcounter{mypage}{0}
\newcommand{\incpage}{\addtocounter{mypage}{1}}

\newcommand{\subtopic}{}
\newcommand{\pause}{}
\newcommand{\only}[1]{#1}

\renewcommand{\slides}[1][]{
  %  \clearpage
  \subsection{\topic}
  \index{\topic}
  {\small #1}
  \setcounter{mypage}{0}
  \smallskip\nopagebreak\hrule\medskip
}

\newcommand{\slidesfoot}{
  \bigskip
}

\newcommand{\sublecture}[2]{
  \phantomsection\addcontentsline{toc}{subsubsection}{#1}
  \index{#1}
}

\newcommand{\sublectureHide}[2]{
  \renewcommand{\subtopic}{#1}
}

\newcommand{\key}[1]{
  \phantomsection\addcontentsline{toc}{subsubsection}{#1}
  %\subsubsection{#1}
  \index{#1}
}

\providecommand{\defn}[1]{%
  \textbf{#1}\index{#1}%
}

\newenvironment{slidecore}[1]{
  \incpage
  \subsubsection*{#1}%{\headerfont\noindent\textbf{#1}\\}%
  \vspace{-6ex}%
  \begin{list}{$\bullet$}{\leftmargin4ex \rightmargin0ex \labelsep1ex
    \labelwidth2ex \partopsep0ex \topsep0ex \parsep.5ex \parskip0ex \itemsep0pt}\item[]~\\\nopagebreak%
}{
  \end{list}\nopagebreak%
  {\hfill\tiny \textsf{\arabic{section}.\arabic{subsection}:\arabic{mypage}}}\nopagebreak%
  \smallskip\nopagebreak\hrule
}

\newcommand{\slide}[2]{
  \begin{slidecore}{#1}
    #2
  \end{slidecore}
}

\renewcommand{\exercises}{}
\newcommand{\exercisestitle}{}
\newcommand{\exsection}[1]{\subsubsection{#1}}
\newcommand{\exsubsection}[1]{\paragraph{#1}}
\newcommand{\exerfoot}{\bigskip}

\newcommand{\story}[1]{
  \subsection*{Motivation \& Outline}
  {\storyfont\sf #1}
  \medskip\nopagebreak\hrule
}

\newcounter{savedsection}
\newcommand{\subappendix}{\setcounter{savedsection}{\arabic{section}}\appendix}
\newcommand{\noappendix}{
  \setcounter{section}{\arabic{savedsection}}% restore section number
  \setcounter{subsection}{0}% reset section counter
%  \gdef\@chapapp{\sectionname}% reset section name
  \renewcommand{\thesection}{\arabic{section}}% make section numbers arabic
}

\newenvironment{items}[1][9]{
\par\setlength{\unitlength}{1pt}\fontsize{#1}{#1}\linespread{1.2}\selectfont
\begin{list}{--}{\leftmargin4ex \rightmargin0ex \labelsep1ex \labelwidth2ex
\topsep0pt \parsep0ex \itemsep3pt}
}{
\end{list}
}

\newenvironment{itemS}[1][4ex]{
\par
\tiny
\begin{list}{--}{\leftmargin#1 \rightmargin0ex \labelsep1ex
  \labelwidth2ex \topsep0pt \parsep0ex \itemsep2pt}
}{
\end{list}
}

\newcommand{\Def}[1]{%
\textbf{#1}\index{#1}}%\marginnote{#1}}

  \scripthead
}

\providecommand{\paper}{
  \input{../latex/style-paper}
  \paperhead
}

\providecommand{\note}[1][9pt]{
  \providecommand{\notehead}[2]{
  \documentclass[#1,fleqn,twoside]{article}
  \stdpackages
  \renewcommand{\labelenumi}{{(\roman{enumi})}}
  \renewcommand{\theenumi}{(\roman{enumi})} %for ref

  \renewcommand{\baselinestretch}{#2}
  \renewcommand{\arraystretch}{1.2}
  \renewcommand{\topfraction}{1}
  \renewcommand{\bottomfraction}{1}
  \renewcommand{\textfraction}{0}
  \columnsep 5ex
  \parindent 3ex
  \parskip 1ex

  % Lists and paragraphs
  \parindent 0pt
  \topsep 4pt plus 1pt minus 2pt
  \partopsep 1pt plus 0.5pt minus 0.5pt
  \itemsep 2pt plus 1pt minus 0.5pt
  \parsep 2pt plus 1pt minus 0.5pt
  \parskip .5pc %add _in_ {thebibliography} environment in *.bbl

  \setcounter{tocdepth}{3}
  \setcounter{secnumdepth}{3}

  \geometry{a4paper,hdivide={25mm,*,25mm},vdivide={25mm,*,25mm}}

  \renewcommand{\headrulewidth}{.0pt}\renewcommand{\footrulewidth}{.0pt}\cfoot{}
  \fancyhead[OL,EC]{\it\theauthor---\today}
  \fancyhead[ER]{\leftmark}
  \fancyhead[OR,EL]{\thepage}
  \fancyfoot[EL,OR]{}
  \setlength{\headsep}{10mm}
  %\fancyhead[OL]{\rightmark}
  %\fancyfoot[EL,OR]{}


  %  \usepackage{palatino}

  \newcommand{\codefont}{\helvetica{8}{1.2}{m}{n}}

  \renewenvironment{abstract}{
    \vspace*{5ex}\begin{list}{}{
      \leftmargin3ex
      \rightmargin3ex
      \topsep-\parskip}\item[]
     \hrule\vspace{1.5ex}{\bf Abstract.~}\small}
    {\vspace{2ex}\hrule\end{list}\vspace{5ex}}
    
  \newenvironment{keyword}
    {\par{\it Keywords:~}}
    {}

  \def\makemytitle{%
    \thispagestyle{empty}
    \begin{list}{}{\leftmargin3ex \rightmargin3ex \topsep0ex \parsep0ex}\item[]
      \begin{center}
        {\fontsize{18}{25}\selectfont{\thetitle\\}}\vspace{5ex}

        {\fontsize{14}{16}\selectfont{\theauthor\\}}\vspace{1ex}

        {\footnotesize{\sl \addressFUB}\\ \emailBerlin}

        {\footnotesize \today}

        \vspace{1ex}
        {\small \published}
      \end{center}
    \end{list}
    \renewcommand{\maketitle}{\chapter{\thetitle}}
  }

  \input{../latex/macros}
  \pdflatex

  \graphicspath{{pics/}{../pics/}{pics-local/}}

  \myauthor{Marc Toussaint}
  \date{\today}
}

%%%%%%%%%%%%%%%%%%%%%%%%%%%%%%%%%%%%%%%%%%%%%%%%%%%%%%%%%%%%%%%%%%%%%%%%%%%%%%%%

\newcommand{\notetitle}{
  \begin{document}
  \thispagestyle{empty}
    
  \maketitle

}

\newenvironment{items}[1][9]{
\par\setlength{\unitlength}{1pt}\fontsize{#1}{#1}\linespread{1.2}\selectfont
\begin{list}{--}{\leftmargin4ex \rightmargin0ex \labelsep1ex \labelwidth2ex
\topsep0pt \parsep0ex \itemsep3pt}
}{
\end{list}
}

  \notehead{#1}{1.1}
}

\providecommand{\course}{NO COURSE}
\providecommand{\coursepicture}{NO PICTURE}
\providecommand{\coursedate}{NO DATE}
\providecommand{\topic}{NO TOPIC}
\providecommand{\keywords}{}
\providecommand{\exnum}{NO NUMBER}
\providecommand{\teacher}{Marc Toussaint}

\providecommand{\stdpackages}{
  \usepackage{amsmath}
  \usepackage{amssymb}
  \usepackage{amsfonts}
  \allowdisplaybreaks
  \usepackage{amsthm}
  \usepackage{eucal}
  \usepackage{graphicx}
%  \usepackage{color}
  \usepackage{geometry}
  \usepackage{framed}
  \usepackage{xcolor}
  \definecolor{shadecolor}{gray}{0.9}
  \setlength{\FrameSep}{3pt}
  \usepackage{fancyvrb}
  \fvset{numbers=none,xleftmargin=5ex,fontsize=\small}

  \usepackage{pdfpages}

  \usepackage{multicol} 
  \usepackage{fancyhdr}
}

\providecommand{\addressUSTT}{
  Machine~Learning~\&~Robotics~lab, U~Stuttgart\\\small
  Universit{\"a}tsstra{\ss}e 38, 70569~Stuttgart, Germany
}

\providecommand{\addressTUB}{
  Learning~\&~Intelligent~Systems~Lab, TU~Berlin\\\small
  Marchstr. 23, 10587 Berlin, Germany
}


\renewcommand{\course}{Optimization Algorithms}
\renewcommand{\coursepicture}{optim}
\renewcommand{\coursedate}{Winter 2024/25}

\renewcommand{\topic}{Convex Optimization}
\renewcommand{\keywords}{Convex, quasiconvex, unimodal, convex
optimization problem, linear program (LP), standard form, simplex
algorithm, LP-relaxation of integer linear programs, quadratic programming
(QP), sequential quadratic programming}

\slides

\slidestitle

%%%%%%%%%%%%%%%%%%%%%%%%%%%%%%%%%%%%%%%%%%%%%%%%%%%%%%%%%%%%%%%%%%%%%%%%%%%%%%%%

\key{Function types: covex, quasi-convex, uni-modal}
\slide{Function types}{

\item A set $X\subseteq V$ is defined \defn{convex} iff
\begin{align*}
\forall x,y\in X, a\in[0,1]:~ ax + (1\!-\!a)y \in X
\end{align*}

\item A function is defined \defn{convex} iff
$$\forall x,y\in\RRR^n,a\in[0,1]:~ f(ax + (1\!-\!a)y) \le a~ f(x) + (1\!-\!a)~ f(y)$$

\item A function is \defn{quasiconvex} iff
$$\forall x,y\in\RRR^n,a\in[0,1]:~ f(ax + (1\!-\!a)y) \le \max\{f(x), f(y)\}$$

{\tiny ..alternatively, iff every sublevel set $\{ x | f(x)\le \a\}$
is convex.}

\item We call a function \defn{unimodal} iff it has
only 1 local minimum, which is the global one

}

%%%%%%%%%%%%%%%%%%%%%%%%%%%%%%%%%%%%%%%%%%%%%%%%%%%%%%%%%%%%%%%%%%%%%%%%%%%%%%%%

\slide{}{

\cen{convex ~$\subset$~ quasiconvex ~$\subset$~ unimodal ~$\subset$~ general}

}

%%%%%%%%%%%%%%%%%%%%%%%%%%%%%%%%%%%%%%%%%%%%%%%%%%%%%%%%%%%%%%%%%%%%%%%%%%%%%%%%

%% \slide{Local optimization}{

%% \item So far I avoided making explicit assumptions about problem
%% convexity: To emphasize that all methods we considered -- except for
%% Newton -- are applicable also on non-convex problems.

%% ~

%% \item The methods we considered are \textbf{local} optimization
%% methods, which can be defined as

%% -- a method that adapts the solution locally

%% -- a method that is guaranteed to converge to a local minimum only

%% ~

%% \item Local methods are efficient

%% -- if the problem is (strictly) unimodal ~ (strictly: no plateaux)

%% -- if time is critical and a local optimum is a sufficiently good
%%    solution

%% -- if the algorithm is restarted very often to hit multiple local
%%    optima

%% }

%%%%%%%%%%%%%%%%%%%%%%%%%%%%%%%%%%%%%%%%%%%%%%%%%%%%%%%%%%%%%%%%%%%%%%%%%%%%%%%%

\slide{Properties}{

\item The sum of two confex functions $f_1(x) + f_2(x)$ is also convex

\item A function $f\in\CC^2$ convex $\iff$ $\he f(x)$ pos.-semidef. everywhere

\item $f$ convex $\To$ sublevel sets $\{ x : f ( x ) \le a\}$ are convex

\pause

\item $l(\l)=\min_x L(x,\l)$ is concave! ~ Point-wise minimization:
\begin{items}
\item For each $x$, $L(x,\l)$ is linear in $\l$
\item Think of $L(x,\l)$ as a family of many linear functions
\item At each $\l$, pick the function with lowest value $\to$ concave
\item (Epigraph: The ``region'' $\{ (x,y) : y\le f(x) \}$ below a function; point-wise minimization $\oto$ intersection of epigraphs.)
\end{items}

}

%%%%%%%%%%%%%%%%%%%%%%%%%%%%%%%%%%%%%%%%%%%%%%%%%%%%%%%%%%%%%%%%%%%%%%%%%%%%%%%%

\slide{Convex Mathematical Program (CP)}{


\item \emph{Variant 1:} A mathematical program $\min_x~ f(x) \st g(x)\le 0,~ h(x) = 0$ is convex iff $f$ is convex and the feasible set is convex.

~

\emph{Variant 2:} A mathematical program $\min_x~ f(x) \st g(x)\le 0,~ h(x) =
0$ is convex iff $f$ and every $g_i$ are convex and $h$ is linear.

~

\begin{items}
\item Variant 2 is the stronger and the default definition
\item In variant 1, only $\{x:h(x)=0\}$ needs to be \emph{linear}, and $\{x:g(x)\le 0\}$ needs to be convex
\end{items}

}

%%%%%%%%%%%%%%%%%%%%%%%%%%%%%%%%%%%%%%%%%%%%%%%%%%%%%%%%%%%%%%%%%%%%%%%%%%%%%%%%

\key{Linear program (LP)}
\key{Quadratic program (QP)}
\slide{Linear and Quadratic Programs}{

\item \defn{Linear Program} (LP)
$$\min_x~ c^\T x \st G x \le h,~ Ax=b$$

LP in standard form
$$\min_x~ c^\T x \st x \ge 0,~ Ax=b$$

\item \defn{Quadratic Program} (QP)
$$\min_x~ \half x^\T Q x + c^\T x   \st   G x \le h,~ Ax=b$$
where $Q$ is positive definite.

~

\small
(This is different to a Quadratically Constraint Quadratic Programs (QCQP))

}

%%%%%%%%%%%%%%%%%%%%%%%%%%%%%%%%%%%%%%%%%%%%%%%%%%%%%%%%%%%%%%%%%%%%%%%%%%%%%%%%

\key{LP in standard form}
\slide{Transforming an LP problem into standard form}{

\item LP problem:
$$\min_x~ c^\T x \st G x \le h,~ Ax=b$$

\item Introduce \defn{slack variable}s:
$$\min_{x,\xi}~ c^\T x \st G x + \xi = h,~ Ax=b,~ \xi\ge 0$$

\item Express $x=x^+ - x^-$ with $x^+,x^-\ge 0$:
{\small
\begin{align*}
\min_{x^+,x^-,\xi}~ &c^\T (x^+-x^-)\\
& \st G (x^+-x^-) + \xi = h,~ A(x^+-x^-)=b,~ \xi\ge 0,~ x^+\ge 0,~
 x^-\ge 0
\end{align*}}
where $(x^+,x^-,\xi)\in\RRR^{2n+m}$

~

\item Now this is conform with the standard form
{\tiny with
$\tilde x = (x^+,x^-,\xi)$,
$\tilde A = \mat{ccc}{G & -G & \Id \\ A & -A & 0}$,
$\tilde b = (h, b)$ }
$$\min_{\tilde x}~ c^\T \tilde x \st \tilde x \ge 0,~ \tilde A \tilde x=\tilde b$$

}

%%%%%%%%%%%%%%%%%%%%%%%%%%%%%%%%%%%%%%%%%%%%%%%%%%%%%%%%%%%%%%%%%%%%%%%%%%%%%%%%

\slide{}{

\item A \defn{slack variable} is a variable that is added to an inequality constraint to transform it into an equality. Introducing a slack variable replaces an inequality constraint with an equality constraint and a non-negativity constraint on the slack variable (wikipedia)

}

%%%%%%%%%%%%%%%%%%%%%%%%%%%%%%%%%%%%%%%%%%%%%%%%%%%%%%%%%%%%%%%%%%%%%%%%%%%%%%%%

\slide{Example LPs}{

~

See the exercises 4.8-4.20 of Boyd \& Vandenberghe!

}

%%%%%%%%%%%%%%%%%%%%%%%%%%%%%%%%%%%%%%%%%%%%%%%%%%%%%%%%%%%%%%%%%%%%%%%%%%%%%%%%

\slide{Example QP}{

\small

\item Support Vector Machines. Primal problem:
\begin{align*}
\min_{\b,\xi}
&~ \norm{\b}^2 + C \sum_{i=1}^n \xi_i \st y_i (x_i^\T \b) \ge
1-\xi_i\comma \xi_i\ge 0
\end{align*}
Dual problem:
\begin{align*}
l(\a,\m)
&= \min_{\b,\xi} L(\b,\xi,\a,\m)
 = -{\textstyle\frac{1}{4}} \sum_{i=1}^n \sum_{i'=1}^n \a_i \a_{i'}
 y_i y_{i'} \hat x_i^\T \hat x_{i'} + \sum_{i=1}^n \a_i \\
\max_{\a,\m}
&~ l(\a,\m) \st 0 \le \a_i \le C
\end{align*}
(See ML lecture 5:13 for a derivation.)

\cen{
\showh[.2]{svm_trenngeraden}
\hspace{1cm}
\showh[.25]{svm_margin}
\hspace{2cm}
}

}

%%%%%%%%%%%%%%%%%%%%%%%%%%%%%%%%%%%%%%%%%%%%%%%%%%%%%%%%%%%%%%%%%%%%%%%%%%%%%%%%

\slide{Finding the optimal discriminative function [from ML lecture]}{

\item The constrained problem
$$\min_{\b,\xi}
\norm{\b}^2 + C \sum_{i=1}^n \xi_i \st y_i (x_i^\T \b) \ge 1-\xi_i\comma \xi_i\ge 0$$
is a \textbf{quadratic program} and can be reformulated as the dual
problem, with dual parameters $\a_i$ that indicate whether the
constraint $y_i (x_i^\T \b) \ge 1-\xi_i$ is active. The dual problem
is \textbf{convex}. SVM libraries use, e.g., CPLEX to solve this.

\item For all inactive constraints ($y_i (x_i^\T \b) \ge 1$) the data
point $(x_i, y_i)$ does not directly influence the solution
$\b^*$. Active points are support vectors.

}

%%%%%%%%%%%%%%%%%%%%%%%%%%%%%%%%%%%%%%%%%%%%%%%%%%%%%%%%%%%%%%%%%%%%%%%%%%%%%%%%

\slide{ [from ML lecture] }{
\tiny
\item Let $(x,\xi)$ be the primal variables, $(\a,\m)$ the dual, we
derive the dual problem:
\begin{align}
\min_{\b,\xi}
&~ \norm{\b}^2 + C \sum_{i=1}^n \xi_i \st y_i (x_i^\T \b) \ge
1-\xi_i\comma \xi_i\ge 0 \\
L(\b,\xi,\a,\m)
&= \norm{\b}^2 + C \sum_{i=1}^n \xi_i
 - \sum_{i=1}^n \a_i [y_i (x_i^\T \b) - (1-\xi_i)]
 - \sum_{i=1}^n \m_i \xi_i \label{eq2}\\
\del_\b L
&\overset{!}= 0 \quad\To\quad 2\b=\sum_{i=1}^n \a_i y_i x_i \label{eq3}\\
\del_\xi L
&\overset{!}= 0 \quad\To\quad \forall_i:~ \a_i = C-\mu_i\\
l(\a,\m)
&= \min_{\b,\xi} L(\b,\xi,\a,\m)
 = -{\textstyle\frac{1}{4}} \sum_{i=1}^n \sum_{i'=1}^n \a_i \a_{i'}
 y_i y_{i'} \hat x_i^\T \hat x_{i'} 
 + \sum_{i=1}^n \a_i \label{eq4}\\
\max_{\a,\m}
&~ l(\a,\m) \st 0 \le \a_i \le C
\label{eq5}
\end{align}

\item \refeq{eq2}: Lagrangian (with negative Lagrange terms because of
$\ge$ instead of $\le$~)

\item \refeq{eq3}: the optimal $\b^*$ depends only on $x_i y_i$ for which
$\a_i>0$ $\to$ support vectors

\item \refeq{eq4}: This assumes that $x_i=(1,\hat x_i)$ includes the constant
feature $1$ (so that the statistics become centered)

\item \refeq{eq5}: This is the dual problem. $\mu_i\ge 0$ implies $\a_i\le C$

\item Note: the dual problem only refers to $\hat x_i^\T \hat x_i$
~$\to$~ \textbf{kernelization}

}

%%%%%%%%%%%%%%%%%%%%%%%%%%%%%%%%%%%%%%%%%%%%%%%%%%%%%%%%%%%%%%%%%%%%%%%%%%%%%%%%

\slide{Algorithms for Convex Programming}{

~

\item All the ones we discussed for non-linear optimization!
\begin{items}
\item log barrier  (``interior point method'', ``[central] path following'')
\item augmented Lagrangian
\item primal-dual Newton
\end{items}

~

\item The simplex algorithm, walking on the constraints

~

(The emphasis in the notion of \emph{interior} point methods is to
distinguish from constraint walking methods.)

}

%%%%%%%%%%%%%%%%%%%%%%%%%%%%%%%%%%%%%%%%%%%%%%%%%%%%%%%%%%%%%%%%%%%%%%%%%%%%%%%%

\sublecture{Simplex Algorithm}{
}

%%%%%%%%%%%%%%%%%%%%%%%%%%%%%%%%%%%%%%%%%%%%%%%%%%%%%%%%%%%%%%%%%%%%%%%%%%%%%%%%

\key{Simplex method}
\slide{Simplex Algorithm}{

Georg Dantzig (1947)

{\tiny Note: Not to confuse with the Nelder-Mead method (downhill simplex method)}

~

\item Consider an LP
$$\min_x~ c^\T x \st G x \le h,~ Ax=b$$
%$$\min_x~ c^\T x \st x \ge 0,~ Ax=b$$

\item Note that in a linear program an optimum is always
located at a vertex

{\tiny (If there are multiple optimal, at least one of them is at a vertex.) }

\show[.3]{simplex}

}

%%%%%%%%%%%%%%%%%%%%%%%%%%%%%%%%%%%%%%%%%%%%%%%%%%%%%%%%%%%%%%%%%%%%%%%%%%%%%%%%

\slide{Simplex Algorithm}{

\show[.3]{simplex}

\item The Simplex Algorithm walks along the edges of the \textbf{polytope}, at
every vertex choosing the edge that decreases $c^\T x$ most

\item This either terminates at a vertex, or leads to an unconstrained
edge ($-\infty$ optimum)

~

\item In practise this procedure is done by ``pivoting on the simplex
tableaux''

}

%%%%%%%%%%%%%%%%%%%%%%%%%%%%%%%%%%%%%%%%%%%%%%%%%%%%%%%%%%%%%%%%%%%%%%%%%%%%%%%%

\slide{Simplex Algorithm vs.\ Interior methods}{

\small

\item The simplex algorithm is often efficient, but in worst case
exponential in $n$ and $m$!

{\tiny (In high dimensions constraints may intersect and form edges and vertices in a combinatorial way.)

}

~\pause

\item Sitting on an edge/face/vertex $\oto$ hard decisions on which constraints are active
\begin{items}
\item The simplex algorithm is sequentially making decisions on which constraints might
be active -- by walking through this combinatorial space.
\end{items}

\item Interior point methods do exactly the opposite:
\begin{items}
\item They ``postpone'' (or relax) hard decisions about active/non-active constraints,
\item approach the optimal vertex from the inside of the polytope;
avoiding the polytope surface
%% \item avoid the need to search through a combinatorial space
%% of constraint activities
\item have polynomial worst-case guaranteed
\end{items}

\pause

\item Historically:
\begin{items}
\item Before 50ies: Penalty and barrier methods methods were standard
\item From 50s: Simplex Algorithm
\item From 70s: More theoretical understanding, interior point methods (and more recently Augmented Lagrangian methods) again more popular
\end{items}


}

%%%%%%%%%%%%%%%%%%%%%%%%%%%%%%%%%%%%%%%%%%%%%%%%%%%%%%%%%%%%%%%%%%%%%%%%%%%%%%%%

\sublecture{Sequential Quadratic Programming}{
}

%%%%%%%%%%%%%%%%%%%%%%%%%%%%%%%%%%%%%%%%%%%%%%%%%%%%%%%%%%%%%%%%%%%%%%%%%%%%%%%%

\slide{Quadratic Programming}{

$$\min_x~ \half x^\T Q x + c^\T x   \st   G x \le h,~ Ax=b$$

%(The dual of a QP is again a QP)

~

\item Efficient Algorithms:
\begin{items}
\item Interior point (log barrier)
\item Augmented Lagrangian
\end{items}

~

\item Highly relevant applications:
\begin{items}
\item Support Vector Machines
\item Similar types of max-margin modelling methods
\end{items}

}

%%%%%%%%%%%%%%%%%%%%%%%%%%%%%%%%%%%%%%%%%%%%%%%%%%%%%%%%%%%%%%%%%%%%%%%%%%%%%%%%

\key{Sequential quadratic programming}
\slide{Sequential Quadratic Programming (SQP)}{

\item SQP is another standard method for \textbf{non-linear} programs
\begin{items}
\item It can be understood as generalization of the Newton method to the constrained case:
\item The Newton method for $\min_x f(x)$ approximates $f$ using 2nd-order Taylor, and computes the optimal step $\d^*$ for this approximation
\item SQP approximates costs $f$ and constraints $g,h$ using Taylor, and then computes the optimal step $\d^*$ for this approximation
\end{items}
\pause
\item In each iteration we consider Taylor approximations:
\begin{items}
\item 2nd order for: $f(x+\d) \approx f(x) + \na f(x)^\T\d + \half \d^\T H \d$
\item 1st order for: $g(x+\d) \approx g(x) + \na g(x)^\T\d\comma h(x+\d) \approx h(x) + \na h(x)^\T\d$
\end{items}
\item Then we compute the optimal step $\d^*$ solving the QP:
\begin{equation*}
\min_\d~ f(x) + \na f(x)^\T \d + \half \d^\T \he f(x) \d \st g(x) + \na
g(x)^\T \d\le 0\comma h(x) + \na
h(x)^\T \d=0
\end{equation*}

}

%%%%%%%%%%%%%%%%%%%%%%%%%%%%%%%%%%%%%%%%%%%%%%%%%%%%%%%%%%%%%%%%%%%%%%%%%%%%%%%%

\slide{Sequential Quadratic Programming (SQP)}{

\small

\item If $f$ \emph{were} a 2nd-order polynomial and $g,h$ linear, then $\d^*$
would jump directly to the optimum

\item Otherwise, backtracking line search

~\pause

\item Note: Solving each QP to compute the search step $\d^*$ requires a constrained solver,
which itself might have two nested loops (e.g.\ using log-barrier or
AugLag) $\to$ three nested loops

\item \textbf{But:} To solve the QP-step, you need no queries of the original problem!

$\to$ SQP can be query efficient. It invests in solving an approximate QP to minimize querying the original problem

\item Potentially more prone to non-smoothness (local Taylor might be misleading)

}

%%%%%%%%%%%%%%%%%%%%%%%%%%%%%%%%%%%%%%%%%%%%%%%%%%%%%%%%%%%%%%%%%%%%%%%%%%%%%%%%

\slide{Baseline methods for constrained optimization}{

\item We now learnt about four baseline methods to tackle constrained optimization:
\begin{items}
\item Log barrier method
%\item Squared penalty method (approximate only)
\item Augmented Lagrangian method
\item Primal-dual Newton
\item Sequential Quadratic Programming
\end{items}

}

%%%%%%%%%%%%%%%%%%%%%%%%%%%%%%%%%%%%%%%%%%%%%%%%%%%%%%%%%%%%%%%%%%%%%%%%%%%%%%%%

\sublecture{LP-relaxations of discrete problems*}{
}

%%%%%%%%%%%%%%%%%%%%%%%%%%%%%%%%%%%%%%%%%%%%%%%%%%%%%%%%%%%%%%%%%%%%%%%%%%%%%%%%

\key{LP-relaxations of integer programs}
\slide{Integer linear programming (ILP)}{

\item An integer linear program (for simplicity binary) is
$$\min_x~ c^\T x \st Ax=b,~ x_i \in\{0,1\}$$

~

\item Examples:
\begin{items}
\item Travelling Salesman: $\min_{x_{ij}} \sum_{ij} c_{ij} x_{ij}$ with
$x_{ij}\in\{0,1\}$ and constraints $\forall_j: \sum_i x_{ij}=1$
(columns sum to 1), $\forall_j: \sum_i x_{ji}=1$, $\forall_{ij}:
t_j-t_i \le n-1+n x_{ij}$ (where $t_i$ are additional integer variables).


\item MaxSAT problem: In conjunctive normal form, each clause contributes
an additional variable and a term in the objective function; each
clause contributes a constraint

\item Search the web for \emph{The Power of Semidefinite Programming
Relaxations for MAXSAT}
\end{items}


}

%%%%%%%%%%%%%%%%%%%%%%%%%%%%%%%%%%%%%%%%%%%%%%%%%%%%%%%%%%%%%%%%%%%%%%%%%%%%%%%%

\slide{LP relaxations of integer linear programs}{

\item Instead of solving
$$\min_x c^\T x \st Ax=b,~ x_i \in\{0,1\}$$
we solve
$$\min_x c^\T x \st Ax=b,~ x\in[0,1]$$

~

\item Clearly, the relaxed solution will be a \emph{lower bound} on
the integer solution (sometimes also called ``outer bound'' because $[0,1]\supset\{0,1\}$)

~

\item Computing the relaxed solution is interesting
\begin{items}
\item as an ``approximation'' or initialization to the integer problem
\item in cases where the optimal relaxed solution happens to be
   integer
\item for using the lower bound for \textbf{branch-and-bound} tree search over the discrete variable
\end{items}

}

%%%%%%%%%%%%%%%%%%%%%%%%%%%%%%%%%%%%%%%%%%%%%%%%%%%%%%%%%%%%%%%%%%%%%%%%%%%%%%%%

\slide{Example*:~ MAP inference in MRFs}{

\small

\item Given integer random variables $x_i$,
$i=1,..,n$, a pairwise Markov Random Field (MRF) is defined as
$$f(x) = \sum_{(ij)\in E} f_{ij}(x_i, x_j) + \sum_i f_i(x_i)$$
where $E$ denotes the set of edges. Problem: find $\max_x f(x)$.

{\tiny (Note: any general (non-pairwise) MRF can be converted
into a pair-wise one, blowing up the number of variables)

}

\item Reformulate with indicator variables
$$b_i(x) = [x_i=x] \comma b_{ij}(x,y) = [x_i=x]~ [x_j=y]$$
These are $nm + |E|m^2$ binary variables

\item The indicator variables need to fulfil the constraints
\begin{align*}
b_i(x), b_{ij}(x,y) &\in\{0,1\} \\
\sum_x b_i(x) &= 1 &&\text{because $x_i$ takes eactly one value}\\
\sum_y b_{ij}(x,y) &= b_i(x) &&\text{consistency between indicators}
\end{align*}

}

%%%%%%%%%%%%%%%%%%%%%%%%%%%%%%%%%%%%%%%%%%%%%%%%%%%%%%%%%%%%%%%%%%%%%%%%%%%%%%%%

\slide{Example*:~ MAP inference in MRFs}{

\small

\item Finding $\max_x f(x)$ of a MRF is then equivalent to
$$\max_{b_i(x),b_{ij}(x,y)} \sum_{(ij)\in E}\sum_{x,y}
b_{ij}(x,y)~ f_{ij}(x, y) + \sum_i\sum_x b_i(x)~ f_i(x)$$
such that
$$b_i(x), b_{ij}(x,y) \in\{0,1\} \comma \sum_x b_i(x) =
1 \comma \sum_y b_{ij}(x,y) = b_i(x)$$

~

\item The LP-relaxation replaces the constraint to be
$$b_i(x), b_{ij}(x,y) \in[0,1] \comma \sum_x b_i(x) =
1 \comma \sum_y b_{ij}(x,y) = b_i(x)$$

This set of feasible $b$'s is called \textbf{marginal polytope}
(because it describes the a space of ``probability distributions''
that are marginally consistent (but not necessarily globally normalized!))

}

%%%%%%%%%%%%%%%%%%%%%%%%%%%%%%%%%%%%%%%%%%%%%%%%%%%%%%%%%%%%%%%%%%%%%%%%%%%%%%%%

\slide{Example*:~ MAP inference in MRFs}{

\small

\item Solving the original MAP problem is NP-hard

Solving the LP-relaxation is really efficient

~

\item If the solution of the LP-relaxation turns out to be integer,
we've solved the originally NP-hard problem!

If not, the relaxed problem can be discretized to be a good
initialization for discrete optimization

~

\item For binary attractive MRFs (a common case) the solution will always
be integer

}

%%%%%%%%%%%%%%%%%%%%%%%%%%%%%%%%%%%%%%%%%%%%%%%%%%%%%%%%%%%%%%%%%%%%%%%%%%%%%%%%

\slide{Conclusions}{\label{lastpage}

\item Convex Problems are an important special case
\begin{items}
\item Convergence of backtracking line search $\ot$ bounded Hessian $\to$ convexity
\item Some applications are convex
\end{items}
\item Algorithms for convex programs are same as we discussed before

~

\item Baseline methods for constrained optimization:
\begin{items}
\item Log barrier method
%\item Squared penalty method (approximate only)
\item Augmented Lagrangian method
\item Primal-dual Newton
\item Sequential Quadratic Programming
\end{items}

}

%%%%%%%%%%%%%%%%%%%%%%%%%%%%%%%%%%%%%%%%%%%%%%%%%%%%%%%%%%%%%%%%%%%%%%%%%%%%%%%%

\slidesfoot
