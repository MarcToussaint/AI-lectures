\providecommand{\slides}{
  \newcommand{\slideshead}{
  \newcommand{\thepage}{\arabic{mypage}}
  %beamer
%  \documentclass[t,hyperref={bookmarks=true}]{beamer}
%  \geometry{papersize={171mm,96mm}}
  \documentclass[t,hyperref={bookmarks=true},aspectratio=169]{beamer}
  \setbeamersize{text margin left=5mm}
  \setbeamersize{text margin right=5mm}
  \usetheme{default}
  \usefonttheme[onlymath]{serif}
  \setbeamertemplate{navigation symbols}{}
  \setbeamertemplate{itemize items}{{\color{black}$\bullet$}}

  \newwrite\keyfile

  %\usepackage{palatino}
  \stdpackages
  %\usepackage{tikz} \usetikzlibrary {shapes.geometric} 
  \usepackage{multimedia}
  \usepackage[utf8]{inputenc}

  %%% geometry/spacing issues
  %
  \definecolor{bluecol}{rgb}{0,0,.5}
  \definecolor{greencol}{rgb}{0,.6,0}
  \definecolor{citcol}{rgb}{.4,.4,.4}
  %\renewcommand{\baselinestretch}{1.1}
  \renewcommand{\arraystretch}{1.2}
  \columnsep 0mm

  \columnseprule 0pt
  \parindent 0ex
  \parskip 0ex
  %\setlength{\itemparsep}{3ex}
  %\renewcommand{\labelitemi}{\rule[3pt]{10pt}{10pt}~}
  %\renewcommand{\labelenumi}{\textbf{(\arabic{enumi})}}
  \setbeamertemplate{enumerate item}{(\roman{enumi})}
  \newcommand{\headerfont}{\helvetica{14}{1.5}{b}{n}}
  \newcommand{\slidefont} {\helvetica{11}{1.4}{m}{n}}
  %\newcommand{\codefont} {\helvetica{8}{1.2}{m}{n}}
  \newcommand{\urlfont} {\fontsize{6}{1.1}\selectfont}
  \renewcommand{\small} {\helvetica{10}{1.4}{m}{n}}
  \renewcommand{\tiny} {\helvetica{8}{1.3}{m}{n}}
  \newcommand{\ttiny} {\helvetica{7}{1.3}{m}{n}}

  %%% count pages properly and put the page number in bottom right
  %
  \newcounter{mypage}
  \newcommand{\incpage}{\addtocounter{mypage}{1}\setcounter{page}{\arabic{mypage}}}
  \setcounter{mypage}{0}
  \resetcounteronoverlays{page}

  \pagestyle{fancy}
  %\setlength{\headsep}{10mm}
  %\addtolength{\footheight}{15mm}
  \renewcommand{\headrulewidth}{0pt} %1pt}
  \renewcommand{\footrulewidth}{0pt} %.5pt}
  \cfoot{}
  \rhead{}
  \lhead{}
  %% \lfoot{\vspace*{-3mm}\hspace*{-3mm}\helvetica{5}{1.3}{m}{n}{\texttt{github.com/MarcToussaint/AI-lectures}}}
\lfoot{}
%  \rfoot{{\tiny\textsf{AI -- \topic -- \subtopic -- \arabic{mypage}/\pageref{lastpage}}}}
  %\lfoot{\raisebox{5mm}{\tiny\textsf{\slideauthor}}}
  %\rfoot{\raisebox{5mm}{\tiny\textsf{\slidevenue{} -- \arabic{mypage}/\pageref{lastpage}}}}
  %\rfoot{~\anchor{30,12}{\tiny\textsf{\thepage/\pageref{lastpage}}}}
  %\lfoot{\small\textsf{Marc Toussaint}}
%  \rfoot{\vspace*{-4.5mm}{\tiny\textsf{\color{gray}\topic\ -- \subtopic\ -- \arabic{mypage}/\pageref{lastpage}}}\hspace*{-4mm}}
  \rfoot{\vspace*{-4.5mm}{\tiny\textsf{\color{gray}\topic\ -- \arabic{mypage}/\pageref{lastpage}}}\hspace*{-4mm}}
%  \rfoot{~\anchor{-10,12}{\tiny\textsf{\color{gray}\topic\ --  \arabic{mypage}/\pageref{lastpage}}}}
  \lfoot{\vspace*{-4.5mm}{\hspace*{-3mm}\includegraphics[height=4mm]{LIS-logo-longText}}}

  \definecolor{grey}{rgb}{.8,.8,.8}
  \definecolor{head}{rgb}{.85,.9,.9}
%  \definecolor{blue}{rgb}{.0,.0,.5}
%  \definecolor{green}{rgb}{.0,.5,.0}
  \definecolor{red}{rgb}{.8,.0,.0}
  \newcommand{\inverted}{
    \definecolor{main}{rgb}{1,1,1}
    \color{main}
    \pagecolor[rgb]{.3,.3,.3}
  }
  \input{../latex/macros}

  \graphicspath{{pics/}{../pics/}{pics-local/}}
}

\newcommand{\slidestitle}{
  \title{\course \topic}
  \author{Marc Toussaint}
  \institute{Learning \& Intelligent Systems Lab, TU Berlin}

  \begin{document}


  %% title slide!
  \slide{}{
    \thispagestyle{empty}

    \twocol{.35}{.55}{
      %\vspace*{-5mm}%
      \hspace*{-5mm}%
      \includegraphics[width=1.\columnwidth]{\coursepicture}
    }{\center

      \textbf{\fontsize{17}{20}\selectfont \course}

      ~

      %Lecture
      \topic\\

      \vspace{1cm}

      {\tiny~\emph{\keywords}~\\}

      \vspace{1cm}

      \teacher
      
      Technical University of Berlin

      \coursedate

      ~

    }
  }
}

\newcommand{\slide}[2]{
  \slidefont
  \incpage\begin{frame}
  \addcontentsline{toc}{section}{#1}
  \vfill
  {\headerfont #1} \vspace*{-2ex}
  \begin{itemize}\item[]~\\
    #2
  \end{itemize}
  \vfill
  \end{frame}
}

% use \begin{frame}[fragile] around slidecore!
\newenvironment{slidecore}[1]{
  \slidefont\incpage
  \addcontentsline{toc}{section}{#1}
  \vfill
  {\headerfont #1} \vspace*{-2ex}
  \begin{itemize}\item[]~\\
}{
  \end{itemize}
  \vfill
}

\newcommand{\titleslide}[4][Marc Toussaint]{
  \newcommand{\slideauthor}{#1}
  \newcommand{\slidevenue}{#3}
  \slidefont
  \incpage
  \begin{frame}
  \begin{center}
    \vspace*{15mm}

    {\headerfont #2\\}
        
    \vspace*{10mm}

    #1 \\

    \vspace*{5mm}

    {\small
      Learning \& Intelligent Systems Lab, TU Berlin\\
%      Science of Intelligence Cluster of Excellence, Berlin\\
      Max Planck Fellow, Institute for Intelligent Systems\\ %Physical Reasoning \& Manipulation Lab -- 
%      MIT CSAIL\\
%      mtoussai@mit.edu,~ marc.toussaint@informatik.uni-stuttgart.de

      \vspace*{10mm}

      \emph{#3}
    }

    \vspace*{0mm}

    %\includegraphics[scale=.1]{pics/eushield-fullcolour}

  \end{center}
  \begin{itemize}\item[]~\\
    #4
  \end{itemize}
  \end{frame}
}

\newcommand{\titleslideempty}[2]{
  \slidefont
  \incpage
  \begin{frame}
  \begin{center}
    \vspace*{15mm}

    {\headerfont #1\\}
        
    %% \vspace*{5mm}

    %% {\small\emph{#2}} \\

  \end{center}
  \begin{itemize}\item[]~\\
    #2
  \end{itemize}
  \end{frame}
}

\providecommand{\key}[1]{
  \addtocounter{mypage}{1}
% \immediate\write\keyfile{#1}
  \addtocontents{toc}{\hyperref[key:#1]{#1 (\arabic{mypage})}}
%  \phantomsection\label{key:#1}
%  \index{#1@{\hyperref[key:#1]{#1 (\arabic{mysec}:\arabic{mypage})}}|phantom}
  \addtocounter{mypage}{-1}
}

\providecommand{\course}{}

\providecommand{\subtopic}{}

\providecommand{\sublecture}[2]{
  \renewcommand{\subtopic}{#1}
  \slide{#1}{#2}
}

\providecommand{\sublectureHide}[2]{
  \renewcommand{\subtopic}{#1}
}

\providecommand{\story}[1]{
~

Motivation: {\tiny #1}\clearpage
}

\newenvironment{items}[1][9]{
\par\setlength{\unitlength}{1pt}\fontsize{#1}{#1}\linespread{1.2}\selectfont
\begin{list}{--}{\leftmargin4ex \rightmargin0ex \labelsep1ex \labelwidth2ex
\topsep.7ex \parsep0ex \itemsep3pt}
}{
\end{list}
}

\newenvironment{itemS}[1][4ex]{
\par
\tiny
\begin{list}{--}{\leftmargin#1 \rightmargin0ex \labelsep1ex
  \labelwidth2ex \topsep0pt \parsep0ex \itemsep2pt}
}{
\end{list}
}

\providecommand{\slidesfoot}{
  \end{document}
}


  \slideshead
  %\slidestitle
}

\providecommand{\exercises}{
  \input{../latex/style-exercises}
  \exerciseshead
}

\providecommand{\script}{
  \newcommand{\scripthead}{
  \documentclass[9pt,fleqn,twoside]{article}
  \stdpackages

  \usepackage{makeidx}
  \makeindex

  \usepackage{thmtools}
  \definecolor{shadecolor}{gray}{0.85}
  \declaretheoremstyle[
    %headfont=\normalfont\bfseries,
    %notefont=\mdseries, notebraces={(}{)},
    %bodyfont=\normalfont,
    %postheadspace=0.5em,
    %spaceabove=6pt,
    mdframed={
      %  skipabove=8pt,
      %  skipbelow=6pt,
      hidealllines=true,
      backgroundcolor={shadecolor},
      innertopmargin=8pt,
      %  innerleftmargin=3pt,
      %  innerrightmargin=3pt
    }
  ]{shaded}
  \declaretheorem[style=shaded,within=section,name=Definition]{myDefinition}
  \declaretheorem[style=shaded,within=section,name=Theorem]{myTheorem}
  \declaretheorem[style=shaded,within=section,name=Identities]{Identities}
  \declaretheorem[style=shaded,within=section,name=Example]{myExample}

  \definecolor{grey}{rgb}{.8,.8,.8}
  \definecolor{bluecol}{rgb}{0,0,.5}
  \definecolor{greencol}{rgb}{0,.4,0}
  \definecolor{shadecolor}{gray}{0.9}
  \definecolor{citcol}{rgb}{.4,.4,.4}
  \usepackage[
    %    pdftex%,
    %%    letterpaper,
    %bookmarks,
    bookmarksnumbered,
    colorlinks,
    urlcolor=bluecol,
    citecolor=black,
    linkcolor=bluecol,
    %    pagecolor=bluecol,
    pdfborder={0 0 0},
    %pdfborderstyle={/S/U/W 1},
    %%    backref,     %link from bibliography back to sections
    %%    pagebackref, %link from bibliography back to pages
    %%    pdfstartview=FitH, %fitwidth instead of fit window
    pdfpagemode=UseOutlines, %bookmarks are displayed by acrobat
    pdftitle={\course},
    pdfauthor={Marc Toussaint},
    pdfkeywords={}
  ]{hyperref}
  \DeclareGraphicsExtensions{.pdf,.png,.jpg,.eps}

  %\usepackage{multirow}
  \usepackage{multimedia}
  %\usepackage{marginnote}
  %\setbeamercolor{background canvas}{bg=}

  \usepackage[round]{natbib}
  \bibliographystyle{abbrvnat}

  \renewcommand{\r}{\varrho}
  \renewcommand{\l}{\lambda}
  \renewcommand{\L}{\Lambda}
  \renewcommand{\b}{\beta}
  \renewcommand{\d}{\delta}
  \renewcommand{\k}{\kappa}
  \renewcommand{\t}{\theta}
  \renewcommand{\O}{\Omega}
  \renewcommand{\o}{\omega}
  \renewcommand{\SS}{{\cal S}}
  \renewcommand{\=}{\!=\!}
  %\renewcommand{\boldsymbol}{}
  %\renewcommand{\Chapter}{\chapter}
  %\renewcommand{\Subsection}{\subsection}

  \renewcommand{\baselinestretch}{1.0}
  \geometry{a5paper,headsep=6mm,hdivide={10mm,*,10mm},vdivide={13mm,*,7mm}}

  \fancyhead[OL,ER]{\course, \textit{Marc Toussaint}}
  \fancyhead[OR,EL]{\thepage}
  \fancyhead[C]{}
  \fancyfoot{}
  \pagestyle{fancy}

  \renewcommand{\labelenumi}{{(\roman{enumi})}}
  \renewcommand{\theenumi}{(\roman{enumi})} %for ref
  \parindent 0pt
  \parskip .5pc

  \columnsep 6ex

  \renewcommand{\familydefault}{\sfdefault}

  \newcommand{\headerfont}{\large}%helvetica{12}{1}{b}{n}}
  \newcommand{\slidefont} {}%\helvetica{9}{1.3}{m}{n}}
  \newcommand{\storyfont} {}
  %  \renewcommand{\small}   {\helvetica{8}{1.2}{m}{n}}
  \renewcommand{\tiny}    {\footnotesize}%helvetica{7}{1.1}{m}{n}}
  \newcommand{\ttiny} {\footnotesize}%fontsize{7}{7}\selectfont}
%  \newcommand{\codefont}{\fontsize{6}{6}\selectfont}%helvetica{8}{1.2}{m}{n}}
  \newcommand{\codefont} {\helvetica{8}{1.2}{m}{n}}

  \input{../latex/macros}

  \usepackage{comment}
  \specialcomment{solution}{
    \small
    \begin{shaded}
  }{
    \end{shaded}
  }

  \graphicspath{{pics/}{../pics/}{pics-local/}}

  \mytitle{\course\\Lecture Script}
  \myauthor{Marc Toussaint}
  \date{\coursedate}
}

%%%%%%%%%%%%%%%%%%%%%%%%%%%%%%%%%%%%%%%%%%%%%%%%%%%%%%%%%%%%%%%%%%%%%%%%%%%%%%%%

\newcommand{\scripttitle}{
  \begin{document}
  \maketitle
  %\anchor{100,10}{\includegraphics[width=4cm]{optim}}
%  \vspace*{1cm}
}

%%%%%%%%%%%%%%%%%%%%%%%%%%%%%%%%%%%%%%%%%%%%%%%%%%%%%%%%%%%%%%%%%%%%%%%%%%%%%%%%

\newcounter{mypage}
\setcounter{mypage}{0}
\newcommand{\incpage}{\addtocounter{mypage}{1}}

\newcommand{\subtopic}{}
\newcommand{\pause}{}
\newcommand{\only}[1]{#1}

\renewcommand{\slides}[1][]{
  %  \clearpage
  \subsection{\topic}
  \index{\topic}
  {\small #1}
  \setcounter{mypage}{0}
  \smallskip\nopagebreak\hrule\medskip
}

\newcommand{\slidesfoot}{
  \bigskip
}

\newcommand{\sublecture}[2]{
  \phantomsection\addcontentsline{toc}{subsubsection}{#1}
  \index{#1}
}

\newcommand{\sublectureHide}[2]{
  \renewcommand{\subtopic}{#1}
}

\newcommand{\key}[1]{
  \phantomsection\addcontentsline{toc}{subsubsection}{#1}
  %\subsubsection{#1}
  \index{#1}
}

\providecommand{\defn}[1]{%
  \textbf{#1}\index{#1}%
}

\newenvironment{slidecore}[1]{
  \incpage
  \subsubsection*{#1}%{\headerfont\noindent\textbf{#1}\\}%
  \vspace{-6ex}%
  \begin{list}{$\bullet$}{\leftmargin4ex \rightmargin0ex \labelsep1ex
    \labelwidth2ex \partopsep0ex \topsep0ex \parsep.5ex \parskip0ex \itemsep0pt}\item[]~\\\nopagebreak%
}{
  \end{list}\nopagebreak%
  {\hfill\tiny \textsf{\arabic{section}.\arabic{subsection}:\arabic{mypage}}}\nopagebreak%
  \smallskip\nopagebreak\hrule
}

\newcommand{\slide}[2]{
  \begin{slidecore}{#1}
    #2
  \end{slidecore}
}

\renewcommand{\exercises}{}
\newcommand{\exercisestitle}{}
\newcommand{\exsection}[1]{\subsubsection{#1}}
\newcommand{\exsubsection}[1]{\paragraph{#1}}
\newcommand{\exerfoot}{\bigskip}

\newcommand{\story}[1]{
  \subsection*{Motivation \& Outline}
  {\storyfont\sf #1}
  \medskip\nopagebreak\hrule
}

\newcounter{savedsection}
\newcommand{\subappendix}{\setcounter{savedsection}{\arabic{section}}\appendix}
\newcommand{\noappendix}{
  \setcounter{section}{\arabic{savedsection}}% restore section number
  \setcounter{subsection}{0}% reset section counter
%  \gdef\@chapapp{\sectionname}% reset section name
  \renewcommand{\thesection}{\arabic{section}}% make section numbers arabic
}

\newenvironment{items}[1][9]{
\par\setlength{\unitlength}{1pt}\fontsize{#1}{#1}\linespread{1.2}\selectfont
\begin{list}{--}{\leftmargin4ex \rightmargin0ex \labelsep1ex \labelwidth2ex
\topsep0pt \parsep0ex \itemsep3pt}
}{
\end{list}
}

\newenvironment{itemS}[1][4ex]{
\par
\tiny
\begin{list}{--}{\leftmargin#1 \rightmargin0ex \labelsep1ex
  \labelwidth2ex \topsep0pt \parsep0ex \itemsep2pt}
}{
\end{list}
}

\newcommand{\Def}[1]{%
\textbf{#1}\index{#1}}%\marginnote{#1}}

  \scripthead
}

\providecommand{\paper}{
  \input{../latex/style-paper}
  \paperhead
}

\providecommand{\note}[1][9pt]{
  \providecommand{\notehead}[2]{
  \documentclass[#1,fleqn,twoside]{article}
  \stdpackages
  \renewcommand{\labelenumi}{{(\roman{enumi})}}
  \renewcommand{\theenumi}{(\roman{enumi})} %for ref

  \renewcommand{\baselinestretch}{#2}
  \renewcommand{\arraystretch}{1.2}
  \renewcommand{\topfraction}{1}
  \renewcommand{\bottomfraction}{1}
  \renewcommand{\textfraction}{0}
  \columnsep 5ex
  \parindent 3ex
  \parskip 1ex

  % Lists and paragraphs
  \parindent 0pt
  \topsep 4pt plus 1pt minus 2pt
  \partopsep 1pt plus 0.5pt minus 0.5pt
  \itemsep 2pt plus 1pt minus 0.5pt
  \parsep 2pt plus 1pt minus 0.5pt
  \parskip .5pc %add _in_ {thebibliography} environment in *.bbl

  \setcounter{tocdepth}{3}
  \setcounter{secnumdepth}{3}

  \geometry{a4paper,hdivide={25mm,*,25mm},vdivide={25mm,*,25mm}}

  \renewcommand{\headrulewidth}{.0pt}\renewcommand{\footrulewidth}{.0pt}\cfoot{}
  \fancyhead[OL,EC]{\it\theauthor---\today}
  \fancyhead[ER]{\leftmark}
  \fancyhead[OR,EL]{\thepage}
  \fancyfoot[EL,OR]{}
  \setlength{\headsep}{10mm}
  %\fancyhead[OL]{\rightmark}
  %\fancyfoot[EL,OR]{}


  %  \usepackage{palatino}

  \newcommand{\codefont}{\helvetica{8}{1.2}{m}{n}}

  \renewenvironment{abstract}{
    \vspace*{5ex}\begin{list}{}{
      \leftmargin3ex
      \rightmargin3ex
      \topsep-\parskip}\item[]
     \hrule\vspace{1.5ex}{\bf Abstract.~}\small}
    {\vspace{2ex}\hrule\end{list}\vspace{5ex}}
    
  \newenvironment{keyword}
    {\par{\it Keywords:~}}
    {}

  \def\makemytitle{%
    \thispagestyle{empty}
    \begin{list}{}{\leftmargin3ex \rightmargin3ex \topsep0ex \parsep0ex}\item[]
      \begin{center}
        {\fontsize{18}{25}\selectfont{\thetitle\\}}\vspace{5ex}

        {\fontsize{14}{16}\selectfont{\theauthor\\}}\vspace{1ex}

        {\footnotesize{\sl \addressFUB}\\ \emailBerlin}

        {\footnotesize \today}

        \vspace{1ex}
        {\small \published}
      \end{center}
    \end{list}
    \renewcommand{\maketitle}{\chapter{\thetitle}}
  }

  \input{../latex/macros}
  \pdflatex

  \graphicspath{{pics/}{../pics/}{pics-local/}}

  \myauthor{Marc Toussaint}
  \date{\today}
}

%%%%%%%%%%%%%%%%%%%%%%%%%%%%%%%%%%%%%%%%%%%%%%%%%%%%%%%%%%%%%%%%%%%%%%%%%%%%%%%%

\newcommand{\notetitle}{
  \begin{document}
  \thispagestyle{empty}
    
  \maketitle

}

\newenvironment{items}[1][9]{
\par\setlength{\unitlength}{1pt}\fontsize{#1}{#1}\linespread{1.2}\selectfont
\begin{list}{--}{\leftmargin4ex \rightmargin0ex \labelsep1ex \labelwidth2ex
\topsep0pt \parsep0ex \itemsep3pt}
}{
\end{list}
}

  \notehead{#1}{1.1}
}

\providecommand{\course}{NO COURSE}
\providecommand{\coursepicture}{NO PICTURE}
\providecommand{\coursedate}{NO DATE}
\providecommand{\topic}{NO TOPIC}
\providecommand{\keywords}{}
\providecommand{\exnum}{NO NUMBER}
\providecommand{\teacher}{Marc Toussaint}

\providecommand{\stdpackages}{
  \usepackage{amsmath}
  \usepackage{amssymb}
  \usepackage{amsfonts}
  \allowdisplaybreaks
  \usepackage{amsthm}
  \usepackage{eucal}
  \usepackage{graphicx}
%  \usepackage{color}
  \usepackage{geometry}
  \usepackage{framed}
  \usepackage{xcolor}
  \definecolor{shadecolor}{gray}{0.9}
  \setlength{\FrameSep}{3pt}
  \usepackage{fancyvrb}
  \fvset{numbers=none,xleftmargin=5ex,fontsize=\small}

  \usepackage{pdfpages}

  \usepackage{multicol} 
  \usepackage{fancyhdr}
}

\providecommand{\addressUSTT}{
  Machine~Learning~\&~Robotics~lab, U~Stuttgart\\\small
  Universit{\"a}tsstra{\ss}e 38, 70569~Stuttgart, Germany
}

\providecommand{\addressTUB}{
  Learning~\&~Intelligent~Systems~Lab, TU~Berlin\\\small
  Marchstr. 23, 10587 Berlin, Germany
}


\renewcommand{\course}{Optimization Algorithms}
\renewcommand{\coursepicture}{optim}
\renewcommand{\coursedate}{Winter 2024/25}

\renewcommand{\topic}{Stochastic Search \& EDAs}
\renewcommand{\keywords}{}

\slides

\slidestitle

%%%%%%%%%%%%%%%%%%%%%%%%%%%%%%%%%%%%%%%%%%%%%%%%%%%%%%%%%%%%%%%%%%%%%%%%%%%%%%%%

\slide{}{

A core aspect in black-box opt is: \emph{What do we estimate from the data?}
\begin{items}
\item gradient (as in implicit filtering)
\item a local model $f_\t(x)$ (model-based opt.)
\item a distribution $p_\t(x)$ of ``good'' points (EDAs)
%% \item Gradient/Newton descent: nothing, momentum, BFGS-estimate -- mostly relies on immediate local information $f(x), \na f(x), \he f(x)$
%% \item BayesOpt: data $D$ and implied GP $P(f|D)$
%% \pause
%% \item Evolutionary Algorithms: population $D$, or ``population parameters'' $\t$
\end{items}

~

\item The $\t$ is what we extract/capture/maintain from the data of previous evaluations
%% \item A core aspect of designing an optimization algorithm is: What does it \emph{maintain} in terms of knowledge/info/model?

}


%%%%%%%%%%%%%%%%%%%%%%%%%%%%%%%%%%%%%%%%%%%%%%%%%%%%%%%%%%%%%%%%%%%%%%%%%%%%%%%%

\key{General stochastic search}
\slide{A general stochastic search scheme}{

\item A general stochastic search scheme:
\begin{items}
\item The algorithm maintains some information $\t$
\item This $\t$ defines a \emph{search} distribution $p_\t(x)$
\item In each iteration it takes $\l$ samples $\{x_i\}_{i=1}^\l \sim p_\t(x)$
\item Each $x_i$ is evaluated ~ $\to$ ~ new data $D=\{(x_i,f(x_i))\}_{i=1}^\l$
\item \textbf{The new data $D$ is used to update $\t$}
\end{items}

~

\begin{algo}
\Require initial $\t$, function $f(x)$, distribution model $p_\t(x)$, update heuristic $h(\t,D)$
\Ensure final $\t$ and best point $x$
\Repeat
\State Sample $\{x_i\}_{i=1}^\l \sim p_\t(x)$
\State Evaluate samples, $D= \{(x_i,f(x_i))\}_{i=1}^\l$
\State Update $\t \gets h(\t,D)$
\Until $\t$ converges
\end{algo}

}

%%%%%%%%%%%%%%%%%%%%%%%%%%%%%%%%%%%%%%%%%%%%%%%%%%%%%%%%%%%%%%%%%%%%%%%%%%%%%%%%

\key{Evolutionary Algorithms (EAs)}
\slide{Evolutionary Algorithms (EAs)}{

\item EAs can well be described as special kinds of parameterizing $p_\t(x)$ and updating $\t$
\begin{items}
\item The $\t$ typically is a set of good points found so far (parents)

\item Mutation \& Crossover define $p_\t(x)$

\item The samples $D$ are called offspring

\item The $\t$-update is often a selection of the best, or ``fitness-proportional'' or rank-based
\end{items}

~
%% \item In discrete optimization, where $x$ is a string of integers,
%% $\t$ is a population of parents (strings)

%% $p_\t(x)$ is the offspring distribution (via crossover \& mutation)
%% defined by these parents

\item Categories of EAs:
\begin{items}
\item \defn{Evolution Strategies}:~ $x\in\RRR^n$, often Gaussian $p_\t(x)$

\item \defn{Genetic Algorithms}:~ $x\in\{0,1\}^n$, crossover \& mutation define
   $p_\t(x)$

\item \defn{Genetic Programming}:~ $x$ are programs/trees, crossover \& mutation

\item \defn{Estimation of Distribution Algorithms}:~ $\t$ directly defines
   $p_\t(x)$
\end{items}

}

%%%%%%%%%%%%%%%%%%%%%%%%%%%%%%%%%%%%%%%%%%%%%%%%%%%%%%%%%%%%%%%%%%%%%%%%%%%%%%%%

\sublecture{Evolution Strategies \& EDAs}{

(as they address continuous optimization in $\RRR^n$)

}

%%%%%%%%%%%%%%%%%%%%%%%%%%%%%%%%%%%%%%%%%%%%%%%%%%%%%%%%%%%%%%%%%%%%%%%%%%%%%%%%

\key{Evolution Strategies}
\slide{Evolution Strategies:~ Gaussian Search Distribution}{

{\tiny [From 1960s/70s. Rechenberg/Schwefel]}

\item The parameter $\t$ defines a Gaussian search distribution $p_\t(x)$
\item In the simplest case, $\t$ is just the mean $\t=(\hat x)$, assuming fixed $\s^2$:
$$p_\t(x) = \NN(x\|\hat x,\s^2)$$
\item We sample $\l$ ``offspring'' $x\sim p_\t$ to get new data $D$
~

\item What is a reasonable upate heuristic $\t \gets h(\t,D)$?
\pause
\begin{items}
\item \textbf{Selection:}
Given $D=\{(x_i,f(x_i))\}_{i=1}^\l$, select the $\m$ best: $D_\m=\text{bestOf}_\mu(D)$
\item Compute the new mean $\hat x$ from $D_\m$
\end{items}

~\pause

\item This algorithm is called ``$(\m,\l)$-ES'' (Evolution Strategy)
\begin{items}
\item The Gaussian is meant to represent a ``species''
\end{items}

}

%%%%%%%%%%%%%%%%%%%%%%%%%%%%%%%%%%%%%%%%%%%%%%%%%%%%%%%%%%%%%%%%%%%%%%%%%%%%%%%%

\slide{``Elitarian'' Selection: ~ $(\mu+\l)$-ES}{

\item To make search monotonous(!), the algorithm also stores the previous elite $D_\m$
\begin{items}
\item $\t= (\hat x, D_\m)$ now includes the mean $\hat x$ and previously selected
\end{items}

~\pause

\item The update heuristic $\t\gets h(\t, D)$ selects from the union of new and elite:
\begin{items}
\item Select the $\mu$ best $D_\mu \gets \text{bestOf}_\mu(D_\mu \cup D)$
\item Compute the new mean $\hat x$ from $D_\mu$
\end{items}

~\pause

\item Special case: ~ \textbf{$(1+1)$-ES = Greedy Local Search/Hill Climber}

\item Special case: ~ \textbf{$(1+\l)$-ES = Local Search}

}

%%%%%%%%%%%%%%%%%%%%%%%%%%%%%%%%%%%%%%%%%%%%%%%%%%%%%%%%%%%%%%%%%%%%%%%%%%%%%%%%

\slide{}{

\item Assuming a fixed $\s$ and isotropic $\NN(x\|\hat x,\s^2)$ is limiting
\begin{items}
\item No notion of going \emph{forward} (downhill/momentum)
\item No adaptation of $\s$
\item Should steps smaller/larger/correlated depending on local Hessian!
\end{items}

}

%%%%%%%%%%%%%%%%%%%%%%%%%%%%%%%%%%%%%%%%%%%%%%%%%%%%%%%%%%%%%%%%%%%%%%%%%%%%%%%%

\key{Covariance Matrix Adaptation (CMA-ES)}
\slide{Covariance Matrix Adaptation (CMA-ES)}{

\show[.7]{CMA-ES}
{\tiny\hfill [Hansen, N. (2006)]}

}

%%%%%%%%%%%%%%%%%%%%%%%%%%%%%%%%%%%%%%%%%%%%%%%%%%%%%%%%%%%%%%%%%%%%%%%%%%%%%%%%

\slide{Covariance Matrix Adaptation (CMA-ES)}{

\item In Covariance Matrix Adaptation
$$\t=(\hat x,\s,C,\r_\s,\r_c) \comma p_\t(x) = \NN(x\|\hat x,\s^2C)$$
where $C$ is the covariance matrix of the search distribution

\item The $\t$ maintains two more pieces of information: $\r_\s$ and
$\r_c$ capture the ``path'' (motion) of the mean $\hat x$ in recent
iterations

\item Rough outline of the $\t$-update:
\begin{items}
\item Let $D_\mu=\text{bestOf}_\mu(D)$ be the selected

\item Compute the new mean $\hat x$ of $D_\mu$

\item Update $\r_\s$ and $\r_c$ proportional to $\hat x_{k\po} - \hat x_k$

\item Update $\s$ depending on $|\r_\s|$

\item Update $C$ depending on $\r_c\r_c^\T$ (rank-1-update) and $\text{Var}(D_\mu)$
\end{items}

}

%%%%%%%%%%%%%%%%%%%%%%%%%%%%%%%%%%%%%%%%%%%%%%%%%%%%%%%%%%%%%%%%%%%%%%%%%%%%%%%%

\slide{CMA references}{

\cit{Hansen}{The CMA evolution strategy: a comparing review}{2006}

\cit{Hansen et al.}{Evaluating the CMA Evolution Strategy on Multimodal
Test Functions}{PPSN 2004}

\show[.7]{CMA-ES1}

%% ~

%% \item A variant:

%% Igel et al.: A Computational Efficient
%% Covariance Matrix Update and a $(1+1)$-CMA for Evolution
%% Strategies, GECCO 2006.

}

%%%%%%%%%%%%%%%%%%%%%%%%%%%%%%%%%%%%%%%%%%%%%%%%%%%%%%%%%%%%%%%%%%%%%%%%%%%%%%%%

\slide{CMA conclusions}{

\item Good starting point for an off-the-shelf blackbox
algorithm

\item It includes components like estimating the local gradient
($\r_\s, \r_c$), the local ``Hessian'' ($\text{Var}(D_\mu)$), smoothing out
local minima (large populations)

~

\item But is this tackling global optimization?

\cen{``For ``large enough'' populations local minima are avoided''}

(But not really.)

}

%%%%%%%%%%%%%%%%%%%%%%%%%%%%%%%%%%%%%%%%%%%%%%%%%%%%%%%%%%%%%%%%%%%%%%%%%%%%%%%%

\key{Estimation of Distribution Algorithms (EDAs)}
\slide{Estimation of Distribution Algorithms (EDAs)}{

\small

\item In general, $\t$ can model a distribution $p_\t(x)$ for any spaces (also discrete/hybrid) using any distribution representation (Bayesian Networks, probabilistic grammars, generative ML, etc)

\item The update heuristic $\t \gets h(\t, D)$ typically let's ``$p_\t(x)$ estimate $D_\mu$'', e.g.\ by likelihood maximization
\begin{align*}
\t ~\gets~ \argmin_\t~ -\!\! \sum_{x\in D_\mu} \log p_\t(x) + \text{regularization}
\end{align*}
\begin{items}
\item The regularization is important, otherwise the new offspring would ``overfit'' on the previous elite and not explore
\item E.g.\ ensure sufficient entropy
\end{items}

}

%%%%%%%%%%%%%%%%%%%%%%%%%%%%%%%%%%%%%%%%%%%%%%%%%%%%%%%%%%%%%%%%%%%%%%%%%%%%%%%%

\slide{}{

\item Stochastic grammars to ``learn'' a distribution of selected structures

\show[.5]{evoPlants}
{\tiny\hfill [Toussaint, GECCO 2003]}

}

%%%%%%%%%%%%%%%%%%%%%%%%%%%%%%%%%%%%%%%%%%%%%%%%%%%%%%%%%%%%%%%%%%%%%%%%%%%%%%%%

\key{Estimation of Distribution Algorithms (EDAs)}
\slide{Estimation of Distribution Algorithms (EDAs)}{

\item EDAs \emph{learn} correlations and structures in selected
\cit{Agakov,..,Toussaint,..,}{Using Machine Learning to Focus Iterative Optimization}{CGO 2006}
\cit{Toussaint}{Compact representations as a search strategy: Compression EDAs}{Theoretical Computer Science, 2006}
\begin{items}
\item E.g., if in all selected distributions, the 3rd bit equals the
7th bit, then the search distribution $p_\t(x)$ should put higher
probability on such candidates
\item In discrete domains,
graphical models can be used to learn the dependencies between variables, e.g.
\textbf{Bayesian Optimization Algorithm (BOA)}
\item In continuous domains, CMA is an example for an EDA
\end{items}

}

%%%%%%%%%%%%%%%%%%%%%%%%%%%%%%%%%%%%%%%%%%%%%%%%%%%%%%%%%%%%%%%%%%%%%%%%%%%%%%%%

%% \slide{Further Ideas}{

%% \item We could learn a distribution over steps

%% -- which steps have decreased $f$ recently $\to$ model

%% ~~ (Related to ``differential evolution'')

%% ~

%% \item We could learn a distributions over directions only

%% $\to$ sample one $\to$ line search

%% }

%%%%%%%%%%%%%%%%%%%%%%%%%%%%%%%%%%%%%%%%%%%%%%%%%%%%%%%%%%%%%%%%%%%%%%%%%%%%%%%%

\key{Simulated annealing}
\slide{Simulated Annealing ~ (accepts also uphill steps)}{

\small

\item Could be viewed as extension to avoid getting stuck in local optima, which
accepts steps with $f(y)>f(x)$ -- but better viewed as sampling technique (see next page)

~

\begin{algo}
\Require initial point $x$ ($\equiv\t$), function $f(x)$, \textbf{proposal distribution} $q(y|x)$ ($\equiv p_x(y)$)
\State initialilze the temperature $T=1$
\Repeat
\State Sample single $y \sim q(y|x)$
\State Acceptance probability $A
=\min\big\{1,~ e^{\frac{f(x)-f(y)}{T}}  \frac{q(x|y)}{q(y|x)}\big\}$
%=\min\big\{1,~ \frac{e^{-f(y)/T} q(x|y)}{e^{-f(x)/T} q(y|x)}\big\}
\State With probability $A$ update $x \gets y$
\State Decrease $T$, e.g.\ $T \gets (1-\e) T$ for small $\e$
\Until $x$ converges
\end{algo}

\item Typically: $q(y|x) \propto \exp\{-\half(y-x)^2/\s^2\}$

\item Instance of our general scheme for $x\equiv \t$, $p_\t(x) \equiv q(x|\t)$, $\l=1$, update stochastic as above

}

%%%%%%%%%%%%%%%%%%%%%%%%%%%%%%%%%%%%%%%%%%%%%%%%%%%%%%%%%%%%%%%%%%%%%%%%%%%%%%%%

\slide{Simulated Annealing}{

\small

\item Simulated Annealing is a Markov chain Monte Carlo (MCMC) method.
\begin{items}
\item Must read!: \emph{An Introduction to MCMC for Machine Learning}
\item These are iterative methods to sample from a distribution, in
our case
$$p(x) \propto e^{\frac{-f(x)}{T}}$$
\end{items}

\item For a fixed temperature $T$, one can prove that the set of
accepted points is distributed as $p(x)$ (but non-i.i.d.!) The acceptance probability
$$A=\min\big\{1,e^{\frac{f(x)-f(y)}{T}}  \frac{q(x|y)}{q(y|x)}\big\}$$
compares the $f(y)$ and $f(x)$, but also the reversibility of $q(y|x)$

\item When cooling the temperature, samples focus at the extrema.
Guaranteed to sample all extrema \emph{eventually}

%% \item Of high theoretical relevance, less of practical

}

%%%%%%%%%%%%%%%%%%%%%%%%%%%%%%%%%%%%%%%%%%%%%%%%%%%%%%%%%%%%%%%%%%%%%%%%%%%%%%%%

\slide{Simulated Annealing}{

\show[.5]{simulatedAnnealing}
{\tiny\hfill [MCMC introduction (2003)]}

}

%%%%%%%%%%%%%%%%%%%%%%%%%%%%%%%%%%%%%%%%%%%%%%%%%%%%%%%%%%%%%%%%%%%%%%%%%%%%%%%%

\slide{Stochastic search conclusions}{\label{lastpage}

\begin{algo}
\Require initial $\t$, function $f(x)$, distribution model $p_\t(x)$, update heuristic $h(\t,D)$
\Ensure final $\t$ and best point $x$
\Repeat
\State Sample $\{x_i\}_{i=1}^\l \sim p_\t(x)$
\State Evaluate samples, $D= \{(x_i,f(x_i))\}_{i=1}^\l$
\State Update $\t \gets h(\t,D)$
\Until $\t$ converges
\end{algo}

\item The framework is very general

\item Algorithms differ in choice of $\t$,  $p_\t(x)$, and $h(t,D)$

\item The update $h(\t,D)$ ``should train the distribution $p_\t(x)$ to match good points''

}

%%%%%%%%%%%%%%%%%%%%%%%%%%%%%%%%%%%%%%%%%%%%%%%%%%%%%%%%%%%%%%%%%%%%%%%%%%%%%%%%

\slidesfoot
