\providecommand{\slides}{
  \newcommand{\slideshead}{
  \newcommand{\thepage}{\arabic{mypage}}
  %beamer
%  \documentclass[t,hyperref={bookmarks=true}]{beamer}
%  \geometry{papersize={171mm,96mm}}
  \documentclass[t,hyperref={bookmarks=true},aspectratio=169]{beamer}
  \setbeamersize{text margin left=5mm}
  \setbeamersize{text margin right=5mm}
  \usetheme{default}
  \usefonttheme[onlymath]{serif}
  \setbeamertemplate{navigation symbols}{}
  \setbeamertemplate{itemize items}{{\color{black}$\bullet$}}

  \newwrite\keyfile

  %\usepackage{palatino}
  \stdpackages
  %\usepackage{tikz} \usetikzlibrary {shapes.geometric} 
  \usepackage{multimedia}
  \usepackage[utf8]{inputenc}

  %%% geometry/spacing issues
  %
  \definecolor{bluecol}{rgb}{0,0,.5}
  \definecolor{greencol}{rgb}{0,.6,0}
  \definecolor{citcol}{rgb}{.4,.4,.4}
  %\renewcommand{\baselinestretch}{1.1}
  \renewcommand{\arraystretch}{1.2}
  \columnsep 0mm

  \columnseprule 0pt
  \parindent 0ex
  \parskip 0ex
  %\setlength{\itemparsep}{3ex}
  %\renewcommand{\labelitemi}{\rule[3pt]{10pt}{10pt}~}
  %\renewcommand{\labelenumi}{\textbf{(\arabic{enumi})}}
  \setbeamertemplate{enumerate item}{(\roman{enumi})}
  \newcommand{\headerfont}{\helvetica{14}{1.5}{b}{n}}
  \newcommand{\slidefont} {\helvetica{11}{1.4}{m}{n}}
  %\newcommand{\codefont} {\helvetica{8}{1.2}{m}{n}}
  \newcommand{\urlfont} {\fontsize{6}{1.1}\selectfont}
  \renewcommand{\small} {\helvetica{10}{1.4}{m}{n}}
  \renewcommand{\tiny} {\helvetica{8}{1.3}{m}{n}}
  \newcommand{\ttiny} {\helvetica{7}{1.3}{m}{n}}

  %%% count pages properly and put the page number in bottom right
  %
  \newcounter{mypage}
  \newcommand{\incpage}{\addtocounter{mypage}{1}\setcounter{page}{\arabic{mypage}}}
  \setcounter{mypage}{0}
  \resetcounteronoverlays{page}

  \pagestyle{fancy}
  %\setlength{\headsep}{10mm}
  %\addtolength{\footheight}{15mm}
  \renewcommand{\headrulewidth}{0pt} %1pt}
  \renewcommand{\footrulewidth}{0pt} %.5pt}
  \cfoot{}
  \rhead{}
  \lhead{}
  %% \lfoot{\vspace*{-3mm}\hspace*{-3mm}\helvetica{5}{1.3}{m}{n}{\texttt{github.com/MarcToussaint/AI-lectures}}}
\lfoot{}
%  \rfoot{{\tiny\textsf{AI -- \topic -- \subtopic -- \arabic{mypage}/\pageref{lastpage}}}}
  %\lfoot{\raisebox{5mm}{\tiny\textsf{\slideauthor}}}
  %\rfoot{\raisebox{5mm}{\tiny\textsf{\slidevenue{} -- \arabic{mypage}/\pageref{lastpage}}}}
  %\rfoot{~\anchor{30,12}{\tiny\textsf{\thepage/\pageref{lastpage}}}}
  %\lfoot{\small\textsf{Marc Toussaint}}
%  \rfoot{\vspace*{-4.5mm}{\tiny\textsf{\color{gray}\topic\ -- \subtopic\ -- \arabic{mypage}/\pageref{lastpage}}}\hspace*{-4mm}}
  \rfoot{\vspace*{-4.5mm}{\tiny\textsf{\color{gray}\topic\ -- \arabic{mypage}/\pageref{lastpage}}}\hspace*{-4mm}}
%  \rfoot{~\anchor{-10,12}{\tiny\textsf{\color{gray}\topic\ --  \arabic{mypage}/\pageref{lastpage}}}}
  \lfoot{\vspace*{-4.5mm}{\hspace*{-3mm}\includegraphics[height=4mm]{LIS-logo-longText}}}

  \definecolor{grey}{rgb}{.8,.8,.8}
  \definecolor{head}{rgb}{.85,.9,.9}
%  \definecolor{blue}{rgb}{.0,.0,.5}
%  \definecolor{green}{rgb}{.0,.5,.0}
  \definecolor{red}{rgb}{.8,.0,.0}
  \newcommand{\inverted}{
    \definecolor{main}{rgb}{1,1,1}
    \color{main}
    \pagecolor[rgb]{.3,.3,.3}
  }
  \input{../latex/macros}

  \graphicspath{{pics/}{../pics/}{pics-local/}}
}

\newcommand{\slidestitle}{
  \title{\course \topic}
  \author{Marc Toussaint}
  \institute{Learning \& Intelligent Systems Lab, TU Berlin}

  \begin{document}


  %% title slide!
  \slide{}{
    \thispagestyle{empty}

    \twocol{.35}{.55}{
      %\vspace*{-5mm}%
      \hspace*{-5mm}%
      \includegraphics[width=1.\columnwidth]{\coursepicture}
    }{\center

      \textbf{\fontsize{17}{20}\selectfont \course}

      ~

      %Lecture
      \topic\\

      \vspace{1cm}

      {\tiny~\emph{\keywords}~\\}

      \vspace{1cm}

      \teacher
      
      Technical University of Berlin

      \coursedate

      ~

    }
  }
}

\newcommand{\slide}[2]{
  \slidefont
  \incpage\begin{frame}
  \addcontentsline{toc}{section}{#1}
  \vfill
  {\headerfont #1} \vspace*{-2ex}
  \begin{itemize}\item[]~\\
    #2
  \end{itemize}
  \vfill
  \end{frame}
}

% use \begin{frame}[fragile] around slidecore!
\newenvironment{slidecore}[1]{
  \slidefont\incpage
  \addcontentsline{toc}{section}{#1}
  \vfill
  {\headerfont #1} \vspace*{-2ex}
  \begin{itemize}\item[]~\\
}{
  \end{itemize}
  \vfill
}

\newcommand{\titleslide}[4][Marc Toussaint]{
  \newcommand{\slideauthor}{#1}
  \newcommand{\slidevenue}{#3}
  \slidefont
  \incpage
  \begin{frame}
  \begin{center}
    \vspace*{15mm}

    {\headerfont #2\\}
        
    \vspace*{10mm}

    #1 \\

    \vspace*{5mm}

    {\small
      Learning \& Intelligent Systems Lab, TU Berlin\\
%      Science of Intelligence Cluster of Excellence, Berlin\\
      Max Planck Fellow, Institute for Intelligent Systems\\ %Physical Reasoning \& Manipulation Lab -- 
%      MIT CSAIL\\
%      mtoussai@mit.edu,~ marc.toussaint@informatik.uni-stuttgart.de

      \vspace*{10mm}

      \emph{#3}
    }

    \vspace*{0mm}

    %\includegraphics[scale=.1]{pics/eushield-fullcolour}

  \end{center}
  \begin{itemize}\item[]~\\
    #4
  \end{itemize}
  \end{frame}
}

\newcommand{\titleslideempty}[2]{
  \slidefont
  \incpage
  \begin{frame}
  \begin{center}
    \vspace*{15mm}

    {\headerfont #1\\}
        
    %% \vspace*{5mm}

    %% {\small\emph{#2}} \\

  \end{center}
  \begin{itemize}\item[]~\\
    #2
  \end{itemize}
  \end{frame}
}

\providecommand{\key}[1]{
  \addtocounter{mypage}{1}
% \immediate\write\keyfile{#1}
  \addtocontents{toc}{\hyperref[key:#1]{#1 (\arabic{mypage})}}
%  \phantomsection\label{key:#1}
%  \index{#1@{\hyperref[key:#1]{#1 (\arabic{mysec}:\arabic{mypage})}}|phantom}
  \addtocounter{mypage}{-1}
}

\providecommand{\course}{}

\providecommand{\subtopic}{}

\providecommand{\sublecture}[2]{
  \renewcommand{\subtopic}{#1}
  \slide{#1}{#2}
}

\providecommand{\sublectureHide}[2]{
  \renewcommand{\subtopic}{#1}
}

\providecommand{\story}[1]{
~

Motivation: {\tiny #1}\clearpage
}

\newenvironment{items}[1][9]{
\par\setlength{\unitlength}{1pt}\fontsize{#1}{#1}\linespread{1.2}\selectfont
\begin{list}{--}{\leftmargin4ex \rightmargin0ex \labelsep1ex \labelwidth2ex
\topsep.7ex \parsep0ex \itemsep3pt}
}{
\end{list}
}

\newenvironment{itemS}[1][4ex]{
\par
\tiny
\begin{list}{--}{\leftmargin#1 \rightmargin0ex \labelsep1ex
  \labelwidth2ex \topsep0pt \parsep0ex \itemsep2pt}
}{
\end{list}
}

\providecommand{\slidesfoot}{
  \end{document}
}


  \slideshead
  %\slidestitle
}

\providecommand{\exercises}{
  \input{../latex/style-exercises}
  \exerciseshead
}

\providecommand{\script}{
  \newcommand{\scripthead}{
  \documentclass[9pt,fleqn,twoside]{article}
  \stdpackages

  \usepackage{makeidx}
  \makeindex

  \usepackage{thmtools}
  \definecolor{shadecolor}{gray}{0.85}
  \declaretheoremstyle[
    %headfont=\normalfont\bfseries,
    %notefont=\mdseries, notebraces={(}{)},
    %bodyfont=\normalfont,
    %postheadspace=0.5em,
    %spaceabove=6pt,
    mdframed={
      %  skipabove=8pt,
      %  skipbelow=6pt,
      hidealllines=true,
      backgroundcolor={shadecolor},
      innertopmargin=8pt,
      %  innerleftmargin=3pt,
      %  innerrightmargin=3pt
    }
  ]{shaded}
  \declaretheorem[style=shaded,within=section,name=Definition]{myDefinition}
  \declaretheorem[style=shaded,within=section,name=Theorem]{myTheorem}
  \declaretheorem[style=shaded,within=section,name=Identities]{Identities}
  \declaretheorem[style=shaded,within=section,name=Example]{myExample}

  \definecolor{grey}{rgb}{.8,.8,.8}
  \definecolor{bluecol}{rgb}{0,0,.5}
  \definecolor{greencol}{rgb}{0,.4,0}
  \definecolor{shadecolor}{gray}{0.9}
  \definecolor{citcol}{rgb}{.4,.4,.4}
  \usepackage[
    %    pdftex%,
    %%    letterpaper,
    %bookmarks,
    bookmarksnumbered,
    colorlinks,
    urlcolor=bluecol,
    citecolor=black,
    linkcolor=bluecol,
    %    pagecolor=bluecol,
    pdfborder={0 0 0},
    %pdfborderstyle={/S/U/W 1},
    %%    backref,     %link from bibliography back to sections
    %%    pagebackref, %link from bibliography back to pages
    %%    pdfstartview=FitH, %fitwidth instead of fit window
    pdfpagemode=UseOutlines, %bookmarks are displayed by acrobat
    pdftitle={\course},
    pdfauthor={Marc Toussaint},
    pdfkeywords={}
  ]{hyperref}
  \DeclareGraphicsExtensions{.pdf,.png,.jpg,.eps}

  %\usepackage{multirow}
  \usepackage{multimedia}
  %\usepackage{marginnote}
  %\setbeamercolor{background canvas}{bg=}

  \usepackage[round]{natbib}
  \bibliographystyle{abbrvnat}

  \renewcommand{\r}{\varrho}
  \renewcommand{\l}{\lambda}
  \renewcommand{\L}{\Lambda}
  \renewcommand{\b}{\beta}
  \renewcommand{\d}{\delta}
  \renewcommand{\k}{\kappa}
  \renewcommand{\t}{\theta}
  \renewcommand{\O}{\Omega}
  \renewcommand{\o}{\omega}
  \renewcommand{\SS}{{\cal S}}
  \renewcommand{\=}{\!=\!}
  %\renewcommand{\boldsymbol}{}
  %\renewcommand{\Chapter}{\chapter}
  %\renewcommand{\Subsection}{\subsection}

  \renewcommand{\baselinestretch}{1.0}
  \geometry{a5paper,headsep=6mm,hdivide={10mm,*,10mm},vdivide={13mm,*,7mm}}

  \fancyhead[OL,ER]{\course, \textit{Marc Toussaint}}
  \fancyhead[OR,EL]{\thepage}
  \fancyhead[C]{}
  \fancyfoot{}
  \pagestyle{fancy}

  \renewcommand{\labelenumi}{{(\roman{enumi})}}
  \renewcommand{\theenumi}{(\roman{enumi})} %for ref
  \parindent 0pt
  \parskip .5pc

  \columnsep 6ex

  \renewcommand{\familydefault}{\sfdefault}

  \newcommand{\headerfont}{\large}%helvetica{12}{1}{b}{n}}
  \newcommand{\slidefont} {}%\helvetica{9}{1.3}{m}{n}}
  \newcommand{\storyfont} {}
  %  \renewcommand{\small}   {\helvetica{8}{1.2}{m}{n}}
  \renewcommand{\tiny}    {\footnotesize}%helvetica{7}{1.1}{m}{n}}
  \newcommand{\ttiny} {\footnotesize}%fontsize{7}{7}\selectfont}
%  \newcommand{\codefont}{\fontsize{6}{6}\selectfont}%helvetica{8}{1.2}{m}{n}}
  \newcommand{\codefont} {\helvetica{8}{1.2}{m}{n}}

  \input{../latex/macros}

  \usepackage{comment}
  \specialcomment{solution}{
    \small
    \begin{shaded}
  }{
    \end{shaded}
  }

  \graphicspath{{pics/}{../pics/}{pics-local/}}

  \mytitle{\course\\Lecture Script}
  \myauthor{Marc Toussaint}
  \date{\coursedate}
}

%%%%%%%%%%%%%%%%%%%%%%%%%%%%%%%%%%%%%%%%%%%%%%%%%%%%%%%%%%%%%%%%%%%%%%%%%%%%%%%%

\newcommand{\scripttitle}{
  \begin{document}
  \maketitle
  %\anchor{100,10}{\includegraphics[width=4cm]{optim}}
%  \vspace*{1cm}
}

%%%%%%%%%%%%%%%%%%%%%%%%%%%%%%%%%%%%%%%%%%%%%%%%%%%%%%%%%%%%%%%%%%%%%%%%%%%%%%%%

\newcounter{mypage}
\setcounter{mypage}{0}
\newcommand{\incpage}{\addtocounter{mypage}{1}}

\newcommand{\subtopic}{}
\newcommand{\pause}{}
\newcommand{\only}[1]{#1}

\renewcommand{\slides}[1][]{
  %  \clearpage
  \subsection{\topic}
  \index{\topic}
  {\small #1}
  \setcounter{mypage}{0}
  \smallskip\nopagebreak\hrule\medskip
}

\newcommand{\slidesfoot}{
  \bigskip
}

\newcommand{\sublecture}[2]{
  \phantomsection\addcontentsline{toc}{subsubsection}{#1}
  \index{#1}
}

\newcommand{\sublectureHide}[2]{
  \renewcommand{\subtopic}{#1}
}

\newcommand{\key}[1]{
  \phantomsection\addcontentsline{toc}{subsubsection}{#1}
  %\subsubsection{#1}
  \index{#1}
}

\providecommand{\defn}[1]{%
  \textbf{#1}\index{#1}%
}

\newenvironment{slidecore}[1]{
  \incpage
  \subsubsection*{#1}%{\headerfont\noindent\textbf{#1}\\}%
  \vspace{-6ex}%
  \begin{list}{$\bullet$}{\leftmargin4ex \rightmargin0ex \labelsep1ex
    \labelwidth2ex \partopsep0ex \topsep0ex \parsep.5ex \parskip0ex \itemsep0pt}\item[]~\\\nopagebreak%
}{
  \end{list}\nopagebreak%
  {\hfill\tiny \textsf{\arabic{section}.\arabic{subsection}:\arabic{mypage}}}\nopagebreak%
  \smallskip\nopagebreak\hrule
}

\newcommand{\slide}[2]{
  \begin{slidecore}{#1}
    #2
  \end{slidecore}
}

\renewcommand{\exercises}{}
\newcommand{\exercisestitle}{}
\newcommand{\exsection}[1]{\subsubsection{#1}}
\newcommand{\exsubsection}[1]{\paragraph{#1}}
\newcommand{\exerfoot}{\bigskip}

\newcommand{\story}[1]{
  \subsection*{Motivation \& Outline}
  {\storyfont\sf #1}
  \medskip\nopagebreak\hrule
}

\newcounter{savedsection}
\newcommand{\subappendix}{\setcounter{savedsection}{\arabic{section}}\appendix}
\newcommand{\noappendix}{
  \setcounter{section}{\arabic{savedsection}}% restore section number
  \setcounter{subsection}{0}% reset section counter
%  \gdef\@chapapp{\sectionname}% reset section name
  \renewcommand{\thesection}{\arabic{section}}% make section numbers arabic
}

\newenvironment{items}[1][9]{
\par\setlength{\unitlength}{1pt}\fontsize{#1}{#1}\linespread{1.2}\selectfont
\begin{list}{--}{\leftmargin4ex \rightmargin0ex \labelsep1ex \labelwidth2ex
\topsep0pt \parsep0ex \itemsep3pt}
}{
\end{list}
}

\newenvironment{itemS}[1][4ex]{
\par
\tiny
\begin{list}{--}{\leftmargin#1 \rightmargin0ex \labelsep1ex
  \labelwidth2ex \topsep0pt \parsep0ex \itemsep2pt}
}{
\end{list}
}

\newcommand{\Def}[1]{%
\textbf{#1}\index{#1}}%\marginnote{#1}}

  \scripthead
}

\providecommand{\paper}{
  \input{../latex/style-paper}
  \paperhead
}

\providecommand{\note}[1][9pt]{
  \providecommand{\notehead}[2]{
  \documentclass[#1,fleqn,twoside]{article}
  \stdpackages
  \renewcommand{\labelenumi}{{(\roman{enumi})}}
  \renewcommand{\theenumi}{(\roman{enumi})} %for ref

  \renewcommand{\baselinestretch}{#2}
  \renewcommand{\arraystretch}{1.2}
  \renewcommand{\topfraction}{1}
  \renewcommand{\bottomfraction}{1}
  \renewcommand{\textfraction}{0}
  \columnsep 5ex
  \parindent 3ex
  \parskip 1ex

  % Lists and paragraphs
  \parindent 0pt
  \topsep 4pt plus 1pt minus 2pt
  \partopsep 1pt plus 0.5pt minus 0.5pt
  \itemsep 2pt plus 1pt minus 0.5pt
  \parsep 2pt plus 1pt minus 0.5pt
  \parskip .5pc %add _in_ {thebibliography} environment in *.bbl

  \setcounter{tocdepth}{3}
  \setcounter{secnumdepth}{3}

  \geometry{a4paper,hdivide={25mm,*,25mm},vdivide={25mm,*,25mm}}

  \renewcommand{\headrulewidth}{.0pt}\renewcommand{\footrulewidth}{.0pt}\cfoot{}
  \fancyhead[OL,EC]{\it\theauthor---\today}
  \fancyhead[ER]{\leftmark}
  \fancyhead[OR,EL]{\thepage}
  \fancyfoot[EL,OR]{}
  \setlength{\headsep}{10mm}
  %\fancyhead[OL]{\rightmark}
  %\fancyfoot[EL,OR]{}


  %  \usepackage{palatino}

  \newcommand{\codefont}{\helvetica{8}{1.2}{m}{n}}

  \renewenvironment{abstract}{
    \vspace*{5ex}\begin{list}{}{
      \leftmargin3ex
      \rightmargin3ex
      \topsep-\parskip}\item[]
     \hrule\vspace{1.5ex}{\bf Abstract.~}\small}
    {\vspace{2ex}\hrule\end{list}\vspace{5ex}}
    
  \newenvironment{keyword}
    {\par{\it Keywords:~}}
    {}

  \def\makemytitle{%
    \thispagestyle{empty}
    \begin{list}{}{\leftmargin3ex \rightmargin3ex \topsep0ex \parsep0ex}\item[]
      \begin{center}
        {\fontsize{18}{25}\selectfont{\thetitle\\}}\vspace{5ex}

        {\fontsize{14}{16}\selectfont{\theauthor\\}}\vspace{1ex}

        {\footnotesize{\sl \addressFUB}\\ \emailBerlin}

        {\footnotesize \today}

        \vspace{1ex}
        {\small \published}
      \end{center}
    \end{list}
    \renewcommand{\maketitle}{\chapter{\thetitle}}
  }

  \input{../latex/macros}
  \pdflatex

  \graphicspath{{pics/}{../pics/}{pics-local/}}

  \myauthor{Marc Toussaint}
  \date{\today}
}

%%%%%%%%%%%%%%%%%%%%%%%%%%%%%%%%%%%%%%%%%%%%%%%%%%%%%%%%%%%%%%%%%%%%%%%%%%%%%%%%

\newcommand{\notetitle}{
  \begin{document}
  \thispagestyle{empty}
    
  \maketitle

}

\newenvironment{items}[1][9]{
\par\setlength{\unitlength}{1pt}\fontsize{#1}{#1}\linespread{1.2}\selectfont
\begin{list}{--}{\leftmargin4ex \rightmargin0ex \labelsep1ex \labelwidth2ex
\topsep0pt \parsep0ex \itemsep3pt}
}{
\end{list}
}

  \notehead{#1}{1.1}
}

\providecommand{\course}{NO COURSE}
\providecommand{\coursepicture}{NO PICTURE}
\providecommand{\coursedate}{NO DATE}
\providecommand{\topic}{NO TOPIC}
\providecommand{\keywords}{}
\providecommand{\exnum}{NO NUMBER}
\providecommand{\teacher}{Marc Toussaint}

\providecommand{\stdpackages}{
  \usepackage{amsmath}
  \usepackage{amssymb}
  \usepackage{amsfonts}
  \allowdisplaybreaks
  \usepackage{amsthm}
  \usepackage{eucal}
  \usepackage{graphicx}
%  \usepackage{color}
  \usepackage{geometry}
  \usepackage{framed}
  \usepackage{xcolor}
  \definecolor{shadecolor}{gray}{0.9}
  \setlength{\FrameSep}{3pt}
  \usepackage{fancyvrb}
  \fvset{numbers=none,xleftmargin=5ex,fontsize=\small}

  \usepackage{pdfpages}

  \usepackage{multicol} 
  \usepackage{fancyhdr}
}

\providecommand{\addressUSTT}{
  Machine~Learning~\&~Robotics~lab, U~Stuttgart\\\small
  Universit{\"a}tsstra{\ss}e 38, 70569~Stuttgart, Germany
}

\providecommand{\addressTUB}{
  Learning~\&~Intelligent~Systems~Lab, TU~Berlin\\\small
  Marchstr. 23, 10587 Berlin, Germany
}

\note

\title{Lecture Note:\\ Robot Kinematics \& Dynamics}
\author{Marc Toussaint\\\small Learning \& Intelligent Systems Lab, TU Berlin}

\makeatletter
\renewcommand{\@seccntformat}[1]{}
\makeatother

\newcommand{\pa}{\text{par}}
\newcommand{\SO}{\text{SO}}
\newcommand{\SE}{\text{SE}}
\newcommand{\se}{\text{se}}
\newcommand{\ang}{\text{ang}}

\notetitle

%%%%%%%%%%%%%%%%%%%%%%%%%%%%%%%%%%%%%%%%%%%%%%%%%%%%%%%%%%%%%%%%%%%%%%%%%%%%%%%%

This is meant as essentials on robotic kinematics and
dynamics -- developed as background for the \emph{Robot Learning} course.

\subsection{Articulated Multibody System}

A robot is a multibody
system. Each body has a \textbf{pose} $x_i\in\SE(3)$, an inertia $(m_i, I_i)$
with mass $m_i\in\RRR$ and inertia tensor $I_i \in \RRR^{3\times 3}$
sym.pos.def., and a shape $s_i$ (any shape representation that defines
a pairwise signed-distance $d(s_i, s_j)$ is sufficient).

We assume this multibody system is tree-structured, i.e., every body
is linked to a parent body or the world. Body $i$ has a relative
transformations $Q_i \in \SE(3)$ from its parent (or world)
$\pa(i)$.  Given the tree
structure, we can compute the pose $x_i$ of each body simply by
forward chaining all relative transformations all $Q_j$ from world to
$i$. Some robotic systems might not really be tree-structured --
these will first be represented as a tree
and then additional constraints to describe loops are added.

What is special about robots is that some of the relative
transformations have degrees of freedom (dofs) that are
\emph{articulated} (i.e., are motorized/movable). Let $Q_i$ have 1
dofs $q_i \in \RRR^1$, then $Q_i(q_i)$ is a function of this dof. We
stack all dofs of the whole multibody tree into a vector $q\in\RRR^n$,
which is called \textbf{joint vector}. We will discuss in more depth
the concepts of generalized and minimal coordinates below.


\subsection{Forward Kinematics \& Jacobian}

We have defined $x_i\in\SE(3)$ as the pose of body $i$, and
$q\in\RRR^n$ as the dofs of the full multibody system. The mappings $\phi_i: q
\mapsto x_i$ (for any $i$) is what so-called ``forward kinematics'' is
concerned about. We already explained that $x_i = \phi_i(q)$ can be
computed simply by chaining all relative transformations. We call this
the (full) forward kinematics.\footnote{Very often we are not interested to
really compute all poses $x_i$ of the multibody system, but only the
pose of one relevant body $i$ (esp.\ the so-called endeffector or
manipulator), or only the position $x^\pos_i$ or some rotated vector $x^v_i
= R_i v$ (with $R_i \in \SO(3)$ its orientation) for body
$i$. In robotics, the word forward kinematics is used often to refer
only to compute the endeffector pose, position, or orientation.}

The derivative $J_i(q)=\del_q \phi_i(q) ~ \in \se(3) \otimes \RRR^n$ is
of central importance to later solve constraint problems, or also to
relate joint space velocities $\dot q$ to the velocity $\dot x_i \in
\se(3)$ of the $i$th body. Note that elements $(v,w)\in \se(3) \equiv
\RRR^6$ are 6-vectors composed of the linear velocity $v\in\RRR^3$ and
angular velocity $w\in\RRR^3$ (always in world
coordinates). Therefore, we can write $J_i(q) \in \RRR^{6 \times n}$
as a matrix, called \textbf{Jacobian}, and

$$(v_i,w_i) = J_i(q) \dot q$$

gives us the linear and angular velocity of body $i$ when joints have
velocities $\dot q$. In practice, code typically returns separate positional Jacobian $J_i^\pos \in \RRR^{3\times n}$ and angular
Jacobian $J_i^\ang \in \RRR^{3\times n}$. In fact, the core job of a
kinematics engine is (1) to represent the articulated multibody tree,
(2) to forward compute $\phi_i(q)$, and (3) to compute
$J_i^\pos, J_i^\ang$ for any $i$.

Since we know how to compute $\phi_i(q)$, we could use ``autodiff'' to
also compute the derivative $J_i(q)$. However, the columns of the positional and
angular Jacobians can actually be computed very easily and more efficiently by simple
insight of how the local translational/angular velocity of each
joint dof translates to the translational/angular velocity of body
$i$.\footnote{For instance, if $j$ is a rotational (``hinge'') joint
around axis $e$ in the joint's origin frame $x_{\pa(j)}$, and body
$i$ is downstream at current pose $x_i$, then the $j$th column of
$$J^\ang_i$$ is $a = (R_{\pa(j)} e)$ (rotates about the axis of
$j$ in world coordinates), and the $j$th column of $$J^\pos_i$$ is
$$a \times (x^\pos_i - x^\pos_{\pa(j)})$$ (translates with a lever around
the axis). Similar arguments can be made if $j$ is a translational
(``prismatic'') joint, and a bit more complicated arguments if $j$ is
a ball joint parameterized by a quaternion $q_j \in\RRR^4$. Thinking
of other joint types as compositions of these basic joints, this
covers all cases.
}
The Jacobians are typically sparse for large robotic systems (e.g.,
multi-robot systems): Every column of $J_i\in\RRR^{6\times n}$
describes how dof $j\in\{1,..,n\}$ influences body $i$. This column
will be zero if dof $j$ is not between the world and body $i$ in the
tree.

\subsection{Fundamental Kinematics Concepts}

The word ``kinematics'' in general refers to the mathematical
description of the possible motions of a (potentially constrained) multibody
system or mechanism \emph{without considering the forces}.

For a multibody system, the poses $x_{1:m}$ fully describes the
\emph{configuration} of the system. When $x$ is some (potentially
redundant) description of system state, we generally call its
embedding space the \textbf{configuration space} $\XX$. E.g., for our
multibody system $\XX=\SE(3)^m$ is a generic embedding space. However, not all
configurations are possible, e.g.\ because bodies are linked in a multibody
system, body shapes cannot penetrate, or obstacles block parts of the
configuration space. We can imagine the \textbf{feasible config.\ space}
$\XX_\text{fea}$ as a manifold of feasible configurations, potentially
with disconnected components or holes. (Actually also
trans-dimensionality, where some parts have different dimensionality is
possible, but we neglect this here.) When introducing 
coordinates $q\in\RRR^n$ for the feasible space these are called
\textbf{generalized coordinates}. (This should not be confused with the word
\textbf{canonical coordinates}, which is used for coordinates in the phase
space of a system, and usually denoted $(q,p)$.)

We discussed the manifold of feasible
configurations, however kinematics is about describing feasible
\emph{motions} on that manifold. Therefore, formally, kinematics
describes which $\dot q \in T_q \XX$ (in the tangent space of $\XX$)
are feasible, and therefore which paths of motion on the manifold.

A holonomic constraint is of the form $h(q, t)=0$, with $h$ a $d$-dimensional
function (where $t$ allows a dependence
on absolute time, which is hardly ever relevant in our field). In such a system,
the generalized coordinates were not minimal and provide only an
embedding space for the true, lower-dimensional feasible manifold. The
true \textbf{degrees of freedom} of the system are $p=n-d$. \textbf{Minimal
  coordinates} are defined to be generalized coordinates of dimension
$n=p$ (where no further holonomic constraints
exist).\footnote{Sometimes it may be convenient to stick with
non-minimal generalized coordinates: When the feasible manifold is
$S^1$, it might be convenient to use redundant $q=(x,y)$ coordinates
and add the constraint $x^2+y^2=1$ rather than
introducing an angle coordinate and running into the annoying modulo issue. Analogous
for $\SO(3)$ and embedding quaternion coordinates $\RRR^4$ may be more
convenient than minimal coordinates for $\SO(3)$. Further, when we
have a closed loop robotic system, it is convenient to use non-minimal
joint coordinates along a tree and profit from the efficiency of tree-based
kinematics engines, and handle the closed loop constraint
otherwise.}

A non-holonomic constraint is of the form $h(q,\dot q,t)=0$, which is
a general description of any constraints on
possible motions on the feasible manifold. A typical example is a car
or wheel in 2D: We describe the configuration naturally with $x =
(p,\varphi)$ with 2D position $p\in\RRR^2$ and heading angle
$\varphi\in\RRR$. Note that (without obstacles) any configuration is
feasible, so the full configuration space is feasible. As there is no
better alternative, we choose generalized coordinates equally as $q =
x$. However, at any $q$, the positional velocity is constraint to
aligned with the heading direction, $\dot p^\T (\sin\p, \cos\p) = 0$,
which is a non-holonomic constraint.

There are cases where a constraint is naturally expressed as $h(q,\dot
q,t)=0$ and ``looks'' non-holonomic, but actually it is so-called
integratable. This means that, by analyzing integrals of trajectories of
feasible velocities $\dot q(t)$, we understand that the actually
reachable $q$ all lie on a sub-manifold which we could more directly
be described by a holonomic constraint $h(q, t)=0$ (here, the absolute time
$t$ dependence really is important). So, such ``integratable
non-holonomic constraints'' can be reformulated to become holonomic
still describe a holonomic system. The literature describes elaborate
maths (Pfaffian form of constraints) to uniquely decide whether a system is truly
holonomic or non-holonomic. But this is rarely relevant for typical
robotic systems in our field.

\subsection{``Force Kinematics''}

We learned that with 
$(v_i,w_i) = J_i \dot q$ the Jacobian relates joint velocities to body
velocities. Let's do the analogous for forces: Assume we have a wrench
$(f_i,\tau_i)$ directly acting on body $i$, what ``do we feel'' in the
joints, i.e., what torques $u\in\RRR^n$ are propagated into the
joints? The answer is the Jacobian transpose: $u = J^\T
(f_i,\tau_i)$.\footnote{We can derive this by conserved work, or better,
power (=work/time): For joint torques $u$ and velocities $\dot q$ we
would consume power $u^\T \dot q$, which needs to equal the power
received at the body, $(f_i,\tau_i)^\T (v_i,w_i) = (f_i,\tau_i)^\T J
\dot q$. As this holds for any $\dot q$ we have $u^\T = (f_i,\tau_i)^\T J$.}


\subsection{Dynamics}

While kinematics describes which $q$ and $\dot
q$ are feasible, dynamics describes which $\ddot q$ are feasible. For
a passive dynamical system there is just one $\ddot q$ feasible: the
one that follows from the laws of physics. For an articulated robotic
system we can choose to exert torques $u$ in each joint (or some joints), and
thereby create various $\ddot q$. For standard, fully actuated robotic systems
we can command torques $u\in\RRR^n$ in all dofs and create arbitrary
accelerations. However, when our multibody system description includes
passive objects of the environment, or describes a free-floating
(walking/running) robot, the accelerations that can be generated for
all dofs (objects, free-floating body) are highly constrained,
depending esp.\ on contacts and possibilities of force transmission
to objects or the ground through contacts.

Assume we know how we want to accelerate $\ddot q$ the system, and
want to compute the necessary joint torques $u$ to achieve this acceleration. That
is, we want to derive the mapping $(q,\dot
q, \ddot q) \mapsto u$. Given the
Jacobians described above, this is easy to derive:
For each body, we can compute
$(v_i,w_i) = J_i \dot q$ and $(\dot v_i, \dot w_i) = J_i \ddot
q$. I.e., we know how we want each body to accelerate. The
Newton-Euler equation tells us such an acceleration would raise the
following inertia forces at the body:

\begin{align*}
\mat{c}{f_i \\ \tau_i}
&= \mat{c}{m_i \dot v_i \\\bar I_i \dot w_i + w_i\times \bar I_i w_i}
 = M_i J_i \ddot q + c_i ~,
\end{align*}

with $M_i = \diag(m \Id_3, \bar I_i)$, $c_i = (0_3, w_i\times \bar I_i
w_i)$, where $\bar I_i = R_i I_i R_i^\T$ and $I_i$ the inertia tensor in
body coordinates.

Conversely, to counteract these inertia forces we have to apply joint torques
$u = J_i^\T \[M_i J_i \ddot q + c_i\]$ -- this is how ``we feel'' the
inertial in the joints. We can separately consider
gravity: By the same argument we need joint torques $u = J_i^\T g_i$
to counteract gravity, where $g_i = (-m g e_z, 0_3)\in\RRR^6$ has only a
single entry in $-z$ direction. As we have many bodies that are accelerated and need
gravity compensation, we overall have

$$u = \sum_{i=1}^m J_i^\T \[M_i J_i \ddot q + c_i + g_i\] ~.$$

Other texts provide much more lengthy derivations either via the general
Euler-Lagrange equations, or recursive Newton-Euler equations. I find
the above very concise and easy to implement, and efficient with
sparse $J_i$. The first term $\sum_{i=1}^m J_i^\T M_i J_i \ddot q$ can elegantly
also be found in the Euler-Lagrange derivation,%
\footnote{The Euler-Lagrange derivation starts with
$\frac{d}{dt} \frac{\del L}{\del \dot q} - \frac{\del L}{\del q} = u$,
where $L(q,\dot q) = T(q,\dot q) - U(q)$ with the system kinetic energy
$$T(q,\dot q) = \sum_i \half m_i v_i^2 + \half w_i^\T \bar I_i w_i
= \sum_i \half \dot q^\T J_i^\T M_i J_i \dot q,~
M_i = \diag(m_i\Id_3, \bar I_i), $$
and the system potential energy $U(q) = \sum_i g m_i x_i^\text{z}$. When computing the partial derivatives analytically we get something
of the form
$$
u = \frac{d}{dt} \frac{\del L}{\del \dot q} - \frac{\del L}{\del q} 
 = M(q) \ddot q + \dot M \dot q - \frac{\del T}{\del q} + \frac{\del U}{\del q},
$$
where total inertial $M(q) = \sum_i J_i^\T M_i J_i$ is simple to
compute, but the
Coriolis terms are more complicated.
}
%
but the Coriolis terms
$c_i$ are less obvious. Recursive Newton-Euler can be tuned to be
numerically faster, but does not provide this nice general and
invertible form.

In standard notation, general robot dynamics are written in the form

$$u = M(q)~ \ddot q + F(q, \dot q)$$

where, for multi-rigid-body systems, we derived $M(q) = \sum_i J_i^\T M_i
J_i$ and $F(q,\dot q) = \sum_i J_i^\T (c_i + g_i)$.

Keep in mind that only when $M(q)$ is invertible we have a one-to-one
relation between our controls and the system acceleration, and we have
the guarantee that our system accelerates with $\ddot q = M(q)^{-1} u -
F(q,\dot q)$ as desired. $M(q)$ is not invertible if $J_i$ do not have
full rank, e.g., if some body $i$ is not articulated at all and $J_i$
is zero. In this case the equation $u = J_i^\T \[M_i J_i \ddot q +
c_i\]$ says that we feel none of the accelerations of $i$ in our
joints -- and conversely cannot induce any accelerations of
$i$. That's the case when the robot is not in contact with body $i$.


%% That wrench has to come from connections to other bodies --
%% transmitted through the joints. We have the equation
%% \begin{align*}
%% F_i
%% &= F^\text{ext}_i + F^\text{joint}_i
%% - \sum_{j=\text{child(i)}} F^\text{joint}_j ~,
%% \end{align*}
%% where $F^\text{ext}_i$ are potential external forces (e.g.\ gravity, contacts) 
%% directly applying to $i$, and $F^\text{joint}_i$ is the force that the
%% parent $\pa(i)$ applies to $i$ through its joint. This equation is
%% recursive in all $F^\text{joint}_i$, and can easily be resolved
%% starting from leaf bodies. Once we have all joint wrenches, projecting
%% them to the joints' axes gives the torques $u_i = h_i^\T
%% F^\text{joint}_i$ (where $h_i\in\RRR^6$ is the ``joint axis'' in world
%% coordinates, generalized to include prismatic and/or rotational joints to
%% read out forces or torques depending on type).

%% Rather that implementing this recursion directly, we can also setup a
%% sparse linear equation system, that provides more options to be used:
%% We have $\ddot q, u\in\RRR^n$, matrices $F, F^e, F^j\in\RRR^{m\times
%%   6}$, and the equation systems
%% \begin{align}
%%  F = M J \ddot q + c\comma F = F^e + T F^j \comma u=H F^j ~,
%% \end{align}
%% where $M\in\RRR^{m \times 6 \times 6}$ contains $m_i \Id$ and $I_i$,
%% $c$ contains terms $w_i\times I_i w_i$, $F^e$ is defined externally,
%% and $T\in\RRR^{6m\times 6m}$ is very sparse, with $+1$ on the
%% diagonal, and some $-1$ in off-diagonals corresponding to child
%% relations in the tree. Putting this together we get
%% \begin{align}
%% u=H T^{-1}\[ M J \ddot q + c - F^e\] ~,
%% \end{align}
%% and we explicitly established the linear relation between $u$ and
%% $\ddot q$. Since matrices sparse and $T$ tree structured, inversion
%% and all multiplication operations are $O(m)$. Note, however, that all
%% quantities above are in world coordinates! This includes all inertia
%% matricies $I_i \get R_i I_i {R_i}^{-1}$, and joint axes $h_i$,
%% and $F^e$ includes implied torques relative to the world origin.


\subsection{Standard Usage: Waypoint + Reference Motion + Controller}


With the above equations we can accelerate the system in any way we
like -- at least those dofs that are currently articulable. In this
view, the rest is planning: We need to decide how we want to
accelerate the system right now in order to reach some future goal.

There are a myriad of opinions on how robotic control systems and middleware
should be structured. Here is just one version, which I consider a baseline.

Consider that we want to have the robot fulfill a kinematic constraint
$$\phi(q_{t=T}) = y^*$$ at time $t=T$, where $\phi$ is a $d$-dimensional
constraint function that typically depends on some poses $x_i$ of some bodies,
and $y^*\in\RRR^d$ is called a setpoint. A standard robotic system
could address this problem as follows:
\begin{enumerate}
  \item First compute a final robot pose $q_T$ that fulfills constraint 
    $\phi(q_{t=T}) = y^*$ -- that problem is called \textbf{inverse
    kinematics} and discussed below.
  \item Next compute a \emph{reference} motion from current robot pose
    $q_0$ to $q_T$ -- that problem can be addressed with \textbf{path
    finding}, \textbf{trajectory optimization}, or basic interpolation
    with a motion profile.
  \item Finally, determine a control policy $\pi: (x,t) \mapsto u$
    that reactively computes motor commands $u$ to follow the
    reference motion -- that problem can be addressed using \textbf{PD
    control}, inverse dynamics as derived above, \textbf{Riccati},
    or \textbf{model-predictive control (MPC)}.
\end{enumerate}
You could think of these as three different time scales: First rough
future waypoint(s)/goal(s), then continuous motion to next waypoint,
then short-term controls. Continuous replanning/re-estimation can also make (1) and (2) reactive.

Of course, robotics systems do not have to be organized in that
way: Some approaches skip step (1) and let step (2) also solve for the
final configuration (e.g., including the optimization of $q_T$ into
the trajectory optimization problem); or one may skip steps (1) and
(2) and let step (3) handle the full problem (e.g., including the goal
constraint in the MPC formulation, or a basic task-space PD controller
(``operational space control''); which typically looses the power of
path finding and optimization).


\subsection{Inverse Kinematics}

Inverse kinematics (IK) simply means computing $q$ to fulfill $\phi(q) =
y^*$. A proper approach is to formulate this as an NLP (non-linear
mathematical program)

\begin{align}
&\min_{q\in\RRR^n} \norm{q-q_0}^2 \st \phi(q) = y^* \label{eqIK}\\
\text{or}\quad&\min_{q\in\RRR^n} \norm{q-q_0}^2 + \m \norm{\phi(q) -
y^*}^2 \quad\text{for large $\m$} 
\end{align}

and use an efficient NLP solver (e.g.\ Augmented Lagrangian, or SQP,
exploiting potential sparseness of $\Del q \phi$). However, typical
textbooks at length discuss computing IK more low-level. For instance,
when approximating $\phi(q) \approx \phi(q_0) + J (q-q_0)$ as linear
with Jacobian $J$, the analytical solution to the above can be written
as (allowing for $\mu\to\infty$):

\begin{align*}
q^*
&= q_0 +  J^\T (J J^\T + \textstyle\frac{1}{\mu} \Id)^{-1} (y^*-\phi(q_0)) ~.
\end{align*}

In the context of optimization, this is the first Newton step when
initializing optimization at $q_0$. Students sometimes interpret this
equation as a tool to directly generate robot motion: They let the
robot literally execute these Newton steps (scaled by a small factor
$\a$), and then the robot starts moving like the decision variable in
a non-linear optimization problem with small-scaled Newton steps. This
is not proper! Proper IK should really first compute the solution
$q_T$ to
\eqref{eqIK}, and then think about how the robot can actually move
to $q_T$ (e.g.\ using proper optimal control, or reactive spline interpolation, or
a basic but nice motion profile).




\end{document}
