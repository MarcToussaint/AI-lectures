\providecommand{\slides}{
  \newcommand{\slideshead}{
  \newcommand{\thepage}{\arabic{mypage}}
  %beamer
%  \documentclass[t,hyperref={bookmarks=true}]{beamer}
%  \geometry{papersize={171mm,96mm}}
  \documentclass[t,hyperref={bookmarks=true},aspectratio=169]{beamer}
  \setbeamersize{text margin left=5mm}
  \setbeamersize{text margin right=5mm}
  \usetheme{default}
  \usefonttheme[onlymath]{serif}
  \setbeamertemplate{navigation symbols}{}
  \setbeamertemplate{itemize items}{{\color{black}$\bullet$}}

  \newwrite\keyfile

  %\usepackage{palatino}
  \stdpackages
  %\usepackage{tikz} \usetikzlibrary {shapes.geometric} 
  \usepackage{multimedia}
  \usepackage[utf8]{inputenc}

  %%% geometry/spacing issues
  %
  \definecolor{bluecol}{rgb}{0,0,.5}
  \definecolor{greencol}{rgb}{0,.6,0}
  \definecolor{citcol}{rgb}{.4,.4,.4}
  %\renewcommand{\baselinestretch}{1.1}
  \renewcommand{\arraystretch}{1.2}
  \columnsep 0mm

  \columnseprule 0pt
  \parindent 0ex
  \parskip 0ex
  %\setlength{\itemparsep}{3ex}
  %\renewcommand{\labelitemi}{\rule[3pt]{10pt}{10pt}~}
  %\renewcommand{\labelenumi}{\textbf{(\arabic{enumi})}}
  \setbeamertemplate{enumerate item}{(\roman{enumi})}
  \newcommand{\headerfont}{\helvetica{14}{1.5}{b}{n}}
  \newcommand{\slidefont} {\helvetica{11}{1.4}{m}{n}}
  %\newcommand{\codefont} {\helvetica{8}{1.2}{m}{n}}
  \newcommand{\urlfont} {\fontsize{6}{1.1}\selectfont}
  \renewcommand{\small} {\helvetica{10}{1.4}{m}{n}}
  \renewcommand{\tiny} {\helvetica{8}{1.3}{m}{n}}
  \newcommand{\ttiny} {\helvetica{7}{1.3}{m}{n}}

  %%% count pages properly and put the page number in bottom right
  %
  \newcounter{mypage}
  \newcommand{\incpage}{\addtocounter{mypage}{1}\setcounter{page}{\arabic{mypage}}}
  \setcounter{mypage}{0}
  \resetcounteronoverlays{page}

  \pagestyle{fancy}
  %\setlength{\headsep}{10mm}
  %\addtolength{\footheight}{15mm}
  \renewcommand{\headrulewidth}{0pt} %1pt}
  \renewcommand{\footrulewidth}{0pt} %.5pt}
  \cfoot{}
  \rhead{}
  \lhead{}
  %% \lfoot{\vspace*{-3mm}\hspace*{-3mm}\helvetica{5}{1.3}{m}{n}{\texttt{github.com/MarcToussaint/AI-lectures}}}
\lfoot{}
%  \rfoot{{\tiny\textsf{AI -- \topic -- \subtopic -- \arabic{mypage}/\pageref{lastpage}}}}
  %\lfoot{\raisebox{5mm}{\tiny\textsf{\slideauthor}}}
  %\rfoot{\raisebox{5mm}{\tiny\textsf{\slidevenue{} -- \arabic{mypage}/\pageref{lastpage}}}}
  %\rfoot{~\anchor{30,12}{\tiny\textsf{\thepage/\pageref{lastpage}}}}
  %\lfoot{\small\textsf{Marc Toussaint}}
%  \rfoot{\vspace*{-4.5mm}{\tiny\textsf{\color{gray}\topic\ -- \subtopic\ -- \arabic{mypage}/\pageref{lastpage}}}\hspace*{-4mm}}
  \rfoot{\vspace*{-4.5mm}{\tiny\textsf{\color{gray}\topic\ -- \arabic{mypage}/\pageref{lastpage}}}\hspace*{-4mm}}
%  \rfoot{~\anchor{-10,12}{\tiny\textsf{\color{gray}\topic\ --  \arabic{mypage}/\pageref{lastpage}}}}
  \lfoot{\vspace*{-4.5mm}{\hspace*{-3mm}\includegraphics[height=4mm]{LIS-logo-longText}}}

  \definecolor{grey}{rgb}{.8,.8,.8}
  \definecolor{head}{rgb}{.85,.9,.9}
%  \definecolor{blue}{rgb}{.0,.0,.5}
%  \definecolor{green}{rgb}{.0,.5,.0}
  \definecolor{red}{rgb}{.8,.0,.0}
  \newcommand{\inverted}{
    \definecolor{main}{rgb}{1,1,1}
    \color{main}
    \pagecolor[rgb]{.3,.3,.3}
  }
  \input{../latex/macros}

  \graphicspath{{pics/}{../pics/}{pics-local/}}
}

\newcommand{\slidestitle}{
  \title{\course \topic}
  \author{Marc Toussaint}
  \institute{Learning \& Intelligent Systems Lab, TU Berlin}

  \begin{document}


  %% title slide!
  \slide{}{
    \thispagestyle{empty}

    \twocol{.35}{.55}{
      %\vspace*{-5mm}%
      \hspace*{-5mm}%
      \includegraphics[width=1.\columnwidth]{\coursepicture}
    }{\center

      \textbf{\fontsize{17}{20}\selectfont \course}

      ~

      %Lecture
      \topic\\

      \vspace{1cm}

      {\tiny~\emph{\keywords}~\\}

      \vspace{1cm}

      \teacher
      
      Technical University of Berlin

      \coursedate

      ~

    }
  }
}

\newcommand{\slide}[2]{
  \slidefont
  \incpage\begin{frame}
  \addcontentsline{toc}{section}{#1}
  \vfill
  {\headerfont #1} \vspace*{-2ex}
  \begin{itemize}\item[]~\\
    #2
  \end{itemize}
  \vfill
  \end{frame}
}

% use \begin{frame}[fragile] around slidecore!
\newenvironment{slidecore}[1]{
  \slidefont\incpage
  \addcontentsline{toc}{section}{#1}
  \vfill
  {\headerfont #1} \vspace*{-2ex}
  \begin{itemize}\item[]~\\
}{
  \end{itemize}
  \vfill
}

\newcommand{\titleslide}[4][Marc Toussaint]{
  \newcommand{\slideauthor}{#1}
  \newcommand{\slidevenue}{#3}
  \slidefont
  \incpage
  \begin{frame}
  \begin{center}
    \vspace*{15mm}

    {\headerfont #2\\}
        
    \vspace*{10mm}

    #1 \\

    \vspace*{5mm}

    {\small
      Learning \& Intelligent Systems Lab, TU Berlin\\
%      Science of Intelligence Cluster of Excellence, Berlin\\
      Max Planck Fellow, Institute for Intelligent Systems\\ %Physical Reasoning \& Manipulation Lab -- 
%      MIT CSAIL\\
%      mtoussai@mit.edu,~ marc.toussaint@informatik.uni-stuttgart.de

      \vspace*{10mm}

      \emph{#3}
    }

    \vspace*{0mm}

    %\includegraphics[scale=.1]{pics/eushield-fullcolour}

  \end{center}
  \begin{itemize}\item[]~\\
    #4
  \end{itemize}
  \end{frame}
}

\newcommand{\titleslideempty}[2]{
  \slidefont
  \incpage
  \begin{frame}
  \begin{center}
    \vspace*{15mm}

    {\headerfont #1\\}
        
    %% \vspace*{5mm}

    %% {\small\emph{#2}} \\

  \end{center}
  \begin{itemize}\item[]~\\
    #2
  \end{itemize}
  \end{frame}
}

\providecommand{\key}[1]{
  \addtocounter{mypage}{1}
% \immediate\write\keyfile{#1}
  \addtocontents{toc}{\hyperref[key:#1]{#1 (\arabic{mypage})}}
%  \phantomsection\label{key:#1}
%  \index{#1@{\hyperref[key:#1]{#1 (\arabic{mysec}:\arabic{mypage})}}|phantom}
  \addtocounter{mypage}{-1}
}

\providecommand{\course}{}

\providecommand{\subtopic}{}

\providecommand{\sublecture}[2]{
  \renewcommand{\subtopic}{#1}
  \slide{#1}{#2}
}

\providecommand{\sublectureHide}[2]{
  \renewcommand{\subtopic}{#1}
}

\providecommand{\story}[1]{
~

Motivation: {\tiny #1}\clearpage
}

\newenvironment{items}[1][9]{
\par\setlength{\unitlength}{1pt}\fontsize{#1}{#1}\linespread{1.2}\selectfont
\begin{list}{--}{\leftmargin4ex \rightmargin0ex \labelsep1ex \labelwidth2ex
\topsep.7ex \parsep0ex \itemsep3pt}
}{
\end{list}
}

\newenvironment{itemS}[1][4ex]{
\par
\tiny
\begin{list}{--}{\leftmargin#1 \rightmargin0ex \labelsep1ex
  \labelwidth2ex \topsep0pt \parsep0ex \itemsep2pt}
}{
\end{list}
}

\providecommand{\slidesfoot}{
  \end{document}
}


  \slideshead
  %\slidestitle
}

\providecommand{\exercises}{
  \input{../latex/style-exercises}
  \exerciseshead
}

\providecommand{\script}{
  \newcommand{\scripthead}{
  \documentclass[9pt,fleqn,twoside]{article}
  \stdpackages

  \usepackage{makeidx}
  \makeindex

  \usepackage{thmtools}
  \definecolor{shadecolor}{gray}{0.85}
  \declaretheoremstyle[
    %headfont=\normalfont\bfseries,
    %notefont=\mdseries, notebraces={(}{)},
    %bodyfont=\normalfont,
    %postheadspace=0.5em,
    %spaceabove=6pt,
    mdframed={
      %  skipabove=8pt,
      %  skipbelow=6pt,
      hidealllines=true,
      backgroundcolor={shadecolor},
      innertopmargin=8pt,
      %  innerleftmargin=3pt,
      %  innerrightmargin=3pt
    }
  ]{shaded}
  \declaretheorem[style=shaded,within=section,name=Definition]{myDefinition}
  \declaretheorem[style=shaded,within=section,name=Theorem]{myTheorem}
  \declaretheorem[style=shaded,within=section,name=Identities]{Identities}
  \declaretheorem[style=shaded,within=section,name=Example]{myExample}

  \definecolor{grey}{rgb}{.8,.8,.8}
  \definecolor{bluecol}{rgb}{0,0,.5}
  \definecolor{greencol}{rgb}{0,.4,0}
  \definecolor{shadecolor}{gray}{0.9}
  \definecolor{citcol}{rgb}{.4,.4,.4}
  \usepackage[
    %    pdftex%,
    %%    letterpaper,
    %bookmarks,
    bookmarksnumbered,
    colorlinks,
    urlcolor=bluecol,
    citecolor=black,
    linkcolor=bluecol,
    %    pagecolor=bluecol,
    pdfborder={0 0 0},
    %pdfborderstyle={/S/U/W 1},
    %%    backref,     %link from bibliography back to sections
    %%    pagebackref, %link from bibliography back to pages
    %%    pdfstartview=FitH, %fitwidth instead of fit window
    pdfpagemode=UseOutlines, %bookmarks are displayed by acrobat
    pdftitle={\course},
    pdfauthor={Marc Toussaint},
    pdfkeywords={}
  ]{hyperref}
  \DeclareGraphicsExtensions{.pdf,.png,.jpg,.eps}

  %\usepackage{multirow}
  \usepackage{multimedia}
  %\usepackage{marginnote}
  %\setbeamercolor{background canvas}{bg=}

  \usepackage[round]{natbib}
  \bibliographystyle{abbrvnat}

  \renewcommand{\r}{\varrho}
  \renewcommand{\l}{\lambda}
  \renewcommand{\L}{\Lambda}
  \renewcommand{\b}{\beta}
  \renewcommand{\d}{\delta}
  \renewcommand{\k}{\kappa}
  \renewcommand{\t}{\theta}
  \renewcommand{\O}{\Omega}
  \renewcommand{\o}{\omega}
  \renewcommand{\SS}{{\cal S}}
  \renewcommand{\=}{\!=\!}
  %\renewcommand{\boldsymbol}{}
  %\renewcommand{\Chapter}{\chapter}
  %\renewcommand{\Subsection}{\subsection}

  \renewcommand{\baselinestretch}{1.0}
  \geometry{a5paper,headsep=6mm,hdivide={10mm,*,10mm},vdivide={13mm,*,7mm}}

  \fancyhead[OL,ER]{\course, \textit{Marc Toussaint}}
  \fancyhead[OR,EL]{\thepage}
  \fancyhead[C]{}
  \fancyfoot{}
  \pagestyle{fancy}

  \renewcommand{\labelenumi}{{(\roman{enumi})}}
  \renewcommand{\theenumi}{(\roman{enumi})} %for ref
  \parindent 0pt
  \parskip .5pc

  \columnsep 6ex

  \renewcommand{\familydefault}{\sfdefault}

  \newcommand{\headerfont}{\large}%helvetica{12}{1}{b}{n}}
  \newcommand{\slidefont} {}%\helvetica{9}{1.3}{m}{n}}
  \newcommand{\storyfont} {}
  %  \renewcommand{\small}   {\helvetica{8}{1.2}{m}{n}}
  \renewcommand{\tiny}    {\footnotesize}%helvetica{7}{1.1}{m}{n}}
  \newcommand{\ttiny} {\footnotesize}%fontsize{7}{7}\selectfont}
%  \newcommand{\codefont}{\fontsize{6}{6}\selectfont}%helvetica{8}{1.2}{m}{n}}
  \newcommand{\codefont} {\helvetica{8}{1.2}{m}{n}}

  \input{../latex/macros}

  \usepackage{comment}
  \specialcomment{solution}{
    \small
    \begin{shaded}
  }{
    \end{shaded}
  }

  \graphicspath{{pics/}{../pics/}{pics-local/}}

  \mytitle{\course\\Lecture Script}
  \myauthor{Marc Toussaint}
  \date{\coursedate}
}

%%%%%%%%%%%%%%%%%%%%%%%%%%%%%%%%%%%%%%%%%%%%%%%%%%%%%%%%%%%%%%%%%%%%%%%%%%%%%%%%

\newcommand{\scripttitle}{
  \begin{document}
  \maketitle
  %\anchor{100,10}{\includegraphics[width=4cm]{optim}}
%  \vspace*{1cm}
}

%%%%%%%%%%%%%%%%%%%%%%%%%%%%%%%%%%%%%%%%%%%%%%%%%%%%%%%%%%%%%%%%%%%%%%%%%%%%%%%%

\newcounter{mypage}
\setcounter{mypage}{0}
\newcommand{\incpage}{\addtocounter{mypage}{1}}

\newcommand{\subtopic}{}
\newcommand{\pause}{}
\newcommand{\only}[1]{#1}

\renewcommand{\slides}[1][]{
  %  \clearpage
  \subsection{\topic}
  \index{\topic}
  {\small #1}
  \setcounter{mypage}{0}
  \smallskip\nopagebreak\hrule\medskip
}

\newcommand{\slidesfoot}{
  \bigskip
}

\newcommand{\sublecture}[2]{
  \phantomsection\addcontentsline{toc}{subsubsection}{#1}
  \index{#1}
}

\newcommand{\sublectureHide}[2]{
  \renewcommand{\subtopic}{#1}
}

\newcommand{\key}[1]{
  \phantomsection\addcontentsline{toc}{subsubsection}{#1}
  %\subsubsection{#1}
  \index{#1}
}

\providecommand{\defn}[1]{%
  \textbf{#1}\index{#1}%
}

\newenvironment{slidecore}[1]{
  \incpage
  \subsubsection*{#1}%{\headerfont\noindent\textbf{#1}\\}%
  \vspace{-6ex}%
  \begin{list}{$\bullet$}{\leftmargin4ex \rightmargin0ex \labelsep1ex
    \labelwidth2ex \partopsep0ex \topsep0ex \parsep.5ex \parskip0ex \itemsep0pt}\item[]~\\\nopagebreak%
}{
  \end{list}\nopagebreak%
  {\hfill\tiny \textsf{\arabic{section}.\arabic{subsection}:\arabic{mypage}}}\nopagebreak%
  \smallskip\nopagebreak\hrule
}

\newcommand{\slide}[2]{
  \begin{slidecore}{#1}
    #2
  \end{slidecore}
}

\renewcommand{\exercises}{}
\newcommand{\exercisestitle}{}
\newcommand{\exsection}[1]{\subsubsection{#1}}
\newcommand{\exsubsection}[1]{\paragraph{#1}}
\newcommand{\exerfoot}{\bigskip}

\newcommand{\story}[1]{
  \subsection*{Motivation \& Outline}
  {\storyfont\sf #1}
  \medskip\nopagebreak\hrule
}

\newcounter{savedsection}
\newcommand{\subappendix}{\setcounter{savedsection}{\arabic{section}}\appendix}
\newcommand{\noappendix}{
  \setcounter{section}{\arabic{savedsection}}% restore section number
  \setcounter{subsection}{0}% reset section counter
%  \gdef\@chapapp{\sectionname}% reset section name
  \renewcommand{\thesection}{\arabic{section}}% make section numbers arabic
}

\newenvironment{items}[1][9]{
\par\setlength{\unitlength}{1pt}\fontsize{#1}{#1}\linespread{1.2}\selectfont
\begin{list}{--}{\leftmargin4ex \rightmargin0ex \labelsep1ex \labelwidth2ex
\topsep0pt \parsep0ex \itemsep3pt}
}{
\end{list}
}

\newenvironment{itemS}[1][4ex]{
\par
\tiny
\begin{list}{--}{\leftmargin#1 \rightmargin0ex \labelsep1ex
  \labelwidth2ex \topsep0pt \parsep0ex \itemsep2pt}
}{
\end{list}
}

\newcommand{\Def}[1]{%
\textbf{#1}\index{#1}}%\marginnote{#1}}

  \scripthead
}

\providecommand{\paper}{
  \input{../latex/style-paper}
  \paperhead
}

\providecommand{\note}[1][9pt]{
  \providecommand{\notehead}[2]{
  \documentclass[#1,fleqn,twoside]{article}
  \stdpackages
  \renewcommand{\labelenumi}{{(\roman{enumi})}}
  \renewcommand{\theenumi}{(\roman{enumi})} %for ref

  \renewcommand{\baselinestretch}{#2}
  \renewcommand{\arraystretch}{1.2}
  \renewcommand{\topfraction}{1}
  \renewcommand{\bottomfraction}{1}
  \renewcommand{\textfraction}{0}
  \columnsep 5ex
  \parindent 3ex
  \parskip 1ex

  % Lists and paragraphs
  \parindent 0pt
  \topsep 4pt plus 1pt minus 2pt
  \partopsep 1pt plus 0.5pt minus 0.5pt
  \itemsep 2pt plus 1pt minus 0.5pt
  \parsep 2pt plus 1pt minus 0.5pt
  \parskip .5pc %add _in_ {thebibliography} environment in *.bbl

  \setcounter{tocdepth}{3}
  \setcounter{secnumdepth}{3}

  \geometry{a4paper,hdivide={25mm,*,25mm},vdivide={25mm,*,25mm}}

  \renewcommand{\headrulewidth}{.0pt}\renewcommand{\footrulewidth}{.0pt}\cfoot{}
  \fancyhead[OL,EC]{\it\theauthor---\today}
  \fancyhead[ER]{\leftmark}
  \fancyhead[OR,EL]{\thepage}
  \fancyfoot[EL,OR]{}
  \setlength{\headsep}{10mm}
  %\fancyhead[OL]{\rightmark}
  %\fancyfoot[EL,OR]{}


  %  \usepackage{palatino}

  \newcommand{\codefont}{\helvetica{8}{1.2}{m}{n}}

  \renewenvironment{abstract}{
    \vspace*{5ex}\begin{list}{}{
      \leftmargin3ex
      \rightmargin3ex
      \topsep-\parskip}\item[]
     \hrule\vspace{1.5ex}{\bf Abstract.~}\small}
    {\vspace{2ex}\hrule\end{list}\vspace{5ex}}
    
  \newenvironment{keyword}
    {\par{\it Keywords:~}}
    {}

  \def\makemytitle{%
    \thispagestyle{empty}
    \begin{list}{}{\leftmargin3ex \rightmargin3ex \topsep0ex \parsep0ex}\item[]
      \begin{center}
        {\fontsize{18}{25}\selectfont{\thetitle\\}}\vspace{5ex}

        {\fontsize{14}{16}\selectfont{\theauthor\\}}\vspace{1ex}

        {\footnotesize{\sl \addressFUB}\\ \emailBerlin}

        {\footnotesize \today}

        \vspace{1ex}
        {\small \published}
      \end{center}
    \end{list}
    \renewcommand{\maketitle}{\chapter{\thetitle}}
  }

  \input{../latex/macros}
  \pdflatex

  \graphicspath{{pics/}{../pics/}{pics-local/}}

  \myauthor{Marc Toussaint}
  \date{\today}
}

%%%%%%%%%%%%%%%%%%%%%%%%%%%%%%%%%%%%%%%%%%%%%%%%%%%%%%%%%%%%%%%%%%%%%%%%%%%%%%%%

\newcommand{\notetitle}{
  \begin{document}
  \thispagestyle{empty}
    
  \maketitle

}

\newenvironment{items}[1][9]{
\par\setlength{\unitlength}{1pt}\fontsize{#1}{#1}\linespread{1.2}\selectfont
\begin{list}{--}{\leftmargin4ex \rightmargin0ex \labelsep1ex \labelwidth2ex
\topsep0pt \parsep0ex \itemsep3pt}
}{
\end{list}
}

  \notehead{#1}{1.1}
}

\providecommand{\course}{NO COURSE}
\providecommand{\coursepicture}{NO PICTURE}
\providecommand{\coursedate}{NO DATE}
\providecommand{\topic}{NO TOPIC}
\providecommand{\keywords}{}
\providecommand{\exnum}{NO NUMBER}
\providecommand{\teacher}{Marc Toussaint}

\providecommand{\stdpackages}{
  \usepackage{amsmath}
  \usepackage{amssymb}
  \usepackage{amsfonts}
  \allowdisplaybreaks
  \usepackage{amsthm}
  \usepackage{eucal}
  \usepackage{graphicx}
%  \usepackage{color}
  \usepackage{geometry}
  \usepackage{framed}
  \usepackage{xcolor}
  \definecolor{shadecolor}{gray}{0.9}
  \setlength{\FrameSep}{3pt}
  \usepackage{fancyvrb}
  \fvset{numbers=none,xleftmargin=5ex,fontsize=\small}

  \usepackage{pdfpages}

  \usepackage{multicol} 
  \usepackage{fancyhdr}
}

\providecommand{\addressUSTT}{
  Machine~Learning~\&~Robotics~lab, U~Stuttgart\\\small
  Universit{\"a}tsstra{\ss}e 38, 70569~Stuttgart, Germany
}

\providecommand{\addressTUB}{
  Learning~\&~Intelligent~Systems~Lab, TU~Berlin\\\small
  Marchstr. 23, 10587 Berlin, Germany
}

\note

\title{Lecture Note:\\ Quaternions, Exponential Map, and Quaternion Jacobians}
\author{Marc Toussaint\\\small Learning \& Intelligent Systems Lab, TU Berlin}

\renewcommand{\t}{\theta}
\renewcommand{\hat}{\widehat}
\newcommand{\bbg}{{\bar{\bar g}}}
\newcommand{\ul}{\widehat}
\newcommand{\bd}{\boldsymbol}
\newcommand{\SE}{\text{SE}}
\newcommand{\SO}{\text{SO}}
\newcommand{\nor}{\text{normalize}}
\newcommand{\skew}{\text{skew}}

\notetitle

%%%%%%%%%%%%%%%%%%%%%%%%%%%%%%%%%%%%%%%%%%%%%%%%%%%%%%%%%%%%%%%%%%%%%%%%%%%%%%%%

$\SO(2)$ is the space of rotations in the plane $\RRR^2$ (or about a
single axis), which can be
described by an angle $\a\in[0,2\pi]$. But there are good arguments to
instead represent such rotations by a point on $S^1$, which is the
circle in $\RRR^2$. E.g., topologically $\SO(2)$ is equal to $S^1$
and not to the interval $[0,2\pi]$. Points in $S^1$ can also be described as
$e^{i\a} \in\CCC$, which is the circle in the complex
plane. Quaternions similarly represent rotations of $\SO(3)$ as points
on $S^3$, i.e., the sphere in $\RRR^4$. And just as $e^{i\a} \in\CCC$
gives points on the circle $S^1$, the exponential map of $\SO(3)$
gives defines quaternions on $S^3$.

The following will first provide reference equations for quaternions,
but then explain in more detail interpolation between rotations, which
naturally introduces the exponential map directly for quaternions
without need to go deeper into Lie groups. An emphasis is then on
deriving the \emph{quaternion Jacobian}, which describes the
angular velocity that is induced when perturbing a
quaternion parameter.

%%%%%%%%%%%%%%%%%%%%%%%%%%%%%%%%%%%%%%%%%%%%%%%%%%%%%%%%%%%%%%%%%%%%%%%%%%%%%%%%

\section{Reference}

Let's define a quaternion $q\in\RRR^4$ as a 4-vector.\footnote{In the
original meaning, they are defined as $q = q_0 + q_1 i + q_2 j + q_3
k$, with $j,k$ being extensions to the imaginary number $i$. But we
just treat them as 4-vectors.} To represent a rotation, a quaternion
needs to be normalized, i.e., $q\in S^3$. We denote the first entry by
$q_0$ and the last three
entries by $\bar q\equiv q_{1:3} \in\RRR^3$. Let's first summarize
some basic properties for a normalized quaternion:

\begin{align}
\text{rotation angle $\t\in\RRR$:} \quad \label{eqAngle}
& \t = 2 \acos(q_0) = 2\, \text{asin}(|\bar q|) \\
\text{normalized axis $\ul w\in S^2$:} \quad \label{eqAxis}
& \ul w = \Frac{\bar q}{|\bar q|} = \Frac{1}{\sin(\t/2)} \bar q \\
\text{rotation vector $w\in\RRR^3$:} \quad \label{eqLog}
& w = \log(q) = 2 \acos(q_0)~ \Frac{\bar q}{|\bar q|}  \\
\text{quat. from vector $w=\t\ul w$:} \quad \label{eqExp}
& q = \exp(w) = (\cos(\t/2),~ \sin(\t/2)~ \ul w) \\
\text{rotation matrix $R\in\RRR^{3\times 3}$:} \quad \label{eqMatrix}
& R
= \mat{ccc}{
    1-q_{22}-q_{33} & q_{12}-q_{03} &    q_{13}+q_{02} \\
    q_{12}+q_{03} &   1-q_{11}-q_{33} &  q_{23}-q_{01} \\
    q_{13}-q_{02} &   q_{23}+q_{01} &    1-q_{11}-q_{22}
    },~ q_{ij} := 2 q_i q_j \\
\text{quaternion from matrix:} \quad
& q_0 = \Frac12 \sqrt{1+\tr R},~
  q_1 = \Frac{R_{32}-R_{23}}{4 q_0},~
  q_2 = \Frac{R_{13}-R_{31}}{4 q_0},~
  q_3 = \Frac{R_{21}-R_{12}}{4 q_0}  \\
\text{quaternion inverse:} \quad
& q^{-1} = (q_0,-\bar q) \quad \text{or} \quad  q^{-1} = (-q_0, \bar q) \\
\text{quaternion concatenation:} \quad &
q \circ q'
 = (q_0 q'_0 - \bar q^\T \bar q',~
    q_0 \bar q' + q'_0 \bar q + \bar q \times \bar q') \\
\text{vector application, $x\in\RRR^3$:} \quad \label{eqApp}
& q \cdot x = (q \circ (0,x) \circ q^{-1})_{1:3} \quad\text{(or via matrix)}
\end{align}

Eqs.~\eqref{eqLog} and \eqref{eqExp} follow from \eqref{eqAngle}
and \eqref{eqAxis}, but why they are defined as 'log' and 'exp' will become clear below. Concerning the application of a rotation quaternion $q$ on a vector
$x$, equation \eqref{eqApp} is elegant, but computationally less
efficient than first converting to matrix using \eqref{eqMatrix} and
then multiplying to the vector. Note that concatenating quaternions as
well as applying a quaternion to a vector never requires to compute
a $\sin$ or $\cos$, as opposed, e.g. to the Rodriguez' equation
(see below).

\section{Continuously moving from $I$ to $q$ -- exponential and log mappings}

Given a quaternion $q$, how can we continuously move from the $\Id=(1,\bar
0)$ to $q$? The answer is simply given by continuously increasing the
rotation angle \eqref{eqAngle} from $0$ to $\t$. Let $t\in[0,\t]$ be
the interpolation coefficient, we have

\begin{align}
q(t)
&= (\cos(t/2),~ \sin(t/2)~ \ul w) ~. \label{eqExp0}
\end{align}

This ``motion'' from $\Id$ to $q$ is a geodesic (=shortest path) on
$S^3$ (w.r.t. to the natural embedded Euclidean metric of $\RRR^4$),
and it has constant absolute velocity:

\begin{align}\label{eqVel}
\dot q(t)
&= (-\sin(t/2),~ \cos(t/2)~ \ul w)\comma
|\dot q(t)| = \Frac12 ~.
\end{align}

At the origin (for $t=0$), the velocity is
$\dot q(0) = \Frac12 (0, \ul w)$, which shows
 how the rotation axis $\ul w$ provides the tangent vector at
$\Id$. For $t>0$ the velocity \eqref{eqVel}
``rotates along the tangent arc'' on $S^3$. We can show that $\dot q(t)$
can also be written as (using that $\ul w$ and $\bar q(t)$ are colinear):

\begin{align}
\Frac12 (0, \ul w) \circ q(t)
&= \Frac12 (0 q_0(t) - \ul w^\T \bar q(t),~ 0 \bar q(t) + q_0(t) \ul w + \ul w \times \bar q(t)) \\
&= \Frac12 (0 - \sin(t/2),~ \cos(t/2)~ \ul w + 0) ~=~ \dot q(t) ~.  \label{eqDotq}
\end{align}

Eq.~\eqref{eqDotq} is a differential equation of the form $\dot q(t)
= \a q(t)$, but with the multiplication being the group operation $\circ$. The solution
$q(t)$ to this differential equation is defined as the exponential map.\footnote{When describing $\SO(3)$ as a Lie groups,
analogous equations appear when tangent motion is integrated to
continuously move to group elements. The solutionis defined as
$\exp(\cdot)$ of a tangent vector.} Namely, for a vector $w=t\ul w \in
\RRR^3$ we define

\begin{align}
\exp(t \ul w)
&= (\cos(t/2),~ \sin(t/2)~ \ul w) \comma
\To~ \Del t \exp(t \ul w)
 = \Frac12 (0, \ul w) \circ \exp(t \ul w) ~,
\end{align}

fully consistent to \eqref{eqExp0} and \eqref{eqDotq}. Conversely, given a quaternion, we define the log as

\begin{align}
\log(q)
&= 2 \acos(q_0)~ \Frac{\bar q}{|\bar q|} ~.
\end{align}

\section{Continuously moving from $q_A$ to $q_B$ -- interpolation in quaternion space}

Consider two quaternions $q_A, q_B$. When concatenating $q_B = q_{BA} \circ q_A$, you should think of $q_{BA}$ as
describing an ``additional'' rotation in the output coordinates of
$q_A$. This is exactly as multiplying matrices $R_B = R_{BA} R_A$, where $R_{BA}$
describes a rotation from orientation $R_A$ to orientation $R_B$.
The relative rotation $q_B / q_A \equiv q_B \circ q_A^{-1} = q_{BA}$
retrieves exactly this relative rotation from $q_A$ to $q_B$.

There are two ways to interpolate between orientations $q_A$ and
$q_B$. Let $t\in[0,1]$. The \emph{proper} interpolation uses the exponential
map:

\begin{align}
q(t) = \exp(t\log(q_B/q_A)) \circ q_A ~. \label{eqInter}
\end{align}

This first computes the relative transform $q_{BA} = q_B/q_A$,
computes its tangent vector representation $\log(q_{BA})$, and interpolates
from $\Id$ to $q_{BA}$ as we described above. Finally it multiplies
$q_A$ from the right, making all this relative to orientation
$q_A$. For completeness, the velocity is

\begin{align}
\dot q(t)
&= \[\Del t \exp(t\log(q_B/q_A))\] \circ q_A \\
&= \[\Frac12 (0, w_{BA}) \circ \exp(t\log(q_B/q_A))\] \circ q_A \\
&= \Frac12 (0, w_{BA}) \circ q(t) ~. \label{eqInterVel}
\end{align}

As an alternative, we can define the \emph{embedded} interpolation as

\begin{align}
q(t) = \nor((1-t)~ q_A + t~ q_B) ~, \label{eqLinInter}
\end{align}

where $\nor(q) = q / |q|$. This assumes that the two
quaternions are \emph{sign aligned}, i.e. $q_A^\T q_B \ge 0$. This
interpolation seems native. However, it describes the exact
same \emph{path} as the proper interpolation: along the geodesic arc
on $S^3$ from $q_A$ to $q_B$. The only difference  is that it does not have constant absolute velocity on
$S^3$. The non-normalized interpolation $(1-t) q_A + t q_B$ in
$\RRR^4$ has constant velocity, but the projection to $S^3$ modulates
velocities depending on the angle of the tangent space to the
embedded interpolation. But note that, as $q_A^\T q_B \ge 0$, the
angle between $q_A$ and $q_B$ in $\RRR^4$ is at most 90 degrees; and
therefore the angle between the linear interpolation and the tangent
spaces is at most 45 degrees -- in this worst case, the velocities at start
and end are by a factor $1/\sqrt{2}$ larger than in the
middle.

The point here is that interpolation of rotations is very
intuitive: ``straight'' interpolation on the sphere $S^3$ which can
equally described using \eqref{eqLinInter} (with the exponential
map) or the naive embedded interpolation \eqref{eqLinInter}. But only the proper
interpolation with the exponential map has constant absolute velocity.

\section{Angular Jacobian w.r.t. Quaternion Parameters}

Assume we have a quaternion $q\in\RRR^4$ as a \emph{parameter} of some
optimization problem and need a Jacobian w.r.t. $q$. More concretely,
we want to know the angular
velocity $w = J \dot q$ that is induced by a velocity (or infinitesimal
variation) of $q$. To relate to Eq.~\eqref{eqInterVel} we think of
$q=q_A$ as the base orientation, and $\dot q$ as moving from $q_A$ to
$q_B$. Eq.~\eqref{eqInterVel} for $t=0$ becomes

\begin{align}
\dot q &= \Frac12 (0, w) \circ q ~.
\end{align}

This equation holds if $\dot q$ is tangential to $S^3$, and translates
$\dot q$ to an angular vector $w\in\RRR^3$ \emph{relative} to the
orientation $q$ (just how $w_{BA}$ was the angular vector relative to
the base orientation $q_A$). Under these tangential assumptions, we
can directly read out $w$ by multiplying $q^{-1}$ from the right,

\begin{align}
\dot q \circ q^{-1}
&= \half (0,w) \comma w = 2~ [\dot q \circ q^{-1}]_{1:3} ~.
\end{align}

However, in the case where $\dot q$ is non-tangential, i.e., $\dot q^\T q \not=0$,
the change in length of the quaternion does not represent any angular
velocity. One approach is to first tangentialize
$\dot {\ul q} = \dot q - q q^\T \dot q$.
However, when pluging in the tangentialized
$\dot{\ul q}$ we get

\begin{align}
w
&= 2~ [(\dot q - (q^\T \dot q) q)  \circ q^{-1}]_{1:3} \\
&= 2~ [\dot q \circ q^{-1}]_{1:3} - 2~ (q^\T \dot q) [q \circ q^{-1}]_{1:3} ~,
\end{align}

and the latter term is identically zero. So, our ``tangentialization term'' drops out anyway.

To construct the Jacobian, let's assume $\dot q=e_i$ is one of the unit vectors
$e_0,..,e_3\in\RRR^4$. Then we have

\begin{align}
w_i
&= 2~ [e_i\circ q^{-1}]_{1:3} ~.
\end{align}

Further, if $q$ itself was not normalized we want the angular Jacobian
of $\ul q = \Frac1{|q|} q$ w.r.t. $q$. As the above relation between $q$
and $w$ is linear, we have

\begin{align}
w_i
&= \Frac2{|q|}~ [e_i \circ \ul q^{-1}]_{1:3} ~.
\end{align}

Based on this we construct the \textbf{angular quaternion Jacobian}
with 4 columns

\begin{align}
J_{\cdot i}(q) = \Frac2{|q|}~ [e_i \circ \ul q^{-1}]_{1:3} ~,
\end{align}

so that we have $w = J(q)~ \dot q$ for any (also non-normalized, non-tangential) $q$
and $\dot q$.

Again, note that this angular vector $w$ is to be
interpreted \emph{relative} to $q$, in the output space of $q$. To
translate this to a world coordinate angular Jacobian, we need to
transform by $q$ again. To give a more complete example: if we have a
pose $q_B = \nor(q) \circ q_A$ that is parameterized by arbitrary (non-zero)
$q\in\RRR^4$, then $\dot
q$ translates to a world coordinate angular velocity $w = q_B \cdot
J(q) \dot q$ of the pose of $B$.


\section{Random rotations \& ``Gaussians''}

To get a random rotation, we want to uniformly sample from $S^3$. This
is particularly easy by first sampling $q\sim\NNN(0,1)$ in $\RRR^4$
and then normalize.

To sample a rotation ``Gaussian around a mean'' quaternion $q$, we
have two options (akin to the interpolation options): The tangent
space sampling:

\begin{align}
q' = \exp(w) \circ q\comma w = \NN(0,\s^2) \in \RRR^3 ~,
\end{align}

or

\begin{align}
q' = \text{normalize}(q + \d/2)\comma \d = \NN(0,\s^2) \in \RRR^4 ~.
\end{align}

In the limit of small $\s^2$ these become the same, as the Gaussian
projected to the tangent space is the same as the Gaussian in the
tangent space.

\appendix
\section{Similarly: Rotation vector and angular velocity}

To recap, a rotation vector is $w\in\RRR^3$, with rotation axis $\ul w
= \frac{w}{|w|}$, and rotation angle $\t = |w|$ (in
radians, using the right thumb convention).

The application of a rotation described by $w\in\RRR^3$ on a vector
$x\in\RRR^3$ is given as (Rodrigues' formula)

\begin{align}
w \cdot x
 &= \cos(\t)~ x
  + \sin(\t)~ (\ul w\times x)
  + (1-\cos(\t))~ \ul w(\ul w^\T x) ~.
\end{align}

This directly also implies convertion to a rotation matrix, as the
rows of the corresponding rotation matrix are simply $w \cdot e_i$ for
the unit vectors $e_{1:3}\in\RRR^3$. To simplify the notation, we
define the skew matrix of a vector $w\in\RRR^3$ as

\begin{align}
\skew(w) = \mat{ccc}{0 & -w_3 & w_2 \\ w_3 & 0 & -w_1 \\ -w_2 & w_1 & 0} ~.
\end{align}

This allows to express the cross
product as matrix multiplication, $w \times v = \skew(w)~ v$. The rotation matrix $R(w)$ 
corresponding to a given rotation vector $w$ then is:

\begin{align}\label{eqRodriguez}
R(w)
 &= \exp(\skew(w)) \\
 &= \cos(\t)~ \Id + \sin(\t)/\t~ \skew(w) + (1-\cos(\t))/\t^2~ w w^\T
\end{align}

The $\exp$ function is called exponential map (generating a group
element (=rotation matrix) via an element of the Lie algebra (=skew
matrix)). The other formular is called Rodrigues' formular: the first
term is a diagonal matrix ($\Id$ is the 3D identity matrix), the second
terms the skew symmetric part, the last term the symmetric part ($w
w^\T$ is also called outper product).


\textbf{Angular velocity \& derivative of a rotation matrix:} We
represent angular velocities by a vector $w\in\RRR^3$, with rotation axis
$\ul w$, and rotation velocity $|w|$ (in radians per second). When a body's orientation at time
$t$ is described by a rotation matrix $R(t)$ and the body's angular
velocity is $w$, then

\begin{align}\label{eqDotR}
\dot R(t) = \skew(w)~ R(t)~.
\end{align}

(That's intuitive to see for a rotation about the $x$-axis with
velocity 1.) Some insights from this relation: Since $R(t)$ must
always be a rotation matrix (fulfill orthogonality and determinant 1),
its derivative $\dot R(t)$ must also fulfill certain constraints; in
particular it can only live in a 3-dimensional sub-space. It turns out
that the derivative $\dot R$ of a rotation matrix $R$ must always be a
skew symmetric matrix $\skew(w)$ times $R$ -- anything else would be
inconsistent with the contraints of orthogonality and determinant 1.

Note also that, assuming $R(0)=I$, the solution to the differential
equation $\dot R(t) = \skew(w)~ R(t)$ can be written as
$R(t)=\exp(t~ \skew(w))$, where here the exponential function notation
is used to denote a more general so-called exponential map, as used in
the context of Lie groups. It also follows that $R(w)$ from
\eqref{eqRodriguez} is the rotation matrix you get when you rotate for
1 second with angular velocity described by $w$.

\end{document}
