\providecommand{\slides}{
  \newcommand{\slideshead}{
  \newcommand{\thepage}{\arabic{mypage}}
  %beamer
%  \documentclass[t,hyperref={bookmarks=true}]{beamer}
%  \geometry{papersize={171mm,96mm}}
  \documentclass[t,hyperref={bookmarks=true},aspectratio=169]{beamer}
  \setbeamersize{text margin left=5mm}
  \setbeamersize{text margin right=5mm}
  \usetheme{default}
  \usefonttheme[onlymath]{serif}
  \setbeamertemplate{navigation symbols}{}
  \setbeamertemplate{itemize items}{{\color{black}$\bullet$}}

  \newwrite\keyfile

  %\usepackage{palatino}
  \stdpackages
  %\usepackage{tikz} \usetikzlibrary {shapes.geometric} 
  \usepackage{multimedia}
  \usepackage[utf8]{inputenc}

  %%% geometry/spacing issues
  %
  \definecolor{bluecol}{rgb}{0,0,.5}
  \definecolor{greencol}{rgb}{0,.6,0}
  \definecolor{citcol}{rgb}{.4,.4,.4}
  %\renewcommand{\baselinestretch}{1.1}
  \renewcommand{\arraystretch}{1.2}
  \columnsep 0mm

  \columnseprule 0pt
  \parindent 0ex
  \parskip 0ex
  %\setlength{\itemparsep}{3ex}
  %\renewcommand{\labelitemi}{\rule[3pt]{10pt}{10pt}~}
  %\renewcommand{\labelenumi}{\textbf{(\arabic{enumi})}}
  \setbeamertemplate{enumerate item}{(\roman{enumi})}
  \newcommand{\headerfont}{\helvetica{14}{1.5}{b}{n}}
  \newcommand{\slidefont} {\helvetica{11}{1.4}{m}{n}}
  %\newcommand{\codefont} {\helvetica{8}{1.2}{m}{n}}
  \renewcommand{\small} {\helvetica{10}{1.4}{m}{n}}
  \renewcommand{\tiny} {\helvetica{8}{1.3}{m}{n}}
  \newcommand{\ttiny} {\helvetica{7}{1.3}{m}{n}}

  %%% count pages properly and put the page number in bottom right
  %
  \newcounter{mypage}
  \newcommand{\incpage}{\addtocounter{mypage}{1}\setcounter{page}{\arabic{mypage}}}
  \setcounter{mypage}{0}
  \resetcounteronoverlays{page}

  \pagestyle{fancy}
  %\setlength{\headsep}{10mm}
  %\addtolength{\footheight}{15mm}
  \renewcommand{\headrulewidth}{0pt} %1pt}
  \renewcommand{\footrulewidth}{0pt} %.5pt}
  \cfoot{}
  \rhead{}
  \lhead{}
  %% \lfoot{\vspace*{-3mm}\hspace*{-3mm}\helvetica{5}{1.3}{m}{n}{\texttt{github.com/MarcToussaint/AI-lectures}}}
\lfoot{}
%  \rfoot{{\tiny\textsf{AI -- \topic -- \subtopic -- \arabic{mypage}/\pageref{lastpage}}}}
  %\lfoot{\raisebox{5mm}{\tiny\textsf{\slideauthor}}}
  %\rfoot{\raisebox{5mm}{\tiny\textsf{\slidevenue{} -- \arabic{mypage}/\pageref{lastpage}}}}
  %\rfoot{~\anchor{30,12}{\tiny\textsf{\thepage/\pageref{lastpage}}}}
  %\lfoot{\small\textsf{Marc Toussaint}}
  \rfoot{\vspace*{-4.5mm}{\tiny\textsf{\color{gray}\topic\ -- \subtopic\ -- \arabic{mypage}/\pageref{lastpage}}}\hspace*{-4mm}}
%  \rfoot{~\anchor{-10,12}{\tiny\textsf{\color{gray}\topic\ --  \arabic{mypage}/\pageref{lastpage}}}}
  \lfoot{\vspace*{-4.5mm}{\hspace*{-3mm}\includegraphics[height=4mm]{LIS-logo-longText}}}

  \definecolor{grey}{rgb}{.8,.8,.8}
  \definecolor{head}{rgb}{.85,.9,.9}
%  \definecolor{blue}{rgb}{.0,.0,.5}
%  \definecolor{green}{rgb}{.0,.5,.0}
  \definecolor{red}{rgb}{.8,.0,.0}
  \newcommand{\inverted}{
    \definecolor{main}{rgb}{1,1,1}
    \color{main}
    \pagecolor[rgb]{.3,.3,.3}
  }
  %auto-ignore
  \renewcommand{\a}{\alpha}
  \renewcommand{\b}{\beta}
  \renewcommand{\d}{\delta}
    \newcommand{\D}{\Delta}
    \newcommand{\e}{\epsilon}
    \newcommand{\g}{\gamma}
    \newcommand{\G}{\Gamma}
  \renewcommand{\l}{\lambda}
  \renewcommand{\L}{\Lambda}
    \newcommand{\m}{\mu}
    \newcommand{\n}{\nu}
    \newcommand{\N}{\nabla}
  \renewcommand{\k}{\kappa}
  \renewcommand{\o}{\omega}
  \renewcommand{\O}{\Omega}
    \newcommand{\p}{\phi}
    \newcommand{\ph}{\varphi}
  \renewcommand{\P}{\Phi}
  \renewcommand{\r}{\varrho}
    \newcommand{\s}{\sigma}
  \renewcommand{\S}{\Sigma}
  \renewcommand{\t}{\theta}
    \newcommand{\Th}{\Theta}
  %\renewcommand{\v}{\vartheta}
    \newcommand{\x}{\xi}
    \newcommand{\X}{\Xi}
    \newcommand{\Y}{\Upsilon}
    \newcommand{\z}{\zeta}

  \renewcommand{\AA}{{\cal A}}
    \newcommand{\BB}{{\cal B}}
    \newcommand{\CC}{{\cal C}}
    \newcommand{\cc}{{\cal c}}
    \newcommand{\DD}{{\cal D}}
    \newcommand{\EE}{{\cal E}}
    \newcommand{\FF}{{\cal F}}
    \newcommand{\GG}{{\cal G}}
    \newcommand{\HH}{{\cal H}}
    \newcommand{\II}{{\cal I}}
    \newcommand{\KK}{{\cal K}}
    \newcommand{\LL}{{\cal L}}
    \newcommand{\MM}{{\cal M}}
    \newcommand{\NN}{{\cal N}}
    \newcommand{\oNN}{\overline\NN}
    \newcommand{\OO}{{\cal O}}
    \newcommand{\PP}{{\cal P}}
    \newcommand{\QQ}{{\cal Q}}
    \newcommand{\RR}{{\cal R}}
  \renewcommand{\SS}{{\cal S}}
    \newcommand{\TT}{{\cal T}}
    \newcommand{\uu}{{\cal u}}
    \newcommand{\UU}{{\cal U}}
    \newcommand{\VV}{{\cal V}}
    \newcommand{\XX}{{\cal X}}
    \newcommand{\xx}{\mathcal{x}}
    \newcommand{\YY}{{\cal Y}}
    \newcommand{\ZZ}{{\cal Z}}
    \newcommand{\SOSO}{{\cal SO}}
    \newcommand{\GLGL}{{\cal GL}}

    \newcommand{\Ee}{{\rm E}}

  \newcommand{\NNN}{{\mathbb{N}}}
  \newcommand{\III}{{\mathbb{I}}}
  \newcommand{\ZZZ}{{\mathbb{Z}}}
  %\newcommand{\RRR}{{\mathrm{I\!R}}}
  \newcommand{\RRR}{{\mathbb{R}}}
  \newcommand{\SSS}{{\mathbb{S}}}
  \newcommand{\EEE}{{\mathbb{E}}}
  \newcommand{\VVV}{{\mathbb{V}}}
  \newcommand{\CCC}{{\mathbb{C}}}
  \newcommand{\DDD}{{\mathbb{D}}}
  \newcommand{\one}{{{\bf 1}}}
  \newcommand{\eee}{\text{e}}

  \newcommand{\NNNN}{{\overline{\cal N}}}

  \renewcommand{\[}{\Big[}
  \renewcommand{\]}{\Big]}
  \renewcommand{\(}{\Big(}
  \renewcommand{\)}{\Big)}
  \renewcommand{\|}{\,|\,}
  \renewcommand{\;}{\,;\,}
  \renewcommand{\=}{\!=\!}
    \newcommand{\<}{\left\langle}
  \renewcommand{\>}{\right\rangle}

  \newcommand{\na}[1][]{{\nabla_{\!\!#1}}}
  \newcommand{\he}[1][]{{\nabla_{\!\!#1}^2}}
  \newcommand{\Prob}{{\rm Prob}}
  \newcommand{\Dir}{{\rm Dir}}
  \newcommand{\Beta}{{\rm Beta}}
  \newcommand{\Bern}{{\rm Bern}}
  \newcommand{\Bin}{{\rm Bin}}
  \newcommand{\Mult}{{\rm Mult}}
  \newcommand{\Aut}{{\rm Aut}}
  \newcommand{\cor}{{\rm cor}}
  \newcommand{\corr}{{\rm corr}}
  \newcommand{\sd}{{\rm sd}}
  \newcommand{\tr}{{\rm tr}}
  \newcommand{\Tr}{{\rm Tr}}
  \newcommand{\rank}{{\rm rank}}
  \newcommand{\diag}{{\rm diag}}
  \newcommand{\dom}{{\rm dom}}
  \newcommand{\id}{{\rm id}}
  \newcommand{\Id}{{\rm\bf I}}
  \newcommand{\Gl}{{\rm Gl}}
  \renewcommand{\th}{\ensuremath{{}^\text{th}} }
  \newcommand{\lag}{\mathcal{L}}
  \newcommand{\inn}{\rfloor}
  \newcommand{\lie}{\pounds}
  \newcommand{\longto}{\longrightarrow}
  \newcommand{\speer}{\parbox{0.4ex}{\raisebox{0.8ex}{$\nearrow$}}}
  \renewcommand{\dag}{ {}^\dagger }
  \newcommand{\blbox}{\rule{1ex}{1ex}}
  \newcommand{\Ji}{J^\sharp}
  \newcommand{\h}{{}^\star}
  \newcommand{\w}{\wedge}
  \newcommand{\too}{\longrightarrow}
  \newcommand{\oot}{\longleftarrow}
  \newcommand{\To}{\Rightarrow}
  \newcommand{\oT}{\Leftarrow}
  \newcommand{\oTo}{\Leftrightarrow}
  \renewcommand{\iff}{~\Leftrightarrow~}
  \newcommand{\Too}{\;\Longrightarrow\;}
  \newcommand{\oto}{\leftrightarrow}
  \newcommand{\ot}{\leftarrow}
  \newcommand{\ootoo}{\longleftrightarrow}
  \newcommand{\ow}{\stackrel{\circ}\wedge}
  \newcommand{\defeq}{\stackrel{\hspace{0.2ex}{}_\Delta}=}
%  \newcommand{\defeq}{{\overstack\Delta =}}
  \newcommand{\feed}{\nonumber \\}
  \newcommand{\comma}{~,\quad}
  \newcommand{\period}{~.\quad}
  \newcommand{\del}{\partial}
  \newcommand{\Del}[1]{\textstyle\frac{\del}{\del #1}}
  \newcommand{\Hes}[1]{\textstyle\frac{\del^2}{\del #1^2}}

%  \newcommand{\quabla}{\Delta}
  \newcommand{\point}{$\bullet~~$}
  \newcommand{\doubletilde}{ ~ \raisebox{0.3ex}{$\widetilde {}$} \raisebox{0.6ex}{$\widetilde {}$} \!\! }
  \newcommand{\topcirc}{\parbox{0ex}{~\raisebox{2.5ex}{${}^\circ$}}}
  \newcommand{\topdot} {\parbox{0ex}{~\raisebox{2.5ex}{$\cdot$}}}
  \newcommand{\topddot} {\parbox{0ex}{~\raisebox{1.3ex}{$\ddot{~}$}}}
  \newcommand{\sym}{\topcirc}
  \newcommand{\tsum}{\textstyle\sum}
  \newcommand{\st}{~~\text{s.t.}~~}

  \newcommand{\half}{\ensuremath{\frac{1}{2}}}
  \newcommand{\third}{\ensuremath{\frac{1}{3}}}
  \newcommand{\fourth}{\ensuremath{\frac{1}{4}}}

  \newcommand{\ubar}{\underline}
  %\renewcommand{\vec}{\underline}
  \renewcommand{\vec}{\boldsymbol}
  %\renewcommand{\_}{\underset}
  %\renewcommand{\^}{\overset}
  %\renewcommand{\*}{{\rm\raisebox{-.6ex}{\text{*}}{}}}
  \renewcommand{\*}{\text{\footnotesize\raisebox{-.4ex}{*}{}}}

  \newcommand{\gto}{{\raisebox{.5ex}{${}_\rightarrow$}}}
  \newcommand{\gfrom}{{\raisebox{.5ex}{${}_\leftarrow$}}}
  \newcommand{\gnto}{{\raisebox{.5ex}{${}_\nrightarrow$}}}
  \newcommand{\gnfrom}{{\raisebox{.5ex}{${}_\nleftarrow$}}}

  %\newcommand{\RND}{{\SS}}
  %\newcommand{\IF}{\text{if }}
  %\newcommand{\AND}{\textsc{and }}
  %\newcommand{\OR}{\textsc{or }}
  %\newcommand{\XOR}{\textsc{xor }}
  %\newcommand{\NOT}{\textsc{not }}

  %\newcommand{\argmax}[1]{{\rm arg}\!\max_{#1}}
  %\newcommand{\argmin}[1]{{\rm arg}\!\min_{#1}}
  \DeclareMathOperator*{\argmax}{argmax}
  \DeclareMathOperator*{\argmin}{argmin}
  \DeclareMathOperator{\sign}{sign}
  \DeclareMathOperator{\acos}{acos}
  \DeclareMathOperator{\unifies}{unifies}
  \DeclareMathOperator{\Span}{span}
  \newcommand{\ortho}{\perp}
  %\newcommand{\argmax}[1]{\underset{~#1}{\text{argmax}}\;}
  %\newcommand{\argmin}[1]{\underset{~#1}{\text{argmin}}\;}
  \newcommand{\ee}[1]{\ensuremath{\cdot10^{#1}}}
  \newcommand{\sub}[1]{\ensuremath{_{\text{#1}}}}
  \newcommand{\up}[1]{\ensuremath{^{\text{#1}}}}
  \newcommand{\kld}[3][{}]{D_{#1}\big(#2\,\big|\!\big|\,#3\big)}
  %\newcommand{\kld}[2]{D\big(#1:#2\big)}
  \newcommand{\sprod}[2]{\big<#1\,,\,#2\big>}
  \newcommand{\End}{\text{End}}
  \newcommand{\txt}[1]{\quad\text{#1}\quad}
  \newcommand{\Over}[2]{\genfrac{}{}{0pt}{0}{#1}{#2}}
  %\newcommand{\mat}[1]{{\bf #1}}
  \newcommand{\arr}[2]{\hspace*{-.5ex}\begin{array}{#1}#2\end{array}\hspace*{-.5ex}}
  \newcommand{\mat}[3][.9]{
    \renewcommand{\arraystretch}{#1}{\scriptscriptstyle{\left(
      \hspace*{-1ex}\begin{array}{#2}#3\end{array}\hspace*{-1ex}
    \right)}}\renewcommand{\arraystretch}{1.2}
  }
  \newcommand{\Mat}[3][.9]{
    \renewcommand{\arraystretch}{#1}{\scriptscriptstyle{\left[
      \hspace*{-1ex}\begin{array}{#2}#3\end{array}\hspace*{-1ex}
    \right]}}\renewcommand{\arraystretch}{1.2}
  }
  \newcommand{\case}[2][ll]{\left\{\arr{#1}{#2}\right.}
  \newcommand{\seq}[1]{\textsf{\<#1\>}}
  \newcommand{\seqq}[1]{\textsf{#1}}
  \newcommand{\floor}[1]{\lfloor#1\rfloor}
  \newcommand{\Exp}[2][]{\mathbb{E}_{#1}\!\left\{#2\right\}}
  \newcommand{\Expno}[1][]{\mathbb{E}_{#1}}
  \newcommand{\Var}[2][]{\text{Var}_{#1}\{#2\}}
  \newcommand{\cov}[2][]{\text{cov}_{#1}\{#2\}}

  %\newcommand{\Exp}[2]{\left\langle{#2}\right\rangle_{#1}}
  \newcommand{\ex}{\setminus}

  \providecommand{\href}[2]{{\color{blue}USE PDFLATEX!}}
  \providecommand{\url}[2]{\href{#1}{{\color{blue}#2}}}
%  \newcommand{\link}[1]{\href{{\protect #1}}{\texttt{\protect #1}}}
  \newcommand{\anchor}[2]{\begin{picture}(0,0)\put(#1){#2}\end{picture}}
  \newcommand{\pagebox}{\begin{picture}(0,0)\put(-3,-23){
    \textcolor[rgb]{.5,1,.5}{\framebox[\textwidth]{\rule[-\textheight]{0pt}{0pt}}}}
    \end{picture}}

  \newcommand{\hide}[1]{
    \begin{list}{}{\leftmargin0ex \rightmargin0ex \topsep0ex \parsep0ex}
       \helvetica{5}{1}{m}{n}
       \renewcommand{\section}{\par SECTION: }
       \renewcommand{\subsection}{\par SUBSECTION: }
       \item[$~~\blacktriangleright$]
       #1%$\blacktriangleleft~~$
       \message{^^JHIDE--Warning!^^J}
    \end{list}
  }
  %\newcommand{\hide}[1]{{\tt[hide:~}{\footnotesize\sf #1}{\tt]}\message{^^JHIDE--Warning!^^J}}
  \newcommand{\Hide}{\renewcommand{\hide}[1]{\message{^^JHIDE--Warning (hidden)!^^J}}}
  \newcommand{\HIDE}{\renewcommand{\hide}[1]{}}
  \newcommand{\fullhide}[1]{\message{^^JHIDE--Warning (hidden)!^^J}}
  %\newcommand{\todo}[1]{{\tt[TODO: #1]}\message{^^JTODO--Warning: #1^^J}}
  %\newcommand{\Todo}{\renewcommand{\todo}[1]{\message{^^JTODO--Warning (hidden)!^^J}}}
  %\renewcommand{\title}[1]{\renewcommand{\thetitle}{#1}}
  \newcommand{\myauthor}[1]{\author{#1}\newcommand{\theauthor}{#1}}%\@author}
  \newcommand{\mytitle}[1]{\title{#1}\newcommand{\thetitle}{#1}}%\@title}
  \newcommand{\header}{\begin{document}\mytitle\cleardefs}
  \newcommand{\contents}{{\tableofcontents}\renewcommand{\contents}{}}
  \newcommand{\footer}{\small\bibliography{marc,bibs}\end{document}}
  \newcommand{\widepaper}{\usepackage{geometry}\geometry{a4paper,hdivide={25mm,*,25mm},vdivide={25mm,*,25mm}}}
  \newcommand{\moviex}[2]{\movie[externalviewer]{#1}{#2}} %\pdflatex\usepackage{multimedia}
  \newcommand{\rbox}[1]{\fboxrule2mm\fcolorbox[rgb]{1,.85,.85}{1,.85,.85}{#1}}
  \newcommand{\mpage}[2]{{\begin{minipage}{#1\columnwidth}#2\end{minipage}}}
  \newcommand{\redbox}[2]{\fboxrule1mm\fcolorbox[rgb]{1,.7,.7}{1,.7,.7}{\begin{minipage}{#1\columnwidth}\center#2\end{minipage}}}
  \newcommand{\onecol}[2]{
    \begin{minipage}[c]{#1\columnwidth}#2\end{minipage}}
  \newcommand{\twocol}[5][0]{
    \begin{minipage}[c]{#2\columnwidth}#4\end{minipage}\hspace*{#1\columnwidth}%
    \begin{minipage}[c]{#3\columnwidth}#5\end{minipage}}
  \newcommand{\threecol}[7][0]{%
    \begin{minipage}[c]{#2\columnwidth}#5\end{minipage}\hspace*{#1\columnwidth}%
    \begin{minipage}[c]{#3\columnwidth}#6\end{minipage}\hspace*{#1\columnwidth}%
    \begin{minipage}[c]{#4\columnwidth}#7\end{minipage}}
  \newcommand{\threecoltext}[7][c]{
    \begin{minipage}[#1]{#2\textwidth}#5\end{minipage}%
    \begin{minipage}[#1]{#3\textwidth}#6\end{minipage}%
    \begin{minipage}[#1]{#4\textwidth}#7\end{minipage}}
  \newcommand{\threecoltop}[7][0]{%
   \begin{minipage}[t]{#2\columnwidth}#5\end{minipage}\hspace*{#1\columnwidth}%
   \begin{minipage}[t]{#3\columnwidth}#6\end{minipage}\hspace*{#1\columnwidth}%
   \begin{minipage}[t]{#4\columnwidth}#7\end{minipage}}
  \newcommand{\fourcol}[9][0]{%
   \begin{minipage}[c]{#2\columnwidth}#6\end{minipage}\hspace*{#1\columnwidth}%
   \begin{minipage}[c]{#3\columnwidth}#7\end{minipage}\hspace*{#1\columnwidth}%
   \begin{minipage}[c]{#4\columnwidth}#8\end{minipage}\hspace*{#1\columnwidth}%
   \begin{minipage}[c]{#5\columnwidth}#9\end{minipage}}
  \newcommand{\helvetica}[4]{\setlength{\unitlength}{1pt}\fontsize{#1}{#1}\linespread{#2}\usefont{OT1}{phv}{#3}{#4}}
  \newcommand{\helve}[1]{\helvetica{#1}{1.5}{m}{n}}
  \newcommand{\german}{\usepackage[german]{babel}\usepackage[utf8]{inputenc}}
\newcommand{\sans}{\renewcommand{\familydefault}{\sfdefault}}

\newcommand{\norm}[1]{|\!|#1|\!|}
\newcommand{\expr}[1]{[\hspace{-.2ex}[#1]\hspace{-.2ex}]}

\newcommand{\Jwi}{J^\sharp_W}
\newcommand{\THi}{T^\sharp_H}
\newcommand{\Jci}{J^\natural_C}
\newcommand{\hJi}{{\bar J}^\sharp}
\renewcommand{\|}{\,|\,}
\renewcommand{\=}{\!=\!}
\newcommand{\myminus}{{\hspace*{-.0pt}\text{\rm -}\hspace*{-.5pt}}}
\newcommand{\myplus}{{\hspace*{-.0pt}\text{\rm +}\hspace*{-.5pt}}}
\newcommand{\1}{{\myminus1}}
\newcommand{\2}{{\myminus2}}
\newcommand{\3}{{\myminus3}}
\newcommand{\mT}{{\text{\rm -}\hspace*{-1pt}\top}}
\newcommand{\po}{{\myplus1}}
\newcommand{\pt}{{\myplus2}}
%\renewcommand{\-}{\myminus}
%\newcommand{\+}{\myplus}
\newcommand{\T}{{\!\top\!}}
%\newcommand{\T}{{\!\textsf{\scriptsize T}\!}}
\newcommand{\xT}{{\underline x}}
\newcommand{\uT}{{\underline u}}
\newcommand{\zT}{{\underline z}}
\newcommand{\Sum}{\textstyle\sum}
\newcommand{\Int}{\textstyle\int}
\newcommand{\Frac}{\textstyle\frac}
\newcommand{\Max}{\textstyle\max}
\newcommand{\Prod}{\textstyle\prod}


\newenvironment{centy}{
\vspace{15mm}
\large
\hspace*{5mm}
\begin{minipage}{8cm}\it\color{blue}
}{
\end{minipage}
}

\newcommand{\old}{{\text{old}}}
\newcommand{\new}{{\text{new}}}
\newcommand{\MAP}{{\text{MAP}}}
\newcommand{\ML}{{\text{ML}}}

\newcommand{\redArrow}{\quad\anchor{0,-1}{\includegraphics[scale=.5]{figs/redArrow}}}
\newcommand{\pub}[1]{{\color{green}\helvetica{8}{1.3}{m}{n}#1\\}}
\DeclareMathOperator{\opKL}{KL}
\newcommand{\KL}[2]{\opKL\big(#1\,\big|\!\big|\,#2\big)} %\left(#1 |\!| #2\right)}

\renewcommand{\show}[2][.7]{\centerline{\includegraphics[width=#1\columnwidth]{#2}}}
\newcommand{\showh}[2][.8]{\includegraphics[width=#1\columnwidth]{#2}}
\newcommand{\shows}[2][.8]{\centerline{\includegraphics[scale=#1]{#2}}}
\newcommand{\showhs}[2][.8]{\includegraphics[scale=#1]{#2}}
\newcommand{\mov}[2]{\movie[externalviewer]{{\color{blue}\small #1}}{movies/#2}}
\newcommand{\movex}[2]{\movie[externalviewer]{#1}{#2}} %\pdflatex\usepackage{multimedia}
%\newcommand{\movgb}[1]{\hfill\movie[externalviewer]{\small[movie]}{/home/mtoussai/movies/10-goalDirectedBehavior/#1}}
\newcommand{\movh}[3][autostart,loop]{
\movie[#1]{\showh[#2]{#3.jpg}}{#3.mp4}%
\movie[externalviewer]{$\circ$}{#3.mp4}
}
\newcommand{\movc}[3][autostart,loop]{\centerline{\movh[#1]{#2}{#3}}}
\newcommand{\mymovie}[3][]{\centerline{\movie[autostart,loop,#1]{\includegraphics[#1]{#2}}{#3}}}


\newcommand{\cen}[1]{\centerline{#1}}

\newcommand{\citing}[1]{
{\color{citcol}\ttiny#1\par}
}

\newcommand{\cit}[3]{
\par\smallskip
{\color{citcol}\ttiny #1: \emph{#2}. #3 \par}
}

\newcommand{\citurl}[4]{
\par\smallskip
{\color{citcol}\ttiny #1: \protect{\href{#4}{\color{blue}{#2.}}} #3 \par}
}

\newcommand{\cito}[3]{
\par\smallskip
{\color{bluecol}\ttiny #1: \emph{#2}. #3 \par}
}

\newcommand{\redoMacrosInProof}{
  \renewcommand{\d}{\delta}
%  \renewcommand{\|}{\,|\,}
  \renewcommand{\=}{\!=\!}
}

\newcommand{\pdflatex}{
  \definecolor{bluecol}{rgb}{0,0,.5}
  %\definecolor{greencol}{rgb}{0,.4,0}
  \usepackage[
    %    pdftex%,
    %%    letterpaper,
    %bookmarks,
    %bookmarksnumbered,
    colorlinks,
    urlcolor=bluecol,
    citecolor=black,
    linkcolor=bluecol,
    %    pagecolor=bluecol,
    pdfborder={0 0 0},
    %pdfborderstyle={/S/U/W 1},
    %%    backref,     %link from bibliography back to sections
    %%    pagebackref, %link from bibliography back to pages
    %%    pdfstartview=FitH, %fitwidth instead of fit window
    pdfpagemode=UseOutlines, %bookmarks are displayed by acrobat
    %%    pdftitle={\thetitle},
    pdfauthor={Marc Toussaint}
  ]{hyperref}
  \DeclareGraphicsExtensions{.pdf,.png,.jpg,.eps}
  \renewcommand{\r}{\varrho}
  \renewcommand{\l}{\lambda}
  \renewcommand{\L}{\Lambda}
  \renewcommand{\s}{\sigma}
  \renewcommand{\b}{\beta}
  \renewcommand{\d}{\delta}
  \renewcommand{\k}{\kappa}
  \renewcommand{\t}{\theta}
  \renewcommand{\O}{\Omega}
  \renewcommand{\o}{\omega}
  \renewcommand{\SS}{{\cal S}}
  \renewcommand{\=}{\!=\!}
  %\renewcommand{\boldsymbol}{}
  %\renewcommand{\Chapter}{\chapter}
  %\renewcommand{\Subsection}{\subsection}
}

%% \makeatletter
%% \newenvironment{code}{%
%%   \begin{lrbox}{\@tempboxa}\begin{minipage}{1\columnwidth}\codefont
%% }{
%%   \end{minipage}\end{lrbox}%
%%   \colorbox[rgb]{.95,.95,.95}{\usebox{\@tempboxa}}
%% }\makeatother

\newenvironment{code}{%
\codefont
\begin{shaded}
}{
\end{shaded}
}

\newcommand{\signature}{
  \includegraphics[height=6ex]{signatureColor} %unterschrift2}
}

\usepackage{algorithm}
\usepackage{algpseudocode}
\algrenewcommand{\algorithmicrequire}{\textbf{Input:~~}}
\algrenewcommand{\algorithmicensure}{\textbf{Output:}}
\algrenewcommand{\algorithmiccomment}[1]{\qquad\hfill~\hspace*{-5ex}\textit{// #1}}
\algrenewcommand{\alglinenumber}[1]{\helvetica{6}{1.3}{m}{n}#1:}

\newenvironment{algo}[1][8]{
\quad\begin{minipage}{.8\columnwidth}\helvetica{#1}{1.4}{m}{n}
\medskip\hrule\medskip
\begin{algorithmic}[1]
}{
\end{algorithmic}
\medskip\hrule\medskip
\end{minipage}
}

\usepackage{etoolbox}

%%%%%%%%%%%%%%%%%%%%%%%%%%%%%%%%%%%%%%%%%%%%%%%%%%%%%%%%%%%%%%%%%%%%%%%%%%%%%%%%

\usepackage{multirow}
\usepackage{colortbl}
%\setlength{\jot}{0pt}
%\setlength{\mathindent}{1ex}
\usepackage{empheq}

%%%%%%%%%%%%%%%%%%%%%%%%%%%%%%%%%%%%%%%%%%%%%%%%%%%%%%%%%%%%%%%%%%%%%%%%%%%%%%%

\newcommand{\mypause}{\pause}
%\newcommand{\dom}{{\text{dom}}}
\newcommand{\defi}[1]{\textbf{#1}}
\newcommand{\red}[1]{\emph{\color{red}#1}}
%\newcommand{\ul}{\underline}
\newcommand{\pos}{{\textsf{pos}}}
\newcommand{\eff}{{\textsf{eff}}}
\newcommand{\rot}{{\textsf{rot}}}
\newcommand{\veC}{{\textsf{vec}}}
\newcommand{\quat}{{\textsf{quat}}}
\newcommand{\col}{{\textsf{col}}}
\newcommand{\de}[2]{\frac{\partial #1}{\partial #2}}
\newcommand{\target}{{\text{target}}}
\newcommand{\near}{{\text{near}}}
\newcommand{\qfree}{Q_{\text{free}}}
\renewcommand{\vec}{\boldsymbol}
\newcommand{\lft}{\text{left}}
\newcommand{\rgh}{\text{right}}
\DeclareMathOperator{\real}{real}
\newcommand{\prev}{{\text{prev}}}
\newcommand{\TR}[2]{T_{{#1}\to{#2}}}
\newcommand{\RO}[2]{R_{{#1}\to{#2}}}
\newcommand{\liter}{\helvetica{8}{1.1}{m}{n}\parskip 1ex}
\newcommand{\Fc}{\color{green}F}
\newcommand{\muc}{\color{blue}\mu}
\newcommand{\Astar}{A$^*$}

%for AI course:
\providecommand{\defn}[1]{\textbf{#1}}

%for optimization course:
\newcommand{\adec}{\r_\a^-}
\newcommand{\ainc}{\r_\a^+}
\newcommand{\ldec}{\r_\l^-}
\newcommand{\linc}{\r_\l^+}
\newcommand{\minc}{\r_\m^+}
\newcommand{\mdec}{\r_\m^-}
\newcommand{\step}{{\delta}}
\newcommand{\lsstop}{\r_{\text{ls}}}
\newcommand{\stepmax}{\step_{\text{max}}}


\definecolor{boxcol}{rgb}{.85,.9,.92}
\newcommand{\eqbox}[1]{\centerline{\fboxrule0mm\fcolorbox{boxcol}{boxcol}{#1}}}
\newcommand{\movgb}[1]{\hfill\movie[externalviewer]{\small[movie]}{/home/mtoussai/movies/10-goalDirectedBehavior/#1}}
\newcommand{\demo}[1]{{{\color{blue}[\small #1]}}}

\graphicspath{{../pics-robotics/}{../pics-ML/}{../pics-all/}{../pics-all2/}{../pics-Optim/}}
\DeclareGraphicsExtensions{.pdf,.png,.jpg}

%\usepackage{pdfpages}
%\setbeamercolor{background canvas}{bg=}

\newcommand{\SUM}{\texttt{sum}}
\usepackage{float}

%% prevent pagebreaks before environment
\makeatletter
\newcommand{\NewParNoBreak}[1][\parskip]{\par\vspace*{-\parskip}\vspace*{#1}\nobreak\@afterheading}
\makeatother

%\newcommand{\idx}[2]{\label{IKgn}}

%%%%%%%%%%%%%%%%%%%%%%%%%%%%%%%%%%%%%%%%%%%%%%%%%%%%%%%%%%%%%%%%%%%%%%%%%%%%%%%%



%% \newwrite\tempfile
%% \immediate\openout\tempfile=z.keys.tex

%% \renewcommand{\key}[1]{
%% %%   \addtocounter{mypage}{1}
%% \makeatletter
%% \immediate\write\tempfile{\symbol{`\\}}
%% \makeatother
%%   \immediate\write\tempfile{hyperref[key:#1]{#1(\arabic{mypage})}}
%% %%  % \phantomsection\label{key:#1}
%% %%   %\index{#1@{\hyperref[key:#1]{#1 (\arabic{mysec}:\arabic{mypage})}}|phantom}
%% %%   \addtocounter{mypage}{-1}
%% }


  \graphicspath{{pics/}{../pics/}{../latex/pics/}}
}

\newcommand{\lecturetitleslide}{
  \title{\course \topic}
  \author{Marc Toussaint}
  \institute{Learning \& Intelligent Systems Lab, TU Berlin}

  \begin{document}


  %% title slide!
  \slide{}{
    \thispagestyle{empty}

    \twocol{.27}{.6}{
      %\vspace*{-5mm}%
      \hspace*{-5mm}%
      \includegraphics[width=1.\columnwidth]{\coursepicture}
    }{\center

      \textbf{\fontsize{17}{20}\selectfont \course}

      ~

      %Lecture
      \topic\\

      \vspace{1cm}

      {\tiny~\emph{\keywords}~\\}

      \vspace{1cm}

      Marc Toussaint
      
      Technical University of Berlin

      \coursedate

      ~

    }
  }
}

\newcommand{\slide}[2]{
  \slidefont
  \incpage\begin{frame}
  \addcontentsline{toc}{section}{#1}
  \vfill
  {\headerfont #1} \vspace*{-2ex}
  \begin{itemize}\item[]~\\
    #2
  \end{itemize}
  \vfill
  \end{frame}
}

% use \begin{frame}[fragile] around slidecore!
\newenvironment{slidecore}[1]{
  \slidefont\incpage
  \addcontentsline{toc}{section}{#1}
  \vfill
  {\headerfont #1} \vspace*{-2ex}
  \begin{itemize}\item[]~\\
}{
  \end{itemize}
  \vfill
}

\newcommand{\titleslide}[4][Marc Toussaint]{
  \newcommand{\slideauthor}{#1}
  \newcommand{\slidevenue}{#3}
  \slidefont
  \incpage
  \begin{frame}
  \begin{center}
    \vspace*{15mm}

    {\headerfont #2\\}
        
    \vspace*{10mm}

    #1 \\

    \vspace*{5mm}

    {\small
      Learning \& Intelligent Systems Lab, TU Berlin\\
%      Science of Intelligence Cluster of Excellence, Berlin\\
      Max Planck Fellow, Institute for Intelligent Systems\\ %Physical Reasoning \& Manipulation Lab -- 
%      MIT CSAIL\\
%      mtoussai@mit.edu,~ marc.toussaint@informatik.uni-stuttgart.de

      \vspace*{10mm}

      \emph{#3}
    }

    \vspace*{0mm}

    %\includegraphics[scale=.1]{pics/eushield-fullcolour}

  \end{center}
  \begin{itemize}\item[]~\\
    #4
  \end{itemize}
  \end{frame}
}

\newcommand{\titleslideempty}[2]{
  \slidefont
  \incpage
  \begin{frame}
  \begin{center}
    \vspace*{15mm}

    {\headerfont #1\\}
        
    %% \vspace*{5mm}

    %% {\small\emph{#2}} \\

  \end{center}
  \begin{itemize}\item[]~\\
    #2
  \end{itemize}
  \end{frame}
}

\providecommand{\key}[1]{
  \addtocounter{mypage}{1}
% \immediate\write\keyfile{#1}
  \addtocontents{toc}{\hyperref[key:#1]{#1 (\arabic{mypage})}}
%  \phantomsection\label{key:#1}
%  \index{#1@{\hyperref[key:#1]{#1 (\arabic{mysec}:\arabic{mypage})}}|phantom}
  \addtocounter{mypage}{-1}
}

\providecommand{\course}{}

\providecommand{\subtopic}{}

\providecommand{\sublecture}[2]{
  \renewcommand{\subtopic}{#1}
  \slide{#1}{#2}
}

\providecommand{\sublectureHide}[2]{
  \renewcommand{\subtopic}{#1}
}

\providecommand{\story}[1]{
~

Motivation: {\tiny #1}\clearpage
}

\newenvironment{items}[1][9]{
\par\setlength{\unitlength}{1pt}\fontsize{#1}{#1}\linespread{1.2}\selectfont
\begin{list}{--}{\leftmargin4ex \rightmargin0ex \labelsep1ex \labelwidth2ex
\topsep0pt \parsep0ex \itemsep3pt}
}{
\end{list}
}

\newenvironment{itemS}[1][4ex]{
\par
\tiny
\begin{list}{--}{\leftmargin#1 \rightmargin0ex \labelsep1ex
  \labelwidth2ex \topsep0pt \parsep0ex \itemsep2pt}
}{
\end{list}
}

\providecommand{\slidesfoot}{
  \end{document}
}


  \slideshead
  \lecturetitleslide
}

\providecommand{\exercises}{
  \newcommand{\exerciseshead}{
  \documentclass[10pt,fleqn]{article}
  \stdpackages
\usepackage[utf8]{inputenc}

  \definecolor{bluecol}{rgb}{0,0,.5}
  \definecolor{greencol}{rgb}{0,.4,0}
  \definecolor{shadecolor}{gray}{0.9}
  \usepackage[
    %    pdftex%,
    %%    letterpaper,
    %    bookmarks,
    %    bookmarksnumbered,
    colorlinks,
    urlcolor=bluecol,
    citecolor=black,
    linkcolor=bluecol,
    %    pagecolor=bluecol,
    pdfborder={0 0 0},
    %pdfborderstyle={/S/U/W 1},
    %%    backref,     %link from bibliography back to sections
    %%    pagebackref, %link from bibliography back to pages
    %%    pdfstartview=FitH, %fitwidth instead of fit window
    pdfpagemode=UseNone, %UseOutlines, %bookmarks are displayed by acrobat
    pdftitle={\course},
    pdfauthor={Marc Toussaint},
    pdfkeywords={}
  ]{hyperref}
  \DeclareGraphicsExtensions{.pdf,.png,.jpg,.eps}

  \renewcommand{\r}{\varrho}
  \renewcommand{\l}{\lambda}
  \renewcommand{\L}{\Lambda}
  \renewcommand{\b}{\beta}
  \renewcommand{\d}{\delta}
  \renewcommand{\k}{\kappa}
  \renewcommand{\t}{\theta}
  \renewcommand{\O}{\Omega}
  \renewcommand{\o}{\omega}
  \renewcommand{\SS}{{\cal S}}
  \renewcommand{\=}{\!=\!}
  %\renewcommand{\boldsymbol}{}
  %\renewcommand{\Chapter}{\chapter}
  %\renewcommand{\Subsection}{\subsection}

  \renewcommand{\baselinestretch}{1.1}
  \geometry{a4paper,headsep=7mm,hdivide={15mm,*,15mm},vdivide={20mm,*,15mm}}

  \fancyhead[L]{\thetitle, \textit{\teacher}---\coursedate}
  \fancyhead[R]{\thepage}
  \fancyhead[C]{}
  \fancyfoot{}
  \pagestyle{fancy}

  \renewcommand{\labelenumi}{{\alph{enumi})}}

  \parindent 0pt
  \parskip 0.5ex

  \newcommand{\codefont}{\helvetica{8}{1.2}{m}{n}}

  %auto-ignore
  \renewcommand{\a}{\alpha}
  \renewcommand{\b}{\beta}
  \renewcommand{\d}{\delta}
    \newcommand{\D}{\Delta}
    \newcommand{\e}{\epsilon}
    \newcommand{\g}{\gamma}
    \newcommand{\G}{\Gamma}
  \renewcommand{\l}{\lambda}
  \renewcommand{\L}{\Lambda}
    \newcommand{\m}{\mu}
    \newcommand{\n}{\nu}
    \newcommand{\N}{\nabla}
  \renewcommand{\k}{\kappa}
  \renewcommand{\o}{\omega}
  \renewcommand{\O}{\Omega}
    \newcommand{\p}{\phi}
    \newcommand{\ph}{\varphi}
  \renewcommand{\P}{\Phi}
  \renewcommand{\r}{\varrho}
    \newcommand{\s}{\sigma}
  \renewcommand{\S}{\Sigma}
  \renewcommand{\t}{\theta}
    \newcommand{\Th}{\Theta}
  %\renewcommand{\v}{\vartheta}
    \newcommand{\x}{\xi}
    \newcommand{\X}{\Xi}
    \newcommand{\Y}{\Upsilon}
    \newcommand{\z}{\zeta}

  \renewcommand{\AA}{{\cal A}}
    \newcommand{\BB}{{\cal B}}
    \newcommand{\CC}{{\cal C}}
    \newcommand{\cc}{{\cal c}}
    \newcommand{\DD}{{\cal D}}
    \newcommand{\EE}{{\cal E}}
    \newcommand{\FF}{{\cal F}}
    \newcommand{\GG}{{\cal G}}
    \newcommand{\HH}{{\cal H}}
    \newcommand{\II}{{\cal I}}
    \newcommand{\KK}{{\cal K}}
    \newcommand{\LL}{{\cal L}}
    \newcommand{\MM}{{\cal M}}
    \newcommand{\NN}{{\cal N}}
    \newcommand{\oNN}{\overline\NN}
    \newcommand{\OO}{{\cal O}}
    \newcommand{\PP}{{\cal P}}
    \newcommand{\QQ}{{\cal Q}}
    \newcommand{\RR}{{\cal R}}
  \renewcommand{\SS}{{\cal S}}
    \newcommand{\TT}{{\cal T}}
    \newcommand{\uu}{{\cal u}}
    \newcommand{\UU}{{\cal U}}
    \newcommand{\VV}{{\cal V}}
    \newcommand{\XX}{{\cal X}}
    \newcommand{\xx}{\mathcal{x}}
    \newcommand{\YY}{{\cal Y}}
    \newcommand{\ZZ}{{\cal Z}}
    \newcommand{\SOSO}{{\cal SO}}
    \newcommand{\GLGL}{{\cal GL}}

    \newcommand{\Ee}{{\rm E}}

  \newcommand{\NNN}{{\mathbb{N}}}
  \newcommand{\III}{{\mathbb{I}}}
  \newcommand{\ZZZ}{{\mathbb{Z}}}
  %\newcommand{\RRR}{{\mathrm{I\!R}}}
  \newcommand{\RRR}{{\mathbb{R}}}
  \newcommand{\SSS}{{\mathbb{S}}}
  \newcommand{\EEE}{{\mathbb{E}}}
  \newcommand{\VVV}{{\mathbb{V}}}
  \newcommand{\CCC}{{\mathbb{C}}}
  \newcommand{\DDD}{{\mathbb{D}}}
  \newcommand{\one}{{{\bf 1}}}
  \newcommand{\eee}{\text{e}}

  \newcommand{\NNNN}{{\overline{\cal N}}}

  \renewcommand{\[}{\Big[}
  \renewcommand{\]}{\Big]}
  \renewcommand{\(}{\Big(}
  \renewcommand{\)}{\Big)}
  \renewcommand{\|}{\,|\,}
  \renewcommand{\;}{\,;\,}
  \renewcommand{\=}{\!=\!}
    \newcommand{\<}{\left\langle}
  \renewcommand{\>}{\right\rangle}

  \newcommand{\na}[1][]{{\nabla_{\!\!#1}}}
  \newcommand{\he}[1][]{{\nabla_{\!\!#1}^2}}
  \newcommand{\Prob}{{\rm Prob}}
  \newcommand{\Dir}{{\rm Dir}}
  \newcommand{\Beta}{{\rm Beta}}
  \newcommand{\Bern}{{\rm Bern}}
  \newcommand{\Bin}{{\rm Bin}}
  \newcommand{\Mult}{{\rm Mult}}
  \newcommand{\Aut}{{\rm Aut}}
  \newcommand{\cor}{{\rm cor}}
  \newcommand{\corr}{{\rm corr}}
  \newcommand{\sd}{{\rm sd}}
  \newcommand{\tr}{{\rm tr}}
  \newcommand{\Tr}{{\rm Tr}}
  \newcommand{\rank}{{\rm rank}}
  \newcommand{\diag}{{\rm diag}}
  \newcommand{\dom}{{\rm dom}}
  \newcommand{\id}{{\rm id}}
  \newcommand{\Id}{{\rm\bf I}}
  \newcommand{\Gl}{{\rm Gl}}
  \renewcommand{\th}{\ensuremath{{}^\text{th}} }
  \newcommand{\lag}{\mathcal{L}}
  \newcommand{\inn}{\rfloor}
  \newcommand{\lie}{\pounds}
  \newcommand{\longto}{\longrightarrow}
  \newcommand{\speer}{\parbox{0.4ex}{\raisebox{0.8ex}{$\nearrow$}}}
  \renewcommand{\dag}{ {}^\dagger }
  \newcommand{\blbox}{\rule{1ex}{1ex}}
  \newcommand{\Ji}{J^\sharp}
  \newcommand{\h}{{}^\star}
  \newcommand{\w}{\wedge}
  \newcommand{\too}{\longrightarrow}
  \newcommand{\oot}{\longleftarrow}
  \newcommand{\To}{\Rightarrow}
  \newcommand{\oT}{\Leftarrow}
  \newcommand{\oTo}{\Leftrightarrow}
  \renewcommand{\iff}{~\Leftrightarrow~}
  \newcommand{\Too}{\;\Longrightarrow\;}
  \newcommand{\oto}{\leftrightarrow}
  \newcommand{\ot}{\leftarrow}
  \newcommand{\ootoo}{\longleftrightarrow}
  \newcommand{\ow}{\stackrel{\circ}\wedge}
  \newcommand{\defeq}{\stackrel{\hspace{0.2ex}{}_\Delta}=}
%  \newcommand{\defeq}{{\overstack\Delta =}}
  \newcommand{\feed}{\nonumber \\}
  \newcommand{\comma}{~,\quad}
  \newcommand{\period}{~.\quad}
  \newcommand{\del}{\partial}
  \newcommand{\Del}[1]{\textstyle\frac{\del}{\del #1}}
  \newcommand{\Hes}[1]{\textstyle\frac{\del^2}{\del #1^2}}

%  \newcommand{\quabla}{\Delta}
  \newcommand{\point}{$\bullet~~$}
  \newcommand{\doubletilde}{ ~ \raisebox{0.3ex}{$\widetilde {}$} \raisebox{0.6ex}{$\widetilde {}$} \!\! }
  \newcommand{\topcirc}{\parbox{0ex}{~\raisebox{2.5ex}{${}^\circ$}}}
  \newcommand{\topdot} {\parbox{0ex}{~\raisebox{2.5ex}{$\cdot$}}}
  \newcommand{\topddot} {\parbox{0ex}{~\raisebox{1.3ex}{$\ddot{~}$}}}
  \newcommand{\sym}{\topcirc}
  \newcommand{\tsum}{\textstyle\sum}
  \newcommand{\st}{~~\text{s.t.}~~}

  \newcommand{\half}{\ensuremath{\frac{1}{2}}}
  \newcommand{\third}{\ensuremath{\frac{1}{3}}}
  \newcommand{\fourth}{\ensuremath{\frac{1}{4}}}

  \newcommand{\ubar}{\underline}
  %\renewcommand{\vec}{\underline}
  \renewcommand{\vec}{\boldsymbol}
  %\renewcommand{\_}{\underset}
  %\renewcommand{\^}{\overset}
  %\renewcommand{\*}{{\rm\raisebox{-.6ex}{\text{*}}{}}}
  \renewcommand{\*}{\text{\footnotesize\raisebox{-.4ex}{*}{}}}

  \newcommand{\gto}{{\raisebox{.5ex}{${}_\rightarrow$}}}
  \newcommand{\gfrom}{{\raisebox{.5ex}{${}_\leftarrow$}}}
  \newcommand{\gnto}{{\raisebox{.5ex}{${}_\nrightarrow$}}}
  \newcommand{\gnfrom}{{\raisebox{.5ex}{${}_\nleftarrow$}}}

  %\newcommand{\RND}{{\SS}}
  %\newcommand{\IF}{\text{if }}
  %\newcommand{\AND}{\textsc{and }}
  %\newcommand{\OR}{\textsc{or }}
  %\newcommand{\XOR}{\textsc{xor }}
  %\newcommand{\NOT}{\textsc{not }}

  %\newcommand{\argmax}[1]{{\rm arg}\!\max_{#1}}
  %\newcommand{\argmin}[1]{{\rm arg}\!\min_{#1}}
  \DeclareMathOperator*{\argmax}{argmax}
  \DeclareMathOperator*{\argmin}{argmin}
  \DeclareMathOperator{\sign}{sign}
  \DeclareMathOperator{\acos}{acos}
  \DeclareMathOperator{\unifies}{unifies}
  \DeclareMathOperator{\Span}{span}
  \newcommand{\ortho}{\perp}
  %\newcommand{\argmax}[1]{\underset{~#1}{\text{argmax}}\;}
  %\newcommand{\argmin}[1]{\underset{~#1}{\text{argmin}}\;}
  \newcommand{\ee}[1]{\ensuremath{\cdot10^{#1}}}
  \newcommand{\sub}[1]{\ensuremath{_{\text{#1}}}}
  \newcommand{\up}[1]{\ensuremath{^{\text{#1}}}}
  \newcommand{\kld}[3][{}]{D_{#1}\big(#2\,\big|\!\big|\,#3\big)}
  %\newcommand{\kld}[2]{D\big(#1:#2\big)}
  \newcommand{\sprod}[2]{\big<#1\,,\,#2\big>}
  \newcommand{\End}{\text{End}}
  \newcommand{\txt}[1]{\quad\text{#1}\quad}
  \newcommand{\Over}[2]{\genfrac{}{}{0pt}{0}{#1}{#2}}
  %\newcommand{\mat}[1]{{\bf #1}}
  \newcommand{\arr}[2]{\hspace*{-.5ex}\begin{array}{#1}#2\end{array}\hspace*{-.5ex}}
  \newcommand{\mat}[3][.9]{
    \renewcommand{\arraystretch}{#1}{\scriptscriptstyle{\left(
      \hspace*{-1ex}\begin{array}{#2}#3\end{array}\hspace*{-1ex}
    \right)}}\renewcommand{\arraystretch}{1.2}
  }
  \newcommand{\Mat}[3][.9]{
    \renewcommand{\arraystretch}{#1}{\scriptscriptstyle{\left[
      \hspace*{-1ex}\begin{array}{#2}#3\end{array}\hspace*{-1ex}
    \right]}}\renewcommand{\arraystretch}{1.2}
  }
  \newcommand{\case}[2][ll]{\left\{\arr{#1}{#2}\right.}
  \newcommand{\seq}[1]{\textsf{\<#1\>}}
  \newcommand{\seqq}[1]{\textsf{#1}}
  \newcommand{\floor}[1]{\lfloor#1\rfloor}
  \newcommand{\Exp}[2][]{\mathbb{E}_{#1}\!\left\{#2\right\}}
  \newcommand{\Expno}[1][]{\mathbb{E}_{#1}}
  \newcommand{\Var}[2][]{\text{Var}_{#1}\{#2\}}
  \newcommand{\cov}[2][]{\text{cov}_{#1}\{#2\}}

  %\newcommand{\Exp}[2]{\left\langle{#2}\right\rangle_{#1}}
  \newcommand{\ex}{\setminus}

  \providecommand{\href}[2]{{\color{blue}USE PDFLATEX!}}
  \providecommand{\url}[2]{\href{#1}{{\color{blue}#2}}}
%  \newcommand{\link}[1]{\href{{\protect #1}}{\texttt{\protect #1}}}
  \newcommand{\anchor}[2]{\begin{picture}(0,0)\put(#1){#2}\end{picture}}
  \newcommand{\pagebox}{\begin{picture}(0,0)\put(-3,-23){
    \textcolor[rgb]{.5,1,.5}{\framebox[\textwidth]{\rule[-\textheight]{0pt}{0pt}}}}
    \end{picture}}

  \newcommand{\hide}[1]{
    \begin{list}{}{\leftmargin0ex \rightmargin0ex \topsep0ex \parsep0ex}
       \helvetica{5}{1}{m}{n}
       \renewcommand{\section}{\par SECTION: }
       \renewcommand{\subsection}{\par SUBSECTION: }
       \item[$~~\blacktriangleright$]
       #1%$\blacktriangleleft~~$
       \message{^^JHIDE--Warning!^^J}
    \end{list}
  }
  %\newcommand{\hide}[1]{{\tt[hide:~}{\footnotesize\sf #1}{\tt]}\message{^^JHIDE--Warning!^^J}}
  \newcommand{\Hide}{\renewcommand{\hide}[1]{\message{^^JHIDE--Warning (hidden)!^^J}}}
  \newcommand{\HIDE}{\renewcommand{\hide}[1]{}}
  \newcommand{\fullhide}[1]{\message{^^JHIDE--Warning (hidden)!^^J}}
  %\newcommand{\todo}[1]{{\tt[TODO: #1]}\message{^^JTODO--Warning: #1^^J}}
  %\newcommand{\Todo}{\renewcommand{\todo}[1]{\message{^^JTODO--Warning (hidden)!^^J}}}
  %\renewcommand{\title}[1]{\renewcommand{\thetitle}{#1}}
  \newcommand{\myauthor}[1]{\author{#1}\newcommand{\theauthor}{#1}}%\@author}
  \newcommand{\mytitle}[1]{\title{#1}\newcommand{\thetitle}{#1}}%\@title}
  \newcommand{\header}{\begin{document}\mytitle\cleardefs}
  \newcommand{\contents}{{\tableofcontents}\renewcommand{\contents}{}}
  \newcommand{\footer}{\small\bibliography{marc,bibs}\end{document}}
  \newcommand{\widepaper}{\usepackage{geometry}\geometry{a4paper,hdivide={25mm,*,25mm},vdivide={25mm,*,25mm}}}
  \newcommand{\moviex}[2]{\movie[externalviewer]{#1}{#2}} %\pdflatex\usepackage{multimedia}
  \newcommand{\rbox}[1]{\fboxrule2mm\fcolorbox[rgb]{1,.85,.85}{1,.85,.85}{#1}}
  \newcommand{\mpage}[2]{{\begin{minipage}{#1\columnwidth}#2\end{minipage}}}
  \newcommand{\redbox}[2]{\fboxrule1mm\fcolorbox[rgb]{1,.7,.7}{1,.7,.7}{\begin{minipage}{#1\columnwidth}\center#2\end{minipage}}}
  \newcommand{\onecol}[2]{
    \begin{minipage}[c]{#1\columnwidth}#2\end{minipage}}
  \newcommand{\twocol}[5][0]{
    \begin{minipage}[c]{#2\columnwidth}#4\end{minipage}\hspace*{#1\columnwidth}%
    \begin{minipage}[c]{#3\columnwidth}#5\end{minipage}}
  \newcommand{\threecol}[7][0]{%
    \begin{minipage}[c]{#2\columnwidth}#5\end{minipage}\hspace*{#1\columnwidth}%
    \begin{minipage}[c]{#3\columnwidth}#6\end{minipage}\hspace*{#1\columnwidth}%
    \begin{minipage}[c]{#4\columnwidth}#7\end{minipage}}
  \newcommand{\threecoltext}[7][c]{
    \begin{minipage}[#1]{#2\textwidth}#5\end{minipage}%
    \begin{minipage}[#1]{#3\textwidth}#6\end{minipage}%
    \begin{minipage}[#1]{#4\textwidth}#7\end{minipage}}
  \newcommand{\threecoltop}[7][0]{%
   \begin{minipage}[t]{#2\columnwidth}#5\end{minipage}\hspace*{#1\columnwidth}%
   \begin{minipage}[t]{#3\columnwidth}#6\end{minipage}\hspace*{#1\columnwidth}%
   \begin{minipage}[t]{#4\columnwidth}#7\end{minipage}}
  \newcommand{\fourcol}[9][0]{%
   \begin{minipage}[c]{#2\columnwidth}#6\end{minipage}\hspace*{#1\columnwidth}%
   \begin{minipage}[c]{#3\columnwidth}#7\end{minipage}\hspace*{#1\columnwidth}%
   \begin{minipage}[c]{#4\columnwidth}#8\end{minipage}\hspace*{#1\columnwidth}%
   \begin{minipage}[c]{#5\columnwidth}#9\end{minipage}}
  \newcommand{\helvetica}[4]{\setlength{\unitlength}{1pt}\fontsize{#1}{#1}\linespread{#2}\usefont{OT1}{phv}{#3}{#4}}
  \newcommand{\helve}[1]{\helvetica{#1}{1.5}{m}{n}}
  \newcommand{\german}{\usepackage[german]{babel}\usepackage[utf8]{inputenc}}
\newcommand{\sans}{\renewcommand{\familydefault}{\sfdefault}}

\newcommand{\norm}[1]{|\!|#1|\!|}
\newcommand{\expr}[1]{[\hspace{-.2ex}[#1]\hspace{-.2ex}]}

\newcommand{\Jwi}{J^\sharp_W}
\newcommand{\THi}{T^\sharp_H}
\newcommand{\Jci}{J^\natural_C}
\newcommand{\hJi}{{\bar J}^\sharp}
\renewcommand{\|}{\,|\,}
\renewcommand{\=}{\!=\!}
\newcommand{\myminus}{{\hspace*{-.0pt}\text{\rm -}\hspace*{-.5pt}}}
\newcommand{\myplus}{{\hspace*{-.0pt}\text{\rm +}\hspace*{-.5pt}}}
\newcommand{\1}{{\myminus1}}
\newcommand{\2}{{\myminus2}}
\newcommand{\3}{{\myminus3}}
\newcommand{\mT}{{\text{\rm -}\hspace*{-1pt}\top}}
\newcommand{\po}{{\myplus1}}
\newcommand{\pt}{{\myplus2}}
%\renewcommand{\-}{\myminus}
%\newcommand{\+}{\myplus}
\newcommand{\T}{{\!\top\!}}
%\newcommand{\T}{{\!\textsf{\scriptsize T}\!}}
\newcommand{\xT}{{\underline x}}
\newcommand{\uT}{{\underline u}}
\newcommand{\zT}{{\underline z}}
\newcommand{\Sum}{\textstyle\sum}
\newcommand{\Int}{\textstyle\int}
\newcommand{\Frac}{\textstyle\frac}
\newcommand{\Max}{\textstyle\max}
\newcommand{\Prod}{\textstyle\prod}


\newenvironment{centy}{
\vspace{15mm}
\large
\hspace*{5mm}
\begin{minipage}{8cm}\it\color{blue}
}{
\end{minipage}
}

\newcommand{\old}{{\text{old}}}
\newcommand{\new}{{\text{new}}}
\newcommand{\MAP}{{\text{MAP}}}
\newcommand{\ML}{{\text{ML}}}

\newcommand{\redArrow}{\quad\anchor{0,-1}{\includegraphics[scale=.5]{figs/redArrow}}}
\newcommand{\pub}[1]{{\color{green}\helvetica{8}{1.3}{m}{n}#1\\}}
\DeclareMathOperator{\opKL}{KL}
\newcommand{\KL}[2]{\opKL\big(#1\,\big|\!\big|\,#2\big)} %\left(#1 |\!| #2\right)}

\renewcommand{\show}[2][.7]{\centerline{\includegraphics[width=#1\columnwidth]{#2}}}
\newcommand{\showh}[2][.8]{\includegraphics[width=#1\columnwidth]{#2}}
\newcommand{\shows}[2][.8]{\centerline{\includegraphics[scale=#1]{#2}}}
\newcommand{\showhs}[2][.8]{\includegraphics[scale=#1]{#2}}
\newcommand{\mov}[2]{\movie[externalviewer]{{\color{blue}\small #1}}{movies/#2}}
\newcommand{\movex}[2]{\movie[externalviewer]{#1}{#2}} %\pdflatex\usepackage{multimedia}
%\newcommand{\movgb}[1]{\hfill\movie[externalviewer]{\small[movie]}{/home/mtoussai/movies/10-goalDirectedBehavior/#1}}
\newcommand{\movh}[3][autostart,loop]{
\movie[#1]{\showh[#2]{#3.jpg}}{#3.mp4}%
\movie[externalviewer]{$\circ$}{#3.mp4}
}
\newcommand{\movc}[3][autostart,loop]{\centerline{\movh[#1]{#2}{#3}}}
\newcommand{\mymovie}[3][]{\centerline{\movie[autostart,loop,#1]{\includegraphics[#1]{#2}}{#3}}}


\newcommand{\cen}[1]{\centerline{#1}}

\newcommand{\citing}[1]{
{\color{citcol}\ttiny#1\par}
}

\newcommand{\cit}[3]{
\par\smallskip
{\color{citcol}\ttiny #1: \emph{#2}. #3 \par}
}

\newcommand{\citurl}[4]{
\par\smallskip
{\color{citcol}\ttiny #1: \protect{\href{#4}{\color{blue}{#2.}}} #3 \par}
}

\newcommand{\cito}[3]{
\par\smallskip
{\color{bluecol}\ttiny #1: \emph{#2}. #3 \par}
}

\newcommand{\redoMacrosInProof}{
  \renewcommand{\d}{\delta}
%  \renewcommand{\|}{\,|\,}
  \renewcommand{\=}{\!=\!}
}

\newcommand{\pdflatex}{
  \definecolor{bluecol}{rgb}{0,0,.5}
  %\definecolor{greencol}{rgb}{0,.4,0}
  \usepackage[
    %    pdftex%,
    %%    letterpaper,
    %bookmarks,
    %bookmarksnumbered,
    colorlinks,
    urlcolor=bluecol,
    citecolor=black,
    linkcolor=bluecol,
    %    pagecolor=bluecol,
    pdfborder={0 0 0},
    %pdfborderstyle={/S/U/W 1},
    %%    backref,     %link from bibliography back to sections
    %%    pagebackref, %link from bibliography back to pages
    %%    pdfstartview=FitH, %fitwidth instead of fit window
    pdfpagemode=UseOutlines, %bookmarks are displayed by acrobat
    %%    pdftitle={\thetitle},
    pdfauthor={Marc Toussaint}
  ]{hyperref}
  \DeclareGraphicsExtensions{.pdf,.png,.jpg,.eps}
  \renewcommand{\r}{\varrho}
  \renewcommand{\l}{\lambda}
  \renewcommand{\L}{\Lambda}
  \renewcommand{\s}{\sigma}
  \renewcommand{\b}{\beta}
  \renewcommand{\d}{\delta}
  \renewcommand{\k}{\kappa}
  \renewcommand{\t}{\theta}
  \renewcommand{\O}{\Omega}
  \renewcommand{\o}{\omega}
  \renewcommand{\SS}{{\cal S}}
  \renewcommand{\=}{\!=\!}
  %\renewcommand{\boldsymbol}{}
  %\renewcommand{\Chapter}{\chapter}
  %\renewcommand{\Subsection}{\subsection}
}

%% \makeatletter
%% \newenvironment{code}{%
%%   \begin{lrbox}{\@tempboxa}\begin{minipage}{1\columnwidth}\codefont
%% }{
%%   \end{minipage}\end{lrbox}%
%%   \colorbox[rgb]{.95,.95,.95}{\usebox{\@tempboxa}}
%% }\makeatother

\newenvironment{code}{%
\codefont
\begin{shaded}
}{
\end{shaded}
}

\newcommand{\signature}{
  \includegraphics[height=6ex]{signatureColor} %unterschrift2}
}

\usepackage{algorithm}
\usepackage{algpseudocode}
\algrenewcommand{\algorithmicrequire}{\textbf{Input:~~}}
\algrenewcommand{\algorithmicensure}{\textbf{Output:}}
\algrenewcommand{\algorithmiccomment}[1]{\qquad\hfill~\hspace*{-5ex}\textit{// #1}}
\algrenewcommand{\alglinenumber}[1]{\helvetica{6}{1.3}{m}{n}#1:}

\newenvironment{algo}[1][8]{
\quad\begin{minipage}{.8\columnwidth}\helvetica{#1}{1.4}{m}{n}
\medskip\hrule\medskip
\begin{algorithmic}[1]
}{
\end{algorithmic}
\medskip\hrule\medskip
\end{minipage}
}

\usepackage{etoolbox}

%%%%%%%%%%%%%%%%%%%%%%%%%%%%%%%%%%%%%%%%%%%%%%%%%%%%%%%%%%%%%%%%%%%%%%%%%%%%%%%%

\usepackage{multirow}
\usepackage{colortbl}
%\setlength{\jot}{0pt}
%\setlength{\mathindent}{1ex}
\usepackage{empheq}

%%%%%%%%%%%%%%%%%%%%%%%%%%%%%%%%%%%%%%%%%%%%%%%%%%%%%%%%%%%%%%%%%%%%%%%%%%%%%%%

\newcommand{\mypause}{\pause}
%\newcommand{\dom}{{\text{dom}}}
\newcommand{\defi}[1]{\textbf{#1}}
\newcommand{\red}[1]{\emph{\color{red}#1}}
%\newcommand{\ul}{\underline}
\newcommand{\pos}{{\textsf{pos}}}
\newcommand{\eff}{{\textsf{eff}}}
\newcommand{\rot}{{\textsf{rot}}}
\newcommand{\veC}{{\textsf{vec}}}
\newcommand{\quat}{{\textsf{quat}}}
\newcommand{\col}{{\textsf{col}}}
\newcommand{\de}[2]{\frac{\partial #1}{\partial #2}}
\newcommand{\target}{{\text{target}}}
\newcommand{\near}{{\text{near}}}
\newcommand{\qfree}{Q_{\text{free}}}
\renewcommand{\vec}{\boldsymbol}
\newcommand{\lft}{\text{left}}
\newcommand{\rgh}{\text{right}}
\DeclareMathOperator{\real}{real}
\newcommand{\prev}{{\text{prev}}}
\newcommand{\TR}[2]{T_{{#1}\to{#2}}}
\newcommand{\RO}[2]{R_{{#1}\to{#2}}}
\newcommand{\liter}{\helvetica{8}{1.1}{m}{n}\parskip 1ex}
\newcommand{\Fc}{\color{green}F}
\newcommand{\muc}{\color{blue}\mu}
\newcommand{\Astar}{A$^*$}

%for AI course:
\providecommand{\defn}[1]{\textbf{#1}}

%for optimization course:
\newcommand{\adec}{\r_\a^-}
\newcommand{\ainc}{\r_\a^+}
\newcommand{\ldec}{\r_\l^-}
\newcommand{\linc}{\r_\l^+}
\newcommand{\minc}{\r_\m^+}
\newcommand{\mdec}{\r_\m^-}
\newcommand{\step}{{\delta}}
\newcommand{\lsstop}{\r_{\text{ls}}}
\newcommand{\stepmax}{\step_{\text{max}}}


\definecolor{boxcol}{rgb}{.85,.9,.92}
\newcommand{\eqbox}[1]{\centerline{\fboxrule0mm\fcolorbox{boxcol}{boxcol}{#1}}}
\newcommand{\movgb}[1]{\hfill\movie[externalviewer]{\small[movie]}{/home/mtoussai/movies/10-goalDirectedBehavior/#1}}
\newcommand{\demo}[1]{{{\color{blue}[\small #1]}}}

\graphicspath{{../pics-robotics/}{../pics-ML/}{../pics-all/}{../pics-all2/}{../pics-Optim/}}
\DeclareGraphicsExtensions{.pdf,.png,.jpg}

%\usepackage{pdfpages}
%\setbeamercolor{background canvas}{bg=}

\newcommand{\SUM}{\texttt{sum}}
\usepackage{float}

%% prevent pagebreaks before environment
\makeatletter
\newcommand{\NewParNoBreak}[1][\parskip]{\par\vspace*{-\parskip}\vspace*{#1}\nobreak\@afterheading}
\makeatother

%\newcommand{\idx}[2]{\label{IKgn}}

%%%%%%%%%%%%%%%%%%%%%%%%%%%%%%%%%%%%%%%%%%%%%%%%%%%%%%%%%%%%%%%%%%%%%%%%%%%%%%%%



%% \newwrite\tempfile
%% \immediate\openout\tempfile=z.keys.tex

%% \renewcommand{\key}[1]{
%% %%   \addtocounter{mypage}{1}
%% \makeatletter
%% \immediate\write\tempfile{\symbol{`\\}}
%% \makeatother
%%   \immediate\write\tempfile{hyperref[key:#1]{#1(\arabic{mypage})}}
%% %%  % \phantomsection\label{key:#1}
%% %%   %\index{#1@{\hyperref[key:#1]{#1 (\arabic{mysec}:\arabic{mypage})}}|phantom}
%% %%   \addtocounter{mypage}{-1}
%% }


  \DefineShortVerb{\@}

  \usepackage{comment}
  \specialcomment{solution}{
    \small
    \begin{shaded}
  }{
    \end{shaded}
  }
  
  \renewcommand{\hat}{\widehat}
  \newcommand{\bbg}{{\bar{\bar g}}}
  \graphicspath{{pics/}{../latex/pics/}}

  \mytitle{\course\\\exnum}
  \myauthor{\teacher\\ \small\addressTUB}
  \date{\coursedate}
}

  %%%%%%%%%%%%%%%%%%%%%%%%%%%%%%%%%%%%%%%%%%%%%%%%%%%%%%%%%%%%%%%%%%%%%%%%%%%%%%%%

\newcommand{\exercisestitle}{  
  \begin{document}
  \maketitle
}

\newcommand{\exsection}[1]{\section{#1}}

\newcommand{\exerfoot}{
  \end{document}
}

\newenvironment{items}[1][9]{
  \par\setlength{\unitlength}{1pt}\fontsize{#1}{#1}\linespread{1.2}\selectfont
  \begin{list}{--}{\leftmargin4ex \rightmargin0ex \labelsep1ex \labelwidth2ex
      \topsep0pt \parsep0ex \itemsep3pt}
}{
  \end{list}
}

  \exerciseshead
}

\providecommand{\script}{
  \newcommand{\scripthead}{
  \documentclass[9pt,fleqn,twoside]{article}
  \stdpackages

  \usepackage{makeidx}
  \makeindex

  \usepackage{thmtools}
  \definecolor{shadecolor}{gray}{0.85}
  \declaretheoremstyle[
    %headfont=\normalfont\bfseries,
    %notefont=\mdseries, notebraces={(}{)},
    %bodyfont=\normalfont,
    %postheadspace=0.5em,
    %spaceabove=6pt,
    mdframed={
      %  skipabove=8pt,
      %  skipbelow=6pt,
      hidealllines=true,
      backgroundcolor={shadecolor},
      innertopmargin=8pt,
      %  innerleftmargin=3pt,
      %  innerrightmargin=3pt
    }
  ]{shaded}
  \declaretheorem[style=shaded,within=section,name=Definition]{myDefinition}
  \declaretheorem[style=shaded,within=section,name=Theorem]{myTheorem}
  \declaretheorem[style=shaded,within=section,name=Identities]{Identities}
  \declaretheorem[style=shaded,within=section,name=Example]{myExample}

  \definecolor{grey}{rgb}{.8,.8,.8}
  \definecolor{bluecol}{rgb}{0,0,.5}
  \definecolor{greencol}{rgb}{0,.4,0}
  \definecolor{shadecolor}{gray}{0.9}
  \definecolor{citcol}{rgb}{.4,.4,.4}
  \usepackage[
    %    pdftex%,
    %%    letterpaper,
    %bookmarks,
    bookmarksnumbered,
    colorlinks,
    urlcolor=bluecol,
    citecolor=black,
    linkcolor=bluecol,
    %    pagecolor=bluecol,
    pdfborder={0 0 0},
    %pdfborderstyle={/S/U/W 1},
    %%    backref,     %link from bibliography back to sections
    %%    pagebackref, %link from bibliography back to pages
    %%    pdfstartview=FitH, %fitwidth instead of fit window
    pdfpagemode=UseOutlines, %bookmarks are displayed by acrobat
    pdftitle={\course},
    pdfauthor={Marc Toussaint},
    pdfkeywords={}
  ]{hyperref}
  \DeclareGraphicsExtensions{.pdf,.png,.jpg,.eps}

  %\usepackage{multirow}
  \usepackage{multimedia}
  %\usepackage{marginnote}
  %\setbeamercolor{background canvas}{bg=}

  \usepackage[round]{natbib}
  \bibliographystyle{abbrvnat}

  \renewcommand{\r}{\varrho}
  \renewcommand{\l}{\lambda}
  \renewcommand{\L}{\Lambda}
  \renewcommand{\b}{\beta}
  \renewcommand{\d}{\delta}
  \renewcommand{\k}{\kappa}
  \renewcommand{\t}{\theta}
  \renewcommand{\O}{\Omega}
  \renewcommand{\o}{\omega}
  \renewcommand{\SS}{{\cal S}}
  \renewcommand{\=}{\!=\!}
  %\renewcommand{\boldsymbol}{}
  %\renewcommand{\Chapter}{\chapter}
  %\renewcommand{\Subsection}{\subsection}

  \renewcommand{\baselinestretch}{1.0}
  \geometry{a5paper,headsep=6mm,hdivide={10mm,*,10mm},vdivide={13mm,*,7mm}}

  \fancyhead[OL,ER]{\course, \textit{Marc Toussaint}}
  \fancyhead[OR,EL]{\thepage}
  \fancyhead[C]{}
  \fancyfoot{}
  \pagestyle{fancy}

  \renewcommand{\labelenumi}{{(\roman{enumi})}}
  \renewcommand{\theenumi}{(\roman{enumi})} %for ref
  \parindent 0pt
  \parskip .5pc

  \columnsep 6ex

  \renewcommand{\familydefault}{\sfdefault}

  \newcommand{\headerfont}{\large}%helvetica{12}{1}{b}{n}}
  \newcommand{\slidefont} {}%\helvetica{9}{1.3}{m}{n}}
  \newcommand{\storyfont} {}
  %  \renewcommand{\small}   {\helvetica{8}{1.2}{m}{n}}
  \renewcommand{\tiny}    {\footnotesize}%helvetica{7}{1.1}{m}{n}}
  \newcommand{\ttiny} {\footnotesize}%fontsize{7}{7}\selectfont}
%  \newcommand{\codefont}{\fontsize{6}{6}\selectfont}%helvetica{8}{1.2}{m}{n}}
  \newcommand{\codefont} {\helvetica{8}{1.2}{m}{n}}

  %auto-ignore
  \renewcommand{\a}{\alpha}
  \renewcommand{\b}{\beta}
  \renewcommand{\d}{\delta}
    \newcommand{\D}{\Delta}
    \newcommand{\e}{\epsilon}
    \newcommand{\g}{\gamma}
    \newcommand{\G}{\Gamma}
  \renewcommand{\l}{\lambda}
  \renewcommand{\L}{\Lambda}
    \newcommand{\m}{\mu}
    \newcommand{\n}{\nu}
    \newcommand{\N}{\nabla}
  \renewcommand{\k}{\kappa}
  \renewcommand{\o}{\omega}
  \renewcommand{\O}{\Omega}
    \newcommand{\p}{\phi}
    \newcommand{\ph}{\varphi}
  \renewcommand{\P}{\Phi}
  \renewcommand{\r}{\varrho}
    \newcommand{\s}{\sigma}
  \renewcommand{\S}{\Sigma}
  \renewcommand{\t}{\theta}
    \newcommand{\Th}{\Theta}
  %\renewcommand{\v}{\vartheta}
    \newcommand{\x}{\xi}
    \newcommand{\X}{\Xi}
    \newcommand{\Y}{\Upsilon}
    \newcommand{\z}{\zeta}

  \renewcommand{\AA}{{\cal A}}
    \newcommand{\BB}{{\cal B}}
    \newcommand{\CC}{{\cal C}}
    \newcommand{\cc}{{\cal c}}
    \newcommand{\DD}{{\cal D}}
    \newcommand{\EE}{{\cal E}}
    \newcommand{\FF}{{\cal F}}
    \newcommand{\GG}{{\cal G}}
    \newcommand{\HH}{{\cal H}}
    \newcommand{\II}{{\cal I}}
    \newcommand{\KK}{{\cal K}}
    \newcommand{\LL}{{\cal L}}
    \newcommand{\MM}{{\cal M}}
    \newcommand{\NN}{{\cal N}}
    \newcommand{\oNN}{\overline\NN}
    \newcommand{\OO}{{\cal O}}
    \newcommand{\PP}{{\cal P}}
    \newcommand{\QQ}{{\cal Q}}
    \newcommand{\RR}{{\cal R}}
  \renewcommand{\SS}{{\cal S}}
    \newcommand{\TT}{{\cal T}}
    \newcommand{\uu}{{\cal u}}
    \newcommand{\UU}{{\cal U}}
    \newcommand{\VV}{{\cal V}}
    \newcommand{\XX}{{\cal X}}
    \newcommand{\xx}{\mathcal{x}}
    \newcommand{\YY}{{\cal Y}}
    \newcommand{\ZZ}{{\cal Z}}
    \newcommand{\SOSO}{{\cal SO}}
    \newcommand{\GLGL}{{\cal GL}}

    \newcommand{\Ee}{{\rm E}}

  \newcommand{\NNN}{{\mathbb{N}}}
  \newcommand{\III}{{\mathbb{I}}}
  \newcommand{\ZZZ}{{\mathbb{Z}}}
  %\newcommand{\RRR}{{\mathrm{I\!R}}}
  \newcommand{\RRR}{{\mathbb{R}}}
  \newcommand{\SSS}{{\mathbb{S}}}
  \newcommand{\EEE}{{\mathbb{E}}}
  \newcommand{\VVV}{{\mathbb{V}}}
  \newcommand{\CCC}{{\mathbb{C}}}
  \newcommand{\DDD}{{\mathbb{D}}}
  \newcommand{\one}{{{\bf 1}}}
  \newcommand{\eee}{\text{e}}

  \newcommand{\NNNN}{{\overline{\cal N}}}

  \renewcommand{\[}{\Big[}
  \renewcommand{\]}{\Big]}
  \renewcommand{\(}{\Big(}
  \renewcommand{\)}{\Big)}
  \renewcommand{\|}{\,|\,}
  \renewcommand{\;}{\,;\,}
  \renewcommand{\=}{\!=\!}
    \newcommand{\<}{\left\langle}
  \renewcommand{\>}{\right\rangle}

  \newcommand{\na}[1][]{{\nabla_{\!\!#1}}}
  \newcommand{\he}[1][]{{\nabla_{\!\!#1}^2}}
  \newcommand{\Prob}{{\rm Prob}}
  \newcommand{\Dir}{{\rm Dir}}
  \newcommand{\Beta}{{\rm Beta}}
  \newcommand{\Bern}{{\rm Bern}}
  \newcommand{\Bin}{{\rm Bin}}
  \newcommand{\Mult}{{\rm Mult}}
  \newcommand{\Aut}{{\rm Aut}}
  \newcommand{\cor}{{\rm cor}}
  \newcommand{\corr}{{\rm corr}}
  \newcommand{\sd}{{\rm sd}}
  \newcommand{\tr}{{\rm tr}}
  \newcommand{\Tr}{{\rm Tr}}
  \newcommand{\rank}{{\rm rank}}
  \newcommand{\diag}{{\rm diag}}
  \newcommand{\dom}{{\rm dom}}
  \newcommand{\id}{{\rm id}}
  \newcommand{\Id}{{\rm\bf I}}
  \newcommand{\Gl}{{\rm Gl}}
  \renewcommand{\th}{\ensuremath{{}^\text{th}} }
  \newcommand{\lag}{\mathcal{L}}
  \newcommand{\inn}{\rfloor}
  \newcommand{\lie}{\pounds}
  \newcommand{\longto}{\longrightarrow}
  \newcommand{\speer}{\parbox{0.4ex}{\raisebox{0.8ex}{$\nearrow$}}}
  \renewcommand{\dag}{ {}^\dagger }
  \newcommand{\blbox}{\rule{1ex}{1ex}}
  \newcommand{\Ji}{J^\sharp}
  \newcommand{\h}{{}^\star}
  \newcommand{\w}{\wedge}
  \newcommand{\too}{\longrightarrow}
  \newcommand{\oot}{\longleftarrow}
  \newcommand{\To}{\Rightarrow}
  \newcommand{\oT}{\Leftarrow}
  \newcommand{\oTo}{\Leftrightarrow}
  \renewcommand{\iff}{~\Leftrightarrow~}
  \newcommand{\Too}{\;\Longrightarrow\;}
  \newcommand{\oto}{\leftrightarrow}
  \newcommand{\ot}{\leftarrow}
  \newcommand{\ootoo}{\longleftrightarrow}
  \newcommand{\ow}{\stackrel{\circ}\wedge}
  \newcommand{\defeq}{\stackrel{\hspace{0.2ex}{}_\Delta}=}
%  \newcommand{\defeq}{{\overstack\Delta =}}
  \newcommand{\feed}{\nonumber \\}
  \newcommand{\comma}{~,\quad}
  \newcommand{\period}{~.\quad}
  \newcommand{\del}{\partial}
  \newcommand{\Del}[1]{\textstyle\frac{\del}{\del #1}}
  \newcommand{\Hes}[1]{\textstyle\frac{\del^2}{\del #1^2}}

%  \newcommand{\quabla}{\Delta}
  \newcommand{\point}{$\bullet~~$}
  \newcommand{\doubletilde}{ ~ \raisebox{0.3ex}{$\widetilde {}$} \raisebox{0.6ex}{$\widetilde {}$} \!\! }
  \newcommand{\topcirc}{\parbox{0ex}{~\raisebox{2.5ex}{${}^\circ$}}}
  \newcommand{\topdot} {\parbox{0ex}{~\raisebox{2.5ex}{$\cdot$}}}
  \newcommand{\topddot} {\parbox{0ex}{~\raisebox{1.3ex}{$\ddot{~}$}}}
  \newcommand{\sym}{\topcirc}
  \newcommand{\tsum}{\textstyle\sum}
  \newcommand{\st}{~~\text{s.t.}~~}

  \newcommand{\half}{\ensuremath{\frac{1}{2}}}
  \newcommand{\third}{\ensuremath{\frac{1}{3}}}
  \newcommand{\fourth}{\ensuremath{\frac{1}{4}}}

  \newcommand{\ubar}{\underline}
  %\renewcommand{\vec}{\underline}
  \renewcommand{\vec}{\boldsymbol}
  %\renewcommand{\_}{\underset}
  %\renewcommand{\^}{\overset}
  %\renewcommand{\*}{{\rm\raisebox{-.6ex}{\text{*}}{}}}
  \renewcommand{\*}{\text{\footnotesize\raisebox{-.4ex}{*}{}}}

  \newcommand{\gto}{{\raisebox{.5ex}{${}_\rightarrow$}}}
  \newcommand{\gfrom}{{\raisebox{.5ex}{${}_\leftarrow$}}}
  \newcommand{\gnto}{{\raisebox{.5ex}{${}_\nrightarrow$}}}
  \newcommand{\gnfrom}{{\raisebox{.5ex}{${}_\nleftarrow$}}}

  %\newcommand{\RND}{{\SS}}
  %\newcommand{\IF}{\text{if }}
  %\newcommand{\AND}{\textsc{and }}
  %\newcommand{\OR}{\textsc{or }}
  %\newcommand{\XOR}{\textsc{xor }}
  %\newcommand{\NOT}{\textsc{not }}

  %\newcommand{\argmax}[1]{{\rm arg}\!\max_{#1}}
  %\newcommand{\argmin}[1]{{\rm arg}\!\min_{#1}}
  \DeclareMathOperator*{\argmax}{argmax}
  \DeclareMathOperator*{\argmin}{argmin}
  \DeclareMathOperator{\sign}{sign}
  \DeclareMathOperator{\acos}{acos}
  \DeclareMathOperator{\unifies}{unifies}
  \DeclareMathOperator{\Span}{span}
  \newcommand{\ortho}{\perp}
  %\newcommand{\argmax}[1]{\underset{~#1}{\text{argmax}}\;}
  %\newcommand{\argmin}[1]{\underset{~#1}{\text{argmin}}\;}
  \newcommand{\ee}[1]{\ensuremath{\cdot10^{#1}}}
  \newcommand{\sub}[1]{\ensuremath{_{\text{#1}}}}
  \newcommand{\up}[1]{\ensuremath{^{\text{#1}}}}
  \newcommand{\kld}[3][{}]{D_{#1}\big(#2\,\big|\!\big|\,#3\big)}
  %\newcommand{\kld}[2]{D\big(#1:#2\big)}
  \newcommand{\sprod}[2]{\big<#1\,,\,#2\big>}
  \newcommand{\End}{\text{End}}
  \newcommand{\txt}[1]{\quad\text{#1}\quad}
  \newcommand{\Over}[2]{\genfrac{}{}{0pt}{0}{#1}{#2}}
  %\newcommand{\mat}[1]{{\bf #1}}
  \newcommand{\arr}[2]{\hspace*{-.5ex}\begin{array}{#1}#2\end{array}\hspace*{-.5ex}}
  \newcommand{\mat}[3][.9]{
    \renewcommand{\arraystretch}{#1}{\scriptscriptstyle{\left(
      \hspace*{-1ex}\begin{array}{#2}#3\end{array}\hspace*{-1ex}
    \right)}}\renewcommand{\arraystretch}{1.2}
  }
  \newcommand{\Mat}[3][.9]{
    \renewcommand{\arraystretch}{#1}{\scriptscriptstyle{\left[
      \hspace*{-1ex}\begin{array}{#2}#3\end{array}\hspace*{-1ex}
    \right]}}\renewcommand{\arraystretch}{1.2}
  }
  \newcommand{\case}[2][ll]{\left\{\arr{#1}{#2}\right.}
  \newcommand{\seq}[1]{\textsf{\<#1\>}}
  \newcommand{\seqq}[1]{\textsf{#1}}
  \newcommand{\floor}[1]{\lfloor#1\rfloor}
  \newcommand{\Exp}[2][]{\mathbb{E}_{#1}\!\left\{#2\right\}}
  \newcommand{\Expno}[1][]{\mathbb{E}_{#1}}
  \newcommand{\Var}[2][]{\text{Var}_{#1}\{#2\}}
  \newcommand{\cov}[2][]{\text{cov}_{#1}\{#2\}}

  %\newcommand{\Exp}[2]{\left\langle{#2}\right\rangle_{#1}}
  \newcommand{\ex}{\setminus}

  \providecommand{\href}[2]{{\color{blue}USE PDFLATEX!}}
  \providecommand{\url}[2]{\href{#1}{{\color{blue}#2}}}
%  \newcommand{\link}[1]{\href{{\protect #1}}{\texttt{\protect #1}}}
  \newcommand{\anchor}[2]{\begin{picture}(0,0)\put(#1){#2}\end{picture}}
  \newcommand{\pagebox}{\begin{picture}(0,0)\put(-3,-23){
    \textcolor[rgb]{.5,1,.5}{\framebox[\textwidth]{\rule[-\textheight]{0pt}{0pt}}}}
    \end{picture}}

  \newcommand{\hide}[1]{
    \begin{list}{}{\leftmargin0ex \rightmargin0ex \topsep0ex \parsep0ex}
       \helvetica{5}{1}{m}{n}
       \renewcommand{\section}{\par SECTION: }
       \renewcommand{\subsection}{\par SUBSECTION: }
       \item[$~~\blacktriangleright$]
       #1%$\blacktriangleleft~~$
       \message{^^JHIDE--Warning!^^J}
    \end{list}
  }
  %\newcommand{\hide}[1]{{\tt[hide:~}{\footnotesize\sf #1}{\tt]}\message{^^JHIDE--Warning!^^J}}
  \newcommand{\Hide}{\renewcommand{\hide}[1]{\message{^^JHIDE--Warning (hidden)!^^J}}}
  \newcommand{\HIDE}{\renewcommand{\hide}[1]{}}
  \newcommand{\fullhide}[1]{\message{^^JHIDE--Warning (hidden)!^^J}}
  %\newcommand{\todo}[1]{{\tt[TODO: #1]}\message{^^JTODO--Warning: #1^^J}}
  %\newcommand{\Todo}{\renewcommand{\todo}[1]{\message{^^JTODO--Warning (hidden)!^^J}}}
  %\renewcommand{\title}[1]{\renewcommand{\thetitle}{#1}}
  \newcommand{\myauthor}[1]{\author{#1}\newcommand{\theauthor}{#1}}%\@author}
  \newcommand{\mytitle}[1]{\title{#1}\newcommand{\thetitle}{#1}}%\@title}
  \newcommand{\header}{\begin{document}\mytitle\cleardefs}
  \newcommand{\contents}{{\tableofcontents}\renewcommand{\contents}{}}
  \newcommand{\footer}{\small\bibliography{marc,bibs}\end{document}}
  \newcommand{\widepaper}{\usepackage{geometry}\geometry{a4paper,hdivide={25mm,*,25mm},vdivide={25mm,*,25mm}}}
  \newcommand{\moviex}[2]{\movie[externalviewer]{#1}{#2}} %\pdflatex\usepackage{multimedia}
  \newcommand{\rbox}[1]{\fboxrule2mm\fcolorbox[rgb]{1,.85,.85}{1,.85,.85}{#1}}
  \newcommand{\mpage}[2]{{\begin{minipage}{#1\columnwidth}#2\end{minipage}}}
  \newcommand{\redbox}[2]{\fboxrule1mm\fcolorbox[rgb]{1,.7,.7}{1,.7,.7}{\begin{minipage}{#1\columnwidth}\center#2\end{minipage}}}
  \newcommand{\onecol}[2]{
    \begin{minipage}[c]{#1\columnwidth}#2\end{minipage}}
  \newcommand{\twocol}[5][0]{
    \begin{minipage}[c]{#2\columnwidth}#4\end{minipage}\hspace*{#1\columnwidth}%
    \begin{minipage}[c]{#3\columnwidth}#5\end{minipage}}
  \newcommand{\threecol}[7][0]{%
    \begin{minipage}[c]{#2\columnwidth}#5\end{minipage}\hspace*{#1\columnwidth}%
    \begin{minipage}[c]{#3\columnwidth}#6\end{minipage}\hspace*{#1\columnwidth}%
    \begin{minipage}[c]{#4\columnwidth}#7\end{minipage}}
  \newcommand{\threecoltext}[7][c]{
    \begin{minipage}[#1]{#2\textwidth}#5\end{minipage}%
    \begin{minipage}[#1]{#3\textwidth}#6\end{minipage}%
    \begin{minipage}[#1]{#4\textwidth}#7\end{minipage}}
  \newcommand{\threecoltop}[7][0]{%
   \begin{minipage}[t]{#2\columnwidth}#5\end{minipage}\hspace*{#1\columnwidth}%
   \begin{minipage}[t]{#3\columnwidth}#6\end{minipage}\hspace*{#1\columnwidth}%
   \begin{minipage}[t]{#4\columnwidth}#7\end{minipage}}
  \newcommand{\fourcol}[9][0]{%
   \begin{minipage}[c]{#2\columnwidth}#6\end{minipage}\hspace*{#1\columnwidth}%
   \begin{minipage}[c]{#3\columnwidth}#7\end{minipage}\hspace*{#1\columnwidth}%
   \begin{minipage}[c]{#4\columnwidth}#8\end{minipage}\hspace*{#1\columnwidth}%
   \begin{minipage}[c]{#5\columnwidth}#9\end{minipage}}
  \newcommand{\helvetica}[4]{\setlength{\unitlength}{1pt}\fontsize{#1}{#1}\linespread{#2}\usefont{OT1}{phv}{#3}{#4}}
  \newcommand{\helve}[1]{\helvetica{#1}{1.5}{m}{n}}
  \newcommand{\german}{\usepackage[german]{babel}\usepackage[utf8]{inputenc}}
\newcommand{\sans}{\renewcommand{\familydefault}{\sfdefault}}

\newcommand{\norm}[1]{|\!|#1|\!|}
\newcommand{\expr}[1]{[\hspace{-.2ex}[#1]\hspace{-.2ex}]}

\newcommand{\Jwi}{J^\sharp_W}
\newcommand{\THi}{T^\sharp_H}
\newcommand{\Jci}{J^\natural_C}
\newcommand{\hJi}{{\bar J}^\sharp}
\renewcommand{\|}{\,|\,}
\renewcommand{\=}{\!=\!}
\newcommand{\myminus}{{\hspace*{-.0pt}\text{\rm -}\hspace*{-.5pt}}}
\newcommand{\myplus}{{\hspace*{-.0pt}\text{\rm +}\hspace*{-.5pt}}}
\newcommand{\1}{{\myminus1}}
\newcommand{\2}{{\myminus2}}
\newcommand{\3}{{\myminus3}}
\newcommand{\mT}{{\text{\rm -}\hspace*{-1pt}\top}}
\newcommand{\po}{{\myplus1}}
\newcommand{\pt}{{\myplus2}}
%\renewcommand{\-}{\myminus}
%\newcommand{\+}{\myplus}
\newcommand{\T}{{\!\top\!}}
%\newcommand{\T}{{\!\textsf{\scriptsize T}\!}}
\newcommand{\xT}{{\underline x}}
\newcommand{\uT}{{\underline u}}
\newcommand{\zT}{{\underline z}}
\newcommand{\Sum}{\textstyle\sum}
\newcommand{\Int}{\textstyle\int}
\newcommand{\Frac}{\textstyle\frac}
\newcommand{\Max}{\textstyle\max}
\newcommand{\Prod}{\textstyle\prod}


\newenvironment{centy}{
\vspace{15mm}
\large
\hspace*{5mm}
\begin{minipage}{8cm}\it\color{blue}
}{
\end{minipage}
}

\newcommand{\old}{{\text{old}}}
\newcommand{\new}{{\text{new}}}
\newcommand{\MAP}{{\text{MAP}}}
\newcommand{\ML}{{\text{ML}}}

\newcommand{\redArrow}{\quad\anchor{0,-1}{\includegraphics[scale=.5]{figs/redArrow}}}
\newcommand{\pub}[1]{{\color{green}\helvetica{8}{1.3}{m}{n}#1\\}}
\DeclareMathOperator{\opKL}{KL}
\newcommand{\KL}[2]{\opKL\big(#1\,\big|\!\big|\,#2\big)} %\left(#1 |\!| #2\right)}

\renewcommand{\show}[2][.7]{\centerline{\includegraphics[width=#1\columnwidth]{#2}}}
\newcommand{\showh}[2][.8]{\includegraphics[width=#1\columnwidth]{#2}}
\newcommand{\shows}[2][.8]{\centerline{\includegraphics[scale=#1]{#2}}}
\newcommand{\showhs}[2][.8]{\includegraphics[scale=#1]{#2}}
\newcommand{\mov}[2]{\movie[externalviewer]{{\color{blue}\small #1}}{movies/#2}}
\newcommand{\movex}[2]{\movie[externalviewer]{#1}{#2}} %\pdflatex\usepackage{multimedia}
%\newcommand{\movgb}[1]{\hfill\movie[externalviewer]{\small[movie]}{/home/mtoussai/movies/10-goalDirectedBehavior/#1}}
\newcommand{\movh}[3][autostart,loop]{
\movie[#1]{\showh[#2]{#3.jpg}}{#3.mp4}%
\movie[externalviewer]{$\circ$}{#3.mp4}
}
\newcommand{\movc}[3][autostart,loop]{\centerline{\movh[#1]{#2}{#3}}}
\newcommand{\mymovie}[3][]{\centerline{\movie[autostart,loop,#1]{\includegraphics[#1]{#2}}{#3}}}


\newcommand{\cen}[1]{\centerline{#1}}

\newcommand{\citing}[1]{
{\color{citcol}\ttiny#1\par}
}

\newcommand{\cit}[3]{
\par\smallskip
{\color{citcol}\ttiny #1: \emph{#2}. #3 \par}
}

\newcommand{\citurl}[4]{
\par\smallskip
{\color{citcol}\ttiny #1: \protect{\href{#4}{\color{blue}{#2.}}} #3 \par}
}

\newcommand{\cito}[3]{
\par\smallskip
{\color{bluecol}\ttiny #1: \emph{#2}. #3 \par}
}

\newcommand{\redoMacrosInProof}{
  \renewcommand{\d}{\delta}
%  \renewcommand{\|}{\,|\,}
  \renewcommand{\=}{\!=\!}
}

\newcommand{\pdflatex}{
  \definecolor{bluecol}{rgb}{0,0,.5}
  %\definecolor{greencol}{rgb}{0,.4,0}
  \usepackage[
    %    pdftex%,
    %%    letterpaper,
    %bookmarks,
    %bookmarksnumbered,
    colorlinks,
    urlcolor=bluecol,
    citecolor=black,
    linkcolor=bluecol,
    %    pagecolor=bluecol,
    pdfborder={0 0 0},
    %pdfborderstyle={/S/U/W 1},
    %%    backref,     %link from bibliography back to sections
    %%    pagebackref, %link from bibliography back to pages
    %%    pdfstartview=FitH, %fitwidth instead of fit window
    pdfpagemode=UseOutlines, %bookmarks are displayed by acrobat
    %%    pdftitle={\thetitle},
    pdfauthor={Marc Toussaint}
  ]{hyperref}
  \DeclareGraphicsExtensions{.pdf,.png,.jpg,.eps}
  \renewcommand{\r}{\varrho}
  \renewcommand{\l}{\lambda}
  \renewcommand{\L}{\Lambda}
  \renewcommand{\s}{\sigma}
  \renewcommand{\b}{\beta}
  \renewcommand{\d}{\delta}
  \renewcommand{\k}{\kappa}
  \renewcommand{\t}{\theta}
  \renewcommand{\O}{\Omega}
  \renewcommand{\o}{\omega}
  \renewcommand{\SS}{{\cal S}}
  \renewcommand{\=}{\!=\!}
  %\renewcommand{\boldsymbol}{}
  %\renewcommand{\Chapter}{\chapter}
  %\renewcommand{\Subsection}{\subsection}
}

%% \makeatletter
%% \newenvironment{code}{%
%%   \begin{lrbox}{\@tempboxa}\begin{minipage}{1\columnwidth}\codefont
%% }{
%%   \end{minipage}\end{lrbox}%
%%   \colorbox[rgb]{.95,.95,.95}{\usebox{\@tempboxa}}
%% }\makeatother

\newenvironment{code}{%
\codefont
\begin{shaded}
}{
\end{shaded}
}

\newcommand{\signature}{
  \includegraphics[height=6ex]{signatureColor} %unterschrift2}
}

\usepackage{algorithm}
\usepackage{algpseudocode}
\algrenewcommand{\algorithmicrequire}{\textbf{Input:~~}}
\algrenewcommand{\algorithmicensure}{\textbf{Output:}}
\algrenewcommand{\algorithmiccomment}[1]{\qquad\hfill~\hspace*{-5ex}\textit{// #1}}
\algrenewcommand{\alglinenumber}[1]{\helvetica{6}{1.3}{m}{n}#1:}

\newenvironment{algo}[1][8]{
\quad\begin{minipage}{.8\columnwidth}\helvetica{#1}{1.4}{m}{n}
\medskip\hrule\medskip
\begin{algorithmic}[1]
}{
\end{algorithmic}
\medskip\hrule\medskip
\end{minipage}
}

\usepackage{etoolbox}

%%%%%%%%%%%%%%%%%%%%%%%%%%%%%%%%%%%%%%%%%%%%%%%%%%%%%%%%%%%%%%%%%%%%%%%%%%%%%%%%

\usepackage{multirow}
\usepackage{colortbl}
%\setlength{\jot}{0pt}
%\setlength{\mathindent}{1ex}
\usepackage{empheq}

%%%%%%%%%%%%%%%%%%%%%%%%%%%%%%%%%%%%%%%%%%%%%%%%%%%%%%%%%%%%%%%%%%%%%%%%%%%%%%%

\newcommand{\mypause}{\pause}
%\newcommand{\dom}{{\text{dom}}}
\newcommand{\defi}[1]{\textbf{#1}}
\newcommand{\red}[1]{\emph{\color{red}#1}}
%\newcommand{\ul}{\underline}
\newcommand{\pos}{{\textsf{pos}}}
\newcommand{\eff}{{\textsf{eff}}}
\newcommand{\rot}{{\textsf{rot}}}
\newcommand{\veC}{{\textsf{vec}}}
\newcommand{\quat}{{\textsf{quat}}}
\newcommand{\col}{{\textsf{col}}}
\newcommand{\de}[2]{\frac{\partial #1}{\partial #2}}
\newcommand{\target}{{\text{target}}}
\newcommand{\near}{{\text{near}}}
\newcommand{\qfree}{Q_{\text{free}}}
\renewcommand{\vec}{\boldsymbol}
\newcommand{\lft}{\text{left}}
\newcommand{\rgh}{\text{right}}
\DeclareMathOperator{\real}{real}
\newcommand{\prev}{{\text{prev}}}
\newcommand{\TR}[2]{T_{{#1}\to{#2}}}
\newcommand{\RO}[2]{R_{{#1}\to{#2}}}
\newcommand{\liter}{\helvetica{8}{1.1}{m}{n}\parskip 1ex}
\newcommand{\Fc}{\color{green}F}
\newcommand{\muc}{\color{blue}\mu}
\newcommand{\Astar}{A$^*$}

%for AI course:
\providecommand{\defn}[1]{\textbf{#1}}

%for optimization course:
\newcommand{\adec}{\r_\a^-}
\newcommand{\ainc}{\r_\a^+}
\newcommand{\ldec}{\r_\l^-}
\newcommand{\linc}{\r_\l^+}
\newcommand{\minc}{\r_\m^+}
\newcommand{\mdec}{\r_\m^-}
\newcommand{\step}{{\delta}}
\newcommand{\lsstop}{\r_{\text{ls}}}
\newcommand{\stepmax}{\step_{\text{max}}}


\definecolor{boxcol}{rgb}{.85,.9,.92}
\newcommand{\eqbox}[1]{\centerline{\fboxrule0mm\fcolorbox{boxcol}{boxcol}{#1}}}
\newcommand{\movgb}[1]{\hfill\movie[externalviewer]{\small[movie]}{/home/mtoussai/movies/10-goalDirectedBehavior/#1}}
\newcommand{\demo}[1]{{{\color{blue}[\small #1]}}}

\graphicspath{{../pics-robotics/}{../pics-ML/}{../pics-all/}{../pics-all2/}{../pics-Optim/}}
\DeclareGraphicsExtensions{.pdf,.png,.jpg}

%\usepackage{pdfpages}
%\setbeamercolor{background canvas}{bg=}

\newcommand{\SUM}{\texttt{sum}}
\usepackage{float}

%% prevent pagebreaks before environment
\makeatletter
\newcommand{\NewParNoBreak}[1][\parskip]{\par\vspace*{-\parskip}\vspace*{#1}\nobreak\@afterheading}
\makeatother

%\newcommand{\idx}[2]{\label{IKgn}}

%%%%%%%%%%%%%%%%%%%%%%%%%%%%%%%%%%%%%%%%%%%%%%%%%%%%%%%%%%%%%%%%%%%%%%%%%%%%%%%%



%% \newwrite\tempfile
%% \immediate\openout\tempfile=z.keys.tex

%% \renewcommand{\key}[1]{
%% %%   \addtocounter{mypage}{1}
%% \makeatletter
%% \immediate\write\tempfile{\symbol{`\\}}
%% \makeatother
%%   \immediate\write\tempfile{hyperref[key:#1]{#1(\arabic{mypage})}}
%% %%  % \phantomsection\label{key:#1}
%% %%   %\index{#1@{\hyperref[key:#1]{#1 (\arabic{mysec}:\arabic{mypage})}}|phantom}
%% %%   \addtocounter{mypage}{-1}
%% }


  \usepackage{comment}
  \specialcomment{solution}{
    \small
    \begin{shaded}
  }{
    \end{shaded}
  }

  \graphicspath{{pics/}{../pics/}{pics-local/}}

  \mytitle{\course\\Lecture Script}
  \myauthor{Marc Toussaint}
  \date{\coursedate}
}

%%%%%%%%%%%%%%%%%%%%%%%%%%%%%%%%%%%%%%%%%%%%%%%%%%%%%%%%%%%%%%%%%%%%%%%%%%%%%%%%

\newcommand{\scripttitle}{
  \begin{document}
  \maketitle
  %\anchor{100,10}{\includegraphics[width=4cm]{optim}}
%  \vspace*{1cm}
}

%%%%%%%%%%%%%%%%%%%%%%%%%%%%%%%%%%%%%%%%%%%%%%%%%%%%%%%%%%%%%%%%%%%%%%%%%%%%%%%%

\newcounter{mypage}
\setcounter{mypage}{0}
\newcommand{\incpage}{\addtocounter{mypage}{1}}

\newcommand{\subtopic}{}
\newcommand{\pause}{}
\newcommand{\only}[1]{#1}

\renewcommand{\slides}[1][]{
  %  \clearpage
  \subsection{\topic}
  \index{\topic}
  {\small #1}
  \setcounter{mypage}{0}
  \smallskip\nopagebreak\hrule\medskip
}

\newcommand{\slidesfoot}{
  \bigskip
}

\newcommand{\sublecture}[2]{
  \phantomsection\addcontentsline{toc}{subsubsection}{#1}
  \index{#1}
}

\newcommand{\sublectureHide}[2]{
  \renewcommand{\subtopic}{#1}
}

\newcommand{\key}[1]{
  \phantomsection\addcontentsline{toc}{subsubsection}{#1}
  %\subsubsection{#1}
  \index{#1}
}

\providecommand{\defn}[1]{%
  \textbf{#1}\index{#1}%
}

\newenvironment{slidecore}[1]{
  \incpage
  \subsubsection*{#1}%{\headerfont\noindent\textbf{#1}\\}%
  \vspace{-6ex}%
  \begin{list}{$\bullet$}{\leftmargin4ex \rightmargin0ex \labelsep1ex
    \labelwidth2ex \partopsep0ex \topsep0ex \parsep.5ex \parskip0ex \itemsep0pt}\item[]~\\\nopagebreak%
}{
  \end{list}\nopagebreak%
  {\hfill\tiny \textsf{\arabic{section}.\arabic{subsection}:\arabic{mypage}}}\nopagebreak%
  \smallskip\nopagebreak\hrule
}

\newcommand{\slide}[2]{
  \begin{slidecore}{#1}
    #2
  \end{slidecore}
}

\renewcommand{\exercises}{}
\newcommand{\exercisestitle}{}
\newcommand{\exsection}[1]{\subsubsection{#1}}
\newcommand{\exsubsection}[1]{\paragraph{#1}}
\newcommand{\exerfoot}{\bigskip}

\newcommand{\story}[1]{
  \subsection*{Motivation \& Outline}
  {\storyfont\sf #1}
  \medskip\nopagebreak\hrule
}

\newcounter{savedsection}
\newcommand{\subappendix}{\setcounter{savedsection}{\arabic{section}}\appendix}
\newcommand{\noappendix}{
  \setcounter{section}{\arabic{savedsection}}% restore section number
  \setcounter{subsection}{0}% reset section counter
%  \gdef\@chapapp{\sectionname}% reset section name
  \renewcommand{\thesection}{\arabic{section}}% make section numbers arabic
}

\newenvironment{items}[1][9]{
\par\setlength{\unitlength}{1pt}\fontsize{#1}{#1}\linespread{1.2}\selectfont
\begin{list}{--}{\leftmargin4ex \rightmargin0ex \labelsep1ex \labelwidth2ex
\topsep0pt \parsep0ex \itemsep3pt}
}{
\end{list}
}

\newenvironment{itemS}[1][4ex]{
\par
\tiny
\begin{list}{--}{\leftmargin#1 \rightmargin0ex \labelsep1ex
  \labelwidth2ex \topsep0pt \parsep0ex \itemsep2pt}
}{
\end{list}
}

\newcommand{\Def}[1]{%
\textbf{#1}\index{#1}}%\marginnote{#1}}

  \scripthead
}

\providecommand{\paper}{
  \input{../latex/style-paper}
  \paperhead
}

\providecommand{\note}{
  \providecommand{\notehead}[2]{
  \documentclass[#1,fleqn,twoside]{article}
  \stdpackages
  \renewcommand{\labelenumi}{{(\roman{enumi})}}
  \renewcommand{\theenumi}{(\roman{enumi})} %for ref

  \renewcommand{\baselinestretch}{#2}
  \renewcommand{\arraystretch}{1.2}
  \renewcommand{\topfraction}{1}
  \renewcommand{\bottomfraction}{1}
  \renewcommand{\textfraction}{0}
  \columnsep 5ex
  \parindent 3ex
  \parskip 1ex

  % Lists and paragraphs
  \parindent 0pt
  \topsep 4pt plus 1pt minus 2pt
  \partopsep 1pt plus 0.5pt minus 0.5pt
  \itemsep 2pt plus 1pt minus 0.5pt
  \parsep 2pt plus 1pt minus 0.5pt
  \parskip .5pc %add _in_ {thebibliography} environment in *.bbl

  \setcounter{tocdepth}{3}
  \setcounter{secnumdepth}{3}

  \geometry{a4paper,hdivide={25mm,*,25mm},vdivide={25mm,*,25mm}}

  \renewcommand{\headrulewidth}{.0pt}\renewcommand{\footrulewidth}{.0pt}\cfoot{}
  \fancyhead[OL,EC]{\it\theauthor---\today}
  \fancyhead[ER]{\leftmark}
  \fancyhead[OR,EL]{\thepage}
  \fancyfoot[EL,OR]{}
  \setlength{\headsep}{10mm}
  %\fancyhead[OL]{\rightmark}
  %\fancyfoot[EL,OR]{}


  %  \usepackage{palatino}

  \newcommand{\codefont}{\helvetica{8}{1.2}{m}{n}}

  \renewenvironment{abstract}{
    \vspace*{5ex}\begin{list}{}{
      \leftmargin3ex
      \rightmargin3ex
      \topsep-\parskip}\item[]
     \hrule\vspace{1.5ex}{\bf Abstract.~}\small}
    {\vspace{2ex}\hrule\end{list}\vspace{5ex}}
    
  \newenvironment{keyword}
    {\par{\it Keywords:~}}
    {}

  \def\makemytitle{%
    \thispagestyle{empty}
    \begin{list}{}{\leftmargin3ex \rightmargin3ex \topsep0ex \parsep0ex}\item[]
      \begin{center}
        {\fontsize{18}{25}\selectfont{\thetitle\\}}\vspace{5ex}

        {\fontsize{14}{16}\selectfont{\theauthor\\}}\vspace{1ex}

        {\footnotesize{\sl \addressFUB}\\ \emailBerlin}

        {\footnotesize \today}

        \vspace{1ex}
        {\small \published}
      \end{center}
    \end{list}
    \renewcommand{\maketitle}{\chapter{\thetitle}}
  }

  %auto-ignore
  \renewcommand{\a}{\alpha}
  \renewcommand{\b}{\beta}
  \renewcommand{\d}{\delta}
    \newcommand{\D}{\Delta}
    \newcommand{\e}{\epsilon}
    \newcommand{\g}{\gamma}
    \newcommand{\G}{\Gamma}
  \renewcommand{\l}{\lambda}
  \renewcommand{\L}{\Lambda}
    \newcommand{\m}{\mu}
    \newcommand{\n}{\nu}
    \newcommand{\N}{\nabla}
  \renewcommand{\k}{\kappa}
  \renewcommand{\o}{\omega}
  \renewcommand{\O}{\Omega}
    \newcommand{\p}{\phi}
    \newcommand{\ph}{\varphi}
  \renewcommand{\P}{\Phi}
  \renewcommand{\r}{\varrho}
    \newcommand{\s}{\sigma}
  \renewcommand{\S}{\Sigma}
  \renewcommand{\t}{\theta}
    \newcommand{\Th}{\Theta}
  %\renewcommand{\v}{\vartheta}
    \newcommand{\x}{\xi}
    \newcommand{\X}{\Xi}
    \newcommand{\Y}{\Upsilon}
    \newcommand{\z}{\zeta}

  \renewcommand{\AA}{{\cal A}}
    \newcommand{\BB}{{\cal B}}
    \newcommand{\CC}{{\cal C}}
    \newcommand{\cc}{{\cal c}}
    \newcommand{\DD}{{\cal D}}
    \newcommand{\EE}{{\cal E}}
    \newcommand{\FF}{{\cal F}}
    \newcommand{\GG}{{\cal G}}
    \newcommand{\HH}{{\cal H}}
    \newcommand{\II}{{\cal I}}
    \newcommand{\KK}{{\cal K}}
    \newcommand{\LL}{{\cal L}}
    \newcommand{\MM}{{\cal M}}
    \newcommand{\NN}{{\cal N}}
    \newcommand{\oNN}{\overline\NN}
    \newcommand{\OO}{{\cal O}}
    \newcommand{\PP}{{\cal P}}
    \newcommand{\QQ}{{\cal Q}}
    \newcommand{\RR}{{\cal R}}
  \renewcommand{\SS}{{\cal S}}
    \newcommand{\TT}{{\cal T}}
    \newcommand{\uu}{{\cal u}}
    \newcommand{\UU}{{\cal U}}
    \newcommand{\VV}{{\cal V}}
    \newcommand{\XX}{{\cal X}}
    \newcommand{\xx}{\mathcal{x}}
    \newcommand{\YY}{{\cal Y}}
    \newcommand{\ZZ}{{\cal Z}}
    \newcommand{\SOSO}{{\cal SO}}
    \newcommand{\GLGL}{{\cal GL}}

    \newcommand{\Ee}{{\rm E}}

  \newcommand{\NNN}{{\mathbb{N}}}
  \newcommand{\III}{{\mathbb{I}}}
  \newcommand{\ZZZ}{{\mathbb{Z}}}
  %\newcommand{\RRR}{{\mathrm{I\!R}}}
  \newcommand{\RRR}{{\mathbb{R}}}
  \newcommand{\SSS}{{\mathbb{S}}}
  \newcommand{\EEE}{{\mathbb{E}}}
  \newcommand{\VVV}{{\mathbb{V}}}
  \newcommand{\CCC}{{\mathbb{C}}}
  \newcommand{\DDD}{{\mathbb{D}}}
  \newcommand{\one}{{{\bf 1}}}
  \newcommand{\eee}{\text{e}}

  \newcommand{\NNNN}{{\overline{\cal N}}}

  \renewcommand{\[}{\Big[}
  \renewcommand{\]}{\Big]}
  \renewcommand{\(}{\Big(}
  \renewcommand{\)}{\Big)}
  \renewcommand{\|}{\,|\,}
  \renewcommand{\;}{\,;\,}
  \renewcommand{\=}{\!=\!}
    \newcommand{\<}{\left\langle}
  \renewcommand{\>}{\right\rangle}

  \newcommand{\na}[1][]{{\nabla_{\!\!#1}}}
  \newcommand{\he}[1][]{{\nabla_{\!\!#1}^2}}
  \newcommand{\Prob}{{\rm Prob}}
  \newcommand{\Dir}{{\rm Dir}}
  \newcommand{\Beta}{{\rm Beta}}
  \newcommand{\Bern}{{\rm Bern}}
  \newcommand{\Bin}{{\rm Bin}}
  \newcommand{\Mult}{{\rm Mult}}
  \newcommand{\Aut}{{\rm Aut}}
  \newcommand{\cor}{{\rm cor}}
  \newcommand{\corr}{{\rm corr}}
  \newcommand{\sd}{{\rm sd}}
  \newcommand{\tr}{{\rm tr}}
  \newcommand{\Tr}{{\rm Tr}}
  \newcommand{\rank}{{\rm rank}}
  \newcommand{\diag}{{\rm diag}}
  \newcommand{\dom}{{\rm dom}}
  \newcommand{\id}{{\rm id}}
  \newcommand{\Id}{{\rm\bf I}}
  \newcommand{\Gl}{{\rm Gl}}
  \renewcommand{\th}{\ensuremath{{}^\text{th}} }
  \newcommand{\lag}{\mathcal{L}}
  \newcommand{\inn}{\rfloor}
  \newcommand{\lie}{\pounds}
  \newcommand{\longto}{\longrightarrow}
  \newcommand{\speer}{\parbox{0.4ex}{\raisebox{0.8ex}{$\nearrow$}}}
  \renewcommand{\dag}{ {}^\dagger }
  \newcommand{\blbox}{\rule{1ex}{1ex}}
  \newcommand{\Ji}{J^\sharp}
  \newcommand{\h}{{}^\star}
  \newcommand{\w}{\wedge}
  \newcommand{\too}{\longrightarrow}
  \newcommand{\oot}{\longleftarrow}
  \newcommand{\To}{\Rightarrow}
  \newcommand{\oT}{\Leftarrow}
  \newcommand{\oTo}{\Leftrightarrow}
  \renewcommand{\iff}{~\Leftrightarrow~}
  \newcommand{\Too}{\;\Longrightarrow\;}
  \newcommand{\oto}{\leftrightarrow}
  \newcommand{\ot}{\leftarrow}
  \newcommand{\ootoo}{\longleftrightarrow}
  \newcommand{\ow}{\stackrel{\circ}\wedge}
  \newcommand{\defeq}{\stackrel{\hspace{0.2ex}{}_\Delta}=}
%  \newcommand{\defeq}{{\overstack\Delta =}}
  \newcommand{\feed}{\nonumber \\}
  \newcommand{\comma}{~,\quad}
  \newcommand{\period}{~.\quad}
  \newcommand{\del}{\partial}
  \newcommand{\Del}[1]{\textstyle\frac{\del}{\del #1}}
  \newcommand{\Hes}[1]{\textstyle\frac{\del^2}{\del #1^2}}

%  \newcommand{\quabla}{\Delta}
  \newcommand{\point}{$\bullet~~$}
  \newcommand{\doubletilde}{ ~ \raisebox{0.3ex}{$\widetilde {}$} \raisebox{0.6ex}{$\widetilde {}$} \!\! }
  \newcommand{\topcirc}{\parbox{0ex}{~\raisebox{2.5ex}{${}^\circ$}}}
  \newcommand{\topdot} {\parbox{0ex}{~\raisebox{2.5ex}{$\cdot$}}}
  \newcommand{\topddot} {\parbox{0ex}{~\raisebox{1.3ex}{$\ddot{~}$}}}
  \newcommand{\sym}{\topcirc}
  \newcommand{\tsum}{\textstyle\sum}
  \newcommand{\st}{~~\text{s.t.}~~}

  \newcommand{\half}{\ensuremath{\frac{1}{2}}}
  \newcommand{\third}{\ensuremath{\frac{1}{3}}}
  \newcommand{\fourth}{\ensuremath{\frac{1}{4}}}

  \newcommand{\ubar}{\underline}
  %\renewcommand{\vec}{\underline}
  \renewcommand{\vec}{\boldsymbol}
  %\renewcommand{\_}{\underset}
  %\renewcommand{\^}{\overset}
  %\renewcommand{\*}{{\rm\raisebox{-.6ex}{\text{*}}{}}}
  \renewcommand{\*}{\text{\footnotesize\raisebox{-.4ex}{*}{}}}

  \newcommand{\gto}{{\raisebox{.5ex}{${}_\rightarrow$}}}
  \newcommand{\gfrom}{{\raisebox{.5ex}{${}_\leftarrow$}}}
  \newcommand{\gnto}{{\raisebox{.5ex}{${}_\nrightarrow$}}}
  \newcommand{\gnfrom}{{\raisebox{.5ex}{${}_\nleftarrow$}}}

  %\newcommand{\RND}{{\SS}}
  %\newcommand{\IF}{\text{if }}
  %\newcommand{\AND}{\textsc{and }}
  %\newcommand{\OR}{\textsc{or }}
  %\newcommand{\XOR}{\textsc{xor }}
  %\newcommand{\NOT}{\textsc{not }}

  %\newcommand{\argmax}[1]{{\rm arg}\!\max_{#1}}
  %\newcommand{\argmin}[1]{{\rm arg}\!\min_{#1}}
  \DeclareMathOperator*{\argmax}{argmax}
  \DeclareMathOperator*{\argmin}{argmin}
  \DeclareMathOperator{\sign}{sign}
  \DeclareMathOperator{\acos}{acos}
  \DeclareMathOperator{\unifies}{unifies}
  \DeclareMathOperator{\Span}{span}
  \newcommand{\ortho}{\perp}
  %\newcommand{\argmax}[1]{\underset{~#1}{\text{argmax}}\;}
  %\newcommand{\argmin}[1]{\underset{~#1}{\text{argmin}}\;}
  \newcommand{\ee}[1]{\ensuremath{\cdot10^{#1}}}
  \newcommand{\sub}[1]{\ensuremath{_{\text{#1}}}}
  \newcommand{\up}[1]{\ensuremath{^{\text{#1}}}}
  \newcommand{\kld}[3][{}]{D_{#1}\big(#2\,\big|\!\big|\,#3\big)}
  %\newcommand{\kld}[2]{D\big(#1:#2\big)}
  \newcommand{\sprod}[2]{\big<#1\,,\,#2\big>}
  \newcommand{\End}{\text{End}}
  \newcommand{\txt}[1]{\quad\text{#1}\quad}
  \newcommand{\Over}[2]{\genfrac{}{}{0pt}{0}{#1}{#2}}
  %\newcommand{\mat}[1]{{\bf #1}}
  \newcommand{\arr}[2]{\hspace*{-.5ex}\begin{array}{#1}#2\end{array}\hspace*{-.5ex}}
  \newcommand{\mat}[3][.9]{
    \renewcommand{\arraystretch}{#1}{\scriptscriptstyle{\left(
      \hspace*{-1ex}\begin{array}{#2}#3\end{array}\hspace*{-1ex}
    \right)}}\renewcommand{\arraystretch}{1.2}
  }
  \newcommand{\Mat}[3][.9]{
    \renewcommand{\arraystretch}{#1}{\scriptscriptstyle{\left[
      \hspace*{-1ex}\begin{array}{#2}#3\end{array}\hspace*{-1ex}
    \right]}}\renewcommand{\arraystretch}{1.2}
  }
  \newcommand{\case}[2][ll]{\left\{\arr{#1}{#2}\right.}
  \newcommand{\seq}[1]{\textsf{\<#1\>}}
  \newcommand{\seqq}[1]{\textsf{#1}}
  \newcommand{\floor}[1]{\lfloor#1\rfloor}
  \newcommand{\Exp}[2][]{\mathbb{E}_{#1}\!\left\{#2\right\}}
  \newcommand{\Expno}[1][]{\mathbb{E}_{#1}}
  \newcommand{\Var}[2][]{\text{Var}_{#1}\{#2\}}
  \newcommand{\cov}[2][]{\text{cov}_{#1}\{#2\}}

  %\newcommand{\Exp}[2]{\left\langle{#2}\right\rangle_{#1}}
  \newcommand{\ex}{\setminus}

  \providecommand{\href}[2]{{\color{blue}USE PDFLATEX!}}
  \providecommand{\url}[2]{\href{#1}{{\color{blue}#2}}}
%  \newcommand{\link}[1]{\href{{\protect #1}}{\texttt{\protect #1}}}
  \newcommand{\anchor}[2]{\begin{picture}(0,0)\put(#1){#2}\end{picture}}
  \newcommand{\pagebox}{\begin{picture}(0,0)\put(-3,-23){
    \textcolor[rgb]{.5,1,.5}{\framebox[\textwidth]{\rule[-\textheight]{0pt}{0pt}}}}
    \end{picture}}

  \newcommand{\hide}[1]{
    \begin{list}{}{\leftmargin0ex \rightmargin0ex \topsep0ex \parsep0ex}
       \helvetica{5}{1}{m}{n}
       \renewcommand{\section}{\par SECTION: }
       \renewcommand{\subsection}{\par SUBSECTION: }
       \item[$~~\blacktriangleright$]
       #1%$\blacktriangleleft~~$
       \message{^^JHIDE--Warning!^^J}
    \end{list}
  }
  %\newcommand{\hide}[1]{{\tt[hide:~}{\footnotesize\sf #1}{\tt]}\message{^^JHIDE--Warning!^^J}}
  \newcommand{\Hide}{\renewcommand{\hide}[1]{\message{^^JHIDE--Warning (hidden)!^^J}}}
  \newcommand{\HIDE}{\renewcommand{\hide}[1]{}}
  \newcommand{\fullhide}[1]{\message{^^JHIDE--Warning (hidden)!^^J}}
  %\newcommand{\todo}[1]{{\tt[TODO: #1]}\message{^^JTODO--Warning: #1^^J}}
  %\newcommand{\Todo}{\renewcommand{\todo}[1]{\message{^^JTODO--Warning (hidden)!^^J}}}
  %\renewcommand{\title}[1]{\renewcommand{\thetitle}{#1}}
  \newcommand{\myauthor}[1]{\author{#1}\newcommand{\theauthor}{#1}}%\@author}
  \newcommand{\mytitle}[1]{\title{#1}\newcommand{\thetitle}{#1}}%\@title}
  \newcommand{\header}{\begin{document}\mytitle\cleardefs}
  \newcommand{\contents}{{\tableofcontents}\renewcommand{\contents}{}}
  \newcommand{\footer}{\small\bibliography{marc,bibs}\end{document}}
  \newcommand{\widepaper}{\usepackage{geometry}\geometry{a4paper,hdivide={25mm,*,25mm},vdivide={25mm,*,25mm}}}
  \newcommand{\moviex}[2]{\movie[externalviewer]{#1}{#2}} %\pdflatex\usepackage{multimedia}
  \newcommand{\rbox}[1]{\fboxrule2mm\fcolorbox[rgb]{1,.85,.85}{1,.85,.85}{#1}}
  \newcommand{\mpage}[2]{{\begin{minipage}{#1\columnwidth}#2\end{minipage}}}
  \newcommand{\redbox}[2]{\fboxrule1mm\fcolorbox[rgb]{1,.7,.7}{1,.7,.7}{\begin{minipage}{#1\columnwidth}\center#2\end{minipage}}}
  \newcommand{\onecol}[2]{
    \begin{minipage}[c]{#1\columnwidth}#2\end{minipage}}
  \newcommand{\twocol}[5][0]{
    \begin{minipage}[c]{#2\columnwidth}#4\end{minipage}\hspace*{#1\columnwidth}%
    \begin{minipage}[c]{#3\columnwidth}#5\end{minipage}}
  \newcommand{\threecol}[7][0]{%
    \begin{minipage}[c]{#2\columnwidth}#5\end{minipage}\hspace*{#1\columnwidth}%
    \begin{minipage}[c]{#3\columnwidth}#6\end{minipage}\hspace*{#1\columnwidth}%
    \begin{minipage}[c]{#4\columnwidth}#7\end{minipage}}
  \newcommand{\threecoltext}[7][c]{
    \begin{minipage}[#1]{#2\textwidth}#5\end{minipage}%
    \begin{minipage}[#1]{#3\textwidth}#6\end{minipage}%
    \begin{minipage}[#1]{#4\textwidth}#7\end{minipage}}
  \newcommand{\threecoltop}[7][0]{%
   \begin{minipage}[t]{#2\columnwidth}#5\end{minipage}\hspace*{#1\columnwidth}%
   \begin{minipage}[t]{#3\columnwidth}#6\end{minipage}\hspace*{#1\columnwidth}%
   \begin{minipage}[t]{#4\columnwidth}#7\end{minipage}}
  \newcommand{\fourcol}[9][0]{%
   \begin{minipage}[c]{#2\columnwidth}#6\end{minipage}\hspace*{#1\columnwidth}%
   \begin{minipage}[c]{#3\columnwidth}#7\end{minipage}\hspace*{#1\columnwidth}%
   \begin{minipage}[c]{#4\columnwidth}#8\end{minipage}\hspace*{#1\columnwidth}%
   \begin{minipage}[c]{#5\columnwidth}#9\end{minipage}}
  \newcommand{\helvetica}[4]{\setlength{\unitlength}{1pt}\fontsize{#1}{#1}\linespread{#2}\usefont{OT1}{phv}{#3}{#4}}
  \newcommand{\helve}[1]{\helvetica{#1}{1.5}{m}{n}}
  \newcommand{\german}{\usepackage[german]{babel}\usepackage[utf8]{inputenc}}
\newcommand{\sans}{\renewcommand{\familydefault}{\sfdefault}}

\newcommand{\norm}[1]{|\!|#1|\!|}
\newcommand{\expr}[1]{[\hspace{-.2ex}[#1]\hspace{-.2ex}]}

\newcommand{\Jwi}{J^\sharp_W}
\newcommand{\THi}{T^\sharp_H}
\newcommand{\Jci}{J^\natural_C}
\newcommand{\hJi}{{\bar J}^\sharp}
\renewcommand{\|}{\,|\,}
\renewcommand{\=}{\!=\!}
\newcommand{\myminus}{{\hspace*{-.0pt}\text{\rm -}\hspace*{-.5pt}}}
\newcommand{\myplus}{{\hspace*{-.0pt}\text{\rm +}\hspace*{-.5pt}}}
\newcommand{\1}{{\myminus1}}
\newcommand{\2}{{\myminus2}}
\newcommand{\3}{{\myminus3}}
\newcommand{\mT}{{\text{\rm -}\hspace*{-1pt}\top}}
\newcommand{\po}{{\myplus1}}
\newcommand{\pt}{{\myplus2}}
%\renewcommand{\-}{\myminus}
%\newcommand{\+}{\myplus}
\newcommand{\T}{{\!\top\!}}
%\newcommand{\T}{{\!\textsf{\scriptsize T}\!}}
\newcommand{\xT}{{\underline x}}
\newcommand{\uT}{{\underline u}}
\newcommand{\zT}{{\underline z}}
\newcommand{\Sum}{\textstyle\sum}
\newcommand{\Int}{\textstyle\int}
\newcommand{\Frac}{\textstyle\frac}
\newcommand{\Max}{\textstyle\max}
\newcommand{\Prod}{\textstyle\prod}


\newenvironment{centy}{
\vspace{15mm}
\large
\hspace*{5mm}
\begin{minipage}{8cm}\it\color{blue}
}{
\end{minipage}
}

\newcommand{\old}{{\text{old}}}
\newcommand{\new}{{\text{new}}}
\newcommand{\MAP}{{\text{MAP}}}
\newcommand{\ML}{{\text{ML}}}

\newcommand{\redArrow}{\quad\anchor{0,-1}{\includegraphics[scale=.5]{figs/redArrow}}}
\newcommand{\pub}[1]{{\color{green}\helvetica{8}{1.3}{m}{n}#1\\}}
\DeclareMathOperator{\opKL}{KL}
\newcommand{\KL}[2]{\opKL\big(#1\,\big|\!\big|\,#2\big)} %\left(#1 |\!| #2\right)}

\renewcommand{\show}[2][.7]{\centerline{\includegraphics[width=#1\columnwidth]{#2}}}
\newcommand{\showh}[2][.8]{\includegraphics[width=#1\columnwidth]{#2}}
\newcommand{\shows}[2][.8]{\centerline{\includegraphics[scale=#1]{#2}}}
\newcommand{\showhs}[2][.8]{\includegraphics[scale=#1]{#2}}
\newcommand{\mov}[2]{\movie[externalviewer]{{\color{blue}\small #1}}{movies/#2}}
\newcommand{\movex}[2]{\movie[externalviewer]{#1}{#2}} %\pdflatex\usepackage{multimedia}
%\newcommand{\movgb}[1]{\hfill\movie[externalviewer]{\small[movie]}{/home/mtoussai/movies/10-goalDirectedBehavior/#1}}
\newcommand{\movh}[3][autostart,loop]{
\movie[#1]{\showh[#2]{#3.jpg}}{#3.mp4}%
\movie[externalviewer]{$\circ$}{#3.mp4}
}
\newcommand{\movc}[3][autostart,loop]{\centerline{\movh[#1]{#2}{#3}}}
\newcommand{\mymovie}[3][]{\centerline{\movie[autostart,loop,#1]{\includegraphics[#1]{#2}}{#3}}}


\newcommand{\cen}[1]{\centerline{#1}}

\newcommand{\citing}[1]{
{\color{citcol}\ttiny#1\par}
}

\newcommand{\cit}[3]{
\par\smallskip
{\color{citcol}\ttiny #1: \emph{#2}. #3 \par}
}

\newcommand{\citurl}[4]{
\par\smallskip
{\color{citcol}\ttiny #1: \protect{\href{#4}{\color{blue}{#2.}}} #3 \par}
}

\newcommand{\cito}[3]{
\par\smallskip
{\color{bluecol}\ttiny #1: \emph{#2}. #3 \par}
}

\newcommand{\redoMacrosInProof}{
  \renewcommand{\d}{\delta}
%  \renewcommand{\|}{\,|\,}
  \renewcommand{\=}{\!=\!}
}

\newcommand{\pdflatex}{
  \definecolor{bluecol}{rgb}{0,0,.5}
  %\definecolor{greencol}{rgb}{0,.4,0}
  \usepackage[
    %    pdftex%,
    %%    letterpaper,
    %bookmarks,
    %bookmarksnumbered,
    colorlinks,
    urlcolor=bluecol,
    citecolor=black,
    linkcolor=bluecol,
    %    pagecolor=bluecol,
    pdfborder={0 0 0},
    %pdfborderstyle={/S/U/W 1},
    %%    backref,     %link from bibliography back to sections
    %%    pagebackref, %link from bibliography back to pages
    %%    pdfstartview=FitH, %fitwidth instead of fit window
    pdfpagemode=UseOutlines, %bookmarks are displayed by acrobat
    %%    pdftitle={\thetitle},
    pdfauthor={Marc Toussaint}
  ]{hyperref}
  \DeclareGraphicsExtensions{.pdf,.png,.jpg,.eps}
  \renewcommand{\r}{\varrho}
  \renewcommand{\l}{\lambda}
  \renewcommand{\L}{\Lambda}
  \renewcommand{\s}{\sigma}
  \renewcommand{\b}{\beta}
  \renewcommand{\d}{\delta}
  \renewcommand{\k}{\kappa}
  \renewcommand{\t}{\theta}
  \renewcommand{\O}{\Omega}
  \renewcommand{\o}{\omega}
  \renewcommand{\SS}{{\cal S}}
  \renewcommand{\=}{\!=\!}
  %\renewcommand{\boldsymbol}{}
  %\renewcommand{\Chapter}{\chapter}
  %\renewcommand{\Subsection}{\subsection}
}

%% \makeatletter
%% \newenvironment{code}{%
%%   \begin{lrbox}{\@tempboxa}\begin{minipage}{1\columnwidth}\codefont
%% }{
%%   \end{minipage}\end{lrbox}%
%%   \colorbox[rgb]{.95,.95,.95}{\usebox{\@tempboxa}}
%% }\makeatother

\newenvironment{code}{%
\codefont
\begin{shaded}
}{
\end{shaded}
}

\newcommand{\signature}{
  \includegraphics[height=6ex]{signatureColor} %unterschrift2}
}

\usepackage{algorithm}
\usepackage{algpseudocode}
\algrenewcommand{\algorithmicrequire}{\textbf{Input:~~}}
\algrenewcommand{\algorithmicensure}{\textbf{Output:}}
\algrenewcommand{\algorithmiccomment}[1]{\qquad\hfill~\hspace*{-5ex}\textit{// #1}}
\algrenewcommand{\alglinenumber}[1]{\helvetica{6}{1.3}{m}{n}#1:}

\newenvironment{algo}[1][8]{
\quad\begin{minipage}{.8\columnwidth}\helvetica{#1}{1.4}{m}{n}
\medskip\hrule\medskip
\begin{algorithmic}[1]
}{
\end{algorithmic}
\medskip\hrule\medskip
\end{minipage}
}

\usepackage{etoolbox}

%%%%%%%%%%%%%%%%%%%%%%%%%%%%%%%%%%%%%%%%%%%%%%%%%%%%%%%%%%%%%%%%%%%%%%%%%%%%%%%%

\usepackage{multirow}
\usepackage{colortbl}
%\setlength{\jot}{0pt}
%\setlength{\mathindent}{1ex}
\usepackage{empheq}

%%%%%%%%%%%%%%%%%%%%%%%%%%%%%%%%%%%%%%%%%%%%%%%%%%%%%%%%%%%%%%%%%%%%%%%%%%%%%%%

\newcommand{\mypause}{\pause}
%\newcommand{\dom}{{\text{dom}}}
\newcommand{\defi}[1]{\textbf{#1}}
\newcommand{\red}[1]{\emph{\color{red}#1}}
%\newcommand{\ul}{\underline}
\newcommand{\pos}{{\textsf{pos}}}
\newcommand{\eff}{{\textsf{eff}}}
\newcommand{\rot}{{\textsf{rot}}}
\newcommand{\veC}{{\textsf{vec}}}
\newcommand{\quat}{{\textsf{quat}}}
\newcommand{\col}{{\textsf{col}}}
\newcommand{\de}[2]{\frac{\partial #1}{\partial #2}}
\newcommand{\target}{{\text{target}}}
\newcommand{\near}{{\text{near}}}
\newcommand{\qfree}{Q_{\text{free}}}
\renewcommand{\vec}{\boldsymbol}
\newcommand{\lft}{\text{left}}
\newcommand{\rgh}{\text{right}}
\DeclareMathOperator{\real}{real}
\newcommand{\prev}{{\text{prev}}}
\newcommand{\TR}[2]{T_{{#1}\to{#2}}}
\newcommand{\RO}[2]{R_{{#1}\to{#2}}}
\newcommand{\liter}{\helvetica{8}{1.1}{m}{n}\parskip 1ex}
\newcommand{\Fc}{\color{green}F}
\newcommand{\muc}{\color{blue}\mu}
\newcommand{\Astar}{A$^*$}

%for AI course:
\providecommand{\defn}[1]{\textbf{#1}}

%for optimization course:
\newcommand{\adec}{\r_\a^-}
\newcommand{\ainc}{\r_\a^+}
\newcommand{\ldec}{\r_\l^-}
\newcommand{\linc}{\r_\l^+}
\newcommand{\minc}{\r_\m^+}
\newcommand{\mdec}{\r_\m^-}
\newcommand{\step}{{\delta}}
\newcommand{\lsstop}{\r_{\text{ls}}}
\newcommand{\stepmax}{\step_{\text{max}}}


\definecolor{boxcol}{rgb}{.85,.9,.92}
\newcommand{\eqbox}[1]{\centerline{\fboxrule0mm\fcolorbox{boxcol}{boxcol}{#1}}}
\newcommand{\movgb}[1]{\hfill\movie[externalviewer]{\small[movie]}{/home/mtoussai/movies/10-goalDirectedBehavior/#1}}
\newcommand{\demo}[1]{{{\color{blue}[\small #1]}}}

\graphicspath{{../pics-robotics/}{../pics-ML/}{../pics-all/}{../pics-all2/}{../pics-Optim/}}
\DeclareGraphicsExtensions{.pdf,.png,.jpg}

%\usepackage{pdfpages}
%\setbeamercolor{background canvas}{bg=}

\newcommand{\SUM}{\texttt{sum}}
\usepackage{float}

%% prevent pagebreaks before environment
\makeatletter
\newcommand{\NewParNoBreak}[1][\parskip]{\par\vspace*{-\parskip}\vspace*{#1}\nobreak\@afterheading}
\makeatother

%\newcommand{\idx}[2]{\label{IKgn}}

%%%%%%%%%%%%%%%%%%%%%%%%%%%%%%%%%%%%%%%%%%%%%%%%%%%%%%%%%%%%%%%%%%%%%%%%%%%%%%%%



%% \newwrite\tempfile
%% \immediate\openout\tempfile=z.keys.tex

%% \renewcommand{\key}[1]{
%% %%   \addtocounter{mypage}{1}
%% \makeatletter
%% \immediate\write\tempfile{\symbol{`\\}}
%% \makeatother
%%   \immediate\write\tempfile{hyperref[key:#1]{#1(\arabic{mypage})}}
%% %%  % \phantomsection\label{key:#1}
%% %%   %\index{#1@{\hyperref[key:#1]{#1 (\arabic{mysec}:\arabic{mypage})}}|phantom}
%% %%   \addtocounter{mypage}{-1}
%% }

  \pdflatex

  \graphicspath{{pics/}{../pics/}{pics-local/}}

  \myauthor{Marc Toussaint}
  \date{\today}
}

%%%%%%%%%%%%%%%%%%%%%%%%%%%%%%%%%%%%%%%%%%%%%%%%%%%%%%%%%%%%%%%%%%%%%%%%%%%%%%%%

\newcommand{\notetitle}{
  \begin{document}
  \thispagestyle{empty}
    
  \maketitle

}

\newenvironment{items}[1][9]{
\par\setlength{\unitlength}{1pt}\fontsize{#1}{#1}\linespread{1.2}\selectfont
\begin{list}{--}{\leftmargin4ex \rightmargin0ex \labelsep1ex \labelwidth2ex
\topsep0pt \parsep0ex \itemsep3pt}
}{
\end{list}
}

  \notehead{10}{1.1}
}

\providecommand{\course}{NO COURSE}
\providecommand{\coursepicture}{NO PICTURE}
\providecommand{\coursedate}{NO DATE}
\providecommand{\topic}{NO TOPIC}
\providecommand{\keywords}{NO KEYWORDS}
\providecommand{\exnum}{NO NUMBER}
\providecommand{\teacher}{Marc Toussaint}

\providecommand{\stdpackages}{
  \usepackage{amsmath}
  \usepackage{amssymb}
  \usepackage{amsfonts}
  \allowdisplaybreaks
  \usepackage{amsthm}
  \usepackage{eucal}
  \usepackage{graphicx}
%  \usepackage{color}
  \usepackage{geometry}
  \usepackage{framed}
  \usepackage{xcolor}
  \definecolor{shadecolor}{gray}{0.9}
  \setlength{\FrameSep}{3pt}
  \usepackage{fancyvrb}
  \fvset{numbers=none,xleftmargin=5ex,fontsize=\small}

  \usepackage{pdfpages}

  \usepackage{multicol} 
  \usepackage{fancyhdr}
}

\providecommand{\addressUSTT}{
  Machine~Learning~\&~Robotics~lab, U~Stuttgart\\\small
  Universit{\"a}tsstra{\ss}e 38, 70569~Stuttgart, Germany
}

\providecommand{\addressTUB}{
  Learning~\&~Intelligent~Systems~Lab, TU~Berlin\\\small
  Marchstr. 23, 10587 Berlin, Germany
}

\note

\title{Lecture Note:\\ Splines}
\author{Marc Toussaint\\\small Learning \& Intelligent Systems Lab, TU Berlin}

\makeatletter
\renewcommand{\@seccntformat}[1]{}
\makeatother

\newcommand{\des}{\text{des}}
\renewcommand{\Ref}{\text{ref}}
\renewcommand{\t}{\theta}
\renewcommand{\hat}{\widehat}
\newcommand{\bbg}{{\bar{\bar g}}}

\notetitle

%%%%%%%%%%%%%%%%%%%%%%%%%%%%%%%%%%%%%%%%%%%%%%%%%%%%%%%%%%%%%%%%%%%%%%%%%%%%%%%%

A spline is a piece-wise polynomial path $x:[0,T] \to \RRR^n$. Let's
first clearly distinguish the use of words \emph{knot},
\emph{waypoint}, and \emph{control point}:
\begin{itemize}
\item A \defn{knot} $t_i$ is a point in \emph{time}, $t_i \in \RRR$,
  we assume $t_i \in [0,T]$. For a spline, we have a non-decreasing
  sequence of knots $t_0,..,t_m$ (we assume $t_0=0$ and $t_m=T$) which
  partition the time interval $[0,T]$ into pieces $[t_i, t_{i\po}]$ so
  that the path is polynomial in each piece. Note that we may often
  have double or triple knots, meaning that several consecutive knots
  $t_i = t_{i\po}$ are equal, especially at the beginning and end.
\item A \defn{waypoint} $x_i$ is a point on the path, typically at a
  knot, $x_i = x(t_i)$. So a path really passes through a waypoint. At
  waypoints, we often also care about velocities $v_i$ and
  accelerations $\a_i$, where $v_i = \dot x(t_i)$, $a_i = \ddot x(t_i)$, .
\item A \defn{control point} $z_j$ is (usually) not a point \emph{on}
  the path, but it indirectly defines the path as an linear
  combination of several control points. B-splines, defined below,
  make this explicit.
\end{itemize}

In robotics, there are two main conventions to define and parameterize splines: Hermite splines and B-splines. Hermite splines are defined by the knot sequence and explicitly prescribing waypoints $x_i$ and (for cubic) velocities $v_i$ at each knot (for quintic also acceperations $a_i$). In contrast, B-splines are specified by the knot sequence and $K$ control points $z_j$. As in B-splines we do not need to provide velocities as part of the specification, they are sometimes easier to use in practice. However, the resulting path does not go (exactly) through the provided control points -- the actual waypoints are implicit and ensuring exact prescribed waypoints implies solving a subproblem.

Cubic splines are a common choice in robotics, as they have a still continuous (piece-wise linear) acceleration profile, and therefore limited jerk (3rd time derivative).

In the following we first discuss a single cubic spline-piece as a means of control, then Hermite splines, then B-splines.

\subsection{Single cubic spline for timing-optimal control to a target}

The following discusses a single cubic spline piece and and how to use it for timing-optimal control to a target. Although very simple, the method is a powerful alternative to typical PD-control to a target. It also lays foundations on how timing-optimality can be realized with Hermite splines.

Consider a cubic polynomial $x(t) = a t^3 + b t^2 + c t + d$. Given four boundary conditions $x(0)=x_0, \dot x(0) = v_0, x(\t) = x_\t, \dot x(\t) = v_\t$, the four coefficients are

$$\begin{align}
d &= x_0 ~, \\
c &= \dot x_0 ~, \\
b &= \frac{1}{\t^2}\[ 3(x_\t-x_0) - \t(\dot x_\t + 2 \dot x_0) \] ~, \\
a &= \frac{1}{\t^3}\[ - 2(x_0-x_\t) + \t(\dot x_\t + \dot x_0) \] ~.
\end{align}$$

This cubic spline is in fact the solution to an optimization problem, namely it is the path that minimizes accelerations between these boundary conditions and it can therefore be viewed as the solution to optimal control with acceleration costs:

$$\begin{align}
\min_x~ \int_0^\tau \ddot x(t)^2~ dt 
\quad\st \mat{c}{x(0)\\\dot x(0)}=\mat{c}{x_0\\v_0},~
\mat{c}{x(\tau)\\\dot x(\tau)}=\mat{c}{x_1\\v_1} ~.
\end{align}$$

%% Check with Maxima:
%% \begin{code}
%% \begin{verbatim}
%% x(t) :=a*t^3 +b*t^2 +c*t +d;
%% define(v(t),diff(x(t),t,1));
%% cons: [x(0) = x_0,v(0) = v_0, x(T)=x_T, v(T)=v_T];
%% sol: solve (cons, [a,b,c,d]);
%% \end{verbatim}
%% \end{code}
The minimal costs can analytically be given as

$$\begin{align}
\int_0^T \ddot x(t)^2 dt
%% &= \int_0^T (6 a t + 2 b)^2 ~ dt \\
%% &= \int_0^T (36 a^2 t^2 + 4 b^2 + 24 abt) ~ dt \\
%% &= 4 b^2 [t]_0^T + 24 ab [\half t^2]_0^T + 36 a^2 [\frac{1}{3} t^3]_0^T \\
&= 4 \tau b^2  + 12 \tau^2 ab + 12 \tau^3 a^2 \\
&= \frac{12}{\tau^3}~[(x_1 - x_0)-\frac{\tau}{2}(v_0+v_1)]^2+\frac{1}{\tau}(v_1-v_0)^2 \label{eqLeap}\\
&= \frac{12}{\tau^3} D^\T D + \frac{1}{\tau} V^\T V \comma D := (x_1 - x_0)-\frac{\tau}{2}(v_0+v_1),~ V:=v_1-v_0,~ \\
&= \tilde D^\T \tilde D + \tilde V^\T \tilde V \comma
\tilde D := \sqrt{12}~ \tau^{-\frac{3}{2}}~ D,~ \tilde V := \tau^{-\half}~ V ~,
\label{eqLeapSOS}
\end{align}$$

where we used some help of computer algebra to get this right.

Eq. $\eqref{eqLeap}$ explicitly gives the optimal cost in terms of
boundary conditions $(x_0,v_0,x_1,v_1)$ and time $\tau$. This is a very powerful means to optimize boundary conditions and $\tau$. The following is a simple application that realizes reactive control.

\subsubsection{Single-piece optimal timing control}

Consider the system is in state $(x,\dot x)$ and you want to control it to a reference point $(x_\Ref, \dot x_\Ref=0)$. An obvious approach would be to use a PD law $\ddot x_\des = k_p (x_\Ref - x) + k_d (\dot x_\Ref - \dot x)$ and translate $\ddot x_\des$ to controls using inverse dynamics. By choosing $k_p$ and $k_d$ appropriately one can generate any desired damped/oscillatory-exponential approach behavior.

However, while PD laws are fundamental for low-level optimal control under noise (e.g. as result of the Riccati equation), they are actually not great for generating more macroscopic approach behavior: They are ``only'' exponentially converging, never really reaching the target in a definite time, never providing a clear expected time-to-target. And accelerations seem intuitively too large when far from the set point, and too small when close. (Which is why many heuristics, such as capped PD laws were proposed.)

Instead of imposing a desired PD behavior, we can impose a desired cubic spline behavior, which leads to succinct convergence in a finite expected time-to-target, as well as moderate gains when far. The approach is simply to choose an optimal $\tau$ (time-to-target) that minimizes

$$\begin{align}
\min_{\tau, x}~ \a \tau + \int_0^\tau \ddot x(t)^2~ dt
\end{align}$$

under our boundary conditions, assuming a cubic spline $x(t), t\in[0,\tau]$. Using $\eqref{eqLeap}$, we know the optimal $x$ and optimal control costs for given $\tau$. When $\d = x_\Ref - x$ and $v$ are co-linear (i.e., the system moves towards the target), computer algebra can tell us the optimal $\tau$:

$$\begin{align}\label{eqTimingControl}
  \tau^* = \frac{1}{\a}\[ \sqrt{6 |\d| \a + v^2} - |v| \] ~.
\end{align}$$

If the system has a lateral movement, the analytical solution seems overly complex, but a numerical solution to the least-squares form $\eqref{eqLeapSOS}$ very efficient. However, in practise, using $\eqref{eqTimingControl}$ with scalar  $v \gets (\d^\T v)/|\d|$ for easy timing control of convergence to a target is highly efficient and versatile.

To make this a reactive control scheme, in each control cycle
$$\tau^*$$ is reevaluated and the corresponding cubic spline reference send to
low-level control. If there are no perturbations, the estimated
$\tau^*$ will be the true time-to-target. See the SecMPD paper for details and comparision to PD approach behavior.

%% \begin{align}
%%   \min_\tau~
%%  & \a \tau
%%   + \frac{12}{\tau^3}~[\d-\frac{\tau}{2}v]^2
%%   + \frac{1}{\tau}~ v^2
%%   = \a \tau
%%   + \frac{12}{\tau^3}~ \d^2
%%   - 2 \frac{12}{\tau^3}~\frac{\tau}{2}~ \d^\T v
%%   + \frac{12}{\tau^3}~ \frac{\tau^2}{4}~ v^2
%%   + \frac{1}{\tau}~ v^2  \\
%%  &= \a \tau
%%   + \frac{12}{\tau^3}~ \d^2
%%   - \frac{12}{\tau^2}~ \d^\T v
%%   + \frac{4}{\tau}~ v^2  \\
%%   \del_\tau[\cdot]
%%   &= \a
%%   - \frac{36}{\tau^4}~ \d^2
%%   + \frac{24}{\tau^3}~ \d^\T v
%%   - \frac{4}{\tau^2}~ v^2  \quad \overset{!}= 0\\
%% \end{align}$$


\subsection{Hermite Cubic Splines}

A Hermite cubic spline is specified by the series of non-decreasing
time knots, $t_0,..,t_m \in [0,T]$, $t_0=0, t_m=T$, and the waypoints
$x_i$ \emph{and velocities} $v_i$ at each time knot. There are not
double knots, so the interval $[0,T]$ is split in $m$ cubic pieces,
where the $i$th piece is determined by the boundary conditions
$(x_{i\1},v_{i\1}, x_i, v_i)$ and $\tau_i = t_i - t_{i\1}$.

Specifying the timings (i.e., knots) and velocities of all waypoints is often not easy in terms of a user interface. Therefore the question is whether a series of given weypoints can easily be augmented with optimal timings and waypoint velocities.

Further, since each piece respects boundary conditions, continuity in velocities is ensured. However, note that two pieces might have completely different
accelerations at their joining knots (from the left and the right), and therefore a freely specified Hermite cubic spline is discontinuous in acceleration (has infintite jerk). Conversely, requiring a path in  $\CC^2$ implies continuity constraints in acceleration at each knot. Over the full path, these are $(m\1) \cdot n$ constraints (in $\RRR^n$), which ``kill'' the degrees-of-freedom of all but the start and end velocity. Therefore, requiring continuous accelleration, the kots, waypoints and start/end velocity alone are sufficient to specify the spline -- but in practise the resulting waypoint velocities might not be quite desired, as they might ``go crazy'' when chaining forward the continuous acceleration constraint.

However, optimizing both, timing and waypoint velocities under out
optimal control objective is rather efficient and effective. Note that
the optimal control cost over the full spline is just the sum of
single piece costs $\eqref{eqLeapSOS}$. This represents costs as a
least-squares of differentiable features, where $D$ can be interpreted
as distance to be covered by accelerations, and $V$ as necessary total
acceleration, and the Jacobians of $\tilde D$ and $\tilde V$
w.r.t.\ all boundary conditions and $\tau_i$ are trivial. Exploiting
the least-squares formulation of $\psi$ we can use the Gauss-Newton
approximate Hessian.

As a concequence, it is fairly efficient to solve for $\tau_{1:m}$, $v_{1:m\1}$ given $v_0, v_m, x_{0:m}$ under continuous acceleration constraints subject to total time and control costs.

As a final note, in Hermite quintic splines we need positions $x_i$, velocities $v_i$ and accelerations $a_i$ at each knot, which describe the quintic polynomial pieces between knots. The issues discussed above apply here analogously.

\subsection{B-Splines}

In B-splines, the path $x: [0,T] \to \RRR^n$ is expressed as a linear combination of control points $z_0,.., z_K \in \RRR^n$,

$$\begin{align}\label{bspline}
x(t) = \sum_{i=0}^K B_{i,p}(t)~ z_i ~,
\end{align}$$

where $B_{i,p}: \RRR \to \RRR$ maps the time $t$ to the weighting of the $i$th control point -- it blends in and out the $i$th control point. For any $t$ it  holds that $\sum_{i=0}^K B_{i,p}(t) = 1$, i.e., all the weights $B_{i,p}(t)$ sum to one (as with a probability distribution over $i$), and the path point $x(t)$ is therefore always in the convex hull of control points.

Concerning terminology, actually the functions $B_{i,p}(t)$ are called \defn{B-splines}, not the resulting path $x(t)$. (But in everyday robotics language, one often calls the path a B-spline.) As the linear (scalar) product in $\eqref{bspline}$ is trivial, the maths (and complexity of code) is all about the B-splines $B_{i,p}(t)$, not the path $x(t)$.

The B-spline functions $B_{i,p}(t)$ are fully specified by a non-decreasing series of time knots $t_0,..,t_m \in [0,T]$ and the integer degree $p\in\{0,1,..\}$. Namely, the recursive definition is

$$\begin{align}
B_{i,0}(t) &= [t_i \le t < t_{i\po}] \comma \text{for $0\le i \le m-1$} ~,\\
B_{i,p}(t)
&= \frac{t-t_i}{t_{i+p}-t_i}~ B_{i,p-1}(t)
 +  \frac{t_{i+p+1}-t}{t_{i+p+1}-t_{i+1}}~ B_{i+1,p-1}(t)  \comma \text{for $0\le i \le m-p-1$} ~.
\end{align}$$


The zero-degree B-spline functions $B_{i,0}$ are indicators of $t_i \le t < t_{i\po}$, and $i$ ranges from $i=0,..,m-1$. The 1st-degree B-spline functions $B_{i,1}$ have support in $t_i \le t < t_{i+2}$ and $i$ only ranges in
$i=0,..,m-2$ -- because one can show that the normalization $\sum_{i=0}^{m-2} B_{i,1}(t) = 1$ holds (and for $i>m-2$, the recursion would also not be clearly defined). In general, degree $p$ B-spline functions $B_{i,p}$ have support in $t_i \le t < t_{i+p+1}$ and $i$ ranges from $i=0,..,m+p-1$, which is why we need $K+1$ control points $z_{0:K}$ with

$$\begin{align}
    K = m+p-1 ~,
\end{align}$$

which ensures the normalization property $\sum_{i=0}^K B_{i,p}(t) = 1$ for every degree.

\begin{figure}
\show[.7]{Bsplines.svg}

\caption{
Illustration of B-spline functions for degrees $p=0,..,4$. Above each
plot of functions, a rough illustration of a resulting spline is
provided, where bullets indicate control points. Note that this
illustration implies a localization of control points in time, namely
roughly where the coresponding weighting function (B-spline function)
is highest -- but control points are formally not localized in time,
they are just being linearly combined, $x(t) = \sum_{i=0}^K
B_{i,p}(t)~ z_i$, with different weighting in time. However,
intuitively we can see that for odd degrees, the ``localization in
time'' of control points roughly aligns with knots, while for even
degrees the localization is between knots. Further, the illustrations
assume multi-knots at the start and end (namely $p\po$-fold knots),
which ensures that the spline starts with $z_0$ and ends with
$z_K$. Multiple equal control points $z_{0:p}$ and $z_{K-p:K}$
(illustrated with gray bars) are needed to ensure also zero
vel/acc/jerk at start and end.
}
\end{figure}

\subsubsection{B-spline Matrix for Time Discretized Paths}

Splines describe a continuous path $x(t)$, but often we want to evaluate this path only at a finite number of time slices $t\in \{\hat t_1,..,\hat t_S\} \subset [0,T]$. E.g., this could be a grid of $S=100$ time slices over which we want to optimize using KOMO, and for which we have to compute collision features. Let $x \in \RRR^{S \times n}$ be the time discretized path, and $z \in\RRR^{K\po\times n}$ be the stack of control points. Then the B-spline representation becomes

$$\begin{align}
x = B_p z \comma \text{with } B_p\in\RRR^{S\times K\po},~ B_{p,si} = B_{i,p}(\hat t_s) ~,
\end{align}$$

where $B_p$ is the B-spline matrix of degree $p$ for this particular time grid $\{\hat t_1,..,\hat t_S\}$.

So whenever we have a problem (e.g., NLP) defined over the fine resolution samples $x_s$, the B-spline matrix provides a linear re-parameterization and it is trivial to pull gradients (and Hessians) back to define a problem over $z$. In our code, KOMO defines NLPs over trajectories -- it is trivial to wrap this with a linear B-spline parameterization to then imply a much lower-dimensional NLP over the control points $z$.

\subsubsection{Ensuring B-splines pass through waypoints}

As we emphasized, the control point parameterization is not necessarily intuitive for a user, as the resulting path does not transition through control points. If a user provides a series of waypoints at desired times $\hat t_s$, how can we construct a B-spline to ensure transitioning through these waypoints at the desired times?

The answer is again the matrix equation. Consider the cubic spline case and that the start and end points and times are fixed. Therefore $z_{0:1}$ and $z_{K\1:K}$, as well as knots $t_{0:3}$ and $t_{m-3:m}$ are fixed. The user wants waypoints $x_1,..,x_S$ at times $\hat t_1,..,\hat t_S$ \emph{between} start and end.

We can distribute $S$ knots $t_{4:3+S}$ uniformly between start and end knots (or also at $\hat t_1,..,\hat t_S$), from which it follows we have $m = S+7$, and $K=m-p-1=S+3$, which are $K+1=S+4$ control points in total, of which $4$ are already fixed. So the $S$ middle control points are still free, and matrix inversion gives them from the desired waypoints,

$$\begin{align}
  z_{2:S+1} = B^\1 x_{1:S} \comma \text{with } B \in \RRR^{S \times S},~ B_{si} =  B_{i+1,3}(\hat t_s),~ s,i=1,..,S  ~.
  \end{align}$$



\subsubsection{Ensuring boundary velocities}

Consider again an online control situation where the is in state $(x,\dot x)$ and we want to steer it through future waypoints. In the B-spline representation we have to construct a spline that starts with current state as starting boundary.

For degrees 2 and 3 this is simple to achieve: In both cases we usually have $z_0=z_1$ and $z_{K\1}=z_K$ to ensure zero start and end velocities. Modifying $z_1$ directly leads to the start velocity $\dot x(0) = \dot B_{0,p}(0) z_0 + \dot B_{1,p}(0) z_1$. But because of normalization we have $\dot B_{0,p}(0) = - \dot B_{1,p}(0)$, and therefore 

$$\begin{align}
  \dot x(0) &= \dot B_{0,p}(0) (z_0 - z_1) \\
  z_1 &= z_0 - \frac{\dot x(0)}{\dot B_{0,p}(0)} ~.
\end{align}$$


%% A NURBS (non-uniform rational B-spline) is a B-spline
%% with weighted control points,
%% \begin{align}
%% f(t) = \sum_{i=0}^K \frac{w_i B_{i,p}(t)~ x_i}{\sum_{j=0}^K w_j B_{j,p}(t)} ~.
%% \end{align}$$


\subsubsection{Gradients}

The gradients of a B-spline represented path w.r.t.\ control points are trivial. But the gradients w.r.t.\ the knots are less trivial. Here the basic equations:

$$\begin{align}
B_{i,p}(t)
  &= \frac{t-t_i}{t_{i+p}-t_i} B_{i,p-1}(t)
 +  \frac{t_{i+p+1}-t}{t_{i+p+1}-t_{i+1}} B_{i+1,p-1}(t) \\
&=: v~ B_{i,p-1} + w~ B_{i+1,p-1} \\
\dot B_{i,p}(t)
 &= \frac{1}{t_{i\myplus{}p}-t_i}~ B_{i,p\1}(t)
 + v~ \dot B_{i,p\1}(t)
 - \frac{1}{t_{i\myplus{}p\po}-t_{i\po}}~ B_{i\po,p\1}(t)
 + w~ \dot B_{i\po,p\1}(t) \\
\del_{t_i} B_{i,p}
  &= \[\frac{-1}{t_{i+p}-t_i} + \frac{t-t_i}{(t_{i+p}-t_i)^2}\]~ B_{i,p-1}
   + v~ \del_{t_i} B_{i,p-1} + w~ \del_{t_i} B_{i+1,p-1} \\
  &= \[\frac{-1}{t-t_i} + \frac{1}{t_{i+p}-t_i} \]~ v~ B_{i,p-1} + v~ \del_{t_i} B_{i,p-1} + w~ \del_{t_i} B_{i+1,p-1} \\
\del_{t_{i+1}} B_{i,p}
  &= \[\frac{1}{t_{i+p+1}-t_{i+1}}\]~ w~ B_{i+1,p-1}
      + v~ \del_{t_{i+1}}~ B_{i,p-1} + w~ \del_{t_{i+1}}~ B_{i+1,p-1} \\
\del_{t_{i+p}} B_{i,p}
  &= \[- \frac{1}{t_{i+p}-t_{i}}\]~ v~ B_{i,p-1}
      + v~ \del_{t_{i+p}}~ B_{i,p-1} + w~ \del_{t_{i+p}}~ B_{i+1,p-1} \\
\del_{t_{i+p+1}} B_{i,p}
  &= \[\frac{1}{t_{i+p+1}-t} - \frac{1}{t_{i+p+1}-t_{i+1}}\]~ w~ B_{i+1,p-1}
      + v~ \del_{t_{i+p+1}}~ B_{i,p-1} + w~ \del_{t_{i+p+1}}~ B_{i+1,p-1}
\end{align}$$


\end{document}
