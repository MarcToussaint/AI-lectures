\providecommand{\slides}{
  \newcommand{\slideshead}{
  \newcommand{\thepage}{\arabic{mypage}}
  %beamer
%  \documentclass[t,hyperref={bookmarks=true}]{beamer}
%  \geometry{papersize={171mm,96mm}}
  \documentclass[t,hyperref={bookmarks=true},aspectratio=169]{beamer}
  \setbeamersize{text margin left=5mm}
  \setbeamersize{text margin right=5mm}
  \usetheme{default}
  \usefonttheme[onlymath]{serif}
  \setbeamertemplate{navigation symbols}{}
  \setbeamertemplate{itemize items}{{\color{black}$\bullet$}}

  \newwrite\keyfile

  %\usepackage{palatino}
  \stdpackages
  %\usepackage{tikz} \usetikzlibrary {shapes.geometric} 
  \usepackage{multimedia}
  \usepackage[utf8]{inputenc}

  %%% geometry/spacing issues
  %
  \definecolor{bluecol}{rgb}{0,0,.5}
  \definecolor{greencol}{rgb}{0,.6,0}
  \definecolor{citcol}{rgb}{.4,.4,.4}
  %\renewcommand{\baselinestretch}{1.1}
  \renewcommand{\arraystretch}{1.2}
  \columnsep 0mm

  \columnseprule 0pt
  \parindent 0ex
  \parskip 0ex
  %\setlength{\itemparsep}{3ex}
  %\renewcommand{\labelitemi}{\rule[3pt]{10pt}{10pt}~}
  %\renewcommand{\labelenumi}{\textbf{(\arabic{enumi})}}
  \setbeamertemplate{enumerate item}{(\roman{enumi})}
  \newcommand{\headerfont}{\helvetica{14}{1.5}{b}{n}}
  \newcommand{\slidefont} {\helvetica{11}{1.4}{m}{n}}
  %\newcommand{\codefont} {\helvetica{8}{1.2}{m}{n}}
  \newcommand{\urlfont} {\fontsize{6}{1.1}\selectfont}
  \renewcommand{\small} {\helvetica{10}{1.4}{m}{n}}
  \renewcommand{\tiny} {\helvetica{8}{1.3}{m}{n}}
  \newcommand{\ttiny} {\helvetica{7}{1.3}{m}{n}}

  %%% count pages properly and put the page number in bottom right
  %
  \newcounter{mypage}
  \newcommand{\incpage}{\addtocounter{mypage}{1}\setcounter{page}{\arabic{mypage}}}
  \setcounter{mypage}{0}
  \resetcounteronoverlays{page}

  \pagestyle{fancy}
  %\setlength{\headsep}{10mm}
  %\addtolength{\footheight}{15mm}
  \renewcommand{\headrulewidth}{0pt} %1pt}
  \renewcommand{\footrulewidth}{0pt} %.5pt}
  \cfoot{}
  \rhead{}
  \lhead{}
  %% \lfoot{\vspace*{-3mm}\hspace*{-3mm}\helvetica{5}{1.3}{m}{n}{\texttt{github.com/MarcToussaint/AI-lectures}}}
\lfoot{}
%  \rfoot{{\tiny\textsf{AI -- \topic -- \subtopic -- \arabic{mypage}/\pageref{lastpage}}}}
  %\lfoot{\raisebox{5mm}{\tiny\textsf{\slideauthor}}}
  %\rfoot{\raisebox{5mm}{\tiny\textsf{\slidevenue{} -- \arabic{mypage}/\pageref{lastpage}}}}
  %\rfoot{~\anchor{30,12}{\tiny\textsf{\thepage/\pageref{lastpage}}}}
  %\lfoot{\small\textsf{Marc Toussaint}}
%  \rfoot{\vspace*{-4.5mm}{\tiny\textsf{\color{gray}\topic\ -- \subtopic\ -- \arabic{mypage}/\pageref{lastpage}}}\hspace*{-4mm}}
  \rfoot{\vspace*{-4.5mm}{\tiny\textsf{\color{gray}\topic\ -- \arabic{mypage}/\pageref{lastpage}}}\hspace*{-4mm}}
%  \rfoot{~\anchor{-10,12}{\tiny\textsf{\color{gray}\topic\ --  \arabic{mypage}/\pageref{lastpage}}}}
  \lfoot{\vspace*{-4.5mm}{\hspace*{-3mm}\includegraphics[height=4mm]{LIS-logo-longText}}}

  \definecolor{grey}{rgb}{.8,.8,.8}
  \definecolor{head}{rgb}{.85,.9,.9}
%  \definecolor{blue}{rgb}{.0,.0,.5}
%  \definecolor{green}{rgb}{.0,.5,.0}
  \definecolor{red}{rgb}{.8,.0,.0}
  \newcommand{\inverted}{
    \definecolor{main}{rgb}{1,1,1}
    \color{main}
    \pagecolor[rgb]{.3,.3,.3}
  }
  \input{../latex/macros}

  \graphicspath{{pics/}{../pics/}{pics-local/}}
}

\newcommand{\slidestitle}{
  \title{\course \topic}
  \author{Marc Toussaint}
  \institute{Learning \& Intelligent Systems Lab, TU Berlin}

  \begin{document}


  %% title slide!
  \slide{}{
    \thispagestyle{empty}

    \twocol{.35}{.55}{
      %\vspace*{-5mm}%
      \hspace*{-5mm}%
      \includegraphics[width=1.\columnwidth]{\coursepicture}
    }{\center

      \textbf{\fontsize{17}{20}\selectfont \course}

      ~

      %Lecture
      \topic\\

      \vspace{1cm}

      {\tiny~\emph{\keywords}~\\}

      \vspace{1cm}

      \teacher
      
      Technical University of Berlin

      \coursedate

      ~

    }
  }
}

\newcommand{\slide}[2]{
  \slidefont
  \incpage\begin{frame}
  \addcontentsline{toc}{section}{#1}
  \vfill
  {\headerfont #1} \vspace*{-2ex}
  \begin{itemize}\item[]~\\
    #2
  \end{itemize}
  \vfill
  \end{frame}
}

% use \begin{frame}[fragile] around slidecore!
\newenvironment{slidecore}[1]{
  \slidefont\incpage
  \addcontentsline{toc}{section}{#1}
  \vfill
  {\headerfont #1} \vspace*{-2ex}
  \begin{itemize}\item[]~\\
}{
  \end{itemize}
  \vfill
}

\newcommand{\titleslide}[4][Marc Toussaint]{
  \newcommand{\slideauthor}{#1}
  \newcommand{\slidevenue}{#3}
  \slidefont
  \incpage
  \begin{frame}
  \begin{center}
    \vspace*{15mm}

    {\headerfont #2\\}
        
    \vspace*{10mm}

    #1 \\

    \vspace*{5mm}

    {\small
      Learning \& Intelligent Systems Lab, TU Berlin\\
%      Science of Intelligence Cluster of Excellence, Berlin\\
      Max Planck Fellow, Institute for Intelligent Systems\\ %Physical Reasoning \& Manipulation Lab -- 
%      MIT CSAIL\\
%      mtoussai@mit.edu,~ marc.toussaint@informatik.uni-stuttgart.de

      \vspace*{10mm}

      \emph{#3}
    }

    \vspace*{0mm}

    %\includegraphics[scale=.1]{pics/eushield-fullcolour}

  \end{center}
  \begin{itemize}\item[]~\\
    #4
  \end{itemize}
  \end{frame}
}

\newcommand{\titleslideempty}[2]{
  \slidefont
  \incpage
  \begin{frame}
  \begin{center}
    \vspace*{15mm}

    {\headerfont #1\\}
        
    %% \vspace*{5mm}

    %% {\small\emph{#2}} \\

  \end{center}
  \begin{itemize}\item[]~\\
    #2
  \end{itemize}
  \end{frame}
}

\providecommand{\key}[1]{
  \addtocounter{mypage}{1}
% \immediate\write\keyfile{#1}
  \addtocontents{toc}{\hyperref[key:#1]{#1 (\arabic{mypage})}}
%  \phantomsection\label{key:#1}
%  \index{#1@{\hyperref[key:#1]{#1 (\arabic{mysec}:\arabic{mypage})}}|phantom}
  \addtocounter{mypage}{-1}
}

\providecommand{\course}{}

\providecommand{\subtopic}{}

\providecommand{\sublecture}[2]{
  \renewcommand{\subtopic}{#1}
  \slide{#1}{#2}
}

\providecommand{\sublectureHide}[2]{
  \renewcommand{\subtopic}{#1}
}

\providecommand{\story}[1]{
~

Motivation: {\tiny #1}\clearpage
}

\newenvironment{items}[1][9]{
\par\setlength{\unitlength}{1pt}\fontsize{#1}{#1}\linespread{1.2}\selectfont
\begin{list}{--}{\leftmargin4ex \rightmargin0ex \labelsep1ex \labelwidth2ex
\topsep.7ex \parsep0ex \itemsep3pt}
}{
\end{list}
}

\newenvironment{itemS}[1][4ex]{
\par
\tiny
\begin{list}{--}{\leftmargin#1 \rightmargin0ex \labelsep1ex
  \labelwidth2ex \topsep0pt \parsep0ex \itemsep2pt}
}{
\end{list}
}

\providecommand{\slidesfoot}{
  \end{document}
}


  \slideshead
  %\slidestitle
}

\providecommand{\exercises}{
  \input{../latex/style-exercises}
  \exerciseshead
}

\providecommand{\script}{
  \newcommand{\scripthead}{
  \documentclass[9pt,fleqn,twoside]{article}
  \stdpackages

  \usepackage{makeidx}
  \makeindex

  \usepackage{thmtools}
  \definecolor{shadecolor}{gray}{0.85}
  \declaretheoremstyle[
    %headfont=\normalfont\bfseries,
    %notefont=\mdseries, notebraces={(}{)},
    %bodyfont=\normalfont,
    %postheadspace=0.5em,
    %spaceabove=6pt,
    mdframed={
      %  skipabove=8pt,
      %  skipbelow=6pt,
      hidealllines=true,
      backgroundcolor={shadecolor},
      innertopmargin=8pt,
      %  innerleftmargin=3pt,
      %  innerrightmargin=3pt
    }
  ]{shaded}
  \declaretheorem[style=shaded,within=section,name=Definition]{myDefinition}
  \declaretheorem[style=shaded,within=section,name=Theorem]{myTheorem}
  \declaretheorem[style=shaded,within=section,name=Identities]{Identities}
  \declaretheorem[style=shaded,within=section,name=Example]{myExample}

  \definecolor{grey}{rgb}{.8,.8,.8}
  \definecolor{bluecol}{rgb}{0,0,.5}
  \definecolor{greencol}{rgb}{0,.4,0}
  \definecolor{shadecolor}{gray}{0.9}
  \definecolor{citcol}{rgb}{.4,.4,.4}
  \usepackage[
    %    pdftex%,
    %%    letterpaper,
    %bookmarks,
    bookmarksnumbered,
    colorlinks,
    urlcolor=bluecol,
    citecolor=black,
    linkcolor=bluecol,
    %    pagecolor=bluecol,
    pdfborder={0 0 0},
    %pdfborderstyle={/S/U/W 1},
    %%    backref,     %link from bibliography back to sections
    %%    pagebackref, %link from bibliography back to pages
    %%    pdfstartview=FitH, %fitwidth instead of fit window
    pdfpagemode=UseOutlines, %bookmarks are displayed by acrobat
    pdftitle={\course},
    pdfauthor={Marc Toussaint},
    pdfkeywords={}
  ]{hyperref}
  \DeclareGraphicsExtensions{.pdf,.png,.jpg,.eps}

  %\usepackage{multirow}
  \usepackage{multimedia}
  %\usepackage{marginnote}
  %\setbeamercolor{background canvas}{bg=}

  \usepackage[round]{natbib}
  \bibliographystyle{abbrvnat}

  \renewcommand{\r}{\varrho}
  \renewcommand{\l}{\lambda}
  \renewcommand{\L}{\Lambda}
  \renewcommand{\b}{\beta}
  \renewcommand{\d}{\delta}
  \renewcommand{\k}{\kappa}
  \renewcommand{\t}{\theta}
  \renewcommand{\O}{\Omega}
  \renewcommand{\o}{\omega}
  \renewcommand{\SS}{{\cal S}}
  \renewcommand{\=}{\!=\!}
  %\renewcommand{\boldsymbol}{}
  %\renewcommand{\Chapter}{\chapter}
  %\renewcommand{\Subsection}{\subsection}

  \renewcommand{\baselinestretch}{1.0}
  \geometry{a5paper,headsep=6mm,hdivide={10mm,*,10mm},vdivide={13mm,*,7mm}}

  \fancyhead[OL,ER]{\course, \textit{Marc Toussaint}}
  \fancyhead[OR,EL]{\thepage}
  \fancyhead[C]{}
  \fancyfoot{}
  \pagestyle{fancy}

  \renewcommand{\labelenumi}{{(\roman{enumi})}}
  \renewcommand{\theenumi}{(\roman{enumi})} %for ref
  \parindent 0pt
  \parskip .5pc

  \columnsep 6ex

  \renewcommand{\familydefault}{\sfdefault}

  \newcommand{\headerfont}{\large}%helvetica{12}{1}{b}{n}}
  \newcommand{\slidefont} {}%\helvetica{9}{1.3}{m}{n}}
  \newcommand{\storyfont} {}
  %  \renewcommand{\small}   {\helvetica{8}{1.2}{m}{n}}
  \renewcommand{\tiny}    {\footnotesize}%helvetica{7}{1.1}{m}{n}}
  \newcommand{\ttiny} {\footnotesize}%fontsize{7}{7}\selectfont}
%  \newcommand{\codefont}{\fontsize{6}{6}\selectfont}%helvetica{8}{1.2}{m}{n}}
  \newcommand{\codefont} {\helvetica{8}{1.2}{m}{n}}

  \input{../latex/macros}

  \usepackage{comment}
  \specialcomment{solution}{
    \small
    \begin{shaded}
  }{
    \end{shaded}
  }

  \graphicspath{{pics/}{../pics/}{pics-local/}}

  \mytitle{\course\\Lecture Script}
  \myauthor{Marc Toussaint}
  \date{\coursedate}
}

%%%%%%%%%%%%%%%%%%%%%%%%%%%%%%%%%%%%%%%%%%%%%%%%%%%%%%%%%%%%%%%%%%%%%%%%%%%%%%%%

\newcommand{\scripttitle}{
  \begin{document}
  \maketitle
  %\anchor{100,10}{\includegraphics[width=4cm]{optim}}
%  \vspace*{1cm}
}

%%%%%%%%%%%%%%%%%%%%%%%%%%%%%%%%%%%%%%%%%%%%%%%%%%%%%%%%%%%%%%%%%%%%%%%%%%%%%%%%

\newcounter{mypage}
\setcounter{mypage}{0}
\newcommand{\incpage}{\addtocounter{mypage}{1}}

\newcommand{\subtopic}{}
\newcommand{\pause}{}
\newcommand{\only}[1]{#1}

\renewcommand{\slides}[1][]{
  %  \clearpage
  \subsection{\topic}
  \index{\topic}
  {\small #1}
  \setcounter{mypage}{0}
  \smallskip\nopagebreak\hrule\medskip
}

\newcommand{\slidesfoot}{
  \bigskip
}

\newcommand{\sublecture}[2]{
  \phantomsection\addcontentsline{toc}{subsubsection}{#1}
  \index{#1}
}

\newcommand{\sublectureHide}[2]{
  \renewcommand{\subtopic}{#1}
}

\newcommand{\key}[1]{
  \phantomsection\addcontentsline{toc}{subsubsection}{#1}
  %\subsubsection{#1}
  \index{#1}
}

\providecommand{\defn}[1]{%
  \textbf{#1}\index{#1}%
}

\newenvironment{slidecore}[1]{
  \incpage
  \subsubsection*{#1}%{\headerfont\noindent\textbf{#1}\\}%
  \vspace{-6ex}%
  \begin{list}{$\bullet$}{\leftmargin4ex \rightmargin0ex \labelsep1ex
    \labelwidth2ex \partopsep0ex \topsep0ex \parsep.5ex \parskip0ex \itemsep0pt}\item[]~\\\nopagebreak%
}{
  \end{list}\nopagebreak%
  {\hfill\tiny \textsf{\arabic{section}.\arabic{subsection}:\arabic{mypage}}}\nopagebreak%
  \smallskip\nopagebreak\hrule
}

\newcommand{\slide}[2]{
  \begin{slidecore}{#1}
    #2
  \end{slidecore}
}

\renewcommand{\exercises}{}
\newcommand{\exercisestitle}{}
\newcommand{\exsection}[1]{\subsubsection{#1}}
\newcommand{\exsubsection}[1]{\paragraph{#1}}
\newcommand{\exerfoot}{\bigskip}

\newcommand{\story}[1]{
  \subsection*{Motivation \& Outline}
  {\storyfont\sf #1}
  \medskip\nopagebreak\hrule
}

\newcounter{savedsection}
\newcommand{\subappendix}{\setcounter{savedsection}{\arabic{section}}\appendix}
\newcommand{\noappendix}{
  \setcounter{section}{\arabic{savedsection}}% restore section number
  \setcounter{subsection}{0}% reset section counter
%  \gdef\@chapapp{\sectionname}% reset section name
  \renewcommand{\thesection}{\arabic{section}}% make section numbers arabic
}

\newenvironment{items}[1][9]{
\par\setlength{\unitlength}{1pt}\fontsize{#1}{#1}\linespread{1.2}\selectfont
\begin{list}{--}{\leftmargin4ex \rightmargin0ex \labelsep1ex \labelwidth2ex
\topsep0pt \parsep0ex \itemsep3pt}
}{
\end{list}
}

\newenvironment{itemS}[1][4ex]{
\par
\tiny
\begin{list}{--}{\leftmargin#1 \rightmargin0ex \labelsep1ex
  \labelwidth2ex \topsep0pt \parsep0ex \itemsep2pt}
}{
\end{list}
}

\newcommand{\Def}[1]{%
\textbf{#1}\index{#1}}%\marginnote{#1}}

  \scripthead
}

\providecommand{\paper}{
  \input{../latex/style-paper}
  \paperhead
}

\providecommand{\note}[1][9pt]{
  \providecommand{\notehead}[2]{
  \documentclass[#1,fleqn,twoside]{article}
  \stdpackages
  \renewcommand{\labelenumi}{{(\roman{enumi})}}
  \renewcommand{\theenumi}{(\roman{enumi})} %for ref

  \renewcommand{\baselinestretch}{#2}
  \renewcommand{\arraystretch}{1.2}
  \renewcommand{\topfraction}{1}
  \renewcommand{\bottomfraction}{1}
  \renewcommand{\textfraction}{0}
  \columnsep 5ex
  \parindent 3ex
  \parskip 1ex

  % Lists and paragraphs
  \parindent 0pt
  \topsep 4pt plus 1pt minus 2pt
  \partopsep 1pt plus 0.5pt minus 0.5pt
  \itemsep 2pt plus 1pt minus 0.5pt
  \parsep 2pt plus 1pt minus 0.5pt
  \parskip .5pc %add _in_ {thebibliography} environment in *.bbl

  \setcounter{tocdepth}{3}
  \setcounter{secnumdepth}{3}

  \geometry{a4paper,hdivide={25mm,*,25mm},vdivide={25mm,*,25mm}}

  \renewcommand{\headrulewidth}{.0pt}\renewcommand{\footrulewidth}{.0pt}\cfoot{}
  \fancyhead[OL,EC]{\it\theauthor---\today}
  \fancyhead[ER]{\leftmark}
  \fancyhead[OR,EL]{\thepage}
  \fancyfoot[EL,OR]{}
  \setlength{\headsep}{10mm}
  %\fancyhead[OL]{\rightmark}
  %\fancyfoot[EL,OR]{}


  %  \usepackage{palatino}

  \newcommand{\codefont}{\helvetica{8}{1.2}{m}{n}}

  \renewenvironment{abstract}{
    \vspace*{5ex}\begin{list}{}{
      \leftmargin3ex
      \rightmargin3ex
      \topsep-\parskip}\item[]
     \hrule\vspace{1.5ex}{\bf Abstract.~}\small}
    {\vspace{2ex}\hrule\end{list}\vspace{5ex}}
    
  \newenvironment{keyword}
    {\par{\it Keywords:~}}
    {}

  \def\makemytitle{%
    \thispagestyle{empty}
    \begin{list}{}{\leftmargin3ex \rightmargin3ex \topsep0ex \parsep0ex}\item[]
      \begin{center}
        {\fontsize{18}{25}\selectfont{\thetitle\\}}\vspace{5ex}

        {\fontsize{14}{16}\selectfont{\theauthor\\}}\vspace{1ex}

        {\footnotesize{\sl \addressFUB}\\ \emailBerlin}

        {\footnotesize \today}

        \vspace{1ex}
        {\small \published}
      \end{center}
    \end{list}
    \renewcommand{\maketitle}{\chapter{\thetitle}}
  }

  \input{../latex/macros}
  \pdflatex

  \graphicspath{{pics/}{../pics/}{pics-local/}}

  \myauthor{Marc Toussaint}
  \date{\today}
}

%%%%%%%%%%%%%%%%%%%%%%%%%%%%%%%%%%%%%%%%%%%%%%%%%%%%%%%%%%%%%%%%%%%%%%%%%%%%%%%%

\newcommand{\notetitle}{
  \begin{document}
  \thispagestyle{empty}
    
  \maketitle

}

\newenvironment{items}[1][9]{
\par\setlength{\unitlength}{1pt}\fontsize{#1}{#1}\linespread{1.2}\selectfont
\begin{list}{--}{\leftmargin4ex \rightmargin0ex \labelsep1ex \labelwidth2ex
\topsep0pt \parsep0ex \itemsep3pt}
}{
\end{list}
}

  \notehead{#1}{1.1}
}

\providecommand{\course}{NO COURSE}
\providecommand{\coursepicture}{NO PICTURE}
\providecommand{\coursedate}{NO DATE}
\providecommand{\topic}{NO TOPIC}
\providecommand{\keywords}{}
\providecommand{\exnum}{NO NUMBER}
\providecommand{\teacher}{Marc Toussaint}

\providecommand{\stdpackages}{
  \usepackage{amsmath}
  \usepackage{amssymb}
  \usepackage{amsfonts}
  \allowdisplaybreaks
  \usepackage{amsthm}
  \usepackage{eucal}
  \usepackage{graphicx}
%  \usepackage{color}
  \usepackage{geometry}
  \usepackage{framed}
  \usepackage{xcolor}
  \definecolor{shadecolor}{gray}{0.9}
  \setlength{\FrameSep}{3pt}
  \usepackage{fancyvrb}
  \fvset{numbers=none,xleftmargin=5ex,fontsize=\small}

  \usepackage{pdfpages}

  \usepackage{multicol} 
  \usepackage{fancyhdr}
}

\providecommand{\addressUSTT}{
  Machine~Learning~\&~Robotics~lab, U~Stuttgart\\\small
  Universit{\"a}tsstra{\ss}e 38, 70569~Stuttgart, Germany
}

\providecommand{\addressTUB}{
  Learning~\&~Intelligent~Systems~Lab, TU~Berlin\\\small
  Marchstr. 23, 10587 Berlin, Germany
}


\renewcommand{\course}{Robot Learning}
\renewcommand{\coursepicture}{roblearn.png}
\renewcommand{\coursedate}{Summer 2024}
\renewcommand{\teacher}{Marc Toussaint}

\renewcommand{\topic}{Robotics Essentials}
\renewcommand{\keywords}{}

\slides

\ifthenelse{\isundefined{\scripthead}}{

\providecommand{\info}[1]{\smallskip{\ttiny [#1]\par}}

\usepackage{bibentry}
\nobibliography*

\ifthenelse{\isundefined{\setbeamertemplate}}{}{
  \setbeamertemplate{bibliography item}{\insertbiblabel}
}

\providecommand{\citehere}[1]{{\fontsize{5}{1}\selectfont\bibentry{#1}\par}}

}{}

\providecommand{\SE}{\text{SE}}
\providecommand{\ang}{\text{ang}}
\providecommand{\ul}{\underline}

\slidestitle

%%%%%%%%%%%%%%%%%%%%%%%%%%%%%%%%%%%%%%%%%%%%%%%%%%%%%%%%%%%%%%%%%%%%%%%%%%%%%%%%

\slide{Robotics Essentials Outline}{

~

\item A robot is an articulated multi-body system: ~ kinematics \& dynamics

~

\item Standard Control: ~ IK,~  path finding \& traj. opt, PD \& MPC

}

%%%%%%%%%%%%%%%%%%%%%%%%%%%%%%%%%%%%%%%%%%%%%%%%%%%%%%%%%%%%%%%%%%%%%%%%%%%%%%%%

\key{Articulated Multibody System}
\slide{Robot as Articulated Multibody System}{

\item A robot is a multibody system. Each body
\begin{items}
\item has a pose $x_i\in\SE(3)$
\item has inertia $(m_i, I_i)$ with mass $m_i\in\RRR$ and inertia tensor $I_i \in \RRR^{3\times 3}$ sym.pos.def.
\item has a shape $s_i$ (formally: any representation that defines a pairwise signed-distance $d(s_i, s_j)$)
\end{items}

~

\info{Useful: ``multibody system'' on Wikipedia}

}

%%%%%%%%%%%%%%%%%%%%%%%%%%%%%%%%%%%%%%%%%%%%%%%%%%%%%%%%%%%%%%%%%%%%%%%%%%%%%%%%

\slide{Robot as Articulated Multibody System}{

\item \textbf{Tree structure:}\anchor{200,-35}{\showh[.18]{geo-transforms-2}}

\begin{items}
\item Every body is linked to a parent body or the world
\item We have relative transformations $Q_i \in \SE(3)$ from parent (or world)
%\item \emph{Algorithm:} Computing pose $x_i$ is done by fwd chaining/concatenating all $Q_i$ from world to $i$
\end{items}

\info{If not tree-structured, we only represent a tree and use additional constraints to describe loops $\to$ more involved, but doable}


~\pause


\twocol[.02]{.65}{.28}{

\item \textbf{Articulated Degrees of Freedom (dofs):}
\begin{items}
\item Some of the relative transformations $Q_i$ may have articulated (=motorized) \textbf{dofs} $q$ so that $Q_i(q)$

\info{Different types of joints (hinge, prismatic, universal, ball) have different \# dofs and different mapping from dofs $q \mapsto Q_i(q)$}

\item We stack all dofs of all relative transformations into a single\\ \textbf{joint vector} $q\in\RRR^n$
\end{items}

}{
\showh[1]{kinematics-3}
}

}

%%%%%%%%%%%%%%%%%%%%%%%%%%%%%%%%%%%%%%%%%%%%%%%%%%%%%%%%%%%%%%%%%%%%%%%%%%%%%%%%

\slide{}{

\cen{$x\in\SE(3)^m$: all body poses, \qquad $q\in\RRR^n$:~ joint vector}

~

~

\begin{items}
\item Forward kinematics: $q \mapsto x$,~ $\dot q \mapsto \dot x$,~ $\ddot q \mapsto \ddot x$
\item Forward dynamics: $u \mapsto \ddot q$,~ inverse dynamics: $\ddot q \mapsto u$ \quad ($u\in\RRR^n$: joint torques)
\end{items}

}

%%%%%%%%%%%%%%%%%%%%%%%%%%%%%%%%%%%%%%%%%%%%%%%%%%%%%%%%%%%%%%%%%%%%%%%%%%%%%%%%

\key{Forward Kinematics}
\slide{Forward Kinematics ~ $q \mapsto x$}{

\item Given $q$, what is the pose of any body $i$?

$$q ~ \mapsto ~ \mat{c}{x_1\\x_2\\\vdots \\x_m} = \phi(q) \quad \in \SE(3)^m$$

~

\begin{items}
\item \emph{Algorithm:} First determine all rel.\ trans. $Q_i(q)$, then forward chain them

\item Often one cares only about position/orientation of one particular body $x_i$: the \textbf{``endeffector''}
\end{items}

}

%%%%%%%%%%%%%%%%%%%%%%%%%%%%%%%%%%%%%%%%%%%%%%%%%%%%%%%%%%%%%%%%%%%%%%%%%%%%%%%%

\slide{Forward Velocities \& Jacobian ~ $\dot q \mapsto \dot x$}{

\item Given $\dot q$, what is the linear and angular velocity $(v_i, w_i)$ of any body $i$?

$$\dot q \mapsto \mat{c}{v_1,w_1\\v_2,w_2\\\vdots \\v_m,w_m} = J(q)~ \dot q \quad\in \RRR^{m\times 6} $$

\begin{items}
\item with \textbf{Jacobian} $J(q)=\del_q \phi(q) ~ \in\RRR^{m\times 6\times n}$.

\info{Since, $\phi$ is $\SE(3)$-valued, the Jacobian actually has output in its tangent space $se(3) \equiv \RRR^6$. In practise, code typically provides separate positional Jacobian $J^\pos \in \RRR^{m\times 3\times n}$ and angular
Jacobian $J^\ang \in \RRR^{m\times 3\times n}$.}

\pause

\item Since we know how to compute $\phi(q)$, we can think of $J(q)$ as the ``autodiff'' of it
\item However, positional/angular Jacobians are really very easy to provide without expensive autodiff

\info{In practise, one only needs to figure out the $J^\pos, J^\ang$ for a rotational and translational joint -- all others follow from this.}
%[Algorithmically all $\dot x_i$ can also be computed by forward propagating velocities along the three, without need to compute Jacobian first.]
\end{items}


}

%%%%%%%%%%%%%%%%%%%%%%%%%%%%%%%%%%%%%%%%%%%%%%%%%%%%%%%%%%%%%%%%%%%%%%%%%%%%%%%%

\slide{Forward Accelerations ~ $\ddot q \mapsto \ddot x$}{

\item Given $\ddot q$, what is the linear and angular acceleration $(\dot v_i, \dot w_i)$ of any body $i$?

$$\ddot x = \dot J(q)~ \dot q + J(q)~ \ddot q ~\approx~ J(q)~ \ddot q$$

~

\begin{items}
\item One typically approximates $\dot J = 0$
\end{items}

}

%%%%%%%%%%%%%%%%%%%%%%%%%%%%%%%%%%%%%%%%%%%%%%%%%%%%%%%%%%%%%%%%%%%%%%%%%%%%%%%%

\slide{The word ``kinematics''}{

\info{in parts from Wikipedia}

~

\begin{items}
\item Mathematical description of possible motions of a (constrainted/multibody) system/mechanism \emph{without considering the forces}
\item ``geometry of [possible] motions''
\item Formally: Describe the space (manifold) of possible system poses and all possible paths in that space
\item Read \textbf{generalized coordinates} on wikipedia: Understanding motion in terms of coordinates and (non-)holonomic constraints:
\end{items}

\cen{
\showh[.15]{hexapod}\qquad
\showh[.2]{deltaRobot}\qquad
\showh[.15]{coordinates}
}

}

%%%%%%%%%%%%%%%%%%%%%%%%%%%%%%%%%%%%%%%%%%%%%%%%%%%%%%%%%%%%%%%%%%%%%%%%%%%%%%%%

\key{Inverse Dynamics}
\slide{Inverse dynamics ~ $\ddot q \mapsto u$}{

\item Given $\ddot q$, what joint torques $u$ do we need to generate this $\ddot q$ (accounting for gravity)?

~\pause

\item Coupled Newton-Euler equations: For each body:\anchor{20,-40}{\showh[.3]{featherstoneRNE}\anchor{-35,-7}{\ttiny from Featherstone'14}}
\begin{align*}
F_i = \mat{c}{f_i \\ \tau_i}
&= \mat{c}{m_i \dot v_i \\I_i \dot w_i + w_i\times I_i w_i} \\
F^\text{back}_i
&= F_i - F^\text{ext}_i + \sum_{j=\text{child(i)}} F^\text{back}_j \comma u_i = h_i^\T F^\text{back}_i
\end{align*}

\info{where $F^\text{ext}_i$ are external (e.g.\ gravity) forces; and $F^\text{back}_i$ is the force ``send back through the joint to the parent of $i$''; $h_i$ is the joint axis (picking up the torque)}

\info{Can also be written as linear equation system between $\ddot q$, $F$, $F^\text{back}$, and $u$ (with sparse matrices only) -- and solved/inverted in $O(m)$.}

}

%%%%%%%%%%%%%%%%%%%%%%%%%%%%%%%%%%%%%%%%%%%%%%%%%%%%%%%%%%%%%%%%%%%%%%%%%%%%%%%%

\slide{}{\centering

solved! \quad We can accelerate the thing as we like

~\pause

the rest is planning: How should I accelerate to reach some future goals?

}

%%%%%%%%%%%%%%%%%%%%%%%%%%%%%%%%%%%%%%%%%%%%%%%%%%%%%%%%%%%%%%%%%%%%%%%%%%%%%%%%

\key{Standard Control Stack}
\slide{Standard Template: Waypoint + Reference Motion + Controller}{

\pause

\item Standard problem setting: Control motors, so that at $t=T$ seconds the endeffector $x_i$ is at desired position $y^*\in\RRR^3$, i.e., $\phi(q_{t=T}) = y^*$

\pause

\item Problem decomposition:
\begin{items}
  \item Find a final robot pose $q_T$ that fulfills constraint 
    $\phi(q_{t=T}) = y^*$ -- \textbf{inverse kinematics}
  \item Find a nice \emph{reference} motion from current robot pose $q_0$ to $q_T$ -- \textbf{path finding, trajectory optimization, or trivial interpolation/PD}
  \item Find a control policy $\pi: x_t \mapsto u_t$ that reactively sends motor commands to follow the reference motion -- \textbf{inverse dynamics, PD control, Riccati}
\end{items}

\info{You could think of this as three different time scales: rough future waypoint(s)/goal(s), continuous motion to next waypoint, short-term controls.}

\info{There are other ways to approach this:
You could remove step (1) and shift that issue into (2), or remove (1 \& 2) and shift all issues into (3) - morphing this into other approaches. E.g. directly defining a desired force/acceleration behavior in ``task space'' (=operational space control).}

\info{continuous replanning/re-estimation can also make (1) and (2) reactive.}

}

%%%%%%%%%%%%%%%%%%%%%%%%%%%%%%%%%%%%%%%%%%%%%%%%%%%%%%%%%%%%%%%%%%%%%%%%%%%%%%%%

\key{Inverse Kinematics}
\slide{Inverse Kinematics}{

\item Find $q$ to fulfill $\phi(q) = y^*$ for differentiable fwd kinematics $\phi$.

\begin{align*}
&\min_{q\in\RRR^n} \norm{q-q_0}^2 \st \phi(q) = y^* \\
\text{or}\quad&\min_{q\in\RRR^n} \norm{q-q_0}^2 + \m \norm{\phi(q) - y^*}^2 \quad\text{for large $\m$}
\end{align*}

~

\item Solution for linearized $\phi$:
\begin{align*}
q^*
&= q_0 +  J^\T (J J^\T + \textstyle\frac{1}{\mu} \Id)^\1 (y^*-\phi(q_0))
\end{align*}

~

{\tiny\hfill Python Package: \url{https://marctoussaint.github.io/robotic/}}

}

%%%%%%%%%%%%%%%%%%%%%%%%%%%%%%%%%%%%%%%%%%%%%%%%%%%%%%%%%%%%%%%%%%%%%%%%%%%%%%%%

\slide{Path Finding \& Trajectory Optimization}{

\item Given current $q_0$ and future $q^*$, find a collision free \textbf{path}
\begin{items}
\item Wolfgang H\"onig's \& Andreas Orthey's lecture
\item RRTs, PRMs, under constraints (kinodynamic)
\end{items}

~\pause

\item \textbf{Trajectory} opimization
\begin{items}
\item Time continuous formulation:
\tiny\begin{align*}
  \min_{q(t)} ~& \int_0^T c(q(t),\dot q(t),\ddot q(t))~ dt ~\st~ q(0)=q_0,~ q(T) = q^*, \dot q(0)=\dot q(T)=0 ~,
  \forall_{t\in[0,T]}: \bar\phi(q(t),\dot q(t),\ddot q(t)) \le 0 ~.
\end{align*}

\item Time-discretized, assuming $k$-order Markov coupling terms (KOMO):
\cit{A tutorial on Newton methods for constrained trajectory optimization and relations to SLAM, Gaussian Process smoothing, optimal control, and probabilistic inference}{Marc Toussaint}{Springer 2017}
\end{items}

}

%%%%%%%%%%%%%%%%%%%%%%%%%%%%%%%%%%%%%%%%%%%%%%%%%%%%%%%%%%%%%%%%%%%%%%%%%%%%%%%%

\slide{Control around a Reference}{

\item Use \textbf{Inverse Dynamics} directly
\begin{items}
\item We have $\ddot q^*(t)$ $\to$ map it to controls $u$ directly
\item But what if you're off the reference a bit? \emph{How to steer back?}
\end{items}

\pause

\item Use \textbf{PD law} to accelerate back to reference:
\begin{items}
\item Define a PD law $\ddot q^\text{desired} = \ddot q^*(t) + k_p (q^*(t) - q) + k_d(\dot q^*(t) - \dot q)$ with desired PD behavior back to reference
\item Then use Inv dynamics $\ddot q^\text{desired} \mapsto u$
\item (Also ok, but needs severe tuning: directly define a PD controller $\ddot u = M \ddot q^*(t) + K_p(q^*(t) - q) + K_d(\dot q^*(t) - \dot q)$.)
\end{items}

\pause

\item Use \textbf{Riccati} to get an \textbf{Optimal Linear Regulator} around reference
\begin{items}
\item Define optimal control problem, e.g., $\min_{\pi:q,\dot q\mapsto u} \int_0^T c(q(t), \dot q(t), u(t))~ dt + \phi(x(T))$
\item We can linearize dynamics around reference $\to$ has an analytic solution (Algebraic Riccati eq.)
\item Resulting controller is a ``linear regulator'', i.e., a PD law where matrices $K_p, K_d$ depend on $t$ and are chosen optimally.
\end{items}

}

%%%%%%%%%%%%%%%%%%%%%%%%%%%%%%%%%%%%%%%%%%%%%%%%%%%%%%%%%%%%%%%%%%%%%%%%%%%%%%%%

\key{Model-Predictive Control (MPC)}
\slide{Model-Predictive Control (MPC)}{


\item When getting far away from the reference, linearization of Riccati might break, and PD is too simple

~\pause

\item Continuously replan ($\sim$ 10-1000Hz): re-solve the optimal control problem
\begin{items}
\item Optimal Control problem can also include task constraints directly, not only following a reference
\item As a compromise: typically limit horizon
\end{items}

~

\cen{\textbf{This is a default way of ``thinking control'' in robotics}}

}

%%%%%%%%%%%%%%%%%%%%%%%%%%%%%%%%%%%%%%%%%%%%%%%%%%%%%%%%%%%%%%%%%%%%%%%%%%%%%%%%

\slide{Summary}{

~\pause

\item A robot is an articulated multi-body system
\begin{items}
\item Fwd kinematics: $q \mapsto x$,~ $\dot q \mapsto \dot x$,~ $\ddot q \mapsto \ddot x$
\item Fwd dynamics: $u \mapsto \ddot q$,~ inv dynamics: $\ddot q \mapsto u$ \end{items}

~\pause

\item Standard Control Template:
\begin{items}
\item IK (or constraint solving) to estimate future goal/waypoints
\item Path Finding \& Trajectory Optimization to estimate Reference Motion
\item PD, Linear Regulator, or MPC to control (around the reference)
\end{items}

}

%%%%%%%%%%%%%%%%%%%%%%%%%%%%%%%%%%%%%%%%%%%%%%%%%%%%%%%%%%%%%%%%%%%%%%%%%%%%%%%%

\key{Challenges}
\slide{How far can we get with this approach?}{

~\pause

\item What did we assume to \emph{know}?
\begin{items}
\item Structure of multi-body system, all shapes, inertias
\item All goals/objectives modelled (=programmed) as differentiable costs/constraints
\end{items}

}

%%%%%%%%%%%%%%%%%%%%%%%%%%%%%%%%%%%%%%%%%%%%%%%%%%%%%%%%%%%%%%%%%%%%%%%%%%%%%%%%

\slide{Challenge 1: Interacting with the environment}{

\item If we only care about the \textbf{robot itself} (all goals/objectives/models concern the robot directly) -- the above it totally fine

~\pause

\item Things get challenging when we care about \textbf{interacting with the environment}
\begin{items}
\item Models/goals/objectives of interaction (contact, grasp) are more complicated
\end{items}

}

%%%%%%%%%%%%%%%%%%%%%%%%%%%%%%%%%%%%%%%%%%%%%%%%%%%%%%%%%%%%%%%%%%%%%%%%%%%%%%%%

\slide{Challenge 1: Interacting with the environment}{

\item Example: Locomotion
\begin{items}
\item Interaction: Making contact with the ground to generate ground forces

\item Robot root is not attached to world, but free floating (complicates dynamics a bit)

\item Dynamics heavily influenced by ground forces, which are \emph{contact complementary} hard on-off switching of forces at contact $\to$ hybrid/discrete structure, makes dynamics and solvers much much more complicated (hybrid control)
\end{items}

~\pause

... more complicated than ``vanilla robot'', but still doable

}

%%%%%%%%%%%%%%%%%%%%%%%%%%%%%%%%%%%%%%%%%%%%%%%%%%%%%%%%%%%%%%%%%%%%%%%%%%%%%%%%

\slide{Challenge 1: Interacting with the environment}{

\item Example: Manipulation
\begin{items}
\item Objects in the environment (part of the ``multibody system'') have their own DOFs, but are NOT ``articulated'' with motors: if not grasped or touched, they cannot move $\to$ their Jacobian $\del_q x_i = 0$

\item Hard on-off switching of manipulability; hybrid dynamics \& problem

\item Dynamics of object motions can be much more complicated than (also free-floating) robot dynamics: friction, stiction, slip, non-point contacts

\item Waypoint constraints $\phi(x_t)$ much more complicated (correct grasping of complex shape, pushing, throwing)

\item If objects are deformable, their form becomes DOF (e.g. neural latent code) -- becomes much much more complicated in above approach
\end{items}

~\pause

\item In essence, things become much more complicated, but one still \emph{can} write down essential physics equations of object interaction, and use these equations in above approach

}

%%%%%%%%%%%%%%%%%%%%%%%%%%%%%%%%%%%%%%%%%%%%%%%%%%%%%%%%%%%%%%%%%%%%%%%%%%%%%%%%

\slide{Challenge 2: State Estimation}{

\item All of the above requires to estimate states
\begin{items}
\item $q_0$ (includes pose of a mobile robot)
\item $x_i$ (poses of objects in environment)
\item shapes and inertias in the environment, dynamics parameters (e.g.\ friction)
\end{items}

~

\info{Basic state estimation can often also be formulated as optimization problem (e.g.\ graph-SLAM) -- similar to motion optimization: Find estimates (also of past motion) that is \emph{most consistent} with sensor readings; minimze error between real readings and model-predicted readings. (Or as probabilistic inference.)}

}

%%%%%%%%%%%%%%%%%%%%%%%%%%%%%%%%%%%%%%%%%%%%%%%%%%%%%%%%%%%%%%%%%%%%%%%%%%%%%%%%

\slide{Relation to Robot Learning}{\label{lastpage}

\item On the formal/theory side, they share foundations:
\begin{items}
\item Optimal Control formulation $\oto$ Markov Decision Processes \& Reinforcement Learning
\item More generally: optimality formulations $\to$ learning/black-box opt.\ approaches
\end{items}

~\pause

\item Components can be \emph{replaced} or \emph{shortcut} by learning:
\begin{items}
\item Dynamic modelling $\oto$ system identification
\item Optimal Control (e.g., MPC, Riccati) can be shortcut by learning $V$- or $Q$-function
\item Need of inverse dynamics can be shortcut by learning $Q$-function instead of $V$-function
\item Constraint solving (also IK) can be shortcut by directly learning a policy or sampler that fulfills constraint
\pause
\item \textbf{Shortcut state estimation:} Avoid all state-based models, learn direct sensor-based models (policies, value functions, planners, dynamics, etc)
\item \textbf{End-to-end:} Shortcut the whole approach by learning images $\mapsto$ torques
\end{items}

}

%%%%%%%%%%%%%%%%%%%%%%%%%%%%%%%%%%%%%%%%%%%%%%%%%%%%%%%%%%%%%%%%%%%%%%%%%%%%%%%%

\slidesfoot

