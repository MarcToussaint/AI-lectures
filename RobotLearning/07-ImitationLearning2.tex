\providecommand{\slides}{
  \newcommand{\slideshead}{
  \newcommand{\thepage}{\arabic{mypage}}
  %beamer
%  \documentclass[t,hyperref={bookmarks=true}]{beamer}
%  \geometry{papersize={171mm,96mm}}
  \documentclass[t,hyperref={bookmarks=true},aspectratio=169]{beamer}
  \setbeamersize{text margin left=5mm}
  \setbeamersize{text margin right=5mm}
  \usetheme{default}
  \usefonttheme[onlymath]{serif}
  \setbeamertemplate{navigation symbols}{}
  \setbeamertemplate{itemize items}{{\color{black}$\bullet$}}

  \newwrite\keyfile

  %\usepackage{palatino}
  \stdpackages
  %\usepackage{tikz} \usetikzlibrary {shapes.geometric} 
  \usepackage{multimedia}
  \usepackage[utf8]{inputenc}

  %%% geometry/spacing issues
  %
  \definecolor{bluecol}{rgb}{0,0,.5}
  \definecolor{greencol}{rgb}{0,.6,0}
  \definecolor{citcol}{rgb}{.4,.4,.4}
  %\renewcommand{\baselinestretch}{1.1}
  \renewcommand{\arraystretch}{1.2}
  \columnsep 0mm

  \columnseprule 0pt
  \parindent 0ex
  \parskip 0ex
  %\setlength{\itemparsep}{3ex}
  %\renewcommand{\labelitemi}{\rule[3pt]{10pt}{10pt}~}
  %\renewcommand{\labelenumi}{\textbf{(\arabic{enumi})}}
  \setbeamertemplate{enumerate item}{(\roman{enumi})}
  \newcommand{\headerfont}{\helvetica{14}{1.5}{b}{n}}
  \newcommand{\slidefont} {\helvetica{11}{1.4}{m}{n}}
  %\newcommand{\codefont} {\helvetica{8}{1.2}{m}{n}}
  \newcommand{\urlfont} {\fontsize{6}{1.1}\selectfont}
  \renewcommand{\small} {\helvetica{10}{1.4}{m}{n}}
  \renewcommand{\tiny} {\helvetica{8}{1.3}{m}{n}}
  \newcommand{\ttiny} {\helvetica{7}{1.3}{m}{n}}

  %%% count pages properly and put the page number in bottom right
  %
  \newcounter{mypage}
  \newcommand{\incpage}{\addtocounter{mypage}{1}\setcounter{page}{\arabic{mypage}}}
  \setcounter{mypage}{0}
  \resetcounteronoverlays{page}

  \pagestyle{fancy}
  %\setlength{\headsep}{10mm}
  %\addtolength{\footheight}{15mm}
  \renewcommand{\headrulewidth}{0pt} %1pt}
  \renewcommand{\footrulewidth}{0pt} %.5pt}
  \cfoot{}
  \rhead{}
  \lhead{}
  %% \lfoot{\vspace*{-3mm}\hspace*{-3mm}\helvetica{5}{1.3}{m}{n}{\texttt{github.com/MarcToussaint/AI-lectures}}}
\lfoot{}
%  \rfoot{{\tiny\textsf{AI -- \topic -- \subtopic -- \arabic{mypage}/\pageref{lastpage}}}}
  %\lfoot{\raisebox{5mm}{\tiny\textsf{\slideauthor}}}
  %\rfoot{\raisebox{5mm}{\tiny\textsf{\slidevenue{} -- \arabic{mypage}/\pageref{lastpage}}}}
  %\rfoot{~\anchor{30,12}{\tiny\textsf{\thepage/\pageref{lastpage}}}}
  %\lfoot{\small\textsf{Marc Toussaint}}
%  \rfoot{\vspace*{-4.5mm}{\tiny\textsf{\color{gray}\topic\ -- \subtopic\ -- \arabic{mypage}/\pageref{lastpage}}}\hspace*{-4mm}}
  \rfoot{\vspace*{-4.5mm}{\tiny\textsf{\color{gray}\topic\ -- \arabic{mypage}/\pageref{lastpage}}}\hspace*{-4mm}}
%  \rfoot{~\anchor{-10,12}{\tiny\textsf{\color{gray}\topic\ --  \arabic{mypage}/\pageref{lastpage}}}}
  \lfoot{\vspace*{-4.5mm}{\hspace*{-3mm}\includegraphics[height=4mm]{LIS-logo-longText}}}

  \definecolor{grey}{rgb}{.8,.8,.8}
  \definecolor{head}{rgb}{.85,.9,.9}
%  \definecolor{blue}{rgb}{.0,.0,.5}
%  \definecolor{green}{rgb}{.0,.5,.0}
  \definecolor{red}{rgb}{.8,.0,.0}
  \newcommand{\inverted}{
    \definecolor{main}{rgb}{1,1,1}
    \color{main}
    \pagecolor[rgb]{.3,.3,.3}
  }
  \input{../latex/macros}

  \graphicspath{{pics/}{../pics/}{pics-local/}}
}

\newcommand{\slidestitle}{
  \title{\course \topic}
  \author{Marc Toussaint}
  \institute{Learning \& Intelligent Systems Lab, TU Berlin}

  \begin{document}


  %% title slide!
  \slide{}{
    \thispagestyle{empty}

    \twocol{.35}{.55}{
      %\vspace*{-5mm}%
      \hspace*{-5mm}%
      \includegraphics[width=1.\columnwidth]{\coursepicture}
    }{\center

      \textbf{\fontsize{17}{20}\selectfont \course}

      ~

      %Lecture
      \topic\\

      \vspace{1cm}

      {\tiny~\emph{\keywords}~\\}

      \vspace{1cm}

      \teacher
      
      Technical University of Berlin

      \coursedate

      ~

    }
  }
}

\newcommand{\slide}[2]{
  \slidefont
  \incpage\begin{frame}
  \addcontentsline{toc}{section}{#1}
  \vfill
  {\headerfont #1} \vspace*{-2ex}
  \begin{itemize}\item[]~\\
    #2
  \end{itemize}
  \vfill
  \end{frame}
}

% use \begin{frame}[fragile] around slidecore!
\newenvironment{slidecore}[1]{
  \slidefont\incpage
  \addcontentsline{toc}{section}{#1}
  \vfill
  {\headerfont #1} \vspace*{-2ex}
  \begin{itemize}\item[]~\\
}{
  \end{itemize}
  \vfill
}

\newcommand{\titleslide}[4][Marc Toussaint]{
  \newcommand{\slideauthor}{#1}
  \newcommand{\slidevenue}{#3}
  \slidefont
  \incpage
  \begin{frame}
  \begin{center}
    \vspace*{15mm}

    {\headerfont #2\\}
        
    \vspace*{10mm}

    #1 \\

    \vspace*{5mm}

    {\small
      Learning \& Intelligent Systems Lab, TU Berlin\\
%      Science of Intelligence Cluster of Excellence, Berlin\\
      Max Planck Fellow, Institute for Intelligent Systems\\ %Physical Reasoning \& Manipulation Lab -- 
%      MIT CSAIL\\
%      mtoussai@mit.edu,~ marc.toussaint@informatik.uni-stuttgart.de

      \vspace*{10mm}

      \emph{#3}
    }

    \vspace*{0mm}

    %\includegraphics[scale=.1]{pics/eushield-fullcolour}

  \end{center}
  \begin{itemize}\item[]~\\
    #4
  \end{itemize}
  \end{frame}
}

\newcommand{\titleslideempty}[2]{
  \slidefont
  \incpage
  \begin{frame}
  \begin{center}
    \vspace*{15mm}

    {\headerfont #1\\}
        
    %% \vspace*{5mm}

    %% {\small\emph{#2}} \\

  \end{center}
  \begin{itemize}\item[]~\\
    #2
  \end{itemize}
  \end{frame}
}

\providecommand{\key}[1]{
  \addtocounter{mypage}{1}
% \immediate\write\keyfile{#1}
  \addtocontents{toc}{\hyperref[key:#1]{#1 (\arabic{mypage})}}
%  \phantomsection\label{key:#1}
%  \index{#1@{\hyperref[key:#1]{#1 (\arabic{mysec}:\arabic{mypage})}}|phantom}
  \addtocounter{mypage}{-1}
}

\providecommand{\course}{}

\providecommand{\subtopic}{}

\providecommand{\sublecture}[2]{
  \renewcommand{\subtopic}{#1}
  \slide{#1}{#2}
}

\providecommand{\sublectureHide}[2]{
  \renewcommand{\subtopic}{#1}
}

\providecommand{\story}[1]{
~

Motivation: {\tiny #1}\clearpage
}

\newenvironment{items}[1][9]{
\par\setlength{\unitlength}{1pt}\fontsize{#1}{#1}\linespread{1.2}\selectfont
\begin{list}{--}{\leftmargin4ex \rightmargin0ex \labelsep1ex \labelwidth2ex
\topsep.7ex \parsep0ex \itemsep3pt}
}{
\end{list}
}

\newenvironment{itemS}[1][4ex]{
\par
\tiny
\begin{list}{--}{\leftmargin#1 \rightmargin0ex \labelsep1ex
  \labelwidth2ex \topsep0pt \parsep0ex \itemsep2pt}
}{
\end{list}
}

\providecommand{\slidesfoot}{
  \end{document}
}


  \slideshead
  %\slidestitle
}

\providecommand{\exercises}{
  \input{../latex/style-exercises}
  \exerciseshead
}

\providecommand{\script}{
  \newcommand{\scripthead}{
  \documentclass[9pt,fleqn,twoside]{article}
  \stdpackages

  \usepackage{makeidx}
  \makeindex

  \usepackage{thmtools}
  \definecolor{shadecolor}{gray}{0.85}
  \declaretheoremstyle[
    %headfont=\normalfont\bfseries,
    %notefont=\mdseries, notebraces={(}{)},
    %bodyfont=\normalfont,
    %postheadspace=0.5em,
    %spaceabove=6pt,
    mdframed={
      %  skipabove=8pt,
      %  skipbelow=6pt,
      hidealllines=true,
      backgroundcolor={shadecolor},
      innertopmargin=8pt,
      %  innerleftmargin=3pt,
      %  innerrightmargin=3pt
    }
  ]{shaded}
  \declaretheorem[style=shaded,within=section,name=Definition]{myDefinition}
  \declaretheorem[style=shaded,within=section,name=Theorem]{myTheorem}
  \declaretheorem[style=shaded,within=section,name=Identities]{Identities}
  \declaretheorem[style=shaded,within=section,name=Example]{myExample}

  \definecolor{grey}{rgb}{.8,.8,.8}
  \definecolor{bluecol}{rgb}{0,0,.5}
  \definecolor{greencol}{rgb}{0,.4,0}
  \definecolor{shadecolor}{gray}{0.9}
  \definecolor{citcol}{rgb}{.4,.4,.4}
  \usepackage[
    %    pdftex%,
    %%    letterpaper,
    %bookmarks,
    bookmarksnumbered,
    colorlinks,
    urlcolor=bluecol,
    citecolor=black,
    linkcolor=bluecol,
    %    pagecolor=bluecol,
    pdfborder={0 0 0},
    %pdfborderstyle={/S/U/W 1},
    %%    backref,     %link from bibliography back to sections
    %%    pagebackref, %link from bibliography back to pages
    %%    pdfstartview=FitH, %fitwidth instead of fit window
    pdfpagemode=UseOutlines, %bookmarks are displayed by acrobat
    pdftitle={\course},
    pdfauthor={Marc Toussaint},
    pdfkeywords={}
  ]{hyperref}
  \DeclareGraphicsExtensions{.pdf,.png,.jpg,.eps}

  %\usepackage{multirow}
  \usepackage{multimedia}
  %\usepackage{marginnote}
  %\setbeamercolor{background canvas}{bg=}

  \usepackage[round]{natbib}
  \bibliographystyle{abbrvnat}

  \renewcommand{\r}{\varrho}
  \renewcommand{\l}{\lambda}
  \renewcommand{\L}{\Lambda}
  \renewcommand{\b}{\beta}
  \renewcommand{\d}{\delta}
  \renewcommand{\k}{\kappa}
  \renewcommand{\t}{\theta}
  \renewcommand{\O}{\Omega}
  \renewcommand{\o}{\omega}
  \renewcommand{\SS}{{\cal S}}
  \renewcommand{\=}{\!=\!}
  %\renewcommand{\boldsymbol}{}
  %\renewcommand{\Chapter}{\chapter}
  %\renewcommand{\Subsection}{\subsection}

  \renewcommand{\baselinestretch}{1.0}
  \geometry{a5paper,headsep=6mm,hdivide={10mm,*,10mm},vdivide={13mm,*,7mm}}

  \fancyhead[OL,ER]{\course, \textit{Marc Toussaint}}
  \fancyhead[OR,EL]{\thepage}
  \fancyhead[C]{}
  \fancyfoot{}
  \pagestyle{fancy}

  \renewcommand{\labelenumi}{{(\roman{enumi})}}
  \renewcommand{\theenumi}{(\roman{enumi})} %for ref
  \parindent 0pt
  \parskip .5pc

  \columnsep 6ex

  \renewcommand{\familydefault}{\sfdefault}

  \newcommand{\headerfont}{\large}%helvetica{12}{1}{b}{n}}
  \newcommand{\slidefont} {}%\helvetica{9}{1.3}{m}{n}}
  \newcommand{\storyfont} {}
  %  \renewcommand{\small}   {\helvetica{8}{1.2}{m}{n}}
  \renewcommand{\tiny}    {\footnotesize}%helvetica{7}{1.1}{m}{n}}
  \newcommand{\ttiny} {\footnotesize}%fontsize{7}{7}\selectfont}
%  \newcommand{\codefont}{\fontsize{6}{6}\selectfont}%helvetica{8}{1.2}{m}{n}}
  \newcommand{\codefont} {\helvetica{8}{1.2}{m}{n}}

  \input{../latex/macros}

  \usepackage{comment}
  \specialcomment{solution}{
    \small
    \begin{shaded}
  }{
    \end{shaded}
  }

  \graphicspath{{pics/}{../pics/}{pics-local/}}

  \mytitle{\course\\Lecture Script}
  \myauthor{Marc Toussaint}
  \date{\coursedate}
}

%%%%%%%%%%%%%%%%%%%%%%%%%%%%%%%%%%%%%%%%%%%%%%%%%%%%%%%%%%%%%%%%%%%%%%%%%%%%%%%%

\newcommand{\scripttitle}{
  \begin{document}
  \maketitle
  %\anchor{100,10}{\includegraphics[width=4cm]{optim}}
%  \vspace*{1cm}
}

%%%%%%%%%%%%%%%%%%%%%%%%%%%%%%%%%%%%%%%%%%%%%%%%%%%%%%%%%%%%%%%%%%%%%%%%%%%%%%%%

\newcounter{mypage}
\setcounter{mypage}{0}
\newcommand{\incpage}{\addtocounter{mypage}{1}}

\newcommand{\subtopic}{}
\newcommand{\pause}{}
\newcommand{\only}[1]{#1}

\renewcommand{\slides}[1][]{
  %  \clearpage
  \subsection{\topic}
  \index{\topic}
  {\small #1}
  \setcounter{mypage}{0}
  \smallskip\nopagebreak\hrule\medskip
}

\newcommand{\slidesfoot}{
  \bigskip
}

\newcommand{\sublecture}[2]{
  \phantomsection\addcontentsline{toc}{subsubsection}{#1}
  \index{#1}
}

\newcommand{\sublectureHide}[2]{
  \renewcommand{\subtopic}{#1}
}

\newcommand{\key}[1]{
  \phantomsection\addcontentsline{toc}{subsubsection}{#1}
  %\subsubsection{#1}
  \index{#1}
}

\providecommand{\defn}[1]{%
  \textbf{#1}\index{#1}%
}

\newenvironment{slidecore}[1]{
  \incpage
  \subsubsection*{#1}%{\headerfont\noindent\textbf{#1}\\}%
  \vspace{-6ex}%
  \begin{list}{$\bullet$}{\leftmargin4ex \rightmargin0ex \labelsep1ex
    \labelwidth2ex \partopsep0ex \topsep0ex \parsep.5ex \parskip0ex \itemsep0pt}\item[]~\\\nopagebreak%
}{
  \end{list}\nopagebreak%
  {\hfill\tiny \textsf{\arabic{section}.\arabic{subsection}:\arabic{mypage}}}\nopagebreak%
  \smallskip\nopagebreak\hrule
}

\newcommand{\slide}[2]{
  \begin{slidecore}{#1}
    #2
  \end{slidecore}
}

\renewcommand{\exercises}{}
\newcommand{\exercisestitle}{}
\newcommand{\exsection}[1]{\subsubsection{#1}}
\newcommand{\exsubsection}[1]{\paragraph{#1}}
\newcommand{\exerfoot}{\bigskip}

\newcommand{\story}[1]{
  \subsection*{Motivation \& Outline}
  {\storyfont\sf #1}
  \medskip\nopagebreak\hrule
}

\newcounter{savedsection}
\newcommand{\subappendix}{\setcounter{savedsection}{\arabic{section}}\appendix}
\newcommand{\noappendix}{
  \setcounter{section}{\arabic{savedsection}}% restore section number
  \setcounter{subsection}{0}% reset section counter
%  \gdef\@chapapp{\sectionname}% reset section name
  \renewcommand{\thesection}{\arabic{section}}% make section numbers arabic
}

\newenvironment{items}[1][9]{
\par\setlength{\unitlength}{1pt}\fontsize{#1}{#1}\linespread{1.2}\selectfont
\begin{list}{--}{\leftmargin4ex \rightmargin0ex \labelsep1ex \labelwidth2ex
\topsep0pt \parsep0ex \itemsep3pt}
}{
\end{list}
}

\newenvironment{itemS}[1][4ex]{
\par
\tiny
\begin{list}{--}{\leftmargin#1 \rightmargin0ex \labelsep1ex
  \labelwidth2ex \topsep0pt \parsep0ex \itemsep2pt}
}{
\end{list}
}

\newcommand{\Def}[1]{%
\textbf{#1}\index{#1}}%\marginnote{#1}}

  \scripthead
}

\providecommand{\paper}{
  \input{../latex/style-paper}
  \paperhead
}

\providecommand{\note}[1][9pt]{
  \providecommand{\notehead}[2]{
  \documentclass[#1,fleqn,twoside]{article}
  \stdpackages
  \renewcommand{\labelenumi}{{(\roman{enumi})}}
  \renewcommand{\theenumi}{(\roman{enumi})} %for ref

  \renewcommand{\baselinestretch}{#2}
  \renewcommand{\arraystretch}{1.2}
  \renewcommand{\topfraction}{1}
  \renewcommand{\bottomfraction}{1}
  \renewcommand{\textfraction}{0}
  \columnsep 5ex
  \parindent 3ex
  \parskip 1ex

  % Lists and paragraphs
  \parindent 0pt
  \topsep 4pt plus 1pt minus 2pt
  \partopsep 1pt plus 0.5pt minus 0.5pt
  \itemsep 2pt plus 1pt minus 0.5pt
  \parsep 2pt plus 1pt minus 0.5pt
  \parskip .5pc %add _in_ {thebibliography} environment in *.bbl

  \setcounter{tocdepth}{3}
  \setcounter{secnumdepth}{3}

  \geometry{a4paper,hdivide={25mm,*,25mm},vdivide={25mm,*,25mm}}

  \renewcommand{\headrulewidth}{.0pt}\renewcommand{\footrulewidth}{.0pt}\cfoot{}
  \fancyhead[OL,EC]{\it\theauthor---\today}
  \fancyhead[ER]{\leftmark}
  \fancyhead[OR,EL]{\thepage}
  \fancyfoot[EL,OR]{}
  \setlength{\headsep}{10mm}
  %\fancyhead[OL]{\rightmark}
  %\fancyfoot[EL,OR]{}


  %  \usepackage{palatino}

  \newcommand{\codefont}{\helvetica{8}{1.2}{m}{n}}

  \renewenvironment{abstract}{
    \vspace*{5ex}\begin{list}{}{
      \leftmargin3ex
      \rightmargin3ex
      \topsep-\parskip}\item[]
     \hrule\vspace{1.5ex}{\bf Abstract.~}\small}
    {\vspace{2ex}\hrule\end{list}\vspace{5ex}}
    
  \newenvironment{keyword}
    {\par{\it Keywords:~}}
    {}

  \def\makemytitle{%
    \thispagestyle{empty}
    \begin{list}{}{\leftmargin3ex \rightmargin3ex \topsep0ex \parsep0ex}\item[]
      \begin{center}
        {\fontsize{18}{25}\selectfont{\thetitle\\}}\vspace{5ex}

        {\fontsize{14}{16}\selectfont{\theauthor\\}}\vspace{1ex}

        {\footnotesize{\sl \addressFUB}\\ \emailBerlin}

        {\footnotesize \today}

        \vspace{1ex}
        {\small \published}
      \end{center}
    \end{list}
    \renewcommand{\maketitle}{\chapter{\thetitle}}
  }

  \input{../latex/macros}
  \pdflatex

  \graphicspath{{pics/}{../pics/}{pics-local/}}

  \myauthor{Marc Toussaint}
  \date{\today}
}

%%%%%%%%%%%%%%%%%%%%%%%%%%%%%%%%%%%%%%%%%%%%%%%%%%%%%%%%%%%%%%%%%%%%%%%%%%%%%%%%

\newcommand{\notetitle}{
  \begin{document}
  \thispagestyle{empty}
    
  \maketitle

}

\newenvironment{items}[1][9]{
\par\setlength{\unitlength}{1pt}\fontsize{#1}{#1}\linespread{1.2}\selectfont
\begin{list}{--}{\leftmargin4ex \rightmargin0ex \labelsep1ex \labelwidth2ex
\topsep0pt \parsep0ex \itemsep3pt}
}{
\end{list}
}

  \notehead{#1}{1.1}
}

\providecommand{\course}{NO COURSE}
\providecommand{\coursepicture}{NO PICTURE}
\providecommand{\coursedate}{NO DATE}
\providecommand{\topic}{NO TOPIC}
\providecommand{\keywords}{}
\providecommand{\exnum}{NO NUMBER}
\providecommand{\teacher}{Marc Toussaint}

\providecommand{\stdpackages}{
  \usepackage{amsmath}
  \usepackage{amssymb}
  \usepackage{amsfonts}
  \allowdisplaybreaks
  \usepackage{amsthm}
  \usepackage{eucal}
  \usepackage{graphicx}
%  \usepackage{color}
  \usepackage{geometry}
  \usepackage{framed}
  \usepackage{xcolor}
  \definecolor{shadecolor}{gray}{0.9}
  \setlength{\FrameSep}{3pt}
  \usepackage{fancyvrb}
  \fvset{numbers=none,xleftmargin=5ex,fontsize=\small}

  \usepackage{pdfpages}

  \usepackage{multicol} 
  \usepackage{fancyhdr}
}

\providecommand{\addressUSTT}{
  Machine~Learning~\&~Robotics~lab, U~Stuttgart\\\small
  Universit{\"a}tsstra{\ss}e 38, 70569~Stuttgart, Germany
}

\providecommand{\addressTUB}{
  Learning~\&~Intelligent~Systems~Lab, TU~Berlin\\\small
  Marchstr. 23, 10587 Berlin, Germany
}


\renewcommand{\course}{Robot Learning}
\renewcommand{\coursepicture}{roblearn.png}
\renewcommand{\coursedate}{Summer 2024}
\renewcommand{\teacher}{Wolfgang H{\"o}nig}

\renewcommand{\topic}{Imitation Learning 2}
\renewcommand{\keywords}{Inspired by Guanya Shi's Lecture 6}

\slides

\ifthenelse{\isundefined{\scripthead}}{

\providecommand{\info}[1]{\smallskip{\ttiny [#1]\par}}

\usepackage{bibentry}
\nobibliography*

\ifthenelse{\isundefined{\setbeamertemplate}}{}{
  \setbeamertemplate{bibliography item}{\insertbiblabel}
}

\providecommand{\citehere}[1]{{\fontsize{5}{1}\selectfont\bibentry{#1}\par}}

}{}

\providecommand{\bm}[1]{\boldsymbol{#1}}

\slidestitle

%%%%%%%%%%%%%%%%%%%%%%%%%%%%%%%%%%%%%%%%%%%%%%%%%%%%%%%%%%%%%%%%%%%%%%%%%%%%%%%%

\slide{Recap}{

\item Imitation Learning
    \begin{itemize}
        \item Given: expert demonstration data $D=\{(x^i_{1:T_i}, u^i_{1:T_i})\}_{i=1}^n$
        \item Goal: reproduce demonstrations
    \end{itemize}

\item Main Challenges: 

    \begin{itemize}
        \item Distributional Domain Shift \hspace{1cm} Solutions:
            \begin{itemize}
                \item Behavior Cloning: add noise
                \item DAgger: interactively add additional \emph{expert} data
                \item Trajectory Distribution Learning: rely on controller
            \end{itemize}
        
        \item Data Collection \hspace{1cm} Solutions:
            \begin{itemize}
                \item Humans: teleoperation, kinesthetic teaching, motion capture, videos
                \pause
                \item \textbf{high-effort computations} (w.r.t. to computation or observation), e.g., \emph{Privileged Teacher}
            \end{itemize}
    \end{itemize}

}

%%%%%%%%%%%%%%%%%%%%%%%%%%%%%%%%%%%%%%%%%%%%%%%%%%%%%%%%%%%%%%%%%%%%%%%%%%%%%%%%

\slide{Outline Today}{

\item Data Collection: Privileged Teacher 

~

\item Generative Models

~

\item Case Studies

    \begin{itemize}
        \item Quadrotor Acrobatics
        \item Learning from ALOHA data
        \item Transfer Learning
    \end{itemize}

    % Experts that aren't human
    % quadrotors
    % Learning from ALOHA data
    % Hybrid cases (Alpha Go, relationship to decision making (e.g., Ichter paper))
}

%%%%%%%%%%%%%%%%%%%%%%%%%%%%%%%%%%%%%%%%%%%%%%%%%%%%%%%%%%%%%%%%%%%%%%%%%%%%%%%%
\key{Privileged Teacher}
\slide{Privileged Teacher}{

\item So far we considered to directly learn $\pi_\t: x \mapsto u$ (or $\pi_\t: y \mapsto u$)
\item $y$ might be high-dimensional or unstructured (e.g., RGBD sequences)

\item Key insight: First learn \emph{privileged} policy (``teacher''); use it to generate data for the ``student''
    \begin{enumerate}
        \item Learn $\pi_{\t_1}: z \mapsto u$ (where $z$ contains some ``ground truth'' data, e.g., states, traffic lights, neighbor behavior)
        \item Use $\pi_{\t_1}$ to generate data $D=\{(x^i_{1:T_i}, u^i_{1:T_i})\}_{i=1}^n$
        \item Learn $\pi_{\t_2}: x \mapsto u$
    \end{enumerate}

}

%%%%%%%%%%%%%%%%%%%%%%%%%%%%%%%%%%%%%%%%%%%%%%%%%%%%%%%%%%%%%%%%%%%%%%%%%%%%%%%%

\slide{Privileged Teacher}{

\show[.65]{learningByCheating}

~

\show[.7]{learningByCheatingFig1}

~

\citehere{2020-chen-LearningCheating}
{\urlfont\url{https://youtu.be/u9ZCxxD-UUw}}

}

%%%%%%%%%%%%%%%%%%%%%%%%%%%%%%%%%%%%%%%%%%%%%%%%%%%%%%%%%%%%%%%%%%%%%%%%%%%%%%%%

\slide{Privileged Teacher}{

\item Pros and Cons compared to one-stage IL?

~

\pause

\twocol{.5}{.5}{
    Pros:
    \begin{itemize}
    \item Second stage can be easily trained with DAgger
    \item Data augmentation simple
    \end{itemize}
}{
    Cons
    \begin{itemize}
    \item Simulation-focused
    \item Hierarchical approach (requires domain knowledge)
    \end{itemize}
}

}

%%%%%%%%%%%%%%%%%%%%%%%%%%%%%%%%%%%%%%%%%%%%%%%%%%%%%%%%%%%%%%%%%%%%%%%%%%%%%%%%

\slide{Generative Models}{

\item Generative Model: 
    \begin{itemize}
        \item Input: Data $D=\{d^i\}_{i=1}^n$
        \item Learning: find distribution $p_\t$ such that $d^i \sim p_\t$
        \item Inference: generate novel data $d^* \sim p_\t$
    \end{itemize}

~
\pause
% Interaction
\item What generative models do you know?
\pause
\info{GAN, VAE, Diffusion, for details see:}

\citehere{2024-bishop-DeepLearningFoundations}

~

\item Relationship to IL
    \begin{itemize}
        \item If $D=\{(x^i_{1:T_i}, u^i_{1:T_i})\}_{i=1}^n$, we can learn \emph{conditional} distribution $p_\t(u_t | x_t)$
        \item Can also generate solution trajectories (esp. in combination with ``classic'' methods)
    \end{itemize}

}

%%%%%%%%%%%%%%%%%%%%%%%%%%%%%%%%%%%%%%%%%%%%%%%%%%%%%%%%%%%%%%%%%%%%%%%%%%%%%%%%
\key{GAN}
\slide{Generative Adverserial Network (GAN)}{

\item Train two networks (generator and discriminator)

\twocol{.5}{.4}{
\show[0.9]{bishop_fig17_1}
}{
\citehere{2024-bishop-DeepLearningFoundations}
\citehere{2017-weng-GANWGAN}
}

\item Loss function ($d_\phi$ should be 1 for real data):
\begin{align*}
\max_{\omega} \min_{\phi} -\frac{1}{N_{data}} \sum_{n\in \text{data}} \ln d_\phi(x_n) - \frac{1}{N_{gen}} \sum_{n\in\text{gen}} \ln (1-d_\phi(g_\omega(z_n)))
\end{align*}

% Interaction: what is this loss function -> cross-entropy for binary classification

}

%%%%%%%%%%%%%%%%%%%%%%%%%%%%%%%%%%%%%%%%%%%%%%%%%%%%%%%%%%%%%%%%%%%%%%%%%%%%%%%%

\slide{GAN + Imitation Learning = (GAIL)}{

\twocol{.65}{.3}{
\show[0.9]{GAIL1}
\show[0.9]{GAIL2}
}{
\item Generator is a policy $x\mapsto u$\\
\item Discriminator has $x, u$ as input 
\item Steps:
\begin{enumerate}
    \item \textbf{Rollout/Sample trajectories using generator (=policy)}
    \item Update discriminator
    \item Update policy
\end{enumerate}
}

\citehere{2016-ho-GenerativeAdversarialImitationa}

% Interaction: How is this different from DAgger?

}

%%%%%%%%%%%%%%%%%%%%%%%%%%%%%%%%%%%%%%%%%%%%%%%%%%%%%%%%%%%%%%%%%%%%%%%%%%%%%%%%
\key{VAE}
\slide{Variational Autoencoder (VAE)}{

\item Train two networks (encoder and decoder)

\twocol{.6}{.3}{
\show[1.0]{lilianweng-vae-gaussian}
}{
\citehere{2024-bishop-DeepLearningFoundations}
\citehere{2018-weng-AutoencoderBetaVAE}
\citehere{2024-chan-TutorialDiffusionModels}

{\tiny ML Lecture, slides 8 and 9}
}

\item Loss function:
\begin{align*}
\min_{\theta, \phi} - \mathbb{E}_{\mathbf{z} \sim q_\phi(\mathbf{z}\vert\mathbf{x})} \log p_\theta(\mathbf{x}\vert\mathbf{z}) + D_\text{KL}( q_\phi(\mathbf{z}\vert\mathbf{x}) \| p_\theta(\mathbf{z}) )
\end{align*}

}
%%%%%%%%%%%%%%%%%%%%%%%%%%%%%%%%%%%%%%%%%%%%%%%%%%%%%%%%%%%%%%%%%%%%%%%%%%%%%%%%

\slide{Variational Autoencoder (VAE)}{

\item Training: SGD Updates for both networks

\show[0.5]{bishop-VAE}

\info{There is an error in the Bishop book (Alg. 19.1): $\mu$ and $\sigma$ are swapped at the highlighted line}

\item Inference: Sample from Normal distribution and execute decoder
}
%%%%%%%%%%%%%%%%%%%%%%%%%%%%%%%%%%%%%%%%%%%%%%%%%%%%%%%%%%%%%%%%%%%%%%%%%%%%%%%%

\slide{Variational Autoencoder (VAE) + Imitation Learning}{

\show[0.9]{ichter1}

\cen{\showh[.4]{ichter3}\qquad\showh[.3]{ichter2}}

\citehere{2018-ichter-LearningSamplingDistributions}

}

%%%%%%%%%%%%%%%%%%%%%%%%%%%%%%%%%%%%%%%%%%%%%%%%%%%%%%%%%%%%%%%%%%%%%%%%%%%%%%%%
\key{Diffusion}
\slide{Diffusion}{

\item Train one network that ``removes'' noise

\twocol{.6}{.3}{

\show[0.9]{ddpm_fig2}

Forward diffusion process: sample $\mathbf{x}_{0}$ and add iid Gaussian noise
\begin{align*}
    q(\mathbf{x}_{1:T} \vert \mathbf{x}_0) = \prod^T_{t=1} q(\mathbf{x}_t \vert \mathbf{x}_{t-1})\\
    q(\mathbf{x}_t \vert \mathbf{x}_{t-1}) = \mathcal{N}(\mathbf{x}_t; \sqrt{1 - \beta_t} \mathbf{x}_{t-1}, \beta_t\mathbf{I})
\end{align*}

}{
\citehere{2024-bishop-DeepLearningFoundations}
\citehere{2021-weng-WhatAreDiffusion}
\citehere{2024-chan-TutorialDiffusionModels}
\citehere{2020-ho-DenoisingDiffusionProbabilistic}

{\tiny ML Lecture, slide 11}
}

}

%%%%%%%%%%%%%%%%%%%%%%%%%%%%%%%%%%%%%%%%%%%%%%%%%%%%%%%%%%%%%%%%%%%%%%%%%%%%%%%%

\slide{Diffusion}{

\item Train one network that ``removes'' noise

\twocol{.6}{.3}{

\show[0.9]{ddpm_fig2}

Reverse process: learn $p_\theta(\mathbf{x}_{t-1} \vert \mathbf{x}_t)$
\begin{align*}
p_\theta(\mathbf{x}_{0:T}) = p(\mathbf{x}_T) \prod^T_{t=1} p_\theta(\mathbf{x}_{t-1} \vert \mathbf{x}_t)\\
p_\theta(\mathbf{x}_{t-1} \vert \mathbf{x}_t) = \mathcal{N}(\mathbf{x}_{t-1}; \boldsymbol{\mu}_\theta(\mathbf{x}_t, t), \boldsymbol{\Sigma}_\theta(\mathbf{x}_t, t))
\end{align*}

}{
\citehere{2024-bishop-DeepLearningFoundations}
\citehere{2021-weng-WhatAreDiffusion}
\citehere{2024-chan-TutorialDiffusionModels}
\citehere{2020-ho-DenoisingDiffusionProbabilistic}

{\tiny ML Lecture, slide 11}
}

}

%%%%%%%%%%%%%%%%%%%%%%%%%%%%%%%%%%%%%%%%%%%%%%%%%%%%%%%%%%%%%%%%%%%%%%%%%%%%%%%%

\slide{Diffusion: Training}{

\show[0.65]{bishop_alg20_1}
}

%%%%%%%%%%%%%%%%%%%%%%%%%%%%%%%%%%%%%%%%%%%%%%%%%%%%%%%%%%%%%%%%%%%%%%%%%%%%%%%%

\slide{Diffusion: Sampling}{

\show[0.65]{bishop_alg20_2}
}

%%%%%%%%%%%%%%%%%%%%%%%%%%%%%%%%%%%%%%%%%%%%%%%%%%%%%%%%%%%%%%%%%%%%%%%%%%%%%%%%

\slide{Diffusion + Imitation Learning}{

\show[0.6]{diffusion-policy1}

\show[0.9]{diffusion-policy2}

\citehere{2023-chi-DiffusionPolicyVisuomotora}

}

%%%%%%%%%%%%%%%%%%%%%%%%%%%%%%%%%%%%%%%%%%%%%%%%%%%%%%%%%%%%%%%%%%%%%%%%%%%%%%%%

\slide{Comparison of Generative Models}{

\show[0.6]{lilianweng-generative-overview}

% Interaction
\item What are advantages / disadvantages? (e.g., sample quality, sample efficiency, distribution ``coverage'', ease of training)

}

%%%%%%%%%%%%%%%%%%%%%%%%%%%%%%%%%%%%%%%%%%%%%%%%%%%%%%%%%%%%%%%%%%%%%%%%%%%%%%%%
\key{Case Studies}
\slide{Case Study: Deep Drone Acrobatics}{

\show[0.8]{deepDroneAcrobatics}

\citehere{2020-kaufmann-DeepDroneAcrobatics}

{\urlfont\url{https://youtu.be/2N_wKXQ6MXA}}

% show video

}

%%%%%%%%%%%%%%%%%%%%%%%%%%%%%%%%%%%%%%%%%%%%%%%%%%%%%%%%%%%%%%%%%%%%%%%%%%%%%%%%

\slide{Case Study: Deep Drone Acrobatics}{

\item Input
    \begin{enumerate}
        \item \textbf{Abstraction} of sequence of last camera images (feature tracks)
        \item \textbf{Preprocessed} sequence of IMU data
        \item Reference trajectory
    \end{enumerate}
\item Output
    \begin{itemize}
        \item Desired body rates and thrust (to be tracked by attitude controller)
    \end{itemize}
\item Data
\begin{itemize}
    \item Purely from simulation (privileged expert = optimization-based MPC controller)
\end{itemize}
\item Learning
\begin{itemize}
    \item Privileged Teacher (here: given, not learned from human demonstrations)
    \item DAgger
\end{itemize}
}

%%%%%%%%%%%%%%%%%%%%%%%%%%%%%%%%%%%%%%%%%%%%%%%%%%%%%%%%%%%%%%%%%%%%%%%%%%%%%%%%

\slide{Case Study: Deep Drone Acrobatics}{

~

\show[1.0]{deepDroneAcrobatics_fig4}

~

}

%%%%%%%%%%%%%%%%%%%%%%%%%%%%%%%%%%%%%%%%%%%%%%%%%%%%%%%%%%%%%%%%%%%%%%%%%%%%%%%%

\slide{Case Study: Deep Drone Acrobatics}{

Unique design choices:
\begin{itemize}
    \item Pre-processing of input for \textbf{sim-to-real transfer}
    
    \show[0.6]{DeepDroneAcrobatics_fig5}

    \item Asynchronous network branch inference
    \item Custom DAgger rollout for \textbf{sim-to-real transfer}: only use policy if similar to expert; also include random actions
\end{itemize}

}

%%%%%%%%%%%%%%%%%%%%%%%%%%%%%%%%%%%%%%%%%%%%%%%%%%%%%%%%%%%%%%%%%%%%%%%%%%%%%%%%

\slide{Case Study: Using ALOHA Data}{

\show{aloha1}

\citehere{2023-zhao-LearningFineGrainedBimanualc}

{\urlfont\url{https://tonyzhaozh.github.io/aloha/}}

}

%%%%%%%%%%%%%%%%%%%%%%%%%%%%%%%%%%%%%%%%%%%%%%%%%%%%%%%%%%%%%%%%%%%%%%%%%%%%%%%%

\slide{Case Study: Using ALOHA Data}{

~

\show[1.0]{aloha_fig3}

~

}

%%%%%%%%%%%%%%%%%%%%%%%%%%%%%%%%%%%%%%%%%%%%%%%%%%%%%%%%%%%%%%%%%%%%%%%%%%%%%%%%

\slide{Case Study: Using ALOHA Data}{

\item Conditional Variational Autoencoder (CVAE)
    \begin{itemize}
        \item Encoder: joint positions, expert action sequence ($k >> 1$)
        \item Latent space: $z$ ``style'' (dim=32)
        \item Decoder: observations (4 RGB images), joint positions, ``style'' $z$; output: planned action sequence
    \end{itemize}

\show{aloha_fig4}

}

%%%%%%%%%%%%%%%%%%%%%%%%%%%%%%%%%%%%%%%%%%%%%%%%%%%%%%%%%%%%%%%%%%%%%%%%%%%%%%%%

\slide{Case Study: Using ALOHA Data}{

\twocol{.6}{.4}{

\item Inference: $z$ is always set to 0 (deterministic generator)
\item Key insights: transformer architectures for encoder and decoder; MPC-style encoding (action chunks + temporal ensemble)

\item Fun statistics:
\begin{itemize}
    \item 80 M parameters; 5h training (RTX 2080 Ti); 10ms inference
    \item 50 demonstrations per task (about 20min of data)
\end{itemize}
}{
    \show[0.9]{aloha_fig5}
}

}

%%%%%%%%%%%%%%%%%%%%%%%%%%%%%%%%%%%%%%%%%%%%%%%%%%%%%%%%%%%%%%%%%%%%%%%%%%%%%%%%

\slide{Case Study: Domain Adaptive Imitation Learning (DAIL)}{

\show{DAIL}

\item How to perform a task, given demonstrations from a different domain (viewpoint, embodiment, and/or dynamics mismatch)?


\show[0.6]{DAIL_fig4}

{\urlfont\url{https://youtu.be/l0tc1JCN_1M}}

\citehere{2020-kim-DomainAdaptiveImitation}

}

%%%%%%%%%%%%%%%%%%%%%%%%%%%%%%%%%%%%%%%%%%%%%%%%%%%%%%%%%%%%%%%%%%%%%%%%%%%%%%%%

\slide{Case Study: Domain Adaptive Imitation Learning (DAIL)}{

\item Given: \textbf{unprocessed} examples for the same tasks for robots $x$ and $y$
\begin{itemize}
    \item $D_{x,y}=\{(D_{M_x, T_i}, D_{M_y, T_i}) \}_{i=1}^N$ for $N$ tasks $\{T_i\}_{i=1}^N$
    \item Data is not paired/aligned, i.e., $s_x^{(t)}$ does not ``match'' $s_y^{(t)}$
    
    \show[0.5]{DAIL_fig1a}
\end{itemize}
\item Goal: Given a new demonstration of unseen task $T_j$ for $y$, transfer/execute directly (``zero-shot'') on robot $x$

}

%%%%%%%%%%%%%%%%%%%%%%%%%%%%%%%%%%%%%%%%%%%%%%%%%%%%%%%%%%%%%%%%%%%%%%%%%%%%%%%%

\slide{Case Study: Domain Adaptive Imitation Learning (DAIL)}{

\item Learning Alignment from $D_{x,y}=\{(D_{M_x, T_i}, D_{M_y, T_i}) \}_{i=1}^N$: 
    \begin{enumerate}
        \item Learn $\pi_{y,T_i}^*$ for all $T_i$ (Behavior Cloning)
        \item Learn mapping of states from $x$ to $y$: $f_{\theta_f}: x_x \mapsto x_y$
        \item Learn mapping of actions from $y$ to $x$: $g_{\theta_g} u_y \mapsto u_x$
        \item Learn dynamics/step function of $x$: $P_{\theta_P}^x: x_x, u_x \mapsto x_x$
    \end{enumerate}

}

%%%%%%%%%%%%%%%%%%%%%%%%%%%%%%%%%%%%%%%%%%%%%%%%%%%%%%%%%%%%%%%%%%%%%%%%%%%%%%%%

\slide{Case Study: Domain Adaptive Imitation Learning (DAIL)}{

\item Adaption
    \begin{enumerate}
        \item Learn $\pi_{y,T_j}^*$ for new task $T_j$ (Behavior Cloning)
        \item $\pi_{y,T_i}^*(x_x) = g_{\theta_g}(\pi_{y,T_j}^*(f_{\theta_f}(x_x)))$
    \end{enumerate}

% \item Approach: Generative Adversarial MDP Alignment (GAMA)

\show{DAIL_fig1bc}

}

%%%%%%%%%%%%%%%%%%%%%%%%%%%%%%%%%%%%%%%%%%%%%%%%%%%%%%%%%%%%%%%%%%%%%%%%%%%%%%%%

\slide{Case Study: Domain Adaptive Imitation Learning (DAIL)}{

\item Alignment Approach: Generative Adversarial MDP Alignment (GAMA)
\begin{itemize}
    \item Discriminator tries to separate real transitions ($(x,u) \to x'$) from aligned transitions
    \item ``Generator'' are $f$ and $g$ (deterministic)
\end{itemize}

\show[0.9]{DAIL_alg1}

}

% %%%%%%%%%%%%%%%%%%%%%%%%%%%%%%%%%%%%%%%%%%%%%%%%%%%%%%%%%%%%%%%%%%%%%%%%%%%%%%%%

% \slide{Case Study: NTE}{

% \item TODO: perhaps better in the multi-robot learning class
% \item alpha zero?
% \item the locomotion paper guanya had?
% \item maybe car racing paper?

% }

% %%%%%%%%%%%%%%%%%%%%%%%%%%%%%%%%%%%%%%%%%%%%%%%%%%%%%%%%%%%%%%%%%%%%%%%%%%%%%%%%

% \slide{To Do}{

% privileged teaching examples: locomotion (?), drone acrobatics; put at beginning or end?


% }


%%%%%%%%%%%%%%%%%%%%%%%%%%%%%%%%%%%%%%%%%%%%%%%%%%%%%%%%%%%%%%%%%%%%%%%%%%%%%%%%
%%%%%%%%%%%%%%%%%%%%%%%%%%%%%%%%%%%%%%%%%%%%%%%%%%%%%%%%%%%%%%%%%%%%%%%%%%%%%%%%

\slide{Conclusion}{\label{lastpage}

\item Imitation Learning works well for robotics
    \begin{itemize}
        \item Efficient, effective, stable training
        \item Fast inference
        \item State-of-the-art real-robot results (mobile robots, manipulation, planning)
    \end{itemize}

\item Main challenge: acquire labeled data
    \begin{itemize}
        \item Simulation possible (e.g., make slow algorithms fast) $\Rightarrow$ Use \textbf{DAgger} and/or \textbf{privileged teacher} paradigm
        \item Only real data $\Rightarrow$ intuitive data collection interfaces, powerful generative and sequence models, transfer learning
    \end{itemize}

\item Details can be tricky (what to learn [policy, trajectory, value function], how to represent inputs, network architectures)

\item Not discussed (yet): How to become better than the ``expert'' (notion of reward)

}

%%%%%%%%%%%%%%%%%%%%%%%%%%%%%%%%%%%%%%%%%%%%%%%%%%%%%%%%%%%%%%%%%%%%%%%%%%%%%%%%

\ttiny
\ifthenelse{\isundefined{\scripthead}}{
\bibliographystyle{plainurl-lis}
\bibliography{b2-ImitationLearning}
}{}

\slidesfoot
