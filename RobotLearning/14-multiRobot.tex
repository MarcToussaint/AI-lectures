\providecommand{\slides}{
  \newcommand{\slideshead}{
  \newcommand{\thepage}{\arabic{mypage}}
  %beamer
%  \documentclass[t,hyperref={bookmarks=true}]{beamer}
%  \geometry{papersize={171mm,96mm}}
  \documentclass[t,hyperref={bookmarks=true},aspectratio=169]{beamer}
  \setbeamersize{text margin left=5mm}
  \setbeamersize{text margin right=5mm}
  \usetheme{default}
  \usefonttheme[onlymath]{serif}
  \setbeamertemplate{navigation symbols}{}
  \setbeamertemplate{itemize items}{{\color{black}$\bullet$}}

  \newwrite\keyfile

  %\usepackage{palatino}
  \stdpackages
  %\usepackage{tikz} \usetikzlibrary {shapes.geometric} 
  \usepackage{multimedia}
  \usepackage[utf8]{inputenc}

  %%% geometry/spacing issues
  %
  \definecolor{bluecol}{rgb}{0,0,.5}
  \definecolor{greencol}{rgb}{0,.6,0}
  \definecolor{citcol}{rgb}{.4,.4,.4}
  %\renewcommand{\baselinestretch}{1.1}
  \renewcommand{\arraystretch}{1.2}
  \columnsep 0mm

  \columnseprule 0pt
  \parindent 0ex
  \parskip 0ex
  %\setlength{\itemparsep}{3ex}
  %\renewcommand{\labelitemi}{\rule[3pt]{10pt}{10pt}~}
  %\renewcommand{\labelenumi}{\textbf{(\arabic{enumi})}}
  \setbeamertemplate{enumerate item}{(\roman{enumi})}
  \newcommand{\headerfont}{\helvetica{14}{1.5}{b}{n}}
  \newcommand{\slidefont} {\helvetica{11}{1.4}{m}{n}}
  %\newcommand{\codefont} {\helvetica{8}{1.2}{m}{n}}
  \newcommand{\urlfont} {\fontsize{6}{1.1}\selectfont}
  \renewcommand{\small} {\helvetica{10}{1.4}{m}{n}}
  \renewcommand{\tiny} {\helvetica{8}{1.3}{m}{n}}
  \newcommand{\ttiny} {\helvetica{7}{1.3}{m}{n}}

  %%% count pages properly and put the page number in bottom right
  %
  \newcounter{mypage}
  \newcommand{\incpage}{\addtocounter{mypage}{1}\setcounter{page}{\arabic{mypage}}}
  \setcounter{mypage}{0}
  \resetcounteronoverlays{page}

  \pagestyle{fancy}
  %\setlength{\headsep}{10mm}
  %\addtolength{\footheight}{15mm}
  \renewcommand{\headrulewidth}{0pt} %1pt}
  \renewcommand{\footrulewidth}{0pt} %.5pt}
  \cfoot{}
  \rhead{}
  \lhead{}
  %% \lfoot{\vspace*{-3mm}\hspace*{-3mm}\helvetica{5}{1.3}{m}{n}{\texttt{github.com/MarcToussaint/AI-lectures}}}
\lfoot{}
%  \rfoot{{\tiny\textsf{AI -- \topic -- \subtopic -- \arabic{mypage}/\pageref{lastpage}}}}
  %\lfoot{\raisebox{5mm}{\tiny\textsf{\slideauthor}}}
  %\rfoot{\raisebox{5mm}{\tiny\textsf{\slidevenue{} -- \arabic{mypage}/\pageref{lastpage}}}}
  %\rfoot{~\anchor{30,12}{\tiny\textsf{\thepage/\pageref{lastpage}}}}
  %\lfoot{\small\textsf{Marc Toussaint}}
%  \rfoot{\vspace*{-4.5mm}{\tiny\textsf{\color{gray}\topic\ -- \subtopic\ -- \arabic{mypage}/\pageref{lastpage}}}\hspace*{-4mm}}
  \rfoot{\vspace*{-4.5mm}{\tiny\textsf{\color{gray}\topic\ -- \arabic{mypage}/\pageref{lastpage}}}\hspace*{-4mm}}
%  \rfoot{~\anchor{-10,12}{\tiny\textsf{\color{gray}\topic\ --  \arabic{mypage}/\pageref{lastpage}}}}
  \lfoot{\vspace*{-4.5mm}{\hspace*{-3mm}\includegraphics[height=4mm]{LIS-logo-longText}}}

  \definecolor{grey}{rgb}{.8,.8,.8}
  \definecolor{head}{rgb}{.85,.9,.9}
%  \definecolor{blue}{rgb}{.0,.0,.5}
%  \definecolor{green}{rgb}{.0,.5,.0}
  \definecolor{red}{rgb}{.8,.0,.0}
  \newcommand{\inverted}{
    \definecolor{main}{rgb}{1,1,1}
    \color{main}
    \pagecolor[rgb]{.3,.3,.3}
  }
  \input{../latex/macros}

  \graphicspath{{pics/}{../pics/}{pics-local/}}
}

\newcommand{\slidestitle}{
  \title{\course \topic}
  \author{Marc Toussaint}
  \institute{Learning \& Intelligent Systems Lab, TU Berlin}

  \begin{document}


  %% title slide!
  \slide{}{
    \thispagestyle{empty}

    \twocol{.35}{.55}{
      %\vspace*{-5mm}%
      \hspace*{-5mm}%
      \includegraphics[width=1.\columnwidth]{\coursepicture}
    }{\center

      \textbf{\fontsize{17}{20}\selectfont \course}

      ~

      %Lecture
      \topic\\

      \vspace{1cm}

      {\tiny~\emph{\keywords}~\\}

      \vspace{1cm}

      \teacher
      
      Technical University of Berlin

      \coursedate

      ~

    }
  }
}

\newcommand{\slide}[2]{
  \slidefont
  \incpage\begin{frame}
  \addcontentsline{toc}{section}{#1}
  \vfill
  {\headerfont #1} \vspace*{-2ex}
  \begin{itemize}\item[]~\\
    #2
  \end{itemize}
  \vfill
  \end{frame}
}

% use \begin{frame}[fragile] around slidecore!
\newenvironment{slidecore}[1]{
  \slidefont\incpage
  \addcontentsline{toc}{section}{#1}
  \vfill
  {\headerfont #1} \vspace*{-2ex}
  \begin{itemize}\item[]~\\
}{
  \end{itemize}
  \vfill
}

\newcommand{\titleslide}[4][Marc Toussaint]{
  \newcommand{\slideauthor}{#1}
  \newcommand{\slidevenue}{#3}
  \slidefont
  \incpage
  \begin{frame}
  \begin{center}
    \vspace*{15mm}

    {\headerfont #2\\}
        
    \vspace*{10mm}

    #1 \\

    \vspace*{5mm}

    {\small
      Learning \& Intelligent Systems Lab, TU Berlin\\
%      Science of Intelligence Cluster of Excellence, Berlin\\
      Max Planck Fellow, Institute for Intelligent Systems\\ %Physical Reasoning \& Manipulation Lab -- 
%      MIT CSAIL\\
%      mtoussai@mit.edu,~ marc.toussaint@informatik.uni-stuttgart.de

      \vspace*{10mm}

      \emph{#3}
    }

    \vspace*{0mm}

    %\includegraphics[scale=.1]{pics/eushield-fullcolour}

  \end{center}
  \begin{itemize}\item[]~\\
    #4
  \end{itemize}
  \end{frame}
}

\newcommand{\titleslideempty}[2]{
  \slidefont
  \incpage
  \begin{frame}
  \begin{center}
    \vspace*{15mm}

    {\headerfont #1\\}
        
    %% \vspace*{5mm}

    %% {\small\emph{#2}} \\

  \end{center}
  \begin{itemize}\item[]~\\
    #2
  \end{itemize}
  \end{frame}
}

\providecommand{\key}[1]{
  \addtocounter{mypage}{1}
% \immediate\write\keyfile{#1}
  \addtocontents{toc}{\hyperref[key:#1]{#1 (\arabic{mypage})}}
%  \phantomsection\label{key:#1}
%  \index{#1@{\hyperref[key:#1]{#1 (\arabic{mysec}:\arabic{mypage})}}|phantom}
  \addtocounter{mypage}{-1}
}

\providecommand{\course}{}

\providecommand{\subtopic}{}

\providecommand{\sublecture}[2]{
  \renewcommand{\subtopic}{#1}
  \slide{#1}{#2}
}

\providecommand{\sublectureHide}[2]{
  \renewcommand{\subtopic}{#1}
}

\providecommand{\story}[1]{
~

Motivation: {\tiny #1}\clearpage
}

\newenvironment{items}[1][9]{
\par\setlength{\unitlength}{1pt}\fontsize{#1}{#1}\linespread{1.2}\selectfont
\begin{list}{--}{\leftmargin4ex \rightmargin0ex \labelsep1ex \labelwidth2ex
\topsep.7ex \parsep0ex \itemsep3pt}
}{
\end{list}
}

\newenvironment{itemS}[1][4ex]{
\par
\tiny
\begin{list}{--}{\leftmargin#1 \rightmargin0ex \labelsep1ex
  \labelwidth2ex \topsep0pt \parsep0ex \itemsep2pt}
}{
\end{list}
}

\providecommand{\slidesfoot}{
  \end{document}
}


  \slideshead
  %\slidestitle
}

\providecommand{\exercises}{
  \input{../latex/style-exercises}
  \exerciseshead
}

\providecommand{\script}{
  \newcommand{\scripthead}{
  \documentclass[9pt,fleqn,twoside]{article}
  \stdpackages

  \usepackage{makeidx}
  \makeindex

  \usepackage{thmtools}
  \definecolor{shadecolor}{gray}{0.85}
  \declaretheoremstyle[
    %headfont=\normalfont\bfseries,
    %notefont=\mdseries, notebraces={(}{)},
    %bodyfont=\normalfont,
    %postheadspace=0.5em,
    %spaceabove=6pt,
    mdframed={
      %  skipabove=8pt,
      %  skipbelow=6pt,
      hidealllines=true,
      backgroundcolor={shadecolor},
      innertopmargin=8pt,
      %  innerleftmargin=3pt,
      %  innerrightmargin=3pt
    }
  ]{shaded}
  \declaretheorem[style=shaded,within=section,name=Definition]{myDefinition}
  \declaretheorem[style=shaded,within=section,name=Theorem]{myTheorem}
  \declaretheorem[style=shaded,within=section,name=Identities]{Identities}
  \declaretheorem[style=shaded,within=section,name=Example]{myExample}

  \definecolor{grey}{rgb}{.8,.8,.8}
  \definecolor{bluecol}{rgb}{0,0,.5}
  \definecolor{greencol}{rgb}{0,.4,0}
  \definecolor{shadecolor}{gray}{0.9}
  \definecolor{citcol}{rgb}{.4,.4,.4}
  \usepackage[
    %    pdftex%,
    %%    letterpaper,
    %bookmarks,
    bookmarksnumbered,
    colorlinks,
    urlcolor=bluecol,
    citecolor=black,
    linkcolor=bluecol,
    %    pagecolor=bluecol,
    pdfborder={0 0 0},
    %pdfborderstyle={/S/U/W 1},
    %%    backref,     %link from bibliography back to sections
    %%    pagebackref, %link from bibliography back to pages
    %%    pdfstartview=FitH, %fitwidth instead of fit window
    pdfpagemode=UseOutlines, %bookmarks are displayed by acrobat
    pdftitle={\course},
    pdfauthor={Marc Toussaint},
    pdfkeywords={}
  ]{hyperref}
  \DeclareGraphicsExtensions{.pdf,.png,.jpg,.eps}

  %\usepackage{multirow}
  \usepackage{multimedia}
  %\usepackage{marginnote}
  %\setbeamercolor{background canvas}{bg=}

  \usepackage[round]{natbib}
  \bibliographystyle{abbrvnat}

  \renewcommand{\r}{\varrho}
  \renewcommand{\l}{\lambda}
  \renewcommand{\L}{\Lambda}
  \renewcommand{\b}{\beta}
  \renewcommand{\d}{\delta}
  \renewcommand{\k}{\kappa}
  \renewcommand{\t}{\theta}
  \renewcommand{\O}{\Omega}
  \renewcommand{\o}{\omega}
  \renewcommand{\SS}{{\cal S}}
  \renewcommand{\=}{\!=\!}
  %\renewcommand{\boldsymbol}{}
  %\renewcommand{\Chapter}{\chapter}
  %\renewcommand{\Subsection}{\subsection}

  \renewcommand{\baselinestretch}{1.0}
  \geometry{a5paper,headsep=6mm,hdivide={10mm,*,10mm},vdivide={13mm,*,7mm}}

  \fancyhead[OL,ER]{\course, \textit{Marc Toussaint}}
  \fancyhead[OR,EL]{\thepage}
  \fancyhead[C]{}
  \fancyfoot{}
  \pagestyle{fancy}

  \renewcommand{\labelenumi}{{(\roman{enumi})}}
  \renewcommand{\theenumi}{(\roman{enumi})} %for ref
  \parindent 0pt
  \parskip .5pc

  \columnsep 6ex

  \renewcommand{\familydefault}{\sfdefault}

  \newcommand{\headerfont}{\large}%helvetica{12}{1}{b}{n}}
  \newcommand{\slidefont} {}%\helvetica{9}{1.3}{m}{n}}
  \newcommand{\storyfont} {}
  %  \renewcommand{\small}   {\helvetica{8}{1.2}{m}{n}}
  \renewcommand{\tiny}    {\footnotesize}%helvetica{7}{1.1}{m}{n}}
  \newcommand{\ttiny} {\footnotesize}%fontsize{7}{7}\selectfont}
%  \newcommand{\codefont}{\fontsize{6}{6}\selectfont}%helvetica{8}{1.2}{m}{n}}
  \newcommand{\codefont} {\helvetica{8}{1.2}{m}{n}}

  \input{../latex/macros}

  \usepackage{comment}
  \specialcomment{solution}{
    \small
    \begin{shaded}
  }{
    \end{shaded}
  }

  \graphicspath{{pics/}{../pics/}{pics-local/}}

  \mytitle{\course\\Lecture Script}
  \myauthor{Marc Toussaint}
  \date{\coursedate}
}

%%%%%%%%%%%%%%%%%%%%%%%%%%%%%%%%%%%%%%%%%%%%%%%%%%%%%%%%%%%%%%%%%%%%%%%%%%%%%%%%

\newcommand{\scripttitle}{
  \begin{document}
  \maketitle
  %\anchor{100,10}{\includegraphics[width=4cm]{optim}}
%  \vspace*{1cm}
}

%%%%%%%%%%%%%%%%%%%%%%%%%%%%%%%%%%%%%%%%%%%%%%%%%%%%%%%%%%%%%%%%%%%%%%%%%%%%%%%%

\newcounter{mypage}
\setcounter{mypage}{0}
\newcommand{\incpage}{\addtocounter{mypage}{1}}

\newcommand{\subtopic}{}
\newcommand{\pause}{}
\newcommand{\only}[1]{#1}

\renewcommand{\slides}[1][]{
  %  \clearpage
  \subsection{\topic}
  \index{\topic}
  {\small #1}
  \setcounter{mypage}{0}
  \smallskip\nopagebreak\hrule\medskip
}

\newcommand{\slidesfoot}{
  \bigskip
}

\newcommand{\sublecture}[2]{
  \phantomsection\addcontentsline{toc}{subsubsection}{#1}
  \index{#1}
}

\newcommand{\sublectureHide}[2]{
  \renewcommand{\subtopic}{#1}
}

\newcommand{\key}[1]{
  \phantomsection\addcontentsline{toc}{subsubsection}{#1}
  %\subsubsection{#1}
  \index{#1}
}

\providecommand{\defn}[1]{%
  \textbf{#1}\index{#1}%
}

\newenvironment{slidecore}[1]{
  \incpage
  \subsubsection*{#1}%{\headerfont\noindent\textbf{#1}\\}%
  \vspace{-6ex}%
  \begin{list}{$\bullet$}{\leftmargin4ex \rightmargin0ex \labelsep1ex
    \labelwidth2ex \partopsep0ex \topsep0ex \parsep.5ex \parskip0ex \itemsep0pt}\item[]~\\\nopagebreak%
}{
  \end{list}\nopagebreak%
  {\hfill\tiny \textsf{\arabic{section}.\arabic{subsection}:\arabic{mypage}}}\nopagebreak%
  \smallskip\nopagebreak\hrule
}

\newcommand{\slide}[2]{
  \begin{slidecore}{#1}
    #2
  \end{slidecore}
}

\renewcommand{\exercises}{}
\newcommand{\exercisestitle}{}
\newcommand{\exsection}[1]{\subsubsection{#1}}
\newcommand{\exsubsection}[1]{\paragraph{#1}}
\newcommand{\exerfoot}{\bigskip}

\newcommand{\story}[1]{
  \subsection*{Motivation \& Outline}
  {\storyfont\sf #1}
  \medskip\nopagebreak\hrule
}

\newcounter{savedsection}
\newcommand{\subappendix}{\setcounter{savedsection}{\arabic{section}}\appendix}
\newcommand{\noappendix}{
  \setcounter{section}{\arabic{savedsection}}% restore section number
  \setcounter{subsection}{0}% reset section counter
%  \gdef\@chapapp{\sectionname}% reset section name
  \renewcommand{\thesection}{\arabic{section}}% make section numbers arabic
}

\newenvironment{items}[1][9]{
\par\setlength{\unitlength}{1pt}\fontsize{#1}{#1}\linespread{1.2}\selectfont
\begin{list}{--}{\leftmargin4ex \rightmargin0ex \labelsep1ex \labelwidth2ex
\topsep0pt \parsep0ex \itemsep3pt}
}{
\end{list}
}

\newenvironment{itemS}[1][4ex]{
\par
\tiny
\begin{list}{--}{\leftmargin#1 \rightmargin0ex \labelsep1ex
  \labelwidth2ex \topsep0pt \parsep0ex \itemsep2pt}
}{
\end{list}
}

\newcommand{\Def}[1]{%
\textbf{#1}\index{#1}}%\marginnote{#1}}

  \scripthead
}

\providecommand{\paper}{
  \input{../latex/style-paper}
  \paperhead
}

\providecommand{\note}[1][9pt]{
  \providecommand{\notehead}[2]{
  \documentclass[#1,fleqn,twoside]{article}
  \stdpackages
  \renewcommand{\labelenumi}{{(\roman{enumi})}}
  \renewcommand{\theenumi}{(\roman{enumi})} %for ref

  \renewcommand{\baselinestretch}{#2}
  \renewcommand{\arraystretch}{1.2}
  \renewcommand{\topfraction}{1}
  \renewcommand{\bottomfraction}{1}
  \renewcommand{\textfraction}{0}
  \columnsep 5ex
  \parindent 3ex
  \parskip 1ex

  % Lists and paragraphs
  \parindent 0pt
  \topsep 4pt plus 1pt minus 2pt
  \partopsep 1pt plus 0.5pt minus 0.5pt
  \itemsep 2pt plus 1pt minus 0.5pt
  \parsep 2pt plus 1pt minus 0.5pt
  \parskip .5pc %add _in_ {thebibliography} environment in *.bbl

  \setcounter{tocdepth}{3}
  \setcounter{secnumdepth}{3}

  \geometry{a4paper,hdivide={25mm,*,25mm},vdivide={25mm,*,25mm}}

  \renewcommand{\headrulewidth}{.0pt}\renewcommand{\footrulewidth}{.0pt}\cfoot{}
  \fancyhead[OL,EC]{\it\theauthor---\today}
  \fancyhead[ER]{\leftmark}
  \fancyhead[OR,EL]{\thepage}
  \fancyfoot[EL,OR]{}
  \setlength{\headsep}{10mm}
  %\fancyhead[OL]{\rightmark}
  %\fancyfoot[EL,OR]{}


  %  \usepackage{palatino}

  \newcommand{\codefont}{\helvetica{8}{1.2}{m}{n}}

  \renewenvironment{abstract}{
    \vspace*{5ex}\begin{list}{}{
      \leftmargin3ex
      \rightmargin3ex
      \topsep-\parskip}\item[]
     \hrule\vspace{1.5ex}{\bf Abstract.~}\small}
    {\vspace{2ex}\hrule\end{list}\vspace{5ex}}
    
  \newenvironment{keyword}
    {\par{\it Keywords:~}}
    {}

  \def\makemytitle{%
    \thispagestyle{empty}
    \begin{list}{}{\leftmargin3ex \rightmargin3ex \topsep0ex \parsep0ex}\item[]
      \begin{center}
        {\fontsize{18}{25}\selectfont{\thetitle\\}}\vspace{5ex}

        {\fontsize{14}{16}\selectfont{\theauthor\\}}\vspace{1ex}

        {\footnotesize{\sl \addressFUB}\\ \emailBerlin}

        {\footnotesize \today}

        \vspace{1ex}
        {\small \published}
      \end{center}
    \end{list}
    \renewcommand{\maketitle}{\chapter{\thetitle}}
  }

  \input{../latex/macros}
  \pdflatex

  \graphicspath{{pics/}{../pics/}{pics-local/}}

  \myauthor{Marc Toussaint}
  \date{\today}
}

%%%%%%%%%%%%%%%%%%%%%%%%%%%%%%%%%%%%%%%%%%%%%%%%%%%%%%%%%%%%%%%%%%%%%%%%%%%%%%%%

\newcommand{\notetitle}{
  \begin{document}
  \thispagestyle{empty}
    
  \maketitle

}

\newenvironment{items}[1][9]{
\par\setlength{\unitlength}{1pt}\fontsize{#1}{#1}\linespread{1.2}\selectfont
\begin{list}{--}{\leftmargin4ex \rightmargin0ex \labelsep1ex \labelwidth2ex
\topsep0pt \parsep0ex \itemsep3pt}
}{
\end{list}
}

  \notehead{#1}{1.1}
}

\providecommand{\course}{NO COURSE}
\providecommand{\coursepicture}{NO PICTURE}
\providecommand{\coursedate}{NO DATE}
\providecommand{\topic}{NO TOPIC}
\providecommand{\keywords}{}
\providecommand{\exnum}{NO NUMBER}
\providecommand{\teacher}{Marc Toussaint}

\providecommand{\stdpackages}{
  \usepackage{amsmath}
  \usepackage{amssymb}
  \usepackage{amsfonts}
  \allowdisplaybreaks
  \usepackage{amsthm}
  \usepackage{eucal}
  \usepackage{graphicx}
%  \usepackage{color}
  \usepackage{geometry}
  \usepackage{framed}
  \usepackage{xcolor}
  \definecolor{shadecolor}{gray}{0.9}
  \setlength{\FrameSep}{3pt}
  \usepackage{fancyvrb}
  \fvset{numbers=none,xleftmargin=5ex,fontsize=\small}

  \usepackage{pdfpages}

  \usepackage{multicol} 
  \usepackage{fancyhdr}
}

\providecommand{\addressUSTT}{
  Machine~Learning~\&~Robotics~lab, U~Stuttgart\\\small
  Universit{\"a}tsstra{\ss}e 38, 70569~Stuttgart, Germany
}

\providecommand{\addressTUB}{
  Learning~\&~Intelligent~Systems~Lab, TU~Berlin\\\small
  Marchstr. 23, 10587 Berlin, Germany
}


\renewcommand{\course}{Robot Learning}
\renewcommand{\coursepicture}{roblearn.png}
\renewcommand{\coursedate}{Summer 2024}
\renewcommand{\teacher}{Wolfgang H{\"o}nig}

\renewcommand{\topic}{Multi-Robot Learning}

\slides

\ifthenelse{\isundefined{\scripthead}}{

\providecommand{\info}[1]{\smallskip{\ttiny [#1]\par}}

\usepackage{bibentry}
\nobibliography*

\ifthenelse{\isundefined{\setbeamertemplate}}{}{
  \setbeamertemplate{bibliography item}{\insertbiblabel}
}

\providecommand{\citehere}[1]{{\fontsize{5}{1}\selectfont\bibentry{#1}\par}}

}{}


\providecommand{\bm}[1]{\boldsymbol{#1}}

% Tikz support
\ifthenelse{\isundefined{\scripthead}}{
\usepackage{tikz,tkz-euclide}
\usetikzlibrary{shapes.geometric, arrows, shadows.blur, shadows, shapes.multipart}
\usetikzlibrary{positioning}
\usetikzlibrary{intersections}
\usetikzlibrary{tikzmark}
}{}

% custom definitions
\ifthenelse{\isundefined{\scripthead}}{
  \usepackage{breqn}
  \usepackage{bm}
}{}
\newcommand{\B}[1]{\mathbf{#1}}
\newcommand{\fav}{\B{f}_a}
\newcommand{\faf}{f_a}
\newcommand{\tauav}{\bm{\tau}_a}
\newcommand{\tauaf}{\tau_a}
\newcommand{\famax}{f_{a,\mathrm{max}}}
\newcommand{\tauamax}{\tau_{a,\mathrm{max}}}
\newcommand{\favhat}{\hat{\B{f}}_a}
\newcommand{\tauavhat}{\hat{\bm{\tau}}_a}
\newcommand{\fnom}{\bm{\Phi}}
\newcommand{\set}{\mathbf{r}}
\newcommand{\env}{\mathrm{env}}
\newcommand{\sm}{\mathrm{small}}
\newcommand{\la}{\mathrm{large}}

\slidestitle


\slide{Motivation: Multi-Robot Systems}{

\item Multiple robots (typically in a team) with a common goal

\item Typical promises:
    \begin{itemize}
        \item Achieve goal faster
        \item Achieve goal more robustly
        \item Higher flexibility (esp. heterogeneous systems)
        \item Cheaper (?)
    \end{itemize}

}

%%%%%%%%%%%%%%%%%%%%%%%%%%%%%%%%%%%%%%%%%%%%%%%%%%%%%%%%%%%%%%%%%%%%%%%%%%%%%%%%

\slide{Motivation: Multi-Robot Systems}{

\item Successful (industrial) solutions

    \begin{itemize}
        \item Warehouse logistics (Amazon Robotics, former Kiva systems)

        % https://youtu.be/TUx-ljgB-5Q
        \movh[]{.6}{movies-wolfgang/amazon-warehouses}

        \item Aerial Drone shows (Intel, Verity Studios)
    \end{itemize}
}

%%%%%%%%%%%%%%%%%%%%%%%%%%%%%%%%%%%%%%%%%%%%%%%%%%%%%%%%%%%%%%%%%%%%%%%%%%%%%%%%


\slide{Motivation: Multi-Robot System Challenges}{

\item Controls: additional constraint for inter-robot collision avoidance

\item Decision Making: information sharing, task assignment, curse-of-dimensionality for centralized approaches, safety/robustness for decentralized systems

\item Perception: sensing team members, sensor fusion

}

%%%%%%%%%%%%%%%%%%%%%%%%%%%%%%%%%%%%%%%%%%%%%%%%%%%%%%%%%%%%%%%%%%%%%%%%%%%%%%%%


\slide{Outline}{

\item \textbf{Handling Dynamic Neighbors}

\begin{itemize}
    \item LSTMs
    \item CNNs
    \item DeepSets
    \item Graph Neural Networks
\end{itemize}

\item Multi-Agent Reinforcement Learning (MARL)

\item Discussion / Open Challenges

}

%%%%%%%%%%%%%%%%%%%%%%%%%%%%%%%%%%%%%%%%%%%%%%%%%%%%%%%%%%%%%%%%%%%%%%%%%%%%%%%%

\slide{Dynamic Neighbors}{

\item Team of robots has time-varying neighbors/observations/communication links

\item Often need to learn with time-varying input dimensionality
\begin{itemize}
  \item Example: (Distributed) collision avoidance maps observation of neighboring robots to actions $f(\mathcal Y) \to u$
\end{itemize}

\item Learned functions need to be \textbf{permutation-invariant} and support \textbf{dynamic domain cardinality}

}

%%%%%%%%%%%%%%%%%%%%%%%%%%%%%%%%%%%%%%%%%%%%%%%%%%%%%%%%%%%%%%%%%%%%%%%%%%%%%%%%

\slide{LSTMs \cite{2018-everett-MotionPlanningDynamic}}{

\show{2018-everett-MotionPlanningDynamic.png}

\item Key idea: Feed observations of neighbors into an LSTM (closest neighbor last)

\show[0.5]{2018-everett-MotionPlanningDynamic-fig3.png}

}

%%%%%%%%%%%%%%%%%%%%%%%%%%%%%%%%%%%%%%%%%%%%%%%%%%%%%%%%%%%%%%%%%%%%%%%%%%%%%%%%

\slide{CNNs \cite{2019-sartoretti-PRIMALPathfindingReinforcement}}{

\twocol{.6}{.4}{
  \show[1.0]{2019-sartoretti-PRIMALPathfindingReinforcement.png}

  \begin{itemize}
      \item Key idea: Encode neighbor information as a picture
      \item Videos: {\urlfont\url{https://goo.gl/T627XD}}
    \end{itemize}
}{
    \show{2019-sartoretti-PRIMALPathfindingReinforcement-fig2.png}
}

}

%%%%%%%%%%%%%%%%%%%%%%%%%%%%%%%%%%%%%%%%%%%%%%%%%%%%%%%%%%%%%%%%%%%%%%%%%%%%%%%%
\key{Deep Sets}
\slide{Deep Sets \cite{2017-zaheer-DeepSets}}{

    \begin{itemize}
    \item Any continuous, permutation-invariant function $f(\mathcal X)$ can be approximated:
      \vspace{15mm}
      \begin{equation*}
        f(\mathcal X) \approx 
          \tikzmark{rho}\rho \left( 
            \tikzmark{sum}\sum_{x\in\mathcal X} 
          % \tikz[baseline]{\node[draw=black,ellipse,thick,anchor=base] (phi) {$\phi(x)$};}  
          \tikzmark{phi}\phi(x)  
        \right)
      \end{equation*}
      \vspace{10mm}
    \item Improvement over Convolutional NN (\textbf{CNN}): continuous space, efficiency
    \item Example:\\[2mm]
      \includegraphics[width=0.6cm]{deepsets/digit5.png} + \includegraphics[width=0.6cm]{deepsets/digit4.png} = 9\\[2mm]
      \includegraphics[width=0.6cm]{deepsets/digit9.png} + \includegraphics[width=0.6cm]{deepsets/digit6.png} + \includegraphics[width=0.6cm]{deepsets/digit5.png} = 20

  \end{itemize}

  \begin{tikzpicture}[
    remember picture,
    overlay,
    expl/.style={draw=black,fill=black!10,rounded corners,text width=5cm},
    arrow/.style={line width=1.5mm,->,>=stealth},
  ]
  \node [expl, above right=of pic cs:phi] (hs) {Learns \textbf{representation} of each element};
  \draw[arrow,line width=0.5mm, black, bend right] (hs.west) to ([xshift=0.4em,yshift=0.5em]pic cs:phi);

  \node [expl, below=of pic cs:sum] (sp) {\textbf{superposition} in hidden state};
  \draw[arrow,line width=0.5mm, black] (sp) to ([xshift=0.4em,yshift=-1.5em]pic cs:sum);

  \node [expl, above left=of pic cs:rho] (agg) {Learns \textbf{aggregation} of hidden state};
  \draw[arrow,line width=0.5mm, black, bend left] (agg.east) to ([xshift=0.4em,yshift=0.5em]pic cs:rho);


  \end{tikzpicture}

}

%%%%%%%%%%%%%%%%%%%%%%%%%%%%%%%%%%%%%%%%%%%%%%%%%%%%%%%%%%%%%%%%%%%%%%%%%%%%%%%%

\slide{Case Study: GLAS \cite{2020-riviere-GLASGlobaltoLocalSafe}}{

\item Goal: imitate (slow) centralized controller using only local observations: $\pi: y \mapsto u$

\item Data: Example trajectories by solving many multi-robot motion planning instances with a centralized planner

\item Approach: Behavior Cloning + Privileged Teacher
}

%%%%%%%%%%%%%%%%%%%%%%%%%%%%%%%%%%%%%%%%%%%%%%%%%%%%%%%%%%%%%%%%%%%%%%%%%%%%%%%%

\slide{Case Study: GLAS \cite{2020-riviere-GLASGlobaltoLocalSafe}}{

\show[0.8]{glas/data_expert_gen.png}

}

%%%%%%%%%%%%%%%%%%%%%%%%%%%%%%%%%%%%%%%%%%%%%%%%%%%%%%%%%%%%%%%%%%%%%%%%%%%%%%%%

\slide{Case Study: GLAS \cite{2020-riviere-GLASGlobaltoLocalSafe}}{

\show[0.8]{glas/data_mask_nonlocal.png}

}

%%%%%%%%%%%%%%%%%%%%%%%%%%%%%%%%%%%%%%%%%%%%%%%%%%%%%%%%%%%%%%%%%%%%%%%%%%%%%%%%

\slide{Case Study: GLAS \cite{2020-riviere-GLASGlobaltoLocalSafe}}{

\item Train (5 small feedforward networks trained jointly)

\show[0.8]{glas/architecture_simple.pdf}

}

%%%%%%%%%%%%%%%%%%%%%%%%%%%%%%%%%%%%%%%%%%%%%%%%%%%%%%%%%%%%%%%%%%%%%%%%%%%%%%%%

\slide{Case Study: GLAS \cite{2020-riviere-GLASGlobaltoLocalSafe}}{

\item How would one train this in practice in pyTorch?
\info{variable number of neighbors vs. batching}

} 

%%%%%%%%%%%%%%%%%%%%%%%%%%%%%%%%%%%%%%%%%%%%%%%%%%%%%%%%%%%%%%%%%%%%%%%%%%%%%%%%

\slide{Case Study: Neural-Swarm2 \cite{2022-shi-NeuralSwarm2PlanningControl}}{

\item Goal: predict aerodynamic interaction
\info{unmodeled physics, as a function of neighbors' positions}

\show[0.4]{neuralswarm2/fig1b.jpg}

\item Data: Real flight tests (synchronized trajectories with poses of robots and measured accelerations and motor commands)

\item Approach: Behavior Cloning

}

%%%%%%%%%%%%%%%%%%%%%%%%%%%%%%%%%%%%%%%%%%%%%%%%%%%%%%%%%%%%%%%%%%%%%%%%%%%%%%%%

\slide{Case Study: Neural-Swarm2 \cite{2022-shi-NeuralSwarm2PlanningControl}: Heterogeneous Deep Sets}{

  \twocol{.45}{.45}{
      \vspace{-2.0cm}
      \begin{equation*}
        \favhat^{(i)}
        \approx
        \tikzmark{rhoh}\bm{\rho}_{\mathcal{I}(i)}\left(\tikzmark{sumh}\sum_{k=1}^K\sum_{\B{x}^{(ij)}\in \set_{\mathrm{type}_k}^{(i)}}\tikzmark{phih}\bm{\phi}_{\mathcal{I}(j)}(\B{x}^{(ij)})\right)
      \end{equation*}


      \begin{tikzpicture}[
        remember picture,
        overlay,
        expl/.style={draw=black,fill=black!10,rounded corners,text width=3cm},
        arrow/.style={line width=1.5mm,->,>=stealth},
      ]
      \node [expl, above=of pic cs:phih] (hs) {Learns \textbf{representation} from type $\mathcal{I}(j)$ neighbor};
      \draw[arrow,line width=0.5mm, black] (hs.south) to ([xshift=0.4em,yshift=0.5em]pic cs:phih);

      \node [expl, below=of pic cs:sumh] (sp) {\textbf{superposition} in hidden state};
      \draw[arrow,line width=0.5mm, black] (sp) to ([xshift=0.4em,yshift=-1.5em]pic cs:sumh);

      \node [expl, above=of pic cs:rhoh] (agg) {Learns \textbf{aggregation} for type $\mathcal{I}(i)$};
      \draw[arrow,line width=0.5mm, black] (agg.south) to ([xshift=0.4em,yshift=0.5em]pic cs:rhoh);
      \end{tikzpicture}

  }{
      \includegraphics[width=\linewidth]{neuralswarm2/fig1a.pdf}
      \begin{multline*}
        \fav^{(3)} \approx \bm{\rho}_{\la}\left(\bm{\phi}_{\sm}(\B{x}^{(31)})+\\
        \bm{\phi}_{\sm}(\B{x}^{(32)}) +\bm{\phi}_{\env}(\B{x}^{(34)})\right)
      \end{multline*}
  }
  \begin{itemize}
    \item \textbf{Expressiveness}: can approximate any $K$-Group permutation-invariant function
    \item \textbf{Efficient}: only $2K$ networks need to be trained
  \end{itemize}
}

%%%%%%%%%%%%%%%%%%%%%%%%%%%%%%%%%%%%%%%%%%%%%%%%%%%%%%%%%%%%%%%%%%%%%%%%%%%%%%%%

\slide{Case Study: Neural-Swarm2 \cite{2022-shi-NeuralSwarm2PlanningControl}}{

  \vspace{5mm}
  \twocol{.5}{.15}{
    \includegraphics[height=0.8\textheight]{neuralswarm2/heatmap-ssls.png}
  }{
    \includegraphics[height=0.8\textheight]{neuralswarm2/heatmap-legend.png}
  }
}

%%%%%%%%%%%%%%%%%%%%%%%%%%%%%%%%%%%%%%%%%%%%%%%%%%%%%%%%%%%%%%%%%%%%%%%%%%%%%%%%

\slide{Case Study: Neural-Swarm2 \cite{2022-shi-NeuralSwarm2PlanningControl}}{

\url{https://youtu.be/Y02juH6BDxo}

}

%%%%%%%%%%%%%%%%%%%%%%%%%%%%%%%%%%%%%%%%%%%%%%%%%%%%%%%%%%%%%%%%%%%%%%%%%%%%%%%%
\key{GNNs}
\slide{Graph Neural Networks (GNNs)}{

\item Inspiration: CNNs as graph

\show{2024-bishop-DeepLearningFoundations-fig13.3.png}

\citehere{2024-bishop-DeepLearningFoundations}

}

%%%%%%%%%%%%%%%%%%%%%%%%%%%%%%%%%%%%%%%%%%%%%%%%%%%%%%%%%%%%%%%%%%%%%%%%%%%%%%%%

\slide{Graph Neural Networks (GNNs)}{

\item Graph $\mathcal G = (\mathcal V, \mathcal E)$

\item Basic case: learn features for each node $n\in\mathcal V$

\item Use $L$ layers with $D$-dimensional vector $h_n^{(l)}$
}

%%%%%%%%%%%%%%%%%%%%%%%%%%%%%%%%%%%%%%%%%%%%%%%%%%%%%%%%%%%%%%%%%%%%%%%%%%%%%%%%

\slide{Graph Neural Networks (GNNs)}{

\show{2024-bishop-DeepLearningFoundations-alg13.1.png}

}

%%%%%%%%%%%%%%%%%%%%%%%%%%%%%%%%%%%%%%%%%%%%%%%%%%%%%%%%%%%%%%%%%%%%%%%%%%%%%%%%

\slide{Graph Neural Networks (GNNs)}{

\item Examples for Aggregate/Update:
\begin{itemize}
  \item Aggregate($\{ h_m^{(l)}: m\in \mathcal N(n) \}$) = $MLP_\rho \left( \sum_{m\in\mathcal N(n)} MLP_\phi(h_m^{(l)})\right)$
  \item Update($h_n^{(l)}, z_n^{(l)}$) = $f(W_{self} h_n^{(l)} + W_{neigh} z_n^{(l)} + b)$
\end{itemize}

\item Extensions to have input/output features per edge and graph
  \info{See e.g., \cite{2024-bishop-DeepLearningFoundations}}


\item Training ``as usual'' (on whole graphs)

\item In practice: PyG {\urlfont\url{https://www.pyg.org/}} or DGL {\urlfont\url{https://www.dgl.ai/}}

}

%%%%%%%%%%%%%%%%%%%%%%%%%%%%%%%%%%%%%%%%%%%%%%%%%%%%%%%%%%%%%%%%%%%%%%%%%%%%%%%%

\slide{Case Study: Learning to Communicate for Multi-Robot Path Finding \cite{2020-li-GraphNeuralNetworks}}{

\show[1.0]{2020-li-GraphNeuralNetworks.png}

\item Goal: Learn how to communicate to imitate a centralized Multi-Agent Path Finding expert

\item Data: Trajectories computed by a centralized expert

\item Approach: IL w/ DAgger

}

%%%%%%%%%%%%%%%%%%%%%%%%%%%%%%%%%%%%%%%%%%%%%%%%%%%%%%%%%%%%%%%%%%%%%%%%%%%%%%%%

\slide{Case Study: Learning to Communicate for Multi-Robot Path Finding \cite{2020-li-GraphNeuralNetworks}}{

\show[0.9]{2020-li-GraphNeuralNetworks-fig1.png}

}

%%%%%%%%%%%%%%%%%%%%%%%%%%%%%%%%%%%%%%%%%%%%%%%%%%%%%%%%%%%%%%%%%%%%%%%%%%%%%%%%

\slide{Case Study: Multi-Robot Perception \cite{2022-zhou-MultiRobotCollaborativePerception}}{


\twocol{.4}{.6}{
\show[1.0]{2022-zhou-MultiRobotCollaborativePerception.png}


\item Goal: Learn what to communicate for depth estimation or segmentation

\item Data: Labeled Data mostly from simulator; some from real flights

\item Approach: Behavior Cloning
}{
\show{2022-zhou-MultiRobotCollaborativePerception-fig1.png}

}

\item Video: {\urlfont\url{https://youtu.be/2bdhLI3dqo0}}

}

%%%%%%%%%%%%%%%%%%%%%%%%%%%%%%%%%%%%%%%%%%%%%%%%%%%%%%%%%%%%%%%%%%%%%%%%%%%%%%%%

\slide{GNN Applications}{

\item Flocking (in simulation) \cite{2020-tolstaya-LearningDecentralizedControllers,2021-kortvelesy-ModGNNExpertPolicy,2022-gama-SynthesizingDecentralizedControllers}

\item Navigation (simulation + RL) \cite{2022-yu-LearningControlAdmissibility}

\item Graph Control Barrier Function (simulation + IL w/ DAgger) \cite{2023-zhang-NeuralGraphControl}

\item Learning to Communicate Variations \cite{2021-li-MessageAwareGraphAttention,2022-gama-SynthesizingDecentralizedControllers}

}

%%%%%%%%%%%%%%%%%%%%%%%%%%%%%%%%%%%%%%%%%%%%%%%%%%%%%%%%%%%%%%%%%%%%%%%%%%%%%%%%


\slide{Outline}{

\item Handling Dynamic Neighbors

\begin{itemize}
    \item LSTMs
    \item CNNs
    \item DeepSets
    \item Graph Neural Networks
\end{itemize}

\item \textbf{Multi-Agent Reinforcement Learning (MARL)}

\item Discussion / Open Challenges

}

%%%%%%%%%%%%%%%%%%%%%%%%%%%%%%%%%%%%%%%%%%%%%%%%%%%%%%%%%%%%%%%%%%%%%%%%%%%%%%%%
\key{MARL}
\slide{MARL Definition}{

\item Single Robot: MDP $(\SS, \AA, P, R, P_0, \g)$ with state space $\SS$, action space $\AA$, transition probabilities $P(s_{t\po} \| s_t,a_t)$, reward fct $r_t = R(s_t,a_t)$, initial state distribution $P_0(s_0)$, and discounting factor $\g\in[0,1]$.
\pause

\item Multi-Robot: Markov game $(N, \SS, \AA, P, R, P_0, \g)$ with $N$ robots, $\SS$ \emph{joint} state space, $\AA=A_1 \times A_2 \times \ldots \times A_N$ \emph{joint} action space, reward fct $r_1,\ldots,r_N = R(s,a)$

\item Goal: Find policy (or policies) that maximize expected reward

}

%%%%%%%%%%%%%%%%%%%%%%%%%%%%%%%%%%%%%%%%%%%%%%%%%%%%%%%%%%%%%%%%%%%%%%%%%%%%%%%%

\slide{Rewards}{

\item \textbf{Fully cooperative}: $r_1 = r_2 = \ldots = r_N$
\info{No credit assignment; difficult to train}

\item \textbf{Competitive}: zero-sum games ($\sum_i r_i = 0$), prey-predator games (cooperative per team; competitive per game)

\item \textbf{Mixed Cooperative-Competitive}: (local) reward shaping, to achieve a common goal


}

%%%%%%%%%%%%%%%%%%%%%%%%%%%%%%%%%%%%%%%%%%%%%%%%%%%%%%%%%%%%%%%%%%%%%%%%%%%%%%%%

\slide{Learning}{

\item \textbf{Centralized} model as stacked robot (centralized training \& inference)

\item \textbf{Independent Learning} each robot learns own policy (decentralized training \& inference)

\item \textbf{Centralized Training Decentralized Execution (CTDE)}
}

%%%%%%%%%%%%%%%%%%%%%%%%%%%%%%%%%%%%%%%%%%%%%%%%%%%%%%%%%%%%%%%%%%%%%%%%%%%%%%%%

\slide{Challenges}{

\item Non-Stationarity: if policy of other agents can't be observed, the Markov assumption is violated (e.g., distributed Q-Learning)

\item Scalability: in standard policy gradient algorithms, the probability of estimating the policy gradient correctly might decrease exponentially with the number of agents
\info{Concrete example: appendix of \cite{2017-lowe-MultiAgentActorCriticMixed}}
}

%%%%%%%%%%%%%%%%%%%%%%%%%%%%%%%%%%%%%%%%%%%%%%%%%%%%%%%%%%%%%%%%%%%%%%%%%%%%%%%%

\slide{Approaches}{

\item Centralized critic, e.g., Multi-Agent deep deterministic policy gradient (MADDPG, \cite{2017-lowe-MultiAgentActorCriticMixed})

\item Factorized value functions, e.g., Value Decomposition Networks (VDN, \cite{2018-sunehag-ValueDecompositionNetworksCooperative})

\item Communication Learning
}

%%%%%%%%%%%%%%%%%%%%%%%%%%%%%%%%%%%%%%%%%%%%%%%%%%%%%%%%%%%%%%%%%%%%%%%%%%%%%%%%

\slide{Practical Considerations}{

\item VMAS (Vectorized Multi-Agent Simulator for Collective Robot Learning) {\urlfont\url{https://github.com/proroklab/VectorizedMultiAgentSimulator}}
\info{Simple 2D physics engine build in pyTorch}

\item MARLlib {\urlfont\url{https://github.com/Replicable-MARL/MARLlib}}

\item More Details/Overview about MARL:

\citehere{2022-wang-DistributedReinforcementLearning}

\citehere{2023-orr-MultiAgentDeepReinforcement}
}

%%%%%%%%%%%%%%%%%%%%%%%%%%%%%%%%%%%%%%%%%%%%%%%%%%%%%%%%%%%%%%%%%%%%%%%%%%%%%%%%

\slide{Case Study: Distributed Collision Avoidance (Ground) \cite{2020-fan-DistributedMultirobotCollision}}{

\show[0.6]{2020-fan-DistributedMultirobotCollision.png}

\twocol{.3}{.7}{
  \show[1.0]{2020-fan-DistributedMultirobotCollision-fig4.png}
}{
  \show{2020-fan-DistributedMultirobotCollision-fig3.png}
}

}

%%%%%%%%%%%%%%%%%%%%%%%%%%%%%%%%%%%%%%%%%%%%%%%%%%%%%%%%%%%%%%%%%%%%%%%%%%%%%%%%

\slide{Case Study: Distributed Collision Avoidance (Ground) \cite{2020-fan-DistributedMultirobotCollision}}{

\item Goal: find decentralized policy: $\pi: y, g \mapsto u$

\item Data: Collected in simulation during RL (input LIDAR, relative goal, velocity; output: action)

\item Approach: PPO (centralized learning, decentralized execution; shared policy)

\item Video: {\urlfont\url{https://sites.google.com/view/hybridmrca}}

}

%%%%%%%%%%%%%%%%%%%%%%%%%%%%%%%%%%%%%%%%%%%%%%%%%%%%%%%%%%%%%%%%%%%%%%%%%%%%%%%%

\slide{Case Study: Distributed Collision Avoidance (UAVs) \cite{2024-huang-CollisionAvoidanceNavigation}}{

\item Goal: find decentralized policy: $\pi: y, g \mapsto u$

\item Data: Collected in simulation during RL (input state, nearby obstacles, nearby neighbors; output: thrust per rotor)

\item Approach: IPPO (centralized learning, decentralized execution; shared policy)

\item Video: {\urlfont\url{https://sites.google.com/view/obst-avoid-swarm-rl}}

}

%%%%%%%%%%%%%%%%%%%%%%%%%%%%%%%%%%%%%%%%%%%%%%%%%%%%%%%%%%%%%%%%%%%%%%%%%%%%%%%%

\slide{Case Study: Neural Tree Expansion \cite{2021-riviere-NeuralTreeExpansion}}{

\item Goal: find decentralized policies for multi-team games (e.g., reach-target avoid)


\twocol{.45}{.45}{
\showh[1.0]{2021-riviere-NeuralTreeExpansion-fig1.png}
}{
  \begin{itemize}
    \item Data: Collected with a neural-biased ``expert'' (large Monte-Carlo Tree Search)

    \item Approach: MCTS + IL + DAgger (essentially: AlphaZero in continuous state spaces)
    
    \item Video: {\urlfont\url{https://youtu.be/mklbTfWl7DE}}
  \end{itemize}
}

}

%%%%%%%%%%%%%%%%%%%%%%%%%%%%%%%%%%%%%%%%%%%%%%%%%%%%%%%%%%%%%%%%%%%%%%%%%%%%%%%%


\slide{Outline}{

\item Handling Dynamic Neighbors

\begin{itemize}
    \item LSTMs
    \item CNNs
    \item DeepSets
    \item Graph Neural Networks
\end{itemize}

\item Multi-Agent Reinforcement Learning (MARL)

\item \textbf{Discussion / Open Challenges}

}

%%%%%%%%%%%%%%%%%%%%%%%%%%%%%%%%%%%%%%%%%%%%%%%%%%%%%%%%%%%%%%%%%%%%%%%%%%%%%%%%
\key{DiNNO}

\slide{DiNNO: Distributed Neural Network Optimization \cite{2022-yu-DiNNODistributedNeural}}{

\show[0.45]{2022-yu-DiNNODistributedNeural-fig1.png}

\item Collect data locally, local augmented Lagrangian update, share resulting weights via consensus

\item Works for IL and RL

\item Web: {\urlfont\url{https://msl.stanford.edu/projects/dist_nn_train}}
}

%%%%%%%%%%%%%%%%%%%%%%%%%%%%%%%%%%%%%%%%%%%%%%%%%%%%%%%%%%%%%%%%%%%%%%%%%%%%%%%%

\slide{LLMs and Multi-Robots \cite{2024-chen-WhySolvingMultiagent}}{

\show[0.5]{2024-chen-WhySolvingMultiagent.png}

\item (Arxiv, Jan. 2024)

\show[1.0]{2024-chen-WhySolvingMultiagent-fig1.png }

}

%%%%%%%%%%%%%%%%%%%%%%%%%%%%%%%%%%%%%%%%%%%%%%%%%%%%%%%%%%%%%%%%%%%%%%%%%%%%%%%%


\slide{LLMs and Multi-Robots \cite{2024-chen-WhySolvingMultiagent}}{

\show{2024-chen-WhySolvingMultiagent-fig2.png}

}

%%%%%%%%%%%%%%%%%%%%%%%%%%%%%%%%%%%%%%%%%%%%%%%%%%%%%%%%%%%%%%%%%%%%%%%%%%%%%%%%


\slide{Open Challenges}{

\item Deployment to real robots (especially RL)

\item Safety (esp. partially unknown dynamics, perception)

\item Interpretability (of communication)


}

%%%%%%%%%%%%%%%%%%%%%%%%%%%%%%%%%%%%%%%%%%%%%%%%%%%%%%%%%%%%%%%%%%%%%%%%%%%%%%%%

\slide{Conclusion}{\label{lastpage}

\item Multi-Robot brings new challenges
  \begin{itemize}
    \item Large state space (or violation of Markov assumption)
    \item Dynamic number of neighbors
    \item Reasoning about communication
  \end{itemize}

\item Deep Sets: permutation invariant architecture that is easy to train and computationally efficient
\info{useful for $\pi: x, \mathcal N \mapsto u$}

\item GNN: Generalization of deep sets
\info{useful for learning communication}

\item Learning a decentralized policy from a centralized expert works well (IL + DAgger)

\item Deployment to real robot teams remains challenging

}

%%%%%%%%%%%%%%%%%%%%%%%%%%%%%%%%%%%%%%%%%%%%%%%%%%%%%%%%%%%%%%%%%%%%%%%%%%%%%%%%

\ttiny
\ifthenelse{\isundefined{\scripthead}}{
\bibliographystyle{plainurl-lis}
\bibliography{b8-MultiRobotLearning}
}{}

\slidesfoot
