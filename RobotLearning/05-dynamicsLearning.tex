\providecommand{\slides}{
  \newcommand{\slideshead}{
  \newcommand{\thepage}{\arabic{mypage}}
  %beamer
%  \documentclass[t,hyperref={bookmarks=true}]{beamer}
%  \geometry{papersize={171mm,96mm}}
  \documentclass[t,hyperref={bookmarks=true},aspectratio=169]{beamer}
  \setbeamersize{text margin left=5mm}
  \setbeamersize{text margin right=5mm}
  \usetheme{default}
  \usefonttheme[onlymath]{serif}
  \setbeamertemplate{navigation symbols}{}
  \setbeamertemplate{itemize items}{{\color{black}$\bullet$}}

  \newwrite\keyfile

  %\usepackage{palatino}
  \stdpackages
  %\usepackage{tikz} \usetikzlibrary {shapes.geometric} 
  \usepackage{multimedia}
  \usepackage[utf8]{inputenc}

  %%% geometry/spacing issues
  %
  \definecolor{bluecol}{rgb}{0,0,.5}
  \definecolor{greencol}{rgb}{0,.6,0}
  \definecolor{citcol}{rgb}{.4,.4,.4}
  %\renewcommand{\baselinestretch}{1.1}
  \renewcommand{\arraystretch}{1.2}
  \columnsep 0mm

  \columnseprule 0pt
  \parindent 0ex
  \parskip 0ex
  %\setlength{\itemparsep}{3ex}
  %\renewcommand{\labelitemi}{\rule[3pt]{10pt}{10pt}~}
  %\renewcommand{\labelenumi}{\textbf{(\arabic{enumi})}}
  \setbeamertemplate{enumerate item}{(\roman{enumi})}
  \newcommand{\headerfont}{\helvetica{14}{1.5}{b}{n}}
  \newcommand{\slidefont} {\helvetica{11}{1.4}{m}{n}}
  %\newcommand{\codefont} {\helvetica{8}{1.2}{m}{n}}
  \newcommand{\urlfont} {\fontsize{6}{1.1}\selectfont}
  \renewcommand{\small} {\helvetica{10}{1.4}{m}{n}}
  \renewcommand{\tiny} {\helvetica{8}{1.3}{m}{n}}
  \newcommand{\ttiny} {\helvetica{7}{1.3}{m}{n}}

  %%% count pages properly and put the page number in bottom right
  %
  \newcounter{mypage}
  \newcommand{\incpage}{\addtocounter{mypage}{1}\setcounter{page}{\arabic{mypage}}}
  \setcounter{mypage}{0}
  \resetcounteronoverlays{page}

  \pagestyle{fancy}
  %\setlength{\headsep}{10mm}
  %\addtolength{\footheight}{15mm}
  \renewcommand{\headrulewidth}{0pt} %1pt}
  \renewcommand{\footrulewidth}{0pt} %.5pt}
  \cfoot{}
  \rhead{}
  \lhead{}
  %% \lfoot{\vspace*{-3mm}\hspace*{-3mm}\helvetica{5}{1.3}{m}{n}{\texttt{github.com/MarcToussaint/AI-lectures}}}
\lfoot{}
%  \rfoot{{\tiny\textsf{AI -- \topic -- \subtopic -- \arabic{mypage}/\pageref{lastpage}}}}
  %\lfoot{\raisebox{5mm}{\tiny\textsf{\slideauthor}}}
  %\rfoot{\raisebox{5mm}{\tiny\textsf{\slidevenue{} -- \arabic{mypage}/\pageref{lastpage}}}}
  %\rfoot{~\anchor{30,12}{\tiny\textsf{\thepage/\pageref{lastpage}}}}
  %\lfoot{\small\textsf{Marc Toussaint}}
%  \rfoot{\vspace*{-4.5mm}{\tiny\textsf{\color{gray}\topic\ -- \subtopic\ -- \arabic{mypage}/\pageref{lastpage}}}\hspace*{-4mm}}
  \rfoot{\vspace*{-4.5mm}{\tiny\textsf{\color{gray}\topic\ -- \arabic{mypage}/\pageref{lastpage}}}\hspace*{-4mm}}
%  \rfoot{~\anchor{-10,12}{\tiny\textsf{\color{gray}\topic\ --  \arabic{mypage}/\pageref{lastpage}}}}
  \lfoot{\vspace*{-4.5mm}{\hspace*{-3mm}\includegraphics[height=4mm]{LIS-logo-longText}}}

  \definecolor{grey}{rgb}{.8,.8,.8}
  \definecolor{head}{rgb}{.85,.9,.9}
%  \definecolor{blue}{rgb}{.0,.0,.5}
%  \definecolor{green}{rgb}{.0,.5,.0}
  \definecolor{red}{rgb}{.8,.0,.0}
  \newcommand{\inverted}{
    \definecolor{main}{rgb}{1,1,1}
    \color{main}
    \pagecolor[rgb]{.3,.3,.3}
  }
  \input{../latex/macros}

  \graphicspath{{pics/}{../pics/}{pics-local/}}
}

\newcommand{\slidestitle}{
  \title{\course \topic}
  \author{Marc Toussaint}
  \institute{Learning \& Intelligent Systems Lab, TU Berlin}

  \begin{document}


  %% title slide!
  \slide{}{
    \thispagestyle{empty}

    \twocol{.35}{.55}{
      %\vspace*{-5mm}%
      \hspace*{-5mm}%
      \includegraphics[width=1.\columnwidth]{\coursepicture}
    }{\center

      \textbf{\fontsize{17}{20}\selectfont \course}

      ~

      %Lecture
      \topic\\

      \vspace{1cm}

      {\tiny~\emph{\keywords}~\\}

      \vspace{1cm}

      \teacher
      
      Technical University of Berlin

      \coursedate

      ~

    }
  }
}

\newcommand{\slide}[2]{
  \slidefont
  \incpage\begin{frame}
  \addcontentsline{toc}{section}{#1}
  \vfill
  {\headerfont #1} \vspace*{-2ex}
  \begin{itemize}\item[]~\\
    #2
  \end{itemize}
  \vfill
  \end{frame}
}

% use \begin{frame}[fragile] around slidecore!
\newenvironment{slidecore}[1]{
  \slidefont\incpage
  \addcontentsline{toc}{section}{#1}
  \vfill
  {\headerfont #1} \vspace*{-2ex}
  \begin{itemize}\item[]~\\
}{
  \end{itemize}
  \vfill
}

\newcommand{\titleslide}[4][Marc Toussaint]{
  \newcommand{\slideauthor}{#1}
  \newcommand{\slidevenue}{#3}
  \slidefont
  \incpage
  \begin{frame}
  \begin{center}
    \vspace*{15mm}

    {\headerfont #2\\}
        
    \vspace*{10mm}

    #1 \\

    \vspace*{5mm}

    {\small
      Learning \& Intelligent Systems Lab, TU Berlin\\
%      Science of Intelligence Cluster of Excellence, Berlin\\
      Max Planck Fellow, Institute for Intelligent Systems\\ %Physical Reasoning \& Manipulation Lab -- 
%      MIT CSAIL\\
%      mtoussai@mit.edu,~ marc.toussaint@informatik.uni-stuttgart.de

      \vspace*{10mm}

      \emph{#3}
    }

    \vspace*{0mm}

    %\includegraphics[scale=.1]{pics/eushield-fullcolour}

  \end{center}
  \begin{itemize}\item[]~\\
    #4
  \end{itemize}
  \end{frame}
}

\newcommand{\titleslideempty}[2]{
  \slidefont
  \incpage
  \begin{frame}
  \begin{center}
    \vspace*{15mm}

    {\headerfont #1\\}
        
    %% \vspace*{5mm}

    %% {\small\emph{#2}} \\

  \end{center}
  \begin{itemize}\item[]~\\
    #2
  \end{itemize}
  \end{frame}
}

\providecommand{\key}[1]{
  \addtocounter{mypage}{1}
% \immediate\write\keyfile{#1}
  \addtocontents{toc}{\hyperref[key:#1]{#1 (\arabic{mypage})}}
%  \phantomsection\label{key:#1}
%  \index{#1@{\hyperref[key:#1]{#1 (\arabic{mysec}:\arabic{mypage})}}|phantom}
  \addtocounter{mypage}{-1}
}

\providecommand{\course}{}

\providecommand{\subtopic}{}

\providecommand{\sublecture}[2]{
  \renewcommand{\subtopic}{#1}
  \slide{#1}{#2}
}

\providecommand{\sublectureHide}[2]{
  \renewcommand{\subtopic}{#1}
}

\providecommand{\story}[1]{
~

Motivation: {\tiny #1}\clearpage
}

\newenvironment{items}[1][9]{
\par\setlength{\unitlength}{1pt}\fontsize{#1}{#1}\linespread{1.2}\selectfont
\begin{list}{--}{\leftmargin4ex \rightmargin0ex \labelsep1ex \labelwidth2ex
\topsep.7ex \parsep0ex \itemsep3pt}
}{
\end{list}
}

\newenvironment{itemS}[1][4ex]{
\par
\tiny
\begin{list}{--}{\leftmargin#1 \rightmargin0ex \labelsep1ex
  \labelwidth2ex \topsep0pt \parsep0ex \itemsep2pt}
}{
\end{list}
}

\providecommand{\slidesfoot}{
  \end{document}
}


  \slideshead
  %\slidestitle
}

\providecommand{\exercises}{
  \input{../latex/style-exercises}
  \exerciseshead
}

\providecommand{\script}{
  \newcommand{\scripthead}{
  \documentclass[9pt,fleqn,twoside]{article}
  \stdpackages

  \usepackage{makeidx}
  \makeindex

  \usepackage{thmtools}
  \definecolor{shadecolor}{gray}{0.85}
  \declaretheoremstyle[
    %headfont=\normalfont\bfseries,
    %notefont=\mdseries, notebraces={(}{)},
    %bodyfont=\normalfont,
    %postheadspace=0.5em,
    %spaceabove=6pt,
    mdframed={
      %  skipabove=8pt,
      %  skipbelow=6pt,
      hidealllines=true,
      backgroundcolor={shadecolor},
      innertopmargin=8pt,
      %  innerleftmargin=3pt,
      %  innerrightmargin=3pt
    }
  ]{shaded}
  \declaretheorem[style=shaded,within=section,name=Definition]{myDefinition}
  \declaretheorem[style=shaded,within=section,name=Theorem]{myTheorem}
  \declaretheorem[style=shaded,within=section,name=Identities]{Identities}
  \declaretheorem[style=shaded,within=section,name=Example]{myExample}

  \definecolor{grey}{rgb}{.8,.8,.8}
  \definecolor{bluecol}{rgb}{0,0,.5}
  \definecolor{greencol}{rgb}{0,.4,0}
  \definecolor{shadecolor}{gray}{0.9}
  \definecolor{citcol}{rgb}{.4,.4,.4}
  \usepackage[
    %    pdftex%,
    %%    letterpaper,
    %bookmarks,
    bookmarksnumbered,
    colorlinks,
    urlcolor=bluecol,
    citecolor=black,
    linkcolor=bluecol,
    %    pagecolor=bluecol,
    pdfborder={0 0 0},
    %pdfborderstyle={/S/U/W 1},
    %%    backref,     %link from bibliography back to sections
    %%    pagebackref, %link from bibliography back to pages
    %%    pdfstartview=FitH, %fitwidth instead of fit window
    pdfpagemode=UseOutlines, %bookmarks are displayed by acrobat
    pdftitle={\course},
    pdfauthor={Marc Toussaint},
    pdfkeywords={}
  ]{hyperref}
  \DeclareGraphicsExtensions{.pdf,.png,.jpg,.eps}

  %\usepackage{multirow}
  \usepackage{multimedia}
  %\usepackage{marginnote}
  %\setbeamercolor{background canvas}{bg=}

  \usepackage[round]{natbib}
  \bibliographystyle{abbrvnat}

  \renewcommand{\r}{\varrho}
  \renewcommand{\l}{\lambda}
  \renewcommand{\L}{\Lambda}
  \renewcommand{\b}{\beta}
  \renewcommand{\d}{\delta}
  \renewcommand{\k}{\kappa}
  \renewcommand{\t}{\theta}
  \renewcommand{\O}{\Omega}
  \renewcommand{\o}{\omega}
  \renewcommand{\SS}{{\cal S}}
  \renewcommand{\=}{\!=\!}
  %\renewcommand{\boldsymbol}{}
  %\renewcommand{\Chapter}{\chapter}
  %\renewcommand{\Subsection}{\subsection}

  \renewcommand{\baselinestretch}{1.0}
  \geometry{a5paper,headsep=6mm,hdivide={10mm,*,10mm},vdivide={13mm,*,7mm}}

  \fancyhead[OL,ER]{\course, \textit{Marc Toussaint}}
  \fancyhead[OR,EL]{\thepage}
  \fancyhead[C]{}
  \fancyfoot{}
  \pagestyle{fancy}

  \renewcommand{\labelenumi}{{(\roman{enumi})}}
  \renewcommand{\theenumi}{(\roman{enumi})} %for ref
  \parindent 0pt
  \parskip .5pc

  \columnsep 6ex

  \renewcommand{\familydefault}{\sfdefault}

  \newcommand{\headerfont}{\large}%helvetica{12}{1}{b}{n}}
  \newcommand{\slidefont} {}%\helvetica{9}{1.3}{m}{n}}
  \newcommand{\storyfont} {}
  %  \renewcommand{\small}   {\helvetica{8}{1.2}{m}{n}}
  \renewcommand{\tiny}    {\footnotesize}%helvetica{7}{1.1}{m}{n}}
  \newcommand{\ttiny} {\footnotesize}%fontsize{7}{7}\selectfont}
%  \newcommand{\codefont}{\fontsize{6}{6}\selectfont}%helvetica{8}{1.2}{m}{n}}
  \newcommand{\codefont} {\helvetica{8}{1.2}{m}{n}}

  \input{../latex/macros}

  \usepackage{comment}
  \specialcomment{solution}{
    \small
    \begin{shaded}
  }{
    \end{shaded}
  }

  \graphicspath{{pics/}{../pics/}{pics-local/}}

  \mytitle{\course\\Lecture Script}
  \myauthor{Marc Toussaint}
  \date{\coursedate}
}

%%%%%%%%%%%%%%%%%%%%%%%%%%%%%%%%%%%%%%%%%%%%%%%%%%%%%%%%%%%%%%%%%%%%%%%%%%%%%%%%

\newcommand{\scripttitle}{
  \begin{document}
  \maketitle
  %\anchor{100,10}{\includegraphics[width=4cm]{optim}}
%  \vspace*{1cm}
}

%%%%%%%%%%%%%%%%%%%%%%%%%%%%%%%%%%%%%%%%%%%%%%%%%%%%%%%%%%%%%%%%%%%%%%%%%%%%%%%%

\newcounter{mypage}
\setcounter{mypage}{0}
\newcommand{\incpage}{\addtocounter{mypage}{1}}

\newcommand{\subtopic}{}
\newcommand{\pause}{}
\newcommand{\only}[1]{#1}

\renewcommand{\slides}[1][]{
  %  \clearpage
  \subsection{\topic}
  \index{\topic}
  {\small #1}
  \setcounter{mypage}{0}
  \smallskip\nopagebreak\hrule\medskip
}

\newcommand{\slidesfoot}{
  \bigskip
}

\newcommand{\sublecture}[2]{
  \phantomsection\addcontentsline{toc}{subsubsection}{#1}
  \index{#1}
}

\newcommand{\sublectureHide}[2]{
  \renewcommand{\subtopic}{#1}
}

\newcommand{\key}[1]{
  \phantomsection\addcontentsline{toc}{subsubsection}{#1}
  %\subsubsection{#1}
  \index{#1}
}

\providecommand{\defn}[1]{%
  \textbf{#1}\index{#1}%
}

\newenvironment{slidecore}[1]{
  \incpage
  \subsubsection*{#1}%{\headerfont\noindent\textbf{#1}\\}%
  \vspace{-6ex}%
  \begin{list}{$\bullet$}{\leftmargin4ex \rightmargin0ex \labelsep1ex
    \labelwidth2ex \partopsep0ex \topsep0ex \parsep.5ex \parskip0ex \itemsep0pt}\item[]~\\\nopagebreak%
}{
  \end{list}\nopagebreak%
  {\hfill\tiny \textsf{\arabic{section}.\arabic{subsection}:\arabic{mypage}}}\nopagebreak%
  \smallskip\nopagebreak\hrule
}

\newcommand{\slide}[2]{
  \begin{slidecore}{#1}
    #2
  \end{slidecore}
}

\renewcommand{\exercises}{}
\newcommand{\exercisestitle}{}
\newcommand{\exsection}[1]{\subsubsection{#1}}
\newcommand{\exsubsection}[1]{\paragraph{#1}}
\newcommand{\exerfoot}{\bigskip}

\newcommand{\story}[1]{
  \subsection*{Motivation \& Outline}
  {\storyfont\sf #1}
  \medskip\nopagebreak\hrule
}

\newcounter{savedsection}
\newcommand{\subappendix}{\setcounter{savedsection}{\arabic{section}}\appendix}
\newcommand{\noappendix}{
  \setcounter{section}{\arabic{savedsection}}% restore section number
  \setcounter{subsection}{0}% reset section counter
%  \gdef\@chapapp{\sectionname}% reset section name
  \renewcommand{\thesection}{\arabic{section}}% make section numbers arabic
}

\newenvironment{items}[1][9]{
\par\setlength{\unitlength}{1pt}\fontsize{#1}{#1}\linespread{1.2}\selectfont
\begin{list}{--}{\leftmargin4ex \rightmargin0ex \labelsep1ex \labelwidth2ex
\topsep0pt \parsep0ex \itemsep3pt}
}{
\end{list}
}

\newenvironment{itemS}[1][4ex]{
\par
\tiny
\begin{list}{--}{\leftmargin#1 \rightmargin0ex \labelsep1ex
  \labelwidth2ex \topsep0pt \parsep0ex \itemsep2pt}
}{
\end{list}
}

\newcommand{\Def}[1]{%
\textbf{#1}\index{#1}}%\marginnote{#1}}

  \scripthead
}

\providecommand{\paper}{
  \input{../latex/style-paper}
  \paperhead
}

\providecommand{\note}[1][9pt]{
  \providecommand{\notehead}[2]{
  \documentclass[#1,fleqn,twoside]{article}
  \stdpackages
  \renewcommand{\labelenumi}{{(\roman{enumi})}}
  \renewcommand{\theenumi}{(\roman{enumi})} %for ref

  \renewcommand{\baselinestretch}{#2}
  \renewcommand{\arraystretch}{1.2}
  \renewcommand{\topfraction}{1}
  \renewcommand{\bottomfraction}{1}
  \renewcommand{\textfraction}{0}
  \columnsep 5ex
  \parindent 3ex
  \parskip 1ex

  % Lists and paragraphs
  \parindent 0pt
  \topsep 4pt plus 1pt minus 2pt
  \partopsep 1pt plus 0.5pt minus 0.5pt
  \itemsep 2pt plus 1pt minus 0.5pt
  \parsep 2pt plus 1pt minus 0.5pt
  \parskip .5pc %add _in_ {thebibliography} environment in *.bbl

  \setcounter{tocdepth}{3}
  \setcounter{secnumdepth}{3}

  \geometry{a4paper,hdivide={25mm,*,25mm},vdivide={25mm,*,25mm}}

  \renewcommand{\headrulewidth}{.0pt}\renewcommand{\footrulewidth}{.0pt}\cfoot{}
  \fancyhead[OL,EC]{\it\theauthor---\today}
  \fancyhead[ER]{\leftmark}
  \fancyhead[OR,EL]{\thepage}
  \fancyfoot[EL,OR]{}
  \setlength{\headsep}{10mm}
  %\fancyhead[OL]{\rightmark}
  %\fancyfoot[EL,OR]{}


  %  \usepackage{palatino}

  \newcommand{\codefont}{\helvetica{8}{1.2}{m}{n}}

  \renewenvironment{abstract}{
    \vspace*{5ex}\begin{list}{}{
      \leftmargin3ex
      \rightmargin3ex
      \topsep-\parskip}\item[]
     \hrule\vspace{1.5ex}{\bf Abstract.~}\small}
    {\vspace{2ex}\hrule\end{list}\vspace{5ex}}
    
  \newenvironment{keyword}
    {\par{\it Keywords:~}}
    {}

  \def\makemytitle{%
    \thispagestyle{empty}
    \begin{list}{}{\leftmargin3ex \rightmargin3ex \topsep0ex \parsep0ex}\item[]
      \begin{center}
        {\fontsize{18}{25}\selectfont{\thetitle\\}}\vspace{5ex}

        {\fontsize{14}{16}\selectfont{\theauthor\\}}\vspace{1ex}

        {\footnotesize{\sl \addressFUB}\\ \emailBerlin}

        {\footnotesize \today}

        \vspace{1ex}
        {\small \published}
      \end{center}
    \end{list}
    \renewcommand{\maketitle}{\chapter{\thetitle}}
  }

  \input{../latex/macros}
  \pdflatex

  \graphicspath{{pics/}{../pics/}{pics-local/}}

  \myauthor{Marc Toussaint}
  \date{\today}
}

%%%%%%%%%%%%%%%%%%%%%%%%%%%%%%%%%%%%%%%%%%%%%%%%%%%%%%%%%%%%%%%%%%%%%%%%%%%%%%%%

\newcommand{\notetitle}{
  \begin{document}
  \thispagestyle{empty}
    
  \maketitle

}

\newenvironment{items}[1][9]{
\par\setlength{\unitlength}{1pt}\fontsize{#1}{#1}\linespread{1.2}\selectfont
\begin{list}{--}{\leftmargin4ex \rightmargin0ex \labelsep1ex \labelwidth2ex
\topsep0pt \parsep0ex \itemsep3pt}
}{
\end{list}
}

  \notehead{#1}{1.1}
}

\providecommand{\course}{NO COURSE}
\providecommand{\coursepicture}{NO PICTURE}
\providecommand{\coursedate}{NO DATE}
\providecommand{\topic}{NO TOPIC}
\providecommand{\keywords}{}
\providecommand{\exnum}{NO NUMBER}
\providecommand{\teacher}{Marc Toussaint}

\providecommand{\stdpackages}{
  \usepackage{amsmath}
  \usepackage{amssymb}
  \usepackage{amsfonts}
  \allowdisplaybreaks
  \usepackage{amsthm}
  \usepackage{eucal}
  \usepackage{graphicx}
%  \usepackage{color}
  \usepackage{geometry}
  \usepackage{framed}
  \usepackage{xcolor}
  \definecolor{shadecolor}{gray}{0.9}
  \setlength{\FrameSep}{3pt}
  \usepackage{fancyvrb}
  \fvset{numbers=none,xleftmargin=5ex,fontsize=\small}

  \usepackage{pdfpages}

  \usepackage{multicol} 
  \usepackage{fancyhdr}
}

\providecommand{\addressUSTT}{
  Machine~Learning~\&~Robotics~lab, U~Stuttgart\\\small
  Universit{\"a}tsstra{\ss}e 38, 70569~Stuttgart, Germany
}

\providecommand{\addressTUB}{
  Learning~\&~Intelligent~Systems~Lab, TU~Berlin\\\small
  Marchstr. 23, 10587 Berlin, Germany
}


\renewcommand{\course}{Robot Learning}
\renewcommand{\coursepicture}{roblearn.png}
\renewcommand{\coursedate}{Summer 2024}
\renewcommand{\teacher}{Marc Toussaint \& Wolfgang H{\"o}nig}

\renewcommand{\topic}{Dynamics Learning}
\renewcommand{\keywords}{(aka.\ System Identification, Model Learning)}

\slides

\ifthenelse{\isundefined{\scripthead}}{

\providecommand{\info}[1]{\smallskip{\ttiny [#1]\par}}

\usepackage{bibentry}
\nobibliography*

\ifthenelse{\isundefined{\setbeamertemplate}}{}{
  \setbeamertemplate{bibliography item}{\insertbiblabel}
}

\providecommand{\citehere}[1]{{\fontsize{5}{1}\selectfont\bibentry{#1}\par}}

}{}

\providecommand{\bm}[1]{\boldsymbol{#1}}

\slidestitle

%%%%%%%%%%%%%%%%%%%%%%%%%%%%%%%%%%%%%%%%%%%%%%%%%%%%%%%%%%%%%%%%%%%%%%%%%%%%%%%%

\slide{Outline}{

\item I. What is learned?
\begin{items}
\item Incl. which mapping exactly, model assumption, parameterization, loss function
\end{items}

~

\item II. How is the data generated?

~

\item III. Multirotor Examples

}


%%%%%%%%%%%%%%%%%%%%%%%%%%%%%%%%%%%%%%%%%%%%%%%%%%%%%%%%%%%%%%%%%%%%%%%%%%%%%%%%

\slide{I. What is learned?}{

~\pause

\show[.9]{robLearn1}

}

%%%%%%%%%%%%%%%%%%%%%%%%%%%%%%%%%%%%%%%%%%%%%%%%%%%%%%%%%%%%%%%%%%%%%%%%%%%%%%%%

\slide{Dynamics Learning -- State-based view}{

\item Learning the \emph{state-based} dynamics:

$$ x_t = f(x_{t\1}, u_{t\1}) \qquad\text{or}\qquad p(x_t \| x_{t\1}, u_{t\1}) $$

~\pause

\item Distinguish three cases:
\begin{items}
\item \textbf{Parameter Estimation:} $f$ is assumed physics with unknown physics parameters $\Th$
\item \textbf{Full Regression:} $f$ is learned as regression model
\item \textbf{Residual Dynamics:} learn the difference to a nominal physics model
\end{items}

}

%%%%%%%%%%%%%%%%%%%%%%%%%%%%%%%%%%%%%%%%%%%%%%%%%%%%%%%%%%%%%%%%%%%%%%%%%%%%%%%%

\slide{Dynamics Learning -- Observation-based view}{

\small

\item $x_t$ is the system \emph{state}

\info{Markov Property: We call a variable \emph{state} if the future is conditionally independent on the past when conditioned on state; $I(\text{future}, \text{past} \| \text{state}) = 0$.}

\item Sometimes the true state is not observed (or unknown), only observations $y_t$ are available ($y_t$: sensor readings, or \emph{state estimates} from sensors) \anchor{50,-40}{\showh[.3]{robLearn-dynamics}}

~\pause

\item We need to use the \textbf{history} of observed $y_t,u_t$ to predict next $y_t$!
\item Distinguish three cases:
\begin{items}
\item \textbf{Autoregression:} Learn a direct history-based model $y_t = f(y_{t-H:t}, u_{t-H:t})$
\item \textbf{Recurrent Model:} Learn a recurrent model with latent state $h_t$ (e.g.\ LSTM)
\item \textbf{State-space Model:} Jointly learn embedding/decoding $x\mapsto y$ and latent dynamics $x,u\mapsto x'$ (is also a recurrent model)
\end{items}

}

%%%%%%%%%%%%%%%%%%%%%%%%%%%%%%%%%%%%%%%%%%%%%%%%%%%%%%%%%%%%%%%%%%%%%%%%%%%%%%%%

\slide{}{

\item In summary, six cases we'll discuss more concretely:
\begin{items}
\item state-based dynamics
\begin{items}
\item physical parameter estimation
\item full regression
\item residual dynamics
\end{items}
\item observation-based dynamics
\begin{items}
\item autoregression (NARX)
\item observation-based dynamics -- recurrent model
\item observation-based dynamics -- state-space model
\end{items}
\end{items}

}
%%%%%%%%%%%%%%%%%%%%%%%%%%%%%%%%%%%%%%%%%%%%%%%%%%%%%%%%%%%%%%%%%%%%%%%%%%%%%%%%

\slide{}{

\item Why learn the dynamics?
\begin{items}\tiny
\item Given learned dynamics, we can use planning (MPC) or RL against the learned model to generate controllers
\item Examples in literature: Schaal'02, Deisenroth'15 (PILCO!), Finn'17, Driess'23, Schubert'23
\end{items}

~\pause


\item Quick terminology:
\begin{items}\tiny
\item Dynamics Learning $\oto$ System Identification (in control theory), Model Learning (in model-based RL)
\item In control theory $u_t$ are called \textbf{inputs} and the \emph{observations/measurements} $y_t$ are called \textbf{outputs}
\end{items}

}

%%%%%%%%%%%%%%%%%%%%%%%%%%%%%%%%%%%%%%%%%%%%%%%%%%%%%%%%%%%%%%%%%%%%%%%%%%%%%%%%

\key{Parameter Estimation}
\slide{State Dynamics -- Parameter Estimation}{

\item Assume that dynamics $x_t = f_\Th(x_{t\1}, u_{t\1})$ has unknown physical parameters $\Th$,\pause e.g.:

~

~

\show[.5]{franka-sysId1}
\show[.5]{franka-sysId2}

\citehere{2019-gaz-DynamicIdentificationFranka}

}

%%%%%%%%%%%%%%%%%%%%%%%%%%%%%%%%%%%%%%%%%%%%%%%%%%%%%%%%%%%%%%%%%%%%%%%%%%%%%%%%

\slide{State Dynamics -- Parameter Estimation}{

\item Given data $D=\{(x_t, x_{t\1}, u_{t\1})\}_{t=1}^T$, find parameters

$$\min_\Th \sum_{t} \norm{x_t - f_\Th(x_{t\1}, u_{t\1})}^2$$

~\pause

\item Sometimes, it is possible to describe $f_\Th$ as linear in $\Th$. See Gaz'19!
\begin{items}
\item Then finding optimal $\Th$ leads to a linear least squares problem.
\item Otherwise: Black-box optimization (CMA-ES) or gradient-based (SGD, Gauss-Newton)
\end{items}

}

%%%%%%%%%%%%%%%%%%%%%%%%%%%%%%%%%%%%%%%%%%%%%%%%%%%%%%%%%%%%%%%%%%%%%%%%%%%%%%%%

\key{Dynamics Regression}
\slide{State Dynamics -- Full Regression}{

\item Learn $f_\t$ directly, using some ML regression, e.g.\ (old-fashioned LWR):

~

\show[.35]{schaal-dyn1}
\show[.4]{schaal-dyn2}

\citehere{2002-schaal-ScalableTechniquesNonparametric}

}

%%%%%%%%%%%%%%%%%%%%%%%%%%%%%%%%%%%%%%%%%%%%%%%%%%%%%%%%%%%%%%%%%%%%%%%%%%%%%%%%

\slide{State Dynamics -- Full Regression}{

\item Given data $D=\{(x_t, x_{t\1}, u_{t\1})\}_{i=1:n,t=1:T_i}$, find parameters

$$\min_\t \sum_{t} \norm{x_t - f_\t(x_{t\1}, u_{t\1})}^2$$

$\to$ same formulation as parameter estimation, really.

~\pause

\item Use supervised ML to minimize regression error

}

%%%%%%%%%%%%%%%%%%%%%%%%%%%%%%%%%%%%%%%%%%%%%%%%%%%%%%%%%%%%%%%%%%%%%%%%%%%%%%%%

\slide{State Dynamics -- Full Regression (probabilistic)}{

\item Given data $D=\{(x_t, x_{t\1}, u_{t\1})\}_{i=1:n,t=1:T_i}$, find parameters

$$ \min_\t - \sum_{t} \log p_\t(x_t \| x_{t\1}, u_{t\1}) $$

where $p_t(x_t \| x_{t\1}, u_{t\1})$ is a probabilistic regression, e.g.\ Gaussian Process:

~

\show[.4]{gaussianProcess1}
{\tiny\hfill(from Rasmussen \& Williams)}

\info{Marc Deisenroth's PICLO paper had huge impact: Using learned GP dynamics to derive optimal controls.}

}

%%%%%%%%%%%%%%%%%%%%%%%%%%%%%%%%%%%%%%%%%%%%%%%%%%%%%%%%%%%%%%%%%%%%%%%%%%%%%%%%

\key{Residual Dynamics}
\slide{State Dynamics -- Residual Dynamics}{

\item Given a nominal dynamics $f_M$ (e.g., assumed physics), learn a residual model $f_\t$ to minimze

$$\min_\t \sum_{t} \norm{x_t - [f_M(x_{t\1}, u_{t\1}) + f_\t(x_{t\1}, u_{t\1})]}^2$$

~\pause

\item Examples: Gaz'19, Multirotor Examples

}

%%%%%%%%%%%%%%%%%%%%%%%%%%%%%%%%%%%%%%%%%%%%%%%%%%%%%%%%%%%%%%%%%%%%%%%%%%%%%%%%

\key{Observation-based models (Autoregression, Recurrent, State-Space)}
\slide{Observation-based Dynamics -- Autoregression (NARX)}{

~

\show[.5]{narx}

\citehere{1997-siegelmann-ComputationalCapabilitiesRecurrent}

\begin{items}
\item NARX=``Autoregression with controls'' \quad our notation: $y_t = f_\t(y_{t\myminus H:t\1},u_{t\myminus H:t\1})$
\item developed in time-series modelling, sequence modelling
\end{items}

\pause

\item How long does the history $H$ have to be?
\pause
\item What's the modern version of autoregression?

}

%%%%%%%%%%%%%%%%%%%%%%%%%%%%%%%%%%%%%%%%%%%%%%%%%%%%%%%%%%%%%%%%%%%%%%%%%%%%%%%%

\slide{Observation-based Dynamics -- Autoregression (Transformers)}{

~

\show[.5]{transformer-dyn1}
\show[.5]{transformer-dyn2}

\citehere{2023-schubert-GeneralistDynamicsModel}

}

%%%%%%%%%%%%%%%%%%%%%%%%%%%%%%%%%%%%%%%%%%%%%%%%%%%%%%%%%%%%%%%%%%%%%%%%%%%%%%%%

\slide{Observation-based Dynamics -- Recurrent Model}{

\small

\item Rather than giving the model a history as input, it should \emph{learn} to memorize relevant information, i.e., learn a latent representation for relevant information $\to$ recurrent NN

\pause

\item Train a latent representation $h_t$ to consume history information and predict $y_t$

~

\show[.5]{Recurrent_neural_network_unfold}
\hfill{\tiny (Wikipedia; change in notation: $x\leadsto (y,u), o\leadsto y$)}

\medskip

\item The most common NN architecture is LSTM (better: Gated Recurrent Units):

\cen{\showh[.3]{LSTM}\quad{\tiny (Hochreiter, Schmidthuber, 1997)}}


}

%%%%%%%%%%%%%%%%%%%%%%%%%%%%%%%%%%%%%%%%%%%%%%%%%%%%%%%%%%%%%%%%%%%%%%%%%%%%%%%%

\slide{Observation-based Dynamics -- State-Space Model}{

\item Also a recurrent model, but explicitly assumes latent state $x_t \in \RRR^d$

~

\show[.6]{dyn-stateSpace1}
\show[.3]{dyn-stateSpace2}

\citehere{2018-doerr-ProbabilisticRecurrentStatespace}

}

%%%%%%%%%%%%%%%%%%%%%%%%%%%%%%%%%%%%%%%%%%%%%%%%%%%%%%%%%%%%%%%%%%%%%%%%%%%%%%%%

\slide{Observation-based Dynamics -- State-Space Model}{

\item Jointly train an embedding/decoding $g: x\mapsto y$ and latent dynamics $f:x,u\mapsto x'$:
\begin{align*}
\begin{array}{c@{}c@{~}c@{~}c@{~}c}
x &, \mathbf{u} \overset{f}\mapsto & x' \\[-1ex]
\!g \rotatebox[origin=c]{-90}{$\mapsto$} ~~& & \!\!g \rotatebox[origin=c]{-90}{$\mapsto$}~~ \\[-1ex]
\mathbf{y} & & \mathbf{y'}
  \end{array}
\end{align*}

\item Only $u_{1:T}, y_{1:T}$ are observed! Train model to maximize data likelihood,
$$\log p(y_{1:T} \| u_{1:T}) \ge \text{Evidence Lower Bound (ELBO)} $$
\begin{items}
\item This method trains both, $g$ and $f$, and implicitly \emph{infers} a notion of state $x_t$
\item Technically, use SGD to maximize ELBO
\end{items}

}

%%%%%%%%%%%%%%%%%%%%%%%%%%%%%%%%%%%%%%%%%%%%%%%%%%%%%%%%%%%%%%%%%%%%%%%%%%%%%%%%

\slide{}{

\item More Literature for the six cases provided at the end of these slides...

}
%% \slide{}{

%% In
%% \citehere{1997-siegelmann-ComputationalCapabilitiesRecurrent}

%% \begin{bibunit}[unsrturl]
%% %\setcounter{enumiv}{#1} %%muss in bu?.bbl rein!
%% %\renewcommand{\refname}{\vspace{-\parskip}}\let\chapter\phantom \let\section\phantom
%% \nocite{*}
%% \putbib[b1-DynamicsLearning]
%% \end{bibunit}

%% }

%%%%%%%%%%%%%%%%%%%%%%%%%%%%%%%%%%%%%%%%%%%%%%%%%%%%%%%%%%%%%%%%%%%%%%%%%%%%%%%%

\slide{II. How is the data generated?}{

~\pause

\item Importance of data generation is (mostly) under-acknowledged in papers!

~

\item Ideas to generate \emph{good} data may be more important than ML method details

~\pause

\item What is \emph{good} data?

}

%%%%%%%%%%%%%%%%%%%%%%%%%%%%%%%%%%%%%%%%%%%%%%%%%%%%%%%%%%%%%%%%%%%%%%%%%%%%%%%%

\key{Data Quality}
\slide{Good Data -- in Linear Regression}{

\item Reconsider regression with linear model $f_\t(x) = \bar x^\T \t$, loss
$$L(\t) = \sum_i (y_i - f_\t(x_i))^2 ~+~ \l\norm{\t}^2$$
and solution
$$\t^* = (X^\T X + \l\Id)^\1 X^\T y ~.$$

\item What is good data?

\pause

\item What is the estimator variance $\Var{\t^*}$?
\pause
\begin{items}
\item Assume data with variance $\Var{y}=\s^2 \Id_n$
\pause
\item Then $\Var{\t^*} = (X^\T X+\l I)^\1 \s^2$
\pause
\item Smaller variance via larger $\l$ (but then larger bias), or \textbf{larger $\det(X^\T X)$}!
\end{items}

\pause

\item Good data means reducing variance (=randomness) of estimated model!
\begin{items}
\item large $\det(X^\T X)$ $\oto$ cover input space!\\
\info{Large estimator variance $\oto$ ``Overfitting'': Reducing variance prevents overfitting. Hastie has great section on \emph{shrinkage} methods (=regularization)}
\end{items}

}

%%%%%%%%%%%%%%%%%%%%%%%%%%%%%%%%%%%%%%%%%%%%%%%%%%%%%%%%%%%%%%%%%%%%%%%%%%%%%%%%

\slide{Good Data -- in Linear System Identification}{

~

\show[.5]{control-sysId}

{\urlfont\url{https://ethz.ch/content/dam/ethz/special-interest/mavt/dynamic-systems-n-control/idsc-dam/Lectures/Signals-and-Systems/Lectures/Fall2018/Lecture11_sigsys.pdf}

}

}

%%%%%%%%%%%%%%%%%%%%%%%%%%%%%%%%%%%%%%%%%%%%%%%%%%%%%%%%%%%%%%%%%%%%%%%%%%%%%%%%

\key{Frequency Excitation}
\slide{Good Data -- in Linear System Identification}{

\small
\item Cover the input space $\to$ cover frequency space
\begin{items}
\item Linear dynamics can be Laplace transformed into frequency domain:
$$Y(s) = H(s)~ U(s)$$
\item $U(s)$ are controls; $Y$ observations; $H(s)$ is called \textbf{transfer function} (complex)
\item $H(s)$ can be probed by sending a single control frequence ($U(s) = \d_{ss'}$)
\end{items}

\show[.4]{control-sysId2}

\item In essence: stimulate the system with control frequencies $u(t) = \cos(k t / \tau_0)$ for $k=0,1,..$

\pause

\item Franka SystemId paper [Gaz'19]: Sinusoidal reference motions (Eq.\ 31):
$$ \dot q_{i,\text{des}(t)} = A_i \sin\left(\textstyle\frac{2\pi}{T_i}~ t\right)\comma i\in\{1,..,n\} $$

}

%%%%%%%%%%%%%%%%%%%%%%%%%%%%%%%%%%%%%%%%%%%%%%%%%%%%%%%%%%%%%%%%%%%%%%%%%%%%%%%%

\slide{Good Data -- in general}{

~

\item Think about good state space coverage! ~ (in all variants of Robot Learning)
\begin{items}
\item Frequency coverage in control systems
\item Exploration in RL beyond $\e$-greedy
\item Long-term structured variation (at least pink noise, Ornstein-Uhlenbeck) instead of Brownian motion
\item Explicit exploration: Novelty seeking, information seeking, exploration bonus, Bayesian RL
\end{items}

}

%%%%%%%%%%%%%%%%%%%%%%%%%%%%%%%%%%%%%%%%%%%%%%%%%%%%%%%%%%%%%%%%%%%%%%%%%%%%%%%%

\slide{III. Background: Multirotors}{

\twocol[.02]{.6}{.35}{


\item State $\mathbf{x}=(\mathbf{p}, \mathbf{q}, \mathbf{v}, \mathbf{\omega})^\T$

\item Control $\mathbf{u}_{\Omega}=(\Omega_1,\hdots,\Omega_n)^\T$

\item Forces $\mathbf{f} = \sum_i c_{f_i} \Omega_i \mathbf{z}_{\Omega_i} = \mathbf{F} \mathbf{u}_{\Omega}$,

\item Torques $\bm{\tau} = \sum_i ( c_{f_i} \mathbf{p}_{\Omega_i} \times \mathbf{z}_{\Omega_i} + c_{\tau_i} \mathbf{z}_{\Omega_i} ) {\Omega}_i = \mathbf{M} \mathbf{u}_{\Omega}$

\item Dynamics
$$
   \begin{array}{l}
       \Dot{\mathbf{p}} = \mathbf{v}, \quad\quad\quad\quad~~\,
       %
       m\dot{\mathbf{v}} = m\mathbf{g} + \mathbf{R}(\mathbf{q}) \mathbf{F} \mathbf{u}_{\Omega} +\mathbf{f}_a,\\
       %
       \Dot{\mathbf{q}} = \cfrac{1}{2} \, \mathbf{q} \circ \begin{bmatrix} 0 \\ \bm{\omega} \end{bmatrix},
       %
       \mathbf{J}\Dot{\bm{\omega}} = -\bm{\omega} \times \mathbf{J}\bm{\omega}  + \mathbf{M} \mathbf{u}_{\Omega} + \bm{\tau}_a,
   \end{array}
$$

}{
% from 10.1109/MRA.2012.2206474
\showh[1]{mahony2012fig2}
{\hfill\tiny [Mahony, $\sim$2012]}
}

\info{Propellers create forces and torques, rest is Newton-Euler}

\info{$\mathbf{f}_a$, $\bm{\tau}_a$ can model drag, wind, aerodynamic interactions etc.}

}

%%%%%%%%%%%%%%%%%%%%%%%%%%%%%%%%%%%%%%%%%%%%%%%%%%%%%%%%%%%%%%%%%%%%%%%%%%%%%%%%

\slide{Multirotors: What is learned?}{

\item Parameters that are hard to measure: inertia $\mathbf{J}$, motor params ($c_{f_i}$, $c_{\tau_i}$, delay)

~

\item Residuals $\mathbf{f}_a$, $\bm{\tau}_a$

\info{potentially as a function of the state (e.g., drag) or environment (e.g., downwash)}

\info{potentially non-Markovian, i.e., a function of a history of states}

~

\item Full dynamics model not so much --- Why?

\pause

\info{Impossible to gather data from all states safely}

\info{Rotational symmetries are surprisingly difficult to learn}

}

%%%%%%%%%%%%%%%%%%%%%%%%%%%%%%%%%%%%%%%%%%%%%%%%%%%%%%%%%%%%%%%%%%%%%%%%%%%%%%%%

\slide{Multirotors: How is it ``learned''? (Classic)}{

Estimate parameters with dedicated experiments

\item Inertia: Swing body in different positions and record motion; solve an optimization problem

\showh[0.35]{foerster_fig2_2a}

}

%%%%%%%%%%%%%%%%%%%%%%%%%%%%%%%%%%%%%%%%%%%%%%%%%%%%%%%%%%%%%%%%%%%%%%%%%%%%%%%%

\slide{Multirotors: How is it ``learned''? (Classic)}{

Estimate parameters with dedicated experiments

\item Motors: Use thrust stand (often for a single motor + propeller) + curve fitting

\twocol{.6}{.35}{

\showh[0.6]{foerster_fig3_2}

}{

\showh[0.6]{foerster_fig3_5}

}

}

%%%%%%%%%%%%%%%%%%%%%%%%%%%%%%%%%%%%%%%%%%%%%%%%%%%%%%%%%%%%%%%%%%%%%%%%%%%%%%%%

\slide{Multirotors: How is it ``learned''? (Classic)}{

Estimate parameters with dedicated experiments

\item Drag: Use wind tunnel + curve fitting with ``guessed'' models

\showh[0.4]{foerster_fig4_8}

\citehere{2015-forster-SystemIdentificationCrazyflie}

}

%%%%%%%%%%%%%%%%%%%%%%%%%%%%%%%%%%%%%%%%%%%%%%%%%%%%%%%%%%%%%%%%%%%%%%%%%%%%%%%%

\slide{Multirotors: How is it ``learned''? (Classic)}{

Estimate parameters with dedicated experiments

~

\item Is this learning?

\pause

\info{Yes, since curve fitting is extensively used}

~

\item Advantages and Disadvantages?

\pause

\info{Pros: Physics intuition (explainability); can improve ``important'' parameters if needed; no need to have a flying system}
\info{Cons: Labor and equipment intensive; does not capture unmodeled terms; does not capture the robot as a system}

}

%%%%%%%%%%%%%%%%%%%%%%%%%%%%%%%%%%%%%%%%%%%%%%%%%%%%%%%%%%%%%%%%%%%%%%%%%%%%%%%%

\slide{Multirotors: How is it learned? (Parameter Estimation)}{

\item Assumption: we have a system that can already fly; Can we do better?

\info{Strong assumption, since controllers need models, too}

\item Direct (analytical) optimization

\citehere{2024-eschmann-DataDrivenSystemIdentification}

\info{Will skip the discussion here}

\item Probabilistic formulation (Gaussian noise)

\citehere{2016-burri-MaximumLikelihoodParameter}

}

%%%%%%%%%%%%%%%%%%%%%%%%%%%%%%%%%%%%%%%%%%%%%%%%%%%%%%%%%%%%%%%%%%%%%%%%%%%%%%%%

\slide{Multirotors: How is it learned? (Maximum Likelihood)}{

% \item Assumption: we have a system that can already fly, can we do better?

% \info{Strong assumption, since controllers need models, too}

\item Given: Dataset with trajectory (position, orientation, motor speed), $\mathbf{Z}$; measurements (IMU data, motor commands), $\mathbf{U}$
\item Goal: 

$$
\hat{\mathbf{X}}_{ML}, \hat{\mathbf{\theta}}_{ML} = \argmax_{\hat{\mathbf{X}}, \hat{\mathbf{\theta}}} p(\mathbf{Z}, \mathbf{U}, \hat{\mathbf{X}}, \hat{\mathbf{\theta}})
$$

(parameters to estimate $\hat{\mathbf{\theta}}$; state estimates $\hat{\mathbf{X}}$; probability $p$)

}

%%%%%%%%%%%%%%%%%%%%%%%%%%%%%%%%%%%%%%%%%%%%%%%%%%%%%%%%%%%%%%%%%%%%%%%%%%%%%%%%

\slide{Multirotors: How is it learned? (Maximum Likelihood)}{

\item Assumptions to simplify $p(\mathbf{Z}, \mathbf{U}, \hat{\mathbf{X}}, \hat{\mathbf{\theta}})$
\begin{itemize}
\item White noise (IMU, motors)
\item Access to a prior trajectory $\rightarrow$ linearize around it and reason about ``residuals'' instead
\end{itemize}
\item $p(\cdot)$ becomes a mixture of Gaussians $\rightarrow$ can be maximized by minimizing the negative log-likelihood

\info{essentially a least square problem}

}

%%%%%%%%%%%%%%%%%%%%%%%%%%%%%%%%%%%%%%%%%%%%%%%%%%%%%%%%%%%%%%%%%%%%%%%%%%%%%%%%

\slide{Multirotors: How is it learned? (Maximum Likelihood)}{

\show[0.5]{2018-burri-FrameworkMaximumLikelihood-alg1}

where $\bar y=(\hat{\mathbf{X}}, \hat{\mathbf{\theta}})^\T$ from before

\citehere{2016-burri-MaximumLikelihoodParameter} 
\citehere{2018-burri-FrameworkMaximumLikelihood}

}

%%%%%%%%%%%%%%%%%%%%%%%%%%%%%%%%%%%%%%%%%%%%%%%%%%%%%%%%%%%%%%%%%%%%%%%%%%%%%%%%

\slide{Multirotors: How is it learned? (Supervised Deep NN)}{

\item Basic models do not capture ``complicated'' aerodynamic effects 

~

\item Blade Element Momentum (BEM) work for single rotors (but high computational effort)

~

\item Can we use (more) data to use function approximation instead?\\ Challenges:
    \begin{itemize}
        \item Training/Data efficiency
        \item Inference speed
    \end{itemize}

}

%%%%%%%%%%%%%%%%%%%%%%%%%%%%%%%%%%%%%%%%%%%%%%%%%%%%%%%%%%%%%%%%%%%%%%%%%%%%%%%%

\slide{Multirotors: How is it learned? (Supervised Deep NN)}{

\item Key idea: learn the ``residual physics'', only

\info{Input: past $h$ states and motor commands $\rightarrow$ not Markovian!}
\info{Output: forces and torques that cannot be explained by the basic model(s) ($\mathbf{f}_a$, $\bm{\tau}_a$)}

~

\show[0.6]{2021-bauersfeld-NeuroBEMHybridAerodynamic_fig2}

}

%%%%%%%%%%%%%%%%%%%%%%%%%%%%%%%%%%%%%%%%%%%%%%%%%%%%%%%%%%%%%%%%%%%%%%%%%%%%%%%%

\slide{Multirotors: How is it learned? (Supervised Deep NN)}{

\item ML method: Supervised training --- Where do the labels come frome?

\pause

\info{Solve dynamics for $\mathbf{f}_a$, $\bm{\tau}_a$}

~

\item Architecture
    \begin{itemize}
        \item Input $h=20$ (past 50 ms)
        \item temporal convolutional (TCN) with 25k parameters (MLP and other parameters in ablation)
    \end{itemize}

~

    \item Main takeaway: strong model/physics priors are better


\citehere{2021-bauersfeld-NeuroBEMHybridAerodynamic}

\info{Video: {\urlfont\url{https://youtu.be/Nze1wlfmzTQ}}}

}

%%%%%%%%%%%%%%%%%%%%%%%%%%%%%%%%%%%%%%%%%%%%%%%%%%%%%%%%%%%%%%%%%%%%%%%%%%%%%%%%

\slide{Multirotors: Data Collection}{

\item Motion capture system for accurate position/orientation state estimates

\info{Sampling at 500 Hz, submillimeter accuracy}
\info{Very costly: EUR 20k -- 100k}

\item On-board data logging of IMU

\info{Sampling at 1000 Hz, very noisy}

}

%%%%%%%%%%%%%%%%%%%%%%%%%%%%%%%%%%%%%%%%%%%%%%%%%%%%%%%%%%%%%%%%%%%%%%%%%%%%%%%%

\slide{Multirotors: Data Preprocessing}{

\item Two data sources $\rightarrow$ Synchronization needed (incl. clock skew)
\begin{itemize}
    \item Online Option: Send data to one computer using a low-latency link (and account for link delay)
    \item Offline Option: Solve optimization problem for clock skew and bias
\end{itemize}

~

\item Some derivatives (e.g., $\mathbf{v}$) are not directly observable
\begin{itemize}
    \item Online Option: Use data from an online filter (e.g., Extended Kalman Filter)
    \item Offline Option: Interpolate data (e.g., using splines), use analytical solution of fitted spline
\end{itemize}

~

\item Motor delays (``easy'' to measure)
\begin{itemize}
    \item Option 1: Include it in model explicitly
    \item Option 2: Shift/filter data accordingly
\end{itemize}

}

%%%%%%%%%%%%%%%%%%%%%%%%%%%%%%%%%%%%%%%%%%%%%%%%%%%%%%%%%%%%%%%%%%%%%%%%%%%%%%%%

\slide{Multirotors: Data Quantity}{

\item Maximum Likelihood: 45 sec flight data ``The pilot was careful to excite all axes, especially in yaw direction.''

\item NeuroBEM: 96 flights, 75 min flight data (1.8M data points) (up to 18 $m/s$ and 47 $m/s^2$)

}


% %%%%%%%%%%%%%%%%%%%%%%%%%%%%%%%%%%%%%%%%%%%%%%%%%%%%%%%%%%%%%%%%%%%%%%%%%%%%%%%%

% \slide{IV. Applications}{
% % [WH ]I think we can remove this slide now?

% papers \& videos

% \item NeuroBEM \url{https://youtu.be/Nze1wlfmzTQ} 

% % Should we talk about bootstrapping: learn dynamics from real data, use it to train in simulation via RL, deploy?

% }

%%%%%%%%%%%%%%%%%%%%%%%%%%%%%%%%%%%%%%%%%%%%%%%%%%%%%%%%%%%%%%%%%%%%%%%%%%%%%%%%

\slide{Literature}{

\item State Dynamics -- Parameter Estimation:

\citehere{2015-forster-SystemIdentificationCrazyflie}

\citehere{2024-eschmann-DataDrivenSystemIdentification}

\citehere{2018-burri-FrameworkMaximumLikelihood}

\citehere{2019-gaz-DynamicIdentificationFranka}

\item State Dynamics -- Full Regression:

\citehere{2002-schaal-ScalableTechniquesNonparametric}

\citehere{2015-deisenroth-GaussianProcessesDataEfficient}

}

%%%%%%%%%%%%%%%%%%%%%%%%%%%%%%%%%%%%%%%%%%%%%%%%%%%%%%%%%%%%%%%%%%%%%%%%%%%%%%%%

\slide{Literature}{

\item Observation-based Dynamics -- Autoregression (NARX):

\citehere{1990-chen-NonlinearSystemIdentification}

\citehere{1997-siegelmann-ComputationalCapabilitiesRecurrent}

\item Observation-based Dynamics -- Recurrent Model (also visual!):

\citehere{2021-bauersfeld-NeuroBEMHybridAerodynamic}

\citehere{2017-finn-DeepVisualForesight}

\citehere{2023-driess-LearningMultiobjectDynamics}

\citehere{2023-schubert-GeneralistDynamicsModel}

}

%%%%%%%%%%%%%%%%%%%%%%%%%%%%%%%%%%%%%%%%%%%%%%%%%%%%%%%%%%%%%%%%%%%%%%%%%%%%%%%%

\slide{Literature}{

\item State-Space Models (learning a \emph{state} dynamics based on only observations):

\citehere{2018-doerr-ProbabilisticRecurrentStatespace}

}

%%%%%%%%%%%%%%%%%%%%%%%%%%%%%%%%%%%%%%%%%%%%%%%%%%%%%%%%%%%%%%%%%%%%%%%%%%%%%%%%

\slide{not mentioned...}{\label{lastpage}

\begin{items}
\item Constrained ML models (Geist)
\item Embed to Control
\item Koopman embedding
\item Dual control
\item Safe Exploration
\end{items}

}

%%%%%%%%%%%%%%%%%%%%%%%%%%%%%%%%%%%%%%%%%%%%%%%%%%%%%%%%%%%%%%%%%%%%%%%%%%%%%%%%

\ttiny
\ifthenelse{\isundefined{\scripthead}}{
\bibliographystyle{plainurl-lis}
\bibliography{b1-DynamicsLearning}
}{}

\slidesfoot
