\providecommand{\slides}{
  \newcommand{\slideshead}{
  \newcommand{\thepage}{\arabic{mypage}}
  %beamer
%  \documentclass[t,hyperref={bookmarks=true}]{beamer}
%  \geometry{papersize={171mm,96mm}}
  \documentclass[t,hyperref={bookmarks=true},aspectratio=169]{beamer}
  \setbeamersize{text margin left=5mm}
  \setbeamersize{text margin right=5mm}
  \usetheme{default}
  \usefonttheme[onlymath]{serif}
  \setbeamertemplate{navigation symbols}{}
  \setbeamertemplate{itemize items}{{\color{black}$\bullet$}}

  \newwrite\keyfile

  %\usepackage{palatino}
  \stdpackages
  %\usepackage{tikz} \usetikzlibrary {shapes.geometric} 
  \usepackage{multimedia}
  \usepackage[utf8]{inputenc}

  %%% geometry/spacing issues
  %
  \definecolor{bluecol}{rgb}{0,0,.5}
  \definecolor{greencol}{rgb}{0,.6,0}
  \definecolor{citcol}{rgb}{.4,.4,.4}
  %\renewcommand{\baselinestretch}{1.1}
  \renewcommand{\arraystretch}{1.2}
  \columnsep 0mm

  \columnseprule 0pt
  \parindent 0ex
  \parskip 0ex
  %\setlength{\itemparsep}{3ex}
  %\renewcommand{\labelitemi}{\rule[3pt]{10pt}{10pt}~}
  %\renewcommand{\labelenumi}{\textbf{(\arabic{enumi})}}
  \setbeamertemplate{enumerate item}{(\roman{enumi})}
  \newcommand{\headerfont}{\helvetica{14}{1.5}{b}{n}}
  \newcommand{\slidefont} {\helvetica{11}{1.4}{m}{n}}
  %\newcommand{\codefont} {\helvetica{8}{1.2}{m}{n}}
  \newcommand{\urlfont} {\fontsize{6}{1.1}\selectfont}
  \renewcommand{\small} {\helvetica{10}{1.4}{m}{n}}
  \renewcommand{\tiny} {\helvetica{8}{1.3}{m}{n}}
  \newcommand{\ttiny} {\helvetica{7}{1.3}{m}{n}}

  %%% count pages properly and put the page number in bottom right
  %
  \newcounter{mypage}
  \newcommand{\incpage}{\addtocounter{mypage}{1}\setcounter{page}{\arabic{mypage}}}
  \setcounter{mypage}{0}
  \resetcounteronoverlays{page}

  \pagestyle{fancy}
  %\setlength{\headsep}{10mm}
  %\addtolength{\footheight}{15mm}
  \renewcommand{\headrulewidth}{0pt} %1pt}
  \renewcommand{\footrulewidth}{0pt} %.5pt}
  \cfoot{}
  \rhead{}
  \lhead{}
  %% \lfoot{\vspace*{-3mm}\hspace*{-3mm}\helvetica{5}{1.3}{m}{n}{\texttt{github.com/MarcToussaint/AI-lectures}}}
\lfoot{}
%  \rfoot{{\tiny\textsf{AI -- \topic -- \subtopic -- \arabic{mypage}/\pageref{lastpage}}}}
  %\lfoot{\raisebox{5mm}{\tiny\textsf{\slideauthor}}}
  %\rfoot{\raisebox{5mm}{\tiny\textsf{\slidevenue{} -- \arabic{mypage}/\pageref{lastpage}}}}
  %\rfoot{~\anchor{30,12}{\tiny\textsf{\thepage/\pageref{lastpage}}}}
  %\lfoot{\small\textsf{Marc Toussaint}}
%  \rfoot{\vspace*{-4.5mm}{\tiny\textsf{\color{gray}\topic\ -- \subtopic\ -- \arabic{mypage}/\pageref{lastpage}}}\hspace*{-4mm}}
  \rfoot{\vspace*{-4.5mm}{\tiny\textsf{\color{gray}\topic\ -- \arabic{mypage}/\pageref{lastpage}}}\hspace*{-4mm}}
%  \rfoot{~\anchor{-10,12}{\tiny\textsf{\color{gray}\topic\ --  \arabic{mypage}/\pageref{lastpage}}}}
  \lfoot{\vspace*{-4.5mm}{\hspace*{-3mm}\includegraphics[height=4mm]{LIS-logo-longText}}}

  \definecolor{grey}{rgb}{.8,.8,.8}
  \definecolor{head}{rgb}{.85,.9,.9}
%  \definecolor{blue}{rgb}{.0,.0,.5}
%  \definecolor{green}{rgb}{.0,.5,.0}
  \definecolor{red}{rgb}{.8,.0,.0}
  \newcommand{\inverted}{
    \definecolor{main}{rgb}{1,1,1}
    \color{main}
    \pagecolor[rgb]{.3,.3,.3}
  }
  \input{../latex/macros}

  \graphicspath{{pics/}{../pics/}{pics-local/}}
}

\newcommand{\slidestitle}{
  \title{\course \topic}
  \author{Marc Toussaint}
  \institute{Learning \& Intelligent Systems Lab, TU Berlin}

  \begin{document}


  %% title slide!
  \slide{}{
    \thispagestyle{empty}

    \twocol{.35}{.55}{
      %\vspace*{-5mm}%
      \hspace*{-5mm}%
      \includegraphics[width=1.\columnwidth]{\coursepicture}
    }{\center

      \textbf{\fontsize{17}{20}\selectfont \course}

      ~

      %Lecture
      \topic\\

      \vspace{1cm}

      {\tiny~\emph{\keywords}~\\}

      \vspace{1cm}

      \teacher
      
      Technical University of Berlin

      \coursedate

      ~

    }
  }
}

\newcommand{\slide}[2]{
  \slidefont
  \incpage\begin{frame}
  \addcontentsline{toc}{section}{#1}
  \vfill
  {\headerfont #1} \vspace*{-2ex}
  \begin{itemize}\item[]~\\
    #2
  \end{itemize}
  \vfill
  \end{frame}
}

% use \begin{frame}[fragile] around slidecore!
\newenvironment{slidecore}[1]{
  \slidefont\incpage
  \addcontentsline{toc}{section}{#1}
  \vfill
  {\headerfont #1} \vspace*{-2ex}
  \begin{itemize}\item[]~\\
}{
  \end{itemize}
  \vfill
}

\newcommand{\titleslide}[4][Marc Toussaint]{
  \newcommand{\slideauthor}{#1}
  \newcommand{\slidevenue}{#3}
  \slidefont
  \incpage
  \begin{frame}
  \begin{center}
    \vspace*{15mm}

    {\headerfont #2\\}
        
    \vspace*{10mm}

    #1 \\

    \vspace*{5mm}

    {\small
      Learning \& Intelligent Systems Lab, TU Berlin\\
%      Science of Intelligence Cluster of Excellence, Berlin\\
      Max Planck Fellow, Institute for Intelligent Systems\\ %Physical Reasoning \& Manipulation Lab -- 
%      MIT CSAIL\\
%      mtoussai@mit.edu,~ marc.toussaint@informatik.uni-stuttgart.de

      \vspace*{10mm}

      \emph{#3}
    }

    \vspace*{0mm}

    %\includegraphics[scale=.1]{pics/eushield-fullcolour}

  \end{center}
  \begin{itemize}\item[]~\\
    #4
  \end{itemize}
  \end{frame}
}

\newcommand{\titleslideempty}[2]{
  \slidefont
  \incpage
  \begin{frame}
  \begin{center}
    \vspace*{15mm}

    {\headerfont #1\\}
        
    %% \vspace*{5mm}

    %% {\small\emph{#2}} \\

  \end{center}
  \begin{itemize}\item[]~\\
    #2
  \end{itemize}
  \end{frame}
}

\providecommand{\key}[1]{
  \addtocounter{mypage}{1}
% \immediate\write\keyfile{#1}
  \addtocontents{toc}{\hyperref[key:#1]{#1 (\arabic{mypage})}}
%  \phantomsection\label{key:#1}
%  \index{#1@{\hyperref[key:#1]{#1 (\arabic{mysec}:\arabic{mypage})}}|phantom}
  \addtocounter{mypage}{-1}
}

\providecommand{\course}{}

\providecommand{\subtopic}{}

\providecommand{\sublecture}[2]{
  \renewcommand{\subtopic}{#1}
  \slide{#1}{#2}
}

\providecommand{\sublectureHide}[2]{
  \renewcommand{\subtopic}{#1}
}

\providecommand{\story}[1]{
~

Motivation: {\tiny #1}\clearpage
}

\newenvironment{items}[1][9]{
\par\setlength{\unitlength}{1pt}\fontsize{#1}{#1}\linespread{1.2}\selectfont
\begin{list}{--}{\leftmargin4ex \rightmargin0ex \labelsep1ex \labelwidth2ex
\topsep.7ex \parsep0ex \itemsep3pt}
}{
\end{list}
}

\newenvironment{itemS}[1][4ex]{
\par
\tiny
\begin{list}{--}{\leftmargin#1 \rightmargin0ex \labelsep1ex
  \labelwidth2ex \topsep0pt \parsep0ex \itemsep2pt}
}{
\end{list}
}

\providecommand{\slidesfoot}{
  \end{document}
}


  \slideshead
  %\slidestitle
}

\providecommand{\exercises}{
  \input{../latex/style-exercises}
  \exerciseshead
}

\providecommand{\script}{
  \newcommand{\scripthead}{
  \documentclass[9pt,fleqn,twoside]{article}
  \stdpackages

  \usepackage{makeidx}
  \makeindex

  \usepackage{thmtools}
  \definecolor{shadecolor}{gray}{0.85}
  \declaretheoremstyle[
    %headfont=\normalfont\bfseries,
    %notefont=\mdseries, notebraces={(}{)},
    %bodyfont=\normalfont,
    %postheadspace=0.5em,
    %spaceabove=6pt,
    mdframed={
      %  skipabove=8pt,
      %  skipbelow=6pt,
      hidealllines=true,
      backgroundcolor={shadecolor},
      innertopmargin=8pt,
      %  innerleftmargin=3pt,
      %  innerrightmargin=3pt
    }
  ]{shaded}
  \declaretheorem[style=shaded,within=section,name=Definition]{myDefinition}
  \declaretheorem[style=shaded,within=section,name=Theorem]{myTheorem}
  \declaretheorem[style=shaded,within=section,name=Identities]{Identities}
  \declaretheorem[style=shaded,within=section,name=Example]{myExample}

  \definecolor{grey}{rgb}{.8,.8,.8}
  \definecolor{bluecol}{rgb}{0,0,.5}
  \definecolor{greencol}{rgb}{0,.4,0}
  \definecolor{shadecolor}{gray}{0.9}
  \definecolor{citcol}{rgb}{.4,.4,.4}
  \usepackage[
    %    pdftex%,
    %%    letterpaper,
    %bookmarks,
    bookmarksnumbered,
    colorlinks,
    urlcolor=bluecol,
    citecolor=black,
    linkcolor=bluecol,
    %    pagecolor=bluecol,
    pdfborder={0 0 0},
    %pdfborderstyle={/S/U/W 1},
    %%    backref,     %link from bibliography back to sections
    %%    pagebackref, %link from bibliography back to pages
    %%    pdfstartview=FitH, %fitwidth instead of fit window
    pdfpagemode=UseOutlines, %bookmarks are displayed by acrobat
    pdftitle={\course},
    pdfauthor={Marc Toussaint},
    pdfkeywords={}
  ]{hyperref}
  \DeclareGraphicsExtensions{.pdf,.png,.jpg,.eps}

  %\usepackage{multirow}
  \usepackage{multimedia}
  %\usepackage{marginnote}
  %\setbeamercolor{background canvas}{bg=}

  \usepackage[round]{natbib}
  \bibliographystyle{abbrvnat}

  \renewcommand{\r}{\varrho}
  \renewcommand{\l}{\lambda}
  \renewcommand{\L}{\Lambda}
  \renewcommand{\b}{\beta}
  \renewcommand{\d}{\delta}
  \renewcommand{\k}{\kappa}
  \renewcommand{\t}{\theta}
  \renewcommand{\O}{\Omega}
  \renewcommand{\o}{\omega}
  \renewcommand{\SS}{{\cal S}}
  \renewcommand{\=}{\!=\!}
  %\renewcommand{\boldsymbol}{}
  %\renewcommand{\Chapter}{\chapter}
  %\renewcommand{\Subsection}{\subsection}

  \renewcommand{\baselinestretch}{1.0}
  \geometry{a5paper,headsep=6mm,hdivide={10mm,*,10mm},vdivide={13mm,*,7mm}}

  \fancyhead[OL,ER]{\course, \textit{Marc Toussaint}}
  \fancyhead[OR,EL]{\thepage}
  \fancyhead[C]{}
  \fancyfoot{}
  \pagestyle{fancy}

  \renewcommand{\labelenumi}{{(\roman{enumi})}}
  \renewcommand{\theenumi}{(\roman{enumi})} %for ref
  \parindent 0pt
  \parskip .5pc

  \columnsep 6ex

  \renewcommand{\familydefault}{\sfdefault}

  \newcommand{\headerfont}{\large}%helvetica{12}{1}{b}{n}}
  \newcommand{\slidefont} {}%\helvetica{9}{1.3}{m}{n}}
  \newcommand{\storyfont} {}
  %  \renewcommand{\small}   {\helvetica{8}{1.2}{m}{n}}
  \renewcommand{\tiny}    {\footnotesize}%helvetica{7}{1.1}{m}{n}}
  \newcommand{\ttiny} {\footnotesize}%fontsize{7}{7}\selectfont}
%  \newcommand{\codefont}{\fontsize{6}{6}\selectfont}%helvetica{8}{1.2}{m}{n}}
  \newcommand{\codefont} {\helvetica{8}{1.2}{m}{n}}

  \input{../latex/macros}

  \usepackage{comment}
  \specialcomment{solution}{
    \small
    \begin{shaded}
  }{
    \end{shaded}
  }

  \graphicspath{{pics/}{../pics/}{pics-local/}}

  \mytitle{\course\\Lecture Script}
  \myauthor{Marc Toussaint}
  \date{\coursedate}
}

%%%%%%%%%%%%%%%%%%%%%%%%%%%%%%%%%%%%%%%%%%%%%%%%%%%%%%%%%%%%%%%%%%%%%%%%%%%%%%%%

\newcommand{\scripttitle}{
  \begin{document}
  \maketitle
  %\anchor{100,10}{\includegraphics[width=4cm]{optim}}
%  \vspace*{1cm}
}

%%%%%%%%%%%%%%%%%%%%%%%%%%%%%%%%%%%%%%%%%%%%%%%%%%%%%%%%%%%%%%%%%%%%%%%%%%%%%%%%

\newcounter{mypage}
\setcounter{mypage}{0}
\newcommand{\incpage}{\addtocounter{mypage}{1}}

\newcommand{\subtopic}{}
\newcommand{\pause}{}
\newcommand{\only}[1]{#1}

\renewcommand{\slides}[1][]{
  %  \clearpage
  \subsection{\topic}
  \index{\topic}
  {\small #1}
  \setcounter{mypage}{0}
  \smallskip\nopagebreak\hrule\medskip
}

\newcommand{\slidesfoot}{
  \bigskip
}

\newcommand{\sublecture}[2]{
  \phantomsection\addcontentsline{toc}{subsubsection}{#1}
  \index{#1}
}

\newcommand{\sublectureHide}[2]{
  \renewcommand{\subtopic}{#1}
}

\newcommand{\key}[1]{
  \phantomsection\addcontentsline{toc}{subsubsection}{#1}
  %\subsubsection{#1}
  \index{#1}
}

\providecommand{\defn}[1]{%
  \textbf{#1}\index{#1}%
}

\newenvironment{slidecore}[1]{
  \incpage
  \subsubsection*{#1}%{\headerfont\noindent\textbf{#1}\\}%
  \vspace{-6ex}%
  \begin{list}{$\bullet$}{\leftmargin4ex \rightmargin0ex \labelsep1ex
    \labelwidth2ex \partopsep0ex \topsep0ex \parsep.5ex \parskip0ex \itemsep0pt}\item[]~\\\nopagebreak%
}{
  \end{list}\nopagebreak%
  {\hfill\tiny \textsf{\arabic{section}.\arabic{subsection}:\arabic{mypage}}}\nopagebreak%
  \smallskip\nopagebreak\hrule
}

\newcommand{\slide}[2]{
  \begin{slidecore}{#1}
    #2
  \end{slidecore}
}

\renewcommand{\exercises}{}
\newcommand{\exercisestitle}{}
\newcommand{\exsection}[1]{\subsubsection{#1}}
\newcommand{\exsubsection}[1]{\paragraph{#1}}
\newcommand{\exerfoot}{\bigskip}

\newcommand{\story}[1]{
  \subsection*{Motivation \& Outline}
  {\storyfont\sf #1}
  \medskip\nopagebreak\hrule
}

\newcounter{savedsection}
\newcommand{\subappendix}{\setcounter{savedsection}{\arabic{section}}\appendix}
\newcommand{\noappendix}{
  \setcounter{section}{\arabic{savedsection}}% restore section number
  \setcounter{subsection}{0}% reset section counter
%  \gdef\@chapapp{\sectionname}% reset section name
  \renewcommand{\thesection}{\arabic{section}}% make section numbers arabic
}

\newenvironment{items}[1][9]{
\par\setlength{\unitlength}{1pt}\fontsize{#1}{#1}\linespread{1.2}\selectfont
\begin{list}{--}{\leftmargin4ex \rightmargin0ex \labelsep1ex \labelwidth2ex
\topsep0pt \parsep0ex \itemsep3pt}
}{
\end{list}
}

\newenvironment{itemS}[1][4ex]{
\par
\tiny
\begin{list}{--}{\leftmargin#1 \rightmargin0ex \labelsep1ex
  \labelwidth2ex \topsep0pt \parsep0ex \itemsep2pt}
}{
\end{list}
}

\newcommand{\Def}[1]{%
\textbf{#1}\index{#1}}%\marginnote{#1}}

  \scripthead
}

\providecommand{\paper}{
  \input{../latex/style-paper}
  \paperhead
}

\providecommand{\note}[1][9pt]{
  \providecommand{\notehead}[2]{
  \documentclass[#1,fleqn,twoside]{article}
  \stdpackages
  \renewcommand{\labelenumi}{{(\roman{enumi})}}
  \renewcommand{\theenumi}{(\roman{enumi})} %for ref

  \renewcommand{\baselinestretch}{#2}
  \renewcommand{\arraystretch}{1.2}
  \renewcommand{\topfraction}{1}
  \renewcommand{\bottomfraction}{1}
  \renewcommand{\textfraction}{0}
  \columnsep 5ex
  \parindent 3ex
  \parskip 1ex

  % Lists and paragraphs
  \parindent 0pt
  \topsep 4pt plus 1pt minus 2pt
  \partopsep 1pt plus 0.5pt minus 0.5pt
  \itemsep 2pt plus 1pt minus 0.5pt
  \parsep 2pt plus 1pt minus 0.5pt
  \parskip .5pc %add _in_ {thebibliography} environment in *.bbl

  \setcounter{tocdepth}{3}
  \setcounter{secnumdepth}{3}

  \geometry{a4paper,hdivide={25mm,*,25mm},vdivide={25mm,*,25mm}}

  \renewcommand{\headrulewidth}{.0pt}\renewcommand{\footrulewidth}{.0pt}\cfoot{}
  \fancyhead[OL,EC]{\it\theauthor---\today}
  \fancyhead[ER]{\leftmark}
  \fancyhead[OR,EL]{\thepage}
  \fancyfoot[EL,OR]{}
  \setlength{\headsep}{10mm}
  %\fancyhead[OL]{\rightmark}
  %\fancyfoot[EL,OR]{}


  %  \usepackage{palatino}

  \newcommand{\codefont}{\helvetica{8}{1.2}{m}{n}}

  \renewenvironment{abstract}{
    \vspace*{5ex}\begin{list}{}{
      \leftmargin3ex
      \rightmargin3ex
      \topsep-\parskip}\item[]
     \hrule\vspace{1.5ex}{\bf Abstract.~}\small}
    {\vspace{2ex}\hrule\end{list}\vspace{5ex}}
    
  \newenvironment{keyword}
    {\par{\it Keywords:~}}
    {}

  \def\makemytitle{%
    \thispagestyle{empty}
    \begin{list}{}{\leftmargin3ex \rightmargin3ex \topsep0ex \parsep0ex}\item[]
      \begin{center}
        {\fontsize{18}{25}\selectfont{\thetitle\\}}\vspace{5ex}

        {\fontsize{14}{16}\selectfont{\theauthor\\}}\vspace{1ex}

        {\footnotesize{\sl \addressFUB}\\ \emailBerlin}

        {\footnotesize \today}

        \vspace{1ex}
        {\small \published}
      \end{center}
    \end{list}
    \renewcommand{\maketitle}{\chapter{\thetitle}}
  }

  \input{../latex/macros}
  \pdflatex

  \graphicspath{{pics/}{../pics/}{pics-local/}}

  \myauthor{Marc Toussaint}
  \date{\today}
}

%%%%%%%%%%%%%%%%%%%%%%%%%%%%%%%%%%%%%%%%%%%%%%%%%%%%%%%%%%%%%%%%%%%%%%%%%%%%%%%%

\newcommand{\notetitle}{
  \begin{document}
  \thispagestyle{empty}
    
  \maketitle

}

\newenvironment{items}[1][9]{
\par\setlength{\unitlength}{1pt}\fontsize{#1}{#1}\linespread{1.2}\selectfont
\begin{list}{--}{\leftmargin4ex \rightmargin0ex \labelsep1ex \labelwidth2ex
\topsep0pt \parsep0ex \itemsep3pt}
}{
\end{list}
}

  \notehead{#1}{1.1}
}

\providecommand{\course}{NO COURSE}
\providecommand{\coursepicture}{NO PICTURE}
\providecommand{\coursedate}{NO DATE}
\providecommand{\topic}{NO TOPIC}
\providecommand{\keywords}{}
\providecommand{\exnum}{NO NUMBER}
\providecommand{\teacher}{Marc Toussaint}

\providecommand{\stdpackages}{
  \usepackage{amsmath}
  \usepackage{amssymb}
  \usepackage{amsfonts}
  \allowdisplaybreaks
  \usepackage{amsthm}
  \usepackage{eucal}
  \usepackage{graphicx}
%  \usepackage{color}
  \usepackage{geometry}
  \usepackage{framed}
  \usepackage{xcolor}
  \definecolor{shadecolor}{gray}{0.9}
  \setlength{\FrameSep}{3pt}
  \usepackage{fancyvrb}
  \fvset{numbers=none,xleftmargin=5ex,fontsize=\small}

  \usepackage{pdfpages}

  \usepackage{multicol} 
  \usepackage{fancyhdr}
}

\providecommand{\addressUSTT}{
  Machine~Learning~\&~Robotics~lab, U~Stuttgart\\\small
  Universit{\"a}tsstra{\ss}e 38, 70569~Stuttgart, Germany
}

\providecommand{\addressTUB}{
  Learning~\&~Intelligent~Systems~Lab, TU~Berlin\\\small
  Marchstr. 23, 10587 Berlin, Germany
}


\renewcommand{\course}{Robot Learning}
\renewcommand{\coursepicture}{roblearn.png}
\renewcommand{\coursedate}{Summer 2024}
\renewcommand{\teacher}{Marc Toussaint}

\renewcommand{\topic}{TAMP \& Language}
\renewcommand{\keywords}{}

\slides

\ifthenelse{\isundefined{\scripthead}}{

\providecommand{\info}[1]{\smallskip{\ttiny [#1]\par}}

\usepackage{bibentry}
\nobibliography*

\ifthenelse{\isundefined{\setbeamertemplate}}{}{
  \setbeamertemplate{bibliography item}{\insertbiblabel}
}

\providecommand{\citehere}[1]{{\fontsize{5}{1}\selectfont\bibentry{#1}\par}}

}{}

\providecommand{\SE}{\text{SE}}
\renewcommand{\path}{{\text{path}}}
\renewcommand{\succ}{{\text{succ}}}
\newcommand{\goal}{{\text{goal}}}
\newcommand{\switch}{{\text{switch}}}
\newcommand{\pre}[1]{{\textsf{#1}}}
\newcommand{\rt}{{\mathcal{T}}}
\newcommand{\xv}{{\underline x}}
\newcommand{\secmpc}{{\sc SecMPC}}
\newcommand{\face}[2]{
\begin{minipage}{11mm}
\centering
\showh[1]{faces/#1}\\
\ttiny #2
\end{minipage}
}

\slidestitle

%%%%%%%%%%%%%%%%%%%%%%%%%%%%%%%%%%%%%%%%%%%%%%%%%%%%%%%%%%%%%%%%%%%%%%%%%%%%%%%%

\slide{Remaining Lectures}{

\item June 25: TAMP \& Language
\item July 2: Multi-Robot Learning
\item July 9: Robot Learning Discussion -- Lecture Feedback -- Exam Info

}

%%%%%%%%%%%%%%%%%%%%%%%%%%%%%%%%%%%%%%%%%%%%%%%%%%%%%%%%%%%%%%%%%%%%%%%%%%%%%%%%

\slide{Outline}{

\item Background on Task and Motion Planning (TAMP)

\item Learning in TAMP

\item Language in Robotics

\item LLMs \& TAMP

}

%%%%%%%%%%%%%%%%%%%%%%%%%%%%%%%%%%%%%%%%%%%%%%%%%%%%%%%%%%%%%%%%%%%%%%%%%%%%%%%%

\key{Task and Motion Planning}
\slide{Task and Motion Planning (TAMP) examples:}{

\tiny

\twocol{.5}{.4}{\centering

\movh[]{.7}{movies-marc/14-Mordatch-CIO}\\
Mordatch et al: CIO (SIGGRAPH'12)

\medskip

\movh[]{.7}{movies-marc/20-Garret-PDDLStream}\\
Garrett et al: PDDLStream (ICAPS'20)

}{\centering

\movh[]{.6}{movies-marc/RSS-concat600600}\\
Toussaint at al: LGP (RSS'18)

\medskip

\movh[]{.9}{movies-marc/21-valentingRSSsubmission}\\
Hartmann et al. (IROS 20)
%\movh[]{.4}{movies-marc/OpenAI-game-pushAround}\\
%\hfill OpenAI Hide \& Seek (arxiv'19)

}

%% \medskip

%% \normalsize
%% %% Mujoco?

%% \cen{\emph{What does it take to generate such behavior?}}

}

%%%%%%%%%%%%%%%%%%%%%%%%%%%%%%%%%%%%%%%%%%%%%%%%%%%%%%%%%%%%%%%%%%%%%%%%%%%%%%%%

\slide{Task and Motion Planning (TAMP)}{

~

\item What is the right level of ``abstraction'' to reason about
manipulation?

\pause

\begin{items}
\item Low-level motor commands? (Torques?)
\item Mid-level kinematic commands? (6D endeff target
position/velocity)
\item Actions/skills? (Pick, place, push, throw, hit, \emph{how long
is the list?})
\end{items}

}

%%%%%%%%%%%%%%%%%%%%%%%%%%%%%%%%%%%%%%%%%%%%%%%%%%%%%%%%%%%%%%%%%%%%%%%%%%%%%%%%

\slide{Abstractions}{

\item What does the AI/RL researcher say about abstractions?
\begin{items}
\item Hierarchical MDPs, Options, Hierarchical RL
\item (Classical AI: Landmarks in A* search)
\item Abstraction learning is hard:
\begin{items}
\item Given action primitives $\to$ state abstractions clear
(Konidaris' work)
\item Given state abstractions $\to$ action primitives clear (``skill
discovery'')
\item Classical ideas for state abstractions: identifying bottlenecks
(=doors in configuration space; McGovern, Barto 2001)
\end{items}
\item Modern view: Data-driven: Assume tons of demonstrations and
cluster-segment them
\end{items}

~\pause

\item What does the Roboticist say about abstractions?\pause
\begin{items}
\item Force level, motion level, task level
\item Task level: discrete symbolic state and actions (STRIPS/PDDL)
\end{items}

}

%%%%%%%%%%%%%%%%%%%%%%%%%%%%%%%%%%%%%%%%%%%%%%%%%%%%%%%%%%%%%%%%%%%%%%%%%%%%%%%%

\slide{STRIPS/PDDL}{

\show{task_planning_example}

\medskip

\begin{items}
\item A symbolic state $s_t$ is a set of grounded literals
\item A symbolic action operators defines a precondition and effect
\item Eventually, \textbf{his defines the set of possible successor states
$s_{t\po} \in \succ(s_t)$}
\end{items}

}

%%%%%%%%%%%%%%%%%%%%%%%%%%%%%%%%%%%%%%%%%%%%%%%%%%%%%%%%%%%%%%%%%%%%%%%%%%%%%%%%

\slide{Task and Motion Planning}{

\item Task-level is defined by
\begin{items}
\item symbols (predicates), objects (constants), and action operators
\item initial state $s_0$, goal sentence, action operators imply $\succ(s_t)$
\end{items}

\item Motion-level is defined by
\begin{items}
\item world configuration space $\XX$, goal configurations $\XX_\text{goal}\subseteq\XX$
\item feasible space $\XX_{s,\t} \subseteq\XX$ depending on logic state $s$
and \emph{entry point} $\t$ (action parameter)

\info{$\XX_{s,\t}$ is called \emph{foliation}, or multi-modal space ~
$\to$ ~ \textbf{multi-modal motion planning (MMMP)}}
\end{items}

\item Path-Finding formulation of TAMP:
\begin{items}
\item Find sequence of $(s_i,\tau_i)$ of symbolic states and 
continuous feasible paths $\tau_i$ that lead to goal:
\item Paths: $\tau_i: [0,1] \to \XX_{s_i,\t_i}$
\item Continuity: $\tau_i(0) = \tau_{i\1}(1)$
\item Entry points: $\t_i = \tau_{i\1}(1)$ (e.g.\ action parameter,
grasp, lower-dim feature of $\tau_{i\1}(1)$)
\item Goal: $s_K \models \text{goal}, \tau_K(1) \in \XX_\text{goal}$
\end{items}

\citehere{2021-garrett-IntegratedTaskMotion}

}


%%%%%%%%%%%%%%%%%%%%%%%%%%%%%%%%%%%%%%%%%%%%%%%%%%%%%%%%%%%%%%%%%%%%%%%%%%%%%%%%

\key{Logic-Geometric Program}
\slide{TAMP as Logic-Geometric Program (LGP)}{

\small

\hspace*{-5mm}\twocol[.05]{.55}{.4}{

\tiny
\begin{align*}
\hspace*{-5mm}\min_{s_{1:K} \atop x:[0,KT]\to \XX} & \int_0^{KT} c(\xv(t))~ dt  \\
\hspace*{-5mm}\st~& x(0)=x_0,~ \\%\phi_\goal(X_T)\le0,~\\
& \forall_{t\in[0,T]}:~
   \bar\phi(\xv(t), s_{k(t)})\le0 \\
& \forall_{k\in\{1,..,K\}}:~
  \hat\phi(\xv(t_k), s_{k\1}, s_k)\le0 \\
& s_K \models \text{goal},~ \forall_{k\in\{1,..,K\}}:~ s_k \in \succ(s_{k\1})
\end{align*}

\small

\item {Skeleton} $s_{1:K}$ defines {schedule of physical
   modes}

\item {Constraints} $\hat\phi, \bar\phi$ {define correct physics \textbf{differentiable}}


\ttiny\medskip

[inequalities subsume equalities; $\xv=(x, \dot x, \ddot x)$]

}{

\pause

\show[.75]{tampLogic}

~

\item Solving implies searching over $s_{1:K}$ and solving the corresponding NLP

}

\medskip

%% \cit{Toussaint}{Logic-Geometric Programming: An Optimization-Based Approach to Combined Task and Motion Planning}{IJCAI'15}

%% \cit{Toussaint, Lopes}{Multi-Bound Tree Search for Logic-Geometric Programming}{ICRA'17}

%% \cit{Toussaint, Allen, Smith, Tenenbaum}{Differentiable Physics and Stable Modes for Tool-Use and Manipulation Planning}{R:SS'18}

\citehere{2015-toussaint-LogicGeometricProgrammingOptimizationBased}

\citehere{2018-toussaint-DifferentiablePhysicsStable}
}

%%%%%%%%%%%%%%%%%%%%%%%%%%%%%%%%%%%%%%%%%%%%%%%%%%%%%%%%%%%%%%%%%%%%%%%%%%%%%%%%

\slide{renderings(!) of example solutions...}{

\threecol{.3}{.4}{.25}{\centering

%\vspace*{-5mm}
\movh[]{.9}{movies-marc/RSS-concat600600}%
\anchor{-40,-7}{\ttiny(R:SS 18)}

%~

%%\movh[]{.9}{videos/19-forceBased-pushWithStickFloat3_COMP}
\movh[loop]{.9}{videos/19-forceBased-pushWithStick-good_COMP}

}{\centering

\movh[loop]{.9}{movies-marc/21-valentingRSSsubmission}%
\anchor{-40,-7}{\ttiny(IROS 20)}
%20-IROS-BUGAassembly}

\medskip

\movh[loop]{.75}{videos/19-forceBased-liftRing-dynamic_COMP}%
\anchor{-40,-7}{\ttiny(IROS 20)}

}{\centering

~

\movh[loop]{.9}{movies-marc/20-DeepVisualReasoningData}%
\anchor{-40,-7}{\ttiny(R:SS 20)}


~

\movh[loop]{.9}{videos/19-banana-03}

~

%\movh[]{.9}{videos/19-forceBased-bookOnShelf2_COMP}

}

}

%%%%%%%%%%%%%%%%%%%%%%%%%%%%%%%%%%%%%%%%%%%%%%%%%%%%%%%%%%%%%%%%%%%%%%%%%%%%%%%%

\slide{Abstractions}{

~

\item What does ``LGP'' say about abstractions?
\pause
\begin{items}
\item There are two levels: the convex level (NLP), and the non-convex
(discrete decisions)
\end{items}

}

%%%%%%%%%%%%%%%%%%%%%%%%%%%%%%%%%%%%%%%%%%%%%%%%%%%%%%%%%%%%%%%%%%%%%%%%%%%%%%%%

\slide{Outline}{

\item Intro to Task and Motion Planning (TAMP)

\item \textbf{Learning in TAMP}

\item Language in Robotics

\item LLMs \& TAMP

}

%%%%%%%%%%%%%%%%%%%%%%%%%%%%%%%%%%%%%%%%%%%%%%%%%%%%%%%%%%%%%%%%%%%%%%%%%%%%%%%%

\key{Learning in TAMP}
\slide{Is model-based TAMP a dead end?}{

\item LGP formulates TAMP as model-based optimization problem
\begin{items}
\item Assumption of having a world model is unrealistic ~ (state
estimation from vision ill-posed...)
\item High computation time for large problems -- why plan from
scratch every time?
\end{items}

~\pause

\item Opportunities for learning:
\begin{items}

\item \textbf{Replace exact model by learned constraints $\phi(x)$}
\begin{items}
\item The LGP definition actually only needs constraints
$\phi(x)$, no explicit world model
\item Instead of hand-defining these from a model $\to$ image-conditional neural models $\phi_\t(x; \II)$
\end{items}

\item \textbf{Learn to predict plans}
\begin{items}
\item Instead of solving from scratch, learn to predict promising
actions $a_{1:K}$ from the scene image
\end{items}
\end{items}

}

%%%%%%%%%%%%%%%%%%%%%%%%%%%%%%%%%%%%%%%%%%%%%%%%%%%%%%%%%%%%%%%%%%%%%%%%%%%%%%%%

\slide{}{

\item Replace exact model by learned constraints  $\phi(x)$:

}

%%%%%%%%%%%%%%%%%%%%%%%%%%%%%%%%%%%%%%%%%%%%%%%%%%%%%%%%%%%%%%%%%%%%%%%%%%%%%%%%

\key{Constraints Learning}
\slide{}{

\show{22-ha1}

\medskip

\twocol[.05]{.45}{.45}{

\item Learn $\phi(x,\II)$ with $V$ input images $\II$ s.t.:
\begin{items}
\item $\phi(x; \II)=0$ $\iff$ $x$ is correct grasp
\item $\phi(x; \II)=0$ $\iff$ $x$ is correct hanging
\end{items}

}{

\small

\item Data generating in simulation:
\begin{items}
\item Collect trial-and-error data on correct grasps and hanging
\end{items}

%% \item (Pre-train backbone on shape reconstruction)

}

\medskip

\citehere{2022-ha-DeepVisualConstraintsa}

}

%%%%%%%%%%%%%%%%%%%%%%%%%%%%%%%%%%%%%%%%%%%%%%%%%%%%%%%%%%%%%%%%%%%%%%%%%%%%%%%%

\slide{Deep Visual Constraints: Network Architecture}{

\medskip\small

\twocol[.03]{.4}{.5}{

\showh[.9]{jungsu2}

~

\showh[.9]{jungsu1}

~

\citehere{2022-ha-DeepVisualConstraintsa}

}{

\item Camera views $\II = \{(I^1, K^1), ..., (I^V, K^V)\}$

Wanted: image-based constraint model

\cen{$\phi(x; \II)$}

~%\pause

\item First train a $d$-dimensional \textbf{field representation}

\cen{$y(p; \II) = \frac{1}{V}\Sum_i \text{MLP}(\text{UNet}(I^i, K^i(x)), K^i(x))$}

\medskip
{\tiny [$p\in\RRR^3$, pre-trained for shape decoding (SDF prediction)]}

~%\pause

\item Function is queried at finite set of \emph{interaction points} $p_1(x),..,p_K(x)$ to get the feature

\cen{$\phi(x;\II) = \text{MLP}(y(p_1(x);\II),..,y(p_K(x);\II))$}

\medskip
{\tiny [fine-tuned for manipulation success (trial \& error in sim)]}

}

}

%%%%%%%%%%%%%%%%%%%%%%%%%%%%%%%%%%%%%%%%%%%%%%%%%%%%%%%%%%%%%%%%%%%%%%%%%%%%%%%%

\slide{Deep Visual Constraints}{

{\tiny (No search over skeletons, no reactive MPC, just optimal path for given sequence of constraints.)}

~

%\twocol{.5}{.4}{

\movc[]{.45}{videos/jungsu/MugHaningDemo}

%% }{

%% \movc[]{.7}{videos/jungsu/training-concat}

%% }

}


%%%%%%%%%%%%%%%%%%%%%%%%%%%%%%%%%%%%%%%%%%%%%%%%%%%%%%%%%%%%%%%%%%%%%%%%%%%%%%%%

\slide{Similar: Learn Dynamics Constraints}{

%% \medskip

%% \cen{\showh[.45]{nerf1}\qquad\showh[.25]{nerf2}}

%% ~

\twocol[.05]{.45}{.45}{

~

\show[1]{nerfDyn1}

\citehere{2023-driess-LearningMultiobjectDynamicsa}
%% \face{Danny_Driess}{Danny Driess}\quad
%% $\cdots$\quad
%% \face{russ_tedrake}{Russ Tedrake}

{\urlfont\url{https://dannydriess.github.io/compnerfdyn/}}

~

\show[1]{nerfDyn3}

}{

\show[1]{nerfDyn2}

~\small

\item Each object has a latent code $z_i^t$

\item learn dynamics $z_{1:m}^t \mapsto z_i^{t\po}$!

}

}

%%%%%%%%%%%%%%%%%%%%%%%%%%%%%%%%%%%%%%%%%%%%%%%%%%%%%%%%%%%%%%%%%%%%%%%%%%%%%%%%

\slide{}{

\item Learning to predict plans..

}

%%%%%%%%%%%%%%%%%%%%%%%%%%%%%%%%%%%%%%%%%%%%%%%%%%%%%%%%%%%%%%%%%%%%%%%%%%%%%%%%

\key{Learning to predict plans}
\slide{}{

\twocol[.05]{.4}{.5}{

\show[1]{20-danny1}

\show{20-danny2}

~

\citehere{2020-driess-DeepVisualReasoning}

}{\small

\item Data collection $D = \{ \left(S^i, g^i, a^i_{1:K^i}, F^i\right) \}_{i=1}^n$
\begin{items}
\item with scene $S^i$, goal $g^i$, actions $a^i_{1:K^i}$,
feasibility $F^i$
\item random generated ``in simulation'', \textbf{model-based TAMP
solver used to label feasibility}
\end{items}

~

\item Train a sequential policy:

{\tiny
$\pi(a_k; g, a_{1:k\1}, S) =$\\
$P(\exists_{K>K}\exists_{a_{k\po:K}}: a_{1:K} \text{feasible} \| a_k,
g, a_{1:k\1}, S)$

}
\begin{items}
\item Similar to language model: Predict next
``token'' $a_k$ given previous $a_{1:k\1}$ conditional $g,S$
\end{items}

}

}

%%%%%%%%%%%%%%%%%%%%%%%%%%%%%%%%%%%%%%%%%%%%%%%%%%%%%%%%%%%%%%%%%%%%%%%%%%%%%%%%

\slide{Deep Visual Reasoning: Network Architecture}{

~

\showh[.4]{talk-LGP/danny1}~\showh[.55]{talk-LGP/danny2}

~\small

\item Uses RNN -- modern version would use transformer

\item Special encoding of predicates $\bar a, \bar g$ and references $O$ (as masks)

}

%%%%%%%%%%%%%%%%%%%%%%%%%%%%%%%%%%%%%%%%%%%%%%%%%%%%%%%%%%%%%%%%%%%%%%%%%%%%%%%%

\slide{Deep Visual Reasoning: Results}{

~

\twocol[.05]{.45}{.45}{

\movc[externalviewer]{1.}{movies-marc/20-deepVisualReasoning}

}{


\show[.8]{talk-LGP/forMarc/results1}

\show[.8]{talk-LGP/forMarc/results2}

~\tiny

\item Often, the first proposed action sequence is feasible

}

}


%%%%%%%%%%%%%%%%%%%%%%%%%%%%%%%%%%%%%%%%%%%%%%%%%%%%%%%%%%%%%%%%%%%%%%%%%%%%%%%%

\slide{Outline}{

\item Intro to Task and Motion Planning (TAMP)

\item Learning in TAMP

\item \textbf{Language in Robotics}

\item LLMs \& TAMP

}

%%%%%%%%%%%%%%%%%%%%%%%%%%%%%%%%%%%%%%%%%%%%%%%%%%%%%%%%%%%%%%%%%%%%%%%%%%%%%%%%

\key{Language in Robotics}
\slide{}{

\twocol[.05]{.45}{.45}{

\show[1]{20-tellex1}

~

\citehere{2020-tellex-RobotsThatUse}

}{

\small

\item Great survey on Natural Language Robot Interaction
\begin{items}
\item Using natural language to command robots, set tasks
\item Using natural language to instruct robots, e.g.\ as part of
demonstrations
\item Different to standard NLP or dialog systems: \textbf{language
needs to be physically grounded}
\end{items}

}

}

%%%%%%%%%%%%%%%%%%%%%%%%%%%%%%%%%%%%%%%%%%%%%%%%%%%%%%%%%%%%%%%%%%%%%%%%%%%%%%%%

\slide{Natural Language Robot Interaction: Examples}{

\medskip

\twocol[.05]{.45}{.45}{

\show[.9]{20-tellex3}

\ttiny from \cite{2020-tellex-RobotsThatUse}

}{

\tiny

\item robot asks for help
\item human sets task (with language \& gesture)
\item robot ``reads/comprehends'' wikihow
\item demonstrations via dialog
\item human sets task (nagivation)
\item ...
\item human sets task (object identification)
\item human sets task (navigation)
\item human sets task (manipulation)

}

}

%%%%%%%%%%%%%%%%%%%%%%%%%%%%%%%%%%%%%%%%%%%%%%%%%%%%%%%%%%%%%%%%%%%%%%%%%%%%%%%%

\slide{Natural Language Robot Interaction: Datasets}{

\twocol[.05]{.5}{.35}{

\show[.8]{20-tellex2}

}{

\tiny ``Data sets typically consist of natural language paired with some form of sensor-based context information about the physical
environment''

}

}

%%%%%%%%%%%%%%%%%%%%%%%%%%%%%%%%%%%%%%%%%%%%%%%%%%%%%%%%%%%%%%%%%%%%%%%%%%%%%%%%

\slide{}{

\small

\item Previous survey highlights substantial literature on Natural Language Robot
Interaction \emph{before} rise of LLMs

\medskip

\tiny Example: {\urlfont\url{https://youtu.be/VqSb-ZZuIwI?t=2523}}

}

%%%%%%%%%%%%%%%%%%%%%%%%%%%%%%%%%%%%%%%%%%%%%%%%%%%%%%%%%%%%%%%%%%%%%%%%%%%%%%%%

\key{Language-Image Models (CLIP, CLIPort, SayCan, PaLM-E, RT-2)}
\slide{CLIP (Contrastive Language-Image Pre-training)}{

~

\twocol[.05]{.45}{.45}{

\show[1]{clip1}

~

\citehere{2021-radford-LearningTransferableVisual}

}{

\tiny ``We demonstrate that the simple pre-training task of predict-
ing which caption goes with which image is an
efficient and scalable way to learn SOTA image
representations from scratch on a dataset of 400
million (image, text) pairs collected from the internet.''

~

\show[1]{CLIP1}

}

~

\info{Contrastive Training: ``maximize the cosine similarity of the
image and text embeddings of the $N$ real pairs in the batch while
minimizing the cosine similarity of the embeddings of the $N^2 - N$ incorrect pairings.}

}

%%%%%%%%%%%%%%%%%%%%%%%%%%%%%%%%%%%%%%%%%%%%%%%%%%%%%%%%%%%%%%%%%%%%%%%%%%%%%%%%

\slide{CLIPort}{

~

\twocol[.05]{.45}{.45}{

\show{CLIPort1}

~

\citehere{2022-shridhar-CliportWhatWhere}

{\urlfont\url{https://cliport.github.io/}}

}{

\tiny

``CLIPort: a language-conditioned imitation-learning agent that combines the broad semantic understanding (what) of CLIP
with the spatial precision (where) of Transporter''

}

~

\item Trains a policy $\pi: (y_i, l_l) \mapsto a_t$
\begin{items}
\item top-down orthographic RGB-D $y_t$, language
instruction $l_t$, pick-n-place 2D coordinates $a_t$
\end{items}

}

%%%%%%%%%%%%%%%%%%%%%%%%%%%%%%%%%%%%%%%%%%%%%%%%%%%%%%%%%%%%%%%%%%%%%%%%%%%%%%%%

\slide{SayCan}{

~

\twocol[.05]{.45}{.45}{

Do As I Can, Not As I Say:
Grounding Language in Robotic Affordances

~

\citehere{2023-brohan-CanNotSay}

{\urlfont\url{https://say-can.github.io/}}

}{

\show[1]{SayCan2}

}

~

\item Use a LLM (PaLM) to predict \emph{multiple} actions (with
probabilities)
\item Multiply each option with \emph{affordance prediction}
(= probability of success)

}

%%%%%%%%%%%%%%%%%%%%%%%%%%%%%%%%%%%%%%%%%%%%%%%%%%%%%%%%%%%%%%%%%%%%%%%%%%%%%%%%

\slide{PaLM-E}{

~

\twocol[.05]{.45}{.45}{

\show[1]{palme1}

~


\citehere{2023-driess-PaLMEEmbodiedMultimodala}

{\urlfont\url{https://palm-e.github.io/}}

}{

\show[.8]{palme-LLM}

~\small

\item Input: \emph{Multi-modal sentence:}
\begin{items}
\item Interleaves words, images (with segmentation), vectors, reference-keywords
\item All token-encoded
\item Various image encodings (ViT, object-centric ViT, OSRT, NeRFs pre-trained)
\end{items}

\item Output:
\begin{items}
\item Sequences of action primitives (previously trained, RT-1)
\end{items}

}

}

%%%%%%%%%%%%%%%%%%%%%%%%%%%%%%%%%%%%%%%%%%%%%%%%%%%%%%%%%%%%%%%%%%%%%%%%%%%%%%%%

\slide{}{

~

\tiny

\twocol{.45}{.45}{\centering

\movh[]{.7}{videos/23-palme-planning_4x_compressed}

``Bring me the rice chips from the drawer''

~

\movh[]{.7}{videos/23-palme-green_star}

``Bring me the green star''

}{\centering

\movc[]{.7}{videos/23-palme-red_blocks_to_coffee}

``Push red blocks to the coffee cup''

~

\movh[]{.7}{videos/23-palme-green_blocks_to_turtle}

``Push green blocks to the turtle''

}

}

%%%%%%%%%%%%%%%%%%%%%%%%%%%%%%%%%%%%%%%%%%%%%%%%%%%%%%%%%%%%%%%%%%%%%%%%%%%%%%%%

\slide{Example input/output}{

~\small

\item Prompt:\quad \texttt{Given <img>. Q: How to grasp the green object?.}

Target:\quad \texttt{A: First grasp the orange object and place it on the table, then grasp the green object.}

~

\item Prompt:\quad \texttt{Given <img>. Q: How to stack the white object on top of the red object?.}

Target:\quad \texttt{A: First grasp the green object and place it on the table, then grasp the white object and place it on the red object.}

}

%%%%%%%%%%%%%%%%%%%%%%%%%%%%%%%%%%%%%%%%%%%%%%%%%%%%%%%%%%%%%%%%%%%%%%%%%%%%%%%%

\slide{PaLM-E Evaluations}{

\small

\item Data sets:
\begin{items}
\item \textbf{TAMP data (generated by our LGP-TAMP planner)}
\item Table data (previous RT1 paper)
\item SayCan data
\item Other visual/language data: WebLI, VQA, COCO, etc.
\end{items}

\item Pre-taining:
\begin{items}
\item LLM backbone: language, VQA (WebLI, VQA, COCO)
\item Encodings: reconstruction, auto-encoding
\end{items}

\item Ablation studies:
\begin{items}
\item Varying transformer sizes
\item generalization (to unseen object situations, esp.\ higher number of objects)
\item freezing, refining, full-learning of backbone LLM or encodings
\item with full/partial choice of data sets \& sizes
\item various image encodings
\end{items}

}

%%%%%%%%%%%%%%%%%%%%%%%%%%%%%%%%%%%%%%%%%%%%%%%%%%%%%%%%%%%%%%%%%%%%%%%%%%%%%%%%

\slide{PaLM-E evaluations}{

\twocol{.45}{.45}{\centering

\showh[.9]{palme-ex1}

~

\showh[.9]{palme-ex3}

}{\centering

\showh[.9]{palme-ex2}

~

\showh[.9]{palme-ex5}

}

%\showh[.9]{palme-ex4}

}


%%%%%%%%%%%%%%%%%%%%%%%%%%%%%%%%%%%%%%%%%%%%%%%%%%%%%%%%%%%%%%%%%%%%%%%%%%%%%%%%

\slide{Follow Up: RT-2}{

~

\twocol[.05]{.45}{.45}{

\show{RT2-1}

~

\citehere{2023-zitkovich-Rt2VisionlanguageactionModels}

}{

\show[1]{RT2-2}

~\tiny

\item quasi-continuous actions (trained end-to-end):

\show[1]{RT2-3}

}

}

%%%%%%%%%%%%%%%%%%%%%%%%%%%%%%%%%%%%%%%%%%%%%%%%%%%%%%%%%%%%%%%%%%%%%%%%%%%%%%%%

\slide{Conclusion}{\label{lastpage}

\item Levels of abstraction: Force, motion, task

\item Task and Motion ``Planning'': Core problem formulation of
robotic AI
\begin{items}
\item TAMP theory \& solvers are fully model-based
\item Clear opportunities for learning: constraint learning, learning
to predict plans
\end{items}

\item Language $\oto$ task \& action level
\begin{items}
\item Lots of classical literature on \emph{language grounding}
\item Connecting natural language with typical robot task descriptions (STRIPS/PDDL)
\end{items}

\item Huge recent focus on marrying LLMs + TAMP + robotics

}

%%%%%%%%%%%%%%%%%%%%%%%%%%%%%%%%%%%%%%%%%%%%%%%%%%%%%%%%%%%%%%%%%%%%%%%%%%%%%%%%

\ttiny
\ifthenelse{\isundefined{\scripthead}}{
\bibliographystyle{plainurl-lis}
\bibliography{b7-TampLearning}
}{}

\slidesfoot
