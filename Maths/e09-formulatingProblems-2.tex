%%%%%%%%%%%%%%%%%%%%%%%%%%%%%%%%%%%%%%%%%%%%%%%%%%%%%%%%%%%%%%%%%%%%%%%%%%%%%%%%

\exsection{Network flow problem}

\textbf{Comment:} Solving problems in the real world involves 2 parts:
\begin{enumerate}
\item[1)] formulating the problem as an optimization problem ($\to$ human)
\item[2)] the actual optimization problem ($\to$ algorithm)
\end{enumerate}
The first part is perhaps more interesting -- that's the science
part. But understanding the algorithms helps to better formalize
problems. This and the following exercises are from Boyd \&
Vandenberghe's book
\url{http://www.stanford.edu/~boyd/cvxbook/bv_cvxbook.pdf}. They are only about the first part.


Solve Exercise 4.12 (pdf page 207) from  Boyd \& Vandenberghe,
\emph{Convex Optimization.}

%%%%%%%%%%%%%%%%%%%%%%%%%%%%%%%%%%%%%%%%%%%%%%%%%%%%%%%%%%%%%%%%%%%%%%%%%%%%%%%%

\exsection{Minimum fuel optimal control}

Solve Exercise 4.16 (pdf page 208) from  Boyd \& Vandenberghe,
\emph{Convex Optimization.}

%%%%%%%%%%%%%%%%%%%%%%%%%%%%%%%%%%%%%%%%%%%%%%%%%%%%%%%%%%%%%%%%%%%%%%%%%%%%%%%%

\exsection{Voluntary: Reformulating Norms}

Solve Exercise 4.11 (pdf page 207) from  Boyd \& Vandenberghe,
\emph{Convex Optimization.}

%%%%%%%%%%%%%%%%%%%%%%%%%%%%%%%%%%%%%%%%%%%%%%%%%%%%%%%%%%%%%%%%%%%%%%%%%%%%%%%%

\exsection{Voluntary: Some more examples}

a) Grocery Shopping: You're at the market and you find $n$ offers,
each represented by a set of items $A_i$ and the respective price
$c_i$. Your goal is to buy at least one of each item for as little as
possible. Formulate as an integer LP.

b) Facility Location: There are $n$ facilities with which to satisfy the needs of $m$ clients.  The
cost for opening facility $j$ is $f_j$, and the cost for servicing client $i$
through facility $j$ is $c_{ij}$.  You have to find an optimal way to open
facilities and to associate clients to facilities. Formulate as an ILP.

% in exam
%% c) Taxicab Driver: You're a taxicab driver in hyper-space ($\RRR^d$)
%% and have to service $n$ clients.  Each client $i$ has a known initial
%% position $c_i\in\RRR^d$ and a destination $d_i\in\RRR^d$.  You start
%% out at position $p_0\in\RRR^d$ and have to service all the clients
%% while minimizing fuel use, which is proportional to covered distance.
%% Hyper-space is funny, so the geometry is not Euclidean and distances
%% are Manhattan distances. Formulate as an LP.
